\chapter*{Introduction}

Algebraic $K$-Theory encompasses a variety of mathematical disciplines, natural
settings, questions and tools, and thus also motivations. As an algebraic topologist
I primarily use the motivation by chromatic homotopy theory. I refer the interested
reader to \cite{Rog14} for a general introduction to the red-shift conjecture, while 
focussing on the aspects which centrally motivate my thesis here.

Since algebraic topology formed as an independent mathematical discipline, a major
aspect of it is the study of invariants associated to topological spaces. The most prominent
and easily defined of these invariants are the homotopy groups associated to a topological
space. The higher homotopy groups, i.e., for $n\geq2$, 
give a countable family of abelian groups associated to a space. These
however suffer from the defect of being very hard to compute. Even more drastically,
it is usually a very hard question to establish, if a homotopy group of some fixed degree
is trivial or non-trivial for some given space, unless elementary arguments
force its triviality.

Much easier to compute, but harder to define are (singular) homology groups, defined by 
a variety of constructions up to the 1950s, proved to coincide by an axiomatic approach by 
Eilenberg and Steenrod in 1945. In the 1950s people found that there are more
constructions satisfying all except one of the axioms, thus establishing topological
$K$-theory and bordism as ``extraordinary'' homology theories. 

The Atiyah-Hirzebruch spectral sequence in principle makes it possible to compute
any homology theory on any space, given only its singular homology with $\Z$-coefficients
as input. However, chromatic homotopy theory establishes
that singular homology is the least complex of all homology theories. In particular,
it gives a conceptual reason for the difficulties one encounters, when one tries to actually
fully calculate the Atiyah-Hirzebruch spectral sequence for specific spaces. Cobordism is
at the other extreme, having chromatic complexity $\infty$ in a conceptually satisfying sense. 

In particular one could hope for an iterative approach to understanding invariants on a 
topological space by starting with singular homology on this space, thus at complexity
$0$, and then iterating from complexity $n$ to $n+1$. Chromatic
red-shift as described for instance in \cite{Rog14} is the conjecture that one way to
produce such intermediary theories of increasing complexity is given by 
iterating algebraic $K$-theory on the associated spectra. 

Suslin has established in 1984 that for any separably closed field $F$ its algebraic 
$K$-theory completed at primes $p$ (other than its characteristic) is equivalent to
topological complex $K$-theory. Thus the singular homology with coefficients in $F$
being a theory of complexity $0$ is transformed into a theory of complexity $1$.
By the results of Christian Ausoni, specifically the ones in \cite{AuKku} we know
that $K(ku_p)$ is the spectrum associated to a cohomology theory of complexity $2$.
In general chromatic red-shift predicts that this is part of a pattern, saying that
algebraic $K$-theory raises chromatic complexity by one.

This thesis is specifically concerned with the involutive structures present on these
spectra. Singular homology arises as the Eilenberg-MacLane spectrum of a discrete
rig category, while \cite{Ri2010} shows that on algebraic $K$-theory objects we always
have an involution induced by transposing and inverting matrices. On complex $K$-theory
this involution specialises to the natural involution induced by complex conjugation.
One chromatic step higher this involution describes the operation on $2$ vector bundles
\cite{BDR2004}, which conjugates each transitional vector bundle. On $K$-theory of
complex $K$-theory we can describe this as the involution induced by transposition
and inversion in both iterations of $K$-theory.

In Chapter 1 I recall the most prominent combinatorial models for connective spectra given
by ringlike categories, specifically bimonoidal and bipermutative categories, while in
addition recalling the results of \cite{EM} delooping their classifying spaces. I rewrite
the construction of \cite{EM} in such a way that I can easily generalise it in Chapter 3 to
bicategories.

In Chapter 2 I establish a multiplicative structure on a combinatorial model for modules
over a bipermutative category as already studied additively by Ang\'elica Osorno in
\cite{Os}. I tie in the multiplicative structure with her additive structure in a manner
that makes this module bicategory a ringlike object again.

In Chapter 3 -- the technical core of this thesis -- I set up a multiplicative delooping
analogous to \cite{EM} for permutative bicategories, which generalises the one of \cite{Os},
while allowing a multiplication to be induced by the tensor product defined in chapter 2. 

In Chapter 4 I offer a few partial results, essentially summaries of known results in
various papers, pertaining to the uniqueness of such structures. The delooping of a
module category of a permutative category is sufficiently unique to fix the spectrum
by minimal data as observed by May and Thomason in 1978 \cite{MT1978}. However, I was
unfortunately unable to prove multiplicative uniqueness of this delooping, which would
as a corollary imply that the multiplicative structure of chapter 3 is the same as the
one obtained by iterating the construction of \cite{EM} twice. I do outline two
arguments by which one might approach this conjectural uniqueness.

Chapters 5 and 6 are concerned with the motivating calculational example $K(ku)$. In particular,
since the calculations of \cite{AuKku} are done by trace methods, i.e., by computations along
the natural map $K\rightarrow THH$, I recall the definition of topological Hochschild homology
in chapter 5. Fixing conventions along the way, specifically how an involution on a ring spectrum
induces one on its topological Hochschild homology, we find that the trace is compatible with
the involutions defined in chapters 2 and 5. 

Compare this to the introduction of Dundas \cite{D98}, where he states that the construction of the
trace map in the context of \cite{D98} 
is compatible with involutions induced by the appropriate functors. Thus the result in
chapter 5 establishes that we have internalised the involution in chapter 2 on $K$-theory
and in chapter 5 on topological Hochschild homology in a compatible way.

Finally in Chapter 6 I retrace the calculations of \cite{AuKku} to the extent that I can
establish the effects of the involution on $K(ku)$ on classes, which are not in the kernel
of the trace map.

\section*{Acknowledgements}
Obviously such a thesis cannot be written in a social vacuum, so I want to take this opportunity
to thank a few people.

First and foremost obviously my advisor, Birgit Richter: In addition to creating a safety, from
which I could explore the multiplicative delooping in chapters 2 and 3, she quite regularly helped
me with indispensable advice when I became disoriented mathematically or socially.

I gratefully acknowledge the funding and environment of the Research Training Group 1670. In addition
to the fact that I was able to focus exclusively on my research in the first three years of my
postgraduate studies, being a part of a big group of PhD students made many experiences seem less 
lonely, which would have been frustrating otherwise. Furthermore being entrusted with the position
of spokesperson for the PhD students I gained valuable insight in how to obtain a
group opinion efficiently and also had an in-depth look at the process of application talks, which
made them much more transparent, hence in particular less frightening to me.

It was plain fortunate that in 2012 at the Arolla conference I had the opportunity to meet with and
talk to Robert Bruner, who explained to me the Adams spectral sequence and how to effectively calculate
with it so pleasantly and patiently that I still remember each minute of it fondly.
Also I want to thank Christian Ausoni for giving me a nice reader's guide to his papers, which helped
a lot in reading them correctly and efficiently, thus kickstarting my knowledge of $K(ku)$.

This section could never be complete without mentioning Stephanie Ziegenhagen. I agree it
was a perfect match, when our paths crossed in Algebraic Topology I, which is now 8 years ago. Since
then we have been through so many things together, and experienced a lot of things so integrally 
together that I could never imagine them without you. As this is probably the last thesis, in which
we address acknowledgements to each other, I want to close the circle with: \emph{May the force of the
universal property always save us from coordinates.} :)

Finally I want to extend special thanks to people, who read the draft in various stages of completion:
My parents for catching numerous errors and typos, Birgit Richter in particular for pruning the 
nonsense I had written, Fabian Kirchner for his careful check of the mathematical prose, and finally
Stephanie Ziegenhagen for her fail-safe instinct to find undefined and confusing parts, and attach
the criticism down to the word or letter that causes the confusion.
