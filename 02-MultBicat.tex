\chapter{Bicategories of Matrices}\label{modulbicat}
\section{Osorno's Delooping of $\M$}
Given a permutative category that has a compatible associative multiplication,
one can define its module bicategory \cite{Os}. Furthermore, Osorno
provides a delooping of this bicategory by considering block sums of
matrices and organising these into a $\Gamma$-structure on the module
bicategory. This leads to an associated spectrum given a permutative
bicategory. 

I extend Osorno's result in a multiplicative manner. 
This means I define the bicategory-analogue of bipermutative
categories in this chapter and adapt Osorno's delooping in a 
manner that it has an induced multiplicative structure in the 
next chapter, leading to an $E_\infty$ symmetric ring spectrum.

Given a bimonoidal category $(\R,\oplus,\otimes)$ one can define its
bicategory of matrices as follows:
{\defn{The \textbf{bicategory of matrices} $\M$ associated to a bimonoidal
category $\R$ is given as follows. It has as objects the natural numbers 
$n\in\N_0$, and its morphism categories are:\[\M(n,m):=\begin{cases}
GL_n\R, ~~~&n=m,\\ \emptyset,  & \mathrm{else,}\end{cases}\] with 
$GL_n\R$ the categories of weakly invertible $n\times n$-matrices 
over the bimonoidal coefficients $\R$ (cf. \cite{Os,Ri2010,BDR2004}}).}

For this category to have an associator it is vital that the 
distributivity morphisms of $\R$ are \emph{isomorphisms} 
(cf. equation (4) on p.323 of \cite{Ri2010})! 

In what follows I need that for each bimonoidal category the 
bipermutative category $\Sigma_*$ is part of the bicategory $\M$ 
in a well-behaved way:
{\prop{\label{symmMatr}Consider the map \[\begin{aligned}E_{\bullet}
\colon& \Sigma_n&\rightarrow &~GL_n\R & \sigma \mapsto E_\sigma,\end{aligned}\] 
with $(E_\sigma)_{ij}:=\delta_{i,\sigma j}$. This map satisfies 
$E_{\sigma \tau} = E_{\sigma}E_{\tau}$. It is a faithful 
functor between monoidal categories.

Furthermore there is an action on general matrices: \[(E_\sigma A 
E_\tau)_{ij}=A_{\sigma^{-1}i,\tau j},\] hence in particular: 
\[(E_{\sigma^{-1}}AE_{\sigma})_{ij}=A_{\sigma i, \sigma j} 
\label{indexPerm}.\]}}

Structurally more satisfactory we get the following embedding.
{\prop{\label{strictAss}The category of finite sets as described in 
Example \ref{Fin} includes into the bicategory of modules for each coefficient
category $\R$: \[\Sigma_* \rightarrow \M.\] More explicitly: Consider 
$\Sigma_*$ as a bicategory with discrete morphism categories, then for
each coefficient category $\R$ we get a strict normal functor 
$E_\bullet\colon \Sigma_* \rightarrow \M,$ i.e., $E_\bullet$ strictly
respects identities and compositions \ref{strictn}. \begin{proof} I only give the 
indication of why this is true in my setup. The essential 
point is the strictness of $0$ and $1$ in the coefficient category as 
units, as well as the strict equality $0\cdot a = 0$.\end{proof}}}

This has the following extremely convenient corollary.
{\cor{For each (small) coefficient category $\R$ its module bicategory 
$\M$ has a sub-bicategory which is the faithful image of $E_\bullet$, 
in particular, this sub-bicategory is a $2$-category, so the 
associator restricts to the identity there.}}

{\rem{With these results it is just a minor abuse of notation to 
identify permutations with their images in $\M$, hence I write 
$\sigma = E_\sigma$. In particular the identity matrix of an object
$n\in\M$ is in the image of $E$ and I write $E_n=E_{id_n}$ for the
unit matrix.}}

In \cite{Os} Osorno established that the direct sum of matrices equips 
this bicategory with a well-behaved permutative structure 
\cite[Theorem 4.7]{Os}, and the main result of the paper 
\cite[Theorem 3.6]{Os} states that this can be delooped just as the
classical case \cite{Seg}.

{\thm[{\cite[Theorem 4.7]{Os}}]{\label{OsPlus} The bicategory of matrices $\M$ 
is strictly symmetric monoidal with respect to the \textbf{block sum} of 
matrices: \[\begin{aligned}\boxplus\colon&\M\times\M &\rightarrow&~\M\\
& (n,m)&\mapsto & ~n+m\\&(A,B)&\mapsto &\left(\begin{array}{c|c}A&0\\
		    \hline0&B\end{array}\right).\end{aligned}\]
The symmetry is just the one given by the functor $E_\bullet$ 
defined in Proposition \ref{symmMatr} from $\Sigma_*$ (cf. Example \ref{Fin}), 
i.e.: \[\Sigma_{n+m}\ni c^+_{n,m}=c_+\colon n\boxplus m \rightarrow m\boxplus n .\]}}

This symmetric monoidal structure exhibits the classifying space of
$\M$ as an infinite loop space.

{\thm[{\cite[Theorem 3.6]{Os}}]{\label{plusdeloop} Let $\mathcal{M}$ be a strict 
symmetric monoidal bicategory. Then there is a special $\Gamma$-bicategory 
$\widehat{\mathcal{M}}$ such that:\[\widehat{\mathcal{M}}(1)\cong 
\mathcal{M}.\] Therefore the classifying space 
$|N\mathcal{M}|$ is an infinite loop space upon group completion.}}

I elaborate on the permutative structure and the delooping further in \ref{Mult} 
once my multiplicative matters are in place. In 
particular my main result of this chapter is the following.
{\thm{Given a bipermutative coefficient category $(\R,\oplus,\otimes)$
there are two permutative structures $\boxplus,\boxtimes$ on its 
module bicategory $\M$ that can be arranged into a bipermutative 
bicategory.}}

This bipermutative structure can then be fitted onto the delooping of
$\M$, such that the result is an $E_\infty$-ring spectrum. This is
the content of the next chapter \ref{multbidel}.

{\thm{There is an $E_\infty$ symmetric ring spectrum $H\M$, which is weakly 
equivalent to the spectrum of Osorno's $\Gamma$-space 
$|N\widehat{\M}|$, with multiplication induced by the 
multiplicative structure of $\M$.}}

{\rem{The embeddings of the symmetric groups are compatible with 
direct sum of matrices in the nicest possible way: \[\xymatrix{
\Sigma_n\times \Sigma_m\ar[r]^\sqcup \ar[d] & \Sigma_{m+n}\ar[d]\\
GL_n\R\times GL_m\R \ar[r]^-\boxplus & GL_{n+m}\R.}\] So we have
$\sigma\boxplus\tau=\sigma\sqcup \tau,$ i.e., considering 
permutations as matrices yields that their direct sum is the
disjoint union.}}

\section{Definition - Symmetric Monoidal and Permutative Bicategories}
In what follows I need two types of symmetric monoidal structures.
One is the $E_\infty$-structure which we deloop, thus the one thought
of as additive. The other one gives the induced $E_\infty$-multiplication
on the delooping.

For convenience I use the shorthand $Ob\C=\C_0$. For $1$-cells, 
i.e., objects of morphism categories, when I do not want to refer 
to their source and target I use $Mor\C=\C_1$.

By weakening the definition of $2$-categories to bicategories one has an assortment
of ways how monoidality can be defined for a bicategory. Apart from the weakenings
of unit axioms, one can
impose associativity up to isomorphism, and varying degrees of symmetry. For 
bicategories there is one more degree of symmetry in addition to ``associative,
braided,'' and ``symmetric'', which is called ``sylleptic''. For a detailed discussion
of these notions, as well as a guide to the 7 sources, which incrementally built the
notion of ``monoidal bicategory'' in a fully weakened version, I defer to the PhD thesis
of Christopher Schommer-Pries \cite{SP09}. The original fully weakened definition of 
braided monoidal bicategories goes back to Kapranov and Voevodsky in \cite{KV} as
braided Gray monoids.

{\defn{\label{permbic} A \textbf{permutative bicategory} $(\C,+,0,c_+)$ is a 
bicategory with strict identities, a strict normal \ref{strictn} functor \[+\colon 
\C\times\C \rightarrow \C,\] a chosen additive unit $0\in\C_0$ and a 
strict natural transformation\[c_+\colon +\circ tw\Rightarrow +\] 
for $tw\colon \C\times\C\rightarrow \C\times\C$ the strict isomorphism,
which exchanges factors. 

These satisfy the following identities: \begin{itemize} \item Adding 
$0$ is strictly equal to the identity functor on $\C$: \[0+\_=\_+0=
id_\C.\]\item Addition is strictly associative, i.e., we have an 
equality of functors \[(\_+\_)\circ((\_+\_)\times id) =(\_+\_)\circ(id
\times (\_+\_)),\] \end{itemize} giving a 
well-defined strict normal $n$-fold sum functor for each $n\geq 0$:
\[\sum_n\colon \C^{\times n}\rightarrow \C.\]
Additionally the additive twist has to make the following 
diagrams strictly commutative for every $a,b,c\in\C_0$:
\begin{itemize}\item[]\[\begin{array}{cc} {\xymatrix{a+b+c \ar[r]^{
c_++id_c}\ar[dr]_{c_+} & b+a+c\ar[d]^{id_b+c_+}\\&b+c+a}}&{\xymatrix{
a+b+c\ar[r]^{id_a+c_+}\ar[dr]_{c_+}& a+c+b\ar[d]^{c_++id_b}\\&c+a+b}}
\end{array}\]\item[] \[\xymatrix{a+b \ar@{=}[rr]\ar[dr]_{c_+}&&a+b\\
&b+a,\ar[ur]_{c_+} & }\]\end{itemize} giving a 
unique (strict) natural transformation for every $n\in\N_0$ and 
$\sigma\in\Sigma_n$: \[c_\sigma\colon \sum_n\circ\left({\sigma\colon 
\C^{\times n} \rightarrow \C^{\times n}}\right)\Rightarrow \sum_n\] 
built from composites of $c_+$.\label{uniquecplussigma}}}

{\rem{This is a maximally strictified version of the definition of ``strict symmetric monoidal''
	Ang\'elica Osorno uses in \cite[Definition 3.1]{Os}. She considers more generally a monoidal
	product, which is just a pseudofunctor, as well as a symmetry, which is just pseudonatural.
	Thus in addition to the assumption of strict identities, the strict functoriality of $+$ and
	the strict naturality are stronger conditions to impose.}}

{\defn{\label{mdlbic} A \textbf{symmetric monoidal bicategory} 
$(\C,\cdot,1,c_\cdot)$ consists of a bicategory with strict units 
$\C$, a pseudofunctor \[(\cdot,\Phi)\colon \C\times\C
\rightarrow \C,\] a chosen unit-object $1\in\C_0$, a strong 
pseudonatural transformation \[(c^1_\cdot,c^2_\cdot)\colon \cdot\circ 
tw\Rightarrow \cdot,\] satisfying the following identities:
\begin{itemize}\item Multiplying with $1$ is strictly equal to the 
identity functor on $\C$: \[1\cdot \_=\_\cdot 1=id_\C.\] 
\item Multiplication is strictly associative, i.e., we have a 
strict equality of functors (and their compositors): 
\[(\_\cdot\_)\circ((\_\cdot\_)\times id)=(\_\cdot\_) \circ(id \times 
(\_\cdot\_)),\] giving a well-defined pseudofunctor:
\[(\prod_n,\Phi^{\prod_n})\colon \C^{\times n}\rightarrow \C.\]
In addition the multiplicative twist makes the following
diagrams strictly commute for every $a,b,c\in\C_0$: 
\[\begin{array}{ccc} {\xymatrix{abc \ar[r]^{c_\cdot\cdot id_c}
\ar[dr]_{c_\cdot} & bac\ar[d]^{id_b\cdot c_\cdot}\\&bca}}&{\xymatrix{
abc\ar[r]^{id_a\cdot c_\cdot}\ar[dr]_{c_\cdot}& acb\ar[d]^{c_\cdot
\cdot id_b}\\&cab}}& {\xymatrix{ab \ar@{=}[r]
\ar[dr]_{c_\cdot}  & ab \\  &ba,\ar[u]_{c_\cdot}}}\end{array}\]
which means in more detail that the functors as well as their 
compositors coincide. In particular ${c^1}_\cdot$ squares to the 
identity transformation with identity two-cell, hence ${c^2}_\cdot$
has to square to the identity as well. Again this implies that we have
a unique strong pseudonatural transformation for every $n\in\N_0,
\sigma\in\Sigma_n$:\[c_\sigma\colon \prod_n\circ\left({\sigma\colon 
\C^{\times n} \rightarrow \C^{\times n}}\right)\Rightarrow \prod_n\]
built from composites of $(c^1_\cdot,c^2_\cdot)$.\end{itemize}}}

{\rem{The above notion is precisely the notion of ``strict symmetric monoidal'' Osorno
	gives in \cite{Os} apart from my standing assumption on strict identity $1$-cells
	in the underlying bicategory. Since I consider no other symmetric monoidal structures
	on bicategories than the ones given by block sum and tensor-product on $\M$, I choose
	to drop the attribute ``strict'', since the essential difference to ``permutative'' is
	the \textbf{non-strictness} of the monoidal functor $(\cdot,\Phi)$.}}

{\rem{Do note that the condition that $c^1$ squares to the identity
implies that it is an isomorphism $1$-cell, not just an equivalence as
one might guess.}}

{\rem{In chapter \ref{multbidel} I can much more easily generalise the 
Grothendieck construction as I defined it in \ref{c+}, since the 
additive structure of $\M$ is even a bit stricter than Osorno 
axiomatised, making her delooping apply to more general monoidal
bicategories than mine does. 

I chose the symbols before incorporating the intuition
that I think of permutative structures as additive structures, which we deloop,
while symmetric monoidal structures can potentially give superimposed
multiplications on the delooping.}} 

\section{The Multiplicative Structure on $\M$} \label{Mult} Just as 
in the classical case of commutative rings one should expect the 
module category of a bipermutative category to have a multiplicative 
structure analogous to the tensor product of modules. Since for 
combinatorial reasons I decided to restrict to a coordinatised version
of modules, given by ranks and matrices, the tensor product has to be
one of matrices as well. 

Let me reiterate that the distributivity morphisms
for a bipermutative category are isomorphisms in this
thesis! (Compare Remark \ref{IsoIsoIso}.)

 Given a bipermutative coefficient category 
$(\R,+,\cdot)$ we want to define a tensor product on its 
bicategory of matrices $\M$. Choose an associative 
bijection $\omega_{n,m}\colon \mathbf{n}\times \mathbf{m}\rightarrow \mathbf{nm}$, 
defining a strictly associative monoidal product on $\mathrm{Fin}$,
which represents the cartesian product of finite sets - cf. Example \ref{Fin}.  
For definiteness I set $\omega_{n,m}(i,j):= (i-1)\cdot m + j$ with 
inverse $\theta_{n,m}(i)=(((i-1) ~\mathrm{div} ~m)+1, ((i-1) ~\mathrm{mod}~ m +1),$
where $i~\mathrm{div} ~m := \floor{\frac{i}{m}}$ is the integer part
of division of $i$ by $m$, while $i~\mathrm{mod}~ m$ is the remainder
$r$ for $i=qm+r$ the Euclidean division of $i$ by $m$. 

Recall that this is consistent with the associative smash product 
on $\Fi$ described in Example \ref{Finp}.

{\defn{\label{tensor}
Given a choice of associative bijections $\omega\colon \mathbf{n}
\times \mathbf{m}\rightarrow \mathbf{nm}$ define the tensor product as 
follows: \[\begin{array}{rlll}\boxtimes\colon&\M&\times&\M
\longrightarrow\M\\&(n,m)&\mapsto&nm\\&(A,B)&\mapsto&(A\boxtimes B)_{
\omega(i_1,j_1),\omega(i_2,j_2)}:=A_{i_1,i_2}\cdot B_{j_1,j_2}.\end{array}\] 
The same description applies to the tensor product of 2-cells.}}

The rest of the section is devoted to proving that $\M$ equipped with
this monoidal structure satisfies the axioms given in \ref{mdlbic}.

{\rem{Obviously my choice of $\omega$ is dictated by the choices I
fixed in \ref{Fin} so that I can establish $E_\bullet$ as a 
bipermutative functor.}}

{\ex{For clarity consider the following small example. Let \[A=\left(
\begin{array}{cc} A_{11}&A_{12}\\A_{21}&A_{22}\end{array}\right)\] and
\[B=\left(\begin{array}{cc} B_{11}&B_{12}\\B_{21}&B_{22}\end{array}
\right),\] then with the bijections chosen above we 
have: \[A\boxtimes B=\left(\begin{array}{cc|cc}
A_{11}B_{11} & A_{11}B_{12} &A_{12}B_{11} &A_{12}B_{12} \\
A_{11}B_{21} & A_{11}B_{22} &A_{12}B_{21} &A_{12}B_{22} \\\hline
A_{21}B_{11} & A_{21}B_{12} &A_{22}B_{11} &A_{22}B_{12} \\
A_{21}B_{21} & A_{21}B_{22} &A_{22}B_{21} &A_{22}B_{22} \\
\end{array}\right).\]}}

{\rem{Following this example it is easy to see that the tensor product
as defined in \ref{tensor} respects weakly invertible matrices with
coefficients in a bipermutative category. In particular, if we 
considered a tensor-product of matrices given by columnwise or 
linewise ordering (as opposed to blockwise), any entry $a_{ij}=0$ 
would produce a full zero column or line, hence definitely not a 
(weakly) invertible matrix.}}

{\rem{As I indicated before the structure induced by direct sum of
matrices on $\M$ yields a permutative structure, but the 
tensor-product does not. The fact that $(\M,\boxplus)$ is permutative 
is shown in the proof of \cite[Theorem 4.7]{Os}. The tensor-product already fails
at the first strictness of a permutative bicategory, because $\boxtimes$ 
is not a strict functor, i.e., it only respects composition up to an 
isomorphism two-cell, which is strict if and only if the multiplicative 
symmetry of the coefficients is trivial, hence only for $\R$ an ordinary 
ring (or rig as in the case of $\N$).}}

\subsection{The Matrix Tensor Product is Symmetric Monoidal}
This subsection is devoted to proving that the tensor product equips
$\M$ with a symmetric monoidal structure in all detail. It can be 
skipped safely by the reader without losing any essential information.
I am sure that an elegant short proof by exploiting the functor 
$E_\bullet\colon \Sigma_*\rightarrow\M$ can be devised, but I want to
exhibit the additional strictness the tensor-product on $\M$ satisfies
in explicit detail.

{\lem{\label{tensoristFunktor} The assignment $\boxtimes$ of 
definition \ref{tensor} is a pseudofunctor of bicategories.

In addition we have the following natural identities on $1$-cells:
\[A\boxtimes B = (A\boxtimes id)(id\boxtimes B)\] as well as strict 
compositors:
\[(A^1A^2)\boxtimes id = (A^1\boxtimes id)(A^2\boxtimes id),\]
\[id\boxtimes (B^1B^2) = (id \boxtimes B^1)(id\boxtimes B^2).\]

\begin{proof}Normality is obvious: the identity matrices $E_n$ and 
$E_m$ are sent to \[E_n\boxtimes E_m = E_{nm},\] by strictness of $0$ 
and $1$ in $\R$. The interesting aspect is the compositor 
\[\Phi^\boxtimes\colon (A^1\circ A^2)\boxtimes (B^1\circ B^2)
\Rightarrow (A^1\boxtimes B^1)\circ (A^2\boxtimes B^2).\]
Composition of 1-cells in $\M$ is given by matrix multiplication, 
hence
\[(A^1\circ A^2)\boxtimes(B^1\circ B^2)_{\omega(i_1,j_1),\omega(i_2,j_2)}\]
\[\begin{aligned}&=(A^1\circ A^2)_{i_1,i_2}(B^1\circ B^2)_{j_1,j_2}\\
             &=\left(\sum_k A^1_{i_1,k}A^2_{k,i_2}\right)
\left(\sum_l B^1_{j_1,l}B^2_{l,j_2}\right)\\    
             &\Rightarrow^{\rho^{-1}} \sum_k A^1_{i_1,k}A^2_{k,i_2}
\left(\sum_lB^1_{j_1,l}B^2_{l,j_2}\right)\\
             &\Rightarrow^{\sum_k\lambda^{-1}} 
	\sum_{(k,l)}A^1_{i_1,k}A^2_{k,i_2}B^1_{j_1,l}B^2_{l,j_2}\\
             &\Rightarrow^{\sum_{(k,l)}id\cdot c^{\R}_\otimes\cdot id} 
\sum_{(k,l)} A^1_{i_1,k}B^1_{j_1,l}A^2_{k,i_2}B^2_{l,j_2}\\
             &= \sum_{(k,l)}(A^1\boxtimes B^1)_{\omega(i_1,j_1),\omega(k,l)}
(A^2\boxtimes B^2)_{\omega(k,l),\omega(i_2,j_2)}\\
             &= (A^1\boxtimes B^1)
\circ(A^2\boxtimes B^2)_{\omega(i_1,j_1),\omega(i_2,j_2)}.\end{aligned}\]
So define $\Phi^\boxtimes := (id\cdot c_\otimes\cdot id)\circ\lambda^{
-1}\circ \rho^{-1}$ in the manner described above for each component 
(with summations suppressed because of the appropriate coherences in 
the coefficient category). It is natural, because the involved 
morphisms are natural in $\R$. It is obvious if either $A^1A^2=id$ or 
$B^1B^2=id$ then the involved natural isomorphisms are forced to be 
identities, hence follow the strict identities claimed above.

To see that $\Phi$ is associative I refer the reader to
\cite{Lap}: Given a morphism \[((A^1\circ A^2)\circ A^3)\boxtimes 
((B^1\circ B^2)\circ B^3) \Rightarrow (A^1\circ (A^2\circ A^3))
\boxtimes (B^1\circ (B^2\circ B^3))\] comprised only of structural 
(iso)morphisms of the bipermutative category $\R$ there is a unique 
structural morphism between the given source and target. Since the 
associator of $\M$ is given by structural morphisms of $\R$ and the 
compositor $\Phi^\boxtimes$ of $\boxtimes$ is given by structural 
morphisms of $\R$ as well, this gives that $\Phi^\boxtimes$ is 
associative in the appropriate manner (cf. \cite[p. 4]{Lei}).
\end{proof}}}

{\rem{Since this is the first proof of this type let me emphasise that 
	it is sufficient to consider the compatibilities on $1$-cells, because
	the $2$-cells are any type of $n\times n$-matrix with no additional
	condition. So the calculations on $1$-cells are ``always'' strictly
	natural with respect to $2$-cells.}}

{\rem{The additional strict identities show that the compositor of 
$\boxtimes$ is a result of the natural isomorphism: 
$(id\boxtimes B)(A\boxtimes id)\Rightarrow 
(A\boxtimes id)(id\boxtimes B) = A\boxtimes B.$}}

{\rem{Consistently with the Deligne conjecture for Algebraic 
$K$-theory we see that we need at least a braiding (i.e., an 
$E_2$-structure) on the coefficient category $\R$ to define 
an $E_1=A_\infty$-multiplication on its module category. 
Cf. for instance \cite[Example 3.9]{Bar} and 
\cite[Remarks after C.6.3.5.17]{Lu2}. For a more thorough
survey of the Deligne conjecture on Hochschild cohomology as well as
a survey of its proofs see Section 16 of \cite{MSm}.}}

{\lem{The functor $\boxtimes$ is strictly associative. \begin{proof}
For this proof I fix the specific associative bijections from the beginning
of this section. For $A\in GL_n\R, B\in GL_m\R, C\in GL_l \R$ we have:
\[((A\boxtimes B)\boxtimes C)_{(i_1-1)ml+(j_1-1)l+k_1,
(i_2-1)ml+(j_2-1)l+k_2}\]
\[\begin{aligned}
&= (A\boxtimes B)_{(i_1-1)m+j_1,(i_2-1)m+j_2}\cdot C_{k_1,k_2} \\
&=  A_{i_1,i_2}\cdot B_{j_1,j_2}\cdot C_{k_1,k_2}\\
&=  A_{i_1,i_2}\cdot (B\boxtimes C)_{(j_1-1)l+k_1,(j_2-1)l+k_2}\\
&= (A\boxtimes (B\boxtimes C))_{(i_1-1)ml + (j_1-1)l + k_1,
(i_2-1)ml + (j_2-1)l +k_2}. \end{aligned}\]\end{proof}}}

{\lem{The object $1$ with its identities is a strict unit for 
$\boxtimes$. \begin{proof} We have:
\[(A\boxtimes 1)_{(i_1-1)1+j_1,(i_2-1)1+j_2} = A_{i_1,i_2}\]
for $i_1,i_2=1,\ldots, |A|$ and $j_1=j_2=1$, analogously 
$1\boxtimes A = A$.\end{proof}}}

The following statement can also be thought of as a convention: Just 
as the empty matrix is a strictly neutral element for 
$\boxplus$, it is a strict zero for $\boxtimes$.

{\lem{The object $0$ with its identity considered as the empty matrix 
(of objects and morphisms respectively) is a strict zero for 
$\boxtimes$.}}

I needed some commutativity to show that $\boxtimes$ is a 
functor, it should be much less surprising that it is necessary for
commutativity of $\boxtimes$.

{\lem{The bicategory of matrices $\M$ over a bipermutative coefficient category
$\R$ is symmetric monoidal with respect to $\boxtimes$. \begin{proof}
At this point I borrow the bipermutative structure from $\Sigma_*$ 
(cf. \ref{symmMatr}), let $A\in GL_n\R$, $B\in GL_m \R$, and consider:
\[(c_{n,m}(A\boxtimes B))_{(i_1-1)n + j_1,(i_2-1)m+j_2}
= (A\boxtimes B)_{(j_1-1)m + i_1,(i_2-1)m+j_2} 
= A_{j_1,i_2}B_{i_1,j_2},\]
\[((B\boxtimes A)c_{n,m})_{(i_1-1)n + j_1,(i_2-1)m+j_2}
= (B\boxtimes A)_{(i_1-1)n+j_1,(j_2-1)n+i_2} 
= B_{i_1,j_2}A_{j_1,i_2},\]
these can obviously be transformed into each other by the 
multiplicative twist of $\R$, so the symmetry has as one-cells 
$c_{n,m}\colon nm\rightarrow mn$ and two-cells 
$(C^\boxtimes)_{ij} = c^\R_\cdot~~\forall i,j$.\end{proof}}}

{\ex{Consider this again on $2\times 2$-matrices, i.e., a diagram:
\[\xymatrix{2\cdot 2\drtwocell<\omit>{~c^\R}\ar[r]^{A\boxtimes B} \ar[d]_{c_{2,2}}
	&2\cdot 2\ar[d]^{c_{2,2}}\\
2\cdot 2\ar[r]_{B\boxtimes A} & 2\cdot 2.}\]
Use the identification $c_{2,2}=(23)$ to calculate:
\[\begin{aligned}
(B\boxtimes A)E_{(23)}
&=\left({\begin{array}{cccc}
B_{11}A_{11} & B_{11}A_{12} &B_{12}A_{11} &B_{12}A_{12} \\
B_{11}A_{21} & B_{11}A_{22} &B_{12}A_{21} &B_{12}A_{22} \\
B_{21}A_{11} & B_{21}A_{12} &B_{22}A_{11} &B_{22}A_{12} \\
B_{21}A_{21} & B_{21}A_{22} &B_{22}A_{21} &B_{22}A_{22} \\
\end{array}}\right)\left({\begin{array}{cccc}
1&&&\\
&&1&\\
&1&&\\
&&&1
\end{array}}\right)\\
&=\left({\begin{array}{cccc}
B_{11}A_{11} &B_{12}A_{11} & B_{11}A_{12} &B_{12}A_{12} \\
B_{11}A_{21} &B_{12}A_{21} & B_{11}A_{22} &B_{12}A_{22} \\
B_{21}A_{11} &B_{22}A_{11} & B_{21}A_{12} &B_{22}A_{12} \\
B_{21}A_{21} &B_{22}A_{21} & B_{21}A_{22} &B_{22}A_{22} \\
\end{array}}\right)
\end{aligned}\]
and the other side:
\[\begin{aligned}
E_{(23)}(A\boxtimes B)
&=\left({\begin{array}{cccc}
1&&&\\
&&1&\\
&1&&\\
&&&1
\end{array}}\right)\left({\begin{array}{cccc}
A_{11}B_{11} & A_{11}B_{12} &A_{12}B_{11} &A_{12}B_{12} \\
A_{11}B_{21} & A_{11}B_{22} &A_{12}B_{21} &A_{12}B_{22} \\
A_{21}B_{11} & A_{21}B_{12} &A_{22}B_{11} &A_{22}B_{12} \\
A_{21}B_{21} & A_{21}B_{22} &A_{22}B_{21} &A_{22}B_{22} \\
\end{array}}\right)\\
&=\left({\begin{array}{cccc}
A_{11}B_{11} & A_{11}B_{12} &A_{12}B_{11} &A_{12}B_{12} \\
A_{21}B_{11} & A_{21}B_{12} &A_{22}B_{11} &A_{22}B_{12} \\
A_{11}B_{21} & A_{11}B_{22} &A_{12}B_{21} &A_{12}B_{22} \\
A_{21}B_{21} & A_{21}B_{22} &A_{22}B_{21} &A_{22}B_{22} \\
\end{array}}\right),
\end{aligned}\]
thus we have the $2$-cell given by the multiplicative twist of $\R$ in
each component:
\[\xymatrix{\left({\begin{array}{cccc}
B_{11}A_{11} &B_{12}A_{11} & B_{11}A_{12} &B_{12}A_{12} \\
B_{11}A_{21} &B_{12}A_{21} & B_{11}A_{22} &B_{12}A_{22} \\
B_{21}A_{11} &B_{22}A_{11} & B_{21}A_{12} &B_{22}A_{12} \\
B_{21}A_{21} &B_{22}A_{21} & B_{21}A_{22} &B_{22}A_{22} \\
\end{array}}\right)\ar[d]_{c^\R}\\
 \left({\begin{array}{cccc}
A_{11}B_{11} & A_{11}B_{12} &A_{12}B_{11} &A_{12}B_{12} \\
A_{21}B_{11} & A_{21}B_{12} &A_{22}B_{11} &A_{22}B_{12} \\
A_{11}B_{21} & A_{11}B_{22} &A_{12}B_{21} &A_{12}B_{22} \\
A_{21}B_{21} & A_{21}B_{22} &A_{22}B_{21} &A_{22}B_{22} \\
\end{array}}\right).}\]}}

Let me summarise these results into one big lemma:
{\lem{The module bicategory $\M$ of a bipermutative category $\R$ is 
strictly symmetric monoidal with respect to the tensor product of 
matrices $\boxtimes$, i.e., we have:
\begin{itemize}
\item $(\boxtimes,\Phi^\boxtimes)$ is a pseudofunctor:
\[\boxtimes\colon \M\times\M\rightarrow\M,\]
\item $\boxtimes$ is strictly associative, i.e., 
\[\boxtimes\circ(\boxtimes\times id) 
= \boxtimes\circ (id\times\boxtimes),\]
\item $\boxtimes$ has a strict unit $1$, i.e.,
\[\boxtimes\circ (id\times 1) 
= \boxtimes \circ (1\times id) = id_{\M},\]
\item $\boxtimes$ has a strong symmetry transformation 
$(c^\Sigma, c^\R_\cdot)$.
\end{itemize}

Additionally the symmetry satisfies the following coherences strictly:
\[\begin{array}{cc}
{\xymatrix{l\cdot m\cdot n \ar[dr]_{c^\Sigma_{l,mn}} 
\ar[r]^{c^\Sigma_{l,m}\boxtimes 1_n}
 & m \cdot l\cdot n \ar[d]^{1_m\boxtimes c^\Sigma_{l,n}}\\
 & m\cdot n\cdot l}}
&
{\xymatrix{l\cdot m\cdot n \ar[dr]_{c^\Sigma_{lm,n}} 
\ar[r]^{1_l\boxtimes c^\Sigma_{m,n}}
 & l\cdot n\cdot m \ar[d]^{c^\Sigma_{n,l}\boxtimes 1_m}\\
 & n\cdot l\cdot m.}}\end{array}\]
Furthermore the symmetry is its own inverse:
\[{\xymatrix{
n\cdot m\ar@{=}[rr] \ar[dr]_{c^\Sigma_{n,m}} && n\cdot m \\
 & m\cdot n.\ar[ur]_{c^\Sigma_{m,n}}}}\]

\begin{proof} Each of the properties that do not follow from the 
previous lemmas is just promoted to $\M$ from $\Sigma_*$ by the 
functor $E_\bullet$, so there is nothing new to prove.\end{proof}}}

In summary I have proved that $(\M, \boxtimes, 1, c_\cdot)$
is a strict symmetric monoidal bicategory in the sense also used by 
\cite[Definition 3.1]{Os}. Since the tensor product of matrices satisfies
these strict axioms, it is sufficient for me to consider this type of 
symmetric monoidal bicategory, although it is very probable that 
this class does not cover all equivalence classes of the most general
type of bicategories with a symmetric monoidal structure one could 
devise.

With the tensor-product in place I can state the second strict 
monoidality the functor $E$ satisfies, which quite trivially follows
from the fact that I chose the same bijection for the tensor-product
as I did for the product in $\Sigma_*$.
{\prop{For each bipermutative coefficient category $\R$ the inclusion
\[E\colon \Sigma_* \rightarrow \M\] is strictly symmetric monoidal 
with respect to $\times$ on $\Sigma_*$ and $\boxtimes$ on $\M$.}}

{\rem{With the symmetric monoidal structures on $\M$ settled the remark
	that everything works enriched as well is obligatory. The calculations
	before extend to $2$-cells, because they are defined merely as part of
	the appropriate product-categories with no additional conditions, thus
	reordering $2$-cells as indicated by the $1$-cells is compatible with
	the enrichment.}}

\section{The Bimonoidal Structure on $\M$}
Osorno has proved that $(\M,\boxplus, 0, c_+)$ is a permutative
bicategory (see Theorem \ref{OsPlus}), and in Section \ref{Mult} I elaborate on the 
fact that $(\M,\boxtimes, 1, c_\cdot)$ is a second symmetric monoidal
bicategory structure on $\M$. One would want these to 
interact in a manner analogous to bipermutative $1$-categories. 
This section is devoted to making the analogy precise, and establishing
$\M$ as a bipermutative bicategory.

In \ref{Fin} the choice of a strictly associative functor representing
the product made the left-distributor strict in $\Sigma_*$, i.e., we 
have $\lambda=id$. Here I used the same bijection that fixes this for 
the tensor-product structure in \ref{tensor}. It should be intuitive 
that this makes $E_\bullet$ into a well-behaved bipermutative functor. I
elaborate on that after the appropriate definition for bicategories.

The distributors of the bipermutative structure on $\Sigma_*$
promote to natural transformations in $\M$ without using two-cells.

{\prop{We have strict equalities of one-cells for $A\in GL_n\R$, 
$B\in GL_m\R$, $C\in GL_l\R$: \[(A\boxplus B)\boxtimes C = A\boxtimes C
\boxplus A\boxtimes C\] and \[A\boxtimes(B\boxplus C)c^\Sigma_{m+l,n}
(c^\Sigma_{n,m}\boxplus c^\Sigma_{n,l}) = c^\Sigma_{m+l,n}(
c^\Sigma_{n,m}\boxplus c^\Sigma_{n,l})((A\boxtimes B) \boxplus 
(A\boxtimes C)).\] \begin{proof} I only comment on the strictness, 
which is a result of the fact that the multiplicative twist enters 
twice as a two-cell, hence cancels out.\end{proof}}}

{\ex{Let me elaborate on $l=n=2, m=1$, so we get:
\[c_{m+l,n}=c_{3,2},\] which is \[c_{3,2}((i-1)3+j) = (j-1)2+i\]
i.e., $c_{3,2}$ is the cycle $(2453)$. We have\[c_{2,1}+c_{2,2}=id+
(23)=(45),\]hence\[c_{3,2}(c_{2,1}+c_{2,2})=(2453)(45)=(432).\] 

On one side we find:\[\begin{aligned}&A\boxtimes(B\boxplus C)E_{(432)}
\\&={\left(\begin{array}{cccccc}A_{11}b &&&A_{12}b&&\\
& A_{11}C_{11}&A_{11}C_{12}&&A_{12}C_{11}&A_{12}C_{12}\\
& A_{11}C_{21}&A_{11}C_{22}&&A_{12}C_{21}&A_{12}C_{22}\\
A_{21}b &&&A_{22}b&&\\
& A_{21}C_{11}&A_{21}C_{12}&&A_{22}C_{11}&A_{22}C_{12}\\
& A_{21}C_{21}&A_{21}C_{22}&&A_{22}C_{21}&A_{22}C_{22}\\\end{array}
\right)}{\left(\begin{array}{cccccc}
1&&&&&\\&&1&&&\\&&&1&&\\&1&&&&\\&&&&1&\\&&&&&1\end{array}\right)}\\
&={\left(\begin{array}{cccccc}A_{11}b &A_{12}b&&&&\\
&& A_{11}C_{11}&A_{11}C_{12}&A_{12}C_{11}&A_{12}C_{12}\\
&& A_{11}C_{21}&A_{11}C_{22}&A_{12}C_{21}&A_{12}C_{22}\\
A_{21}b &A_{22}b&&&&\\
&& A_{21}C_{11}&A_{21}C_{12}&A_{22}C_{11}&A_{22}C_{12}\\
&& A_{21}C_{21}&A_{21}C_{22}&A_{22}C_{21}&A_{22}C_{22}\\
\end{array}\right).}\end{aligned}\] On the other side we have:
\[\begin{aligned} &E_{(432)}(A\boxtimes B\boxplus A\boxtimes C)\\
&={\left(\begin{array}{cccccc}1&&&&&\\&&1&&&\\&&&1&&\\&1&&&&\\&&&&1&\\
&&&&&1\end{array}\right)}{\left(\begin{array}{cccccc}
A_{11}b &A_{12}b&&&&\\ 
A_{21}b &A_{22}b&&&&\\
&& A_{11}C_{11}&A_{11}C_{12}&A_{12}C_{11}&A_{12}C_{12}\\
&& A_{11}C_{21}&A_{11}C_{22}&A_{12}C_{21}&A_{12}C_{22}\\
&& A_{21}C_{11}&A_{21}C_{12}&A_{22}C_{11}&A_{22}C_{12}\\
&& A_{21}C_{21}&A_{21}C_{22}&A_{22}C_{21}&A_{22}C_{22}\\
\end{array}\right)}\\&={\left(\begin{array}{cccccc}A_{11}b&A_{12}b&&&&
\\&& A_{11}C_{11}&A_{11}C_{12}&A_{12}C_{11}&A_{12}C_{12}\\
&& A_{11}C_{21}&A_{11}C_{22}&A_{12}C_{21}&A_{12}C_{22}\\
A_{21}b &A_{22}b&&&&\\
&& A_{21}C_{11}&A_{21}C_{12}&A_{22}C_{11}&A_{22}C_{12}\\
&& A_{21}C_{21}&A_{21}C_{22}&A_{22}C_{21}&A_{22}C_{22}\\\end{array}
\right)}=A\boxtimes(B\boxplus C)E_{(432)}.\end{aligned}\]}}

Because of this strictness I define what a bipermutative bicategory is
in close analogy with $1$-categories.

{\defn{\label{bipbic}A \textbf{bipermutative bicategory} $\R$ is a bicategory 
(with strict identities) with two monoidal structures $\boxplus, 
\boxtimes$, an additive symmetry $c_\boxplus$, making $(\R,\boxplus)$ 
into a permutative bicategory (Definition \ref{permbic}), a multiplicative symmetry 
$c_\boxtimes$, making $(\R,\boxtimes)$ into a symmetric monoidal 
bicategory (Definition \ref{mdlbic}), and strictly natural distributivity 
isomorphisms (strictly invertible 1-cells):
\[\lambda\colon a\boxtimes b\boxplus a\boxtimes b' \rightarrow 
a\boxtimes (b\boxplus b'),\] \[\rho\colon a\boxtimes b\boxplus 
a'\boxtimes b \rightarrow (a\boxplus a')\boxtimes b,\]
satisfying the following strict identities of 1-cells:
\begin{enumerate} \item strict zero: \[0\boxtimes a=a\boxtimes 0 =0~~
\forall a\in\R,\] \item $\boxplus$-associativity of distributors:
\[\lambda(\lambda\boxplus id) = \lambda(id\boxplus \lambda),\]
\[\rho(\rho\boxplus id) = \rho(id\boxplus \rho),\]
\item additive symmetry of distributors:
\[(c_\boxplus\boxtimes id)\lambda = \lambda\circ c_\boxplus,\]
\[(id\boxtimes c_+) \rho = \rho c_+,\]
\item $\boxtimes$-associativity of distributors:
\[\lambda = \lambda \circ (\lambda \boxtimes id),\]
\[\rho = \rho\circ (id\boxtimes \rho),\]
\item middle associativity of distributors:
\[\lambda\circ (id\boxtimes \rho)=\rho\circ (\lambda\boxtimes id),\]
\item mixed associativity of distributors:
\[\lambda(\rho\boxplus \rho) =\rho (\lambda\boxplus\lambda)
 (1\boxtimes c_\boxplus \boxtimes 1),\]
\item multiplicative symmetry of distributors:
\[c_\boxtimes \circ \lambda = \rho \circ 
	(c_\boxtimes \boxplus c_\boxtimes).\]\end{enumerate}}}

{\rem{Let me emphasise that I have modelled this definition of 
bipermutative bicategory in such a way that the only thing left to 
show given the two symmetric monoidal structures $\boxplus$ and 
$\boxtimes$ on $\M$ is: There are distributors $\lambda$ and 
$\rho$, they are strict natural transformations and they
satisfy the coherences above. There is no additional data in the form
of coherence $2$-cells involved.}}

{\defn{Define the distributivity $1$-cells for $\M$ as follows:
\[\lambda:=\id=E_\lambda,~~\mathrm{and~~} \rho:=E_{\rho^\Sigma}
=c^\Sigma_{m+l,n}(c^\Sigma_{n,m}\boxplus c^\Sigma_{n,l}),\] with 
identities as 2-cells.}}

With these distributivity $1$-cells I can easily prove the 
following theorem, which I use to summarise all explicit details about
the bipermutative structure of $\M$, because most of it is part of the
lemmas already proven above.

{\thm{For $\R$ a bipermutative $1$-category (possibly enriched over the
symmetric monoidal categories $\mathit{Top}, \mathit{Cat}, 
\mathit{sSet}$), the following is a bipermutative bicategory $\M$ 
(with $2$-cells in the same enrichment):\begin{itemize}
\item $Ob\M = \N_0,$ \item $\M(n,m)=\begin{cases}GL_n\R, &~n=m,\\ 
                     		    \emptyset, & ~n\neq m,\end{cases}$
\item $(A\boxplus B)_{i,j}=\begin{cases}A_{i,j}, &~1\leq i,j\leq |A|,\\ 
                       B_{i-|A|,j-|A|}, & ~1\leq i-|A|,j-|A|\leq |B|,\\
                                        0,\end{cases}$
\item $\boxplus$ is a strict normal \ref{strictn} functor, 
i.e., $(A_1A_2\boxplus B_1B_2)=(A_1\boxplus B_1)(A_2\boxplus B_2)$
and $\id_n\boxplus \id_m = \id_{n+m},$
\item $(A\boxtimes B)_{(i_1-1)|B|+j_1,(i_2-1)|B|+j_2} 
      := A_{i_1,i_2}B_{j_1,j_2},$
\item $\boxtimes$ is a pseudofunctor, i.e., 
$(\id_n\boxtimes \id_m) = \id_{nm}$ and there is a natural isomorphism 
$2$-cell $(A_1\boxtimes B_1)(A_2\boxtimes B_2) \Rightarrow 
(A_1A_2)\boxtimes (B_1B_2)$ given by the adequate composition of 
(both) $\R$-distributors and its multiplicative symmetry $c^\R$ 
(cf. \ref{tensoristFunktor}), \item the matrix $E_{c^+}$ for the 
additive twist in $\Sigma_*$: \[c^+_{n,m}(i)=\begin{cases} 
i+m, &~i\leq n,\\i-n, &~i\geq n+1, \end{cases}\] yields the additive 
twist with $(C^+)_{i,j}=\delta_{i,c^+_{n,m}(j)},$ which is a strict
natural transformation $\boxplus \circ tw \Rightarrow \boxplus$, i.e., 
a pseudonatural transformation with coherence $2$-cells identities,
\item the bijections $c_{n,m}((i-1)m+j) = i + (j-1)n$ yield the 
multiplicative twist with $C_{i,j}=\delta_{i,c_{n,m}(j)}$, which is 
a strong pseudonatural transformation with $2$-cell given by the 
$c^\R$ in each component.\end{itemize} the distributors are given as:
\begin{itemize} \item $\lambda = \id\colon nm+nl \rightarrow n(m+l),$
and \item $\rho = c_{n,m+l}(c_{m,n}+c_{l,n})\colon mn+ln \rightarrow 
(m+l)n,$\end{itemize} and satisfy the coherences of \ref{bipbic}.
\begin{proof} The only thing left to prove is the fact that the 
distributors satisfy the claimed coherences. For that consider the 
functor \[E\colon \Sigma_*\rightarrow \M\] again. I already 
established that it is strictly symmetric monoidal with respect to $\boxplus$,
but given $\boxtimes$ as in \ref{tensor} and $\times$ as in \ref{Fin}
it is obvious that $E$ is also strictly symmetric monoidal with respect to 
these structures. Take particular note that the coherence $2$-cell of
$c_\boxtimes$ does not feature here because the multiplicative 
symmetry of $\R$ is forced to be the identity for the product 
$0\cdot 0 = 0\cdot 1=0$ and $1\cdot 1$ by the axioms of symmetric 
monoidal categories for the first and third case and the additional 
zero-axiom for bipermutative categories.

The distributors are defined as part of the image of $E$ explicitly,
so obviously we have $E_\lambda = \lambda$ and $E_\rho=\rho$, but $E$
is a strict functor of bicategories (and even $2$-categories, if we
restrict our attention to its image), so all coherences these 
distributors satisfy in $\Sigma_*$ directly promote to $\M$ for 
arbitrary bipermutative coefficients $\R$. 

In particular, $E$ is a strict functor of bipermutative bicategories, 
which is strictly additive, strictly multiplicative, and strictly
satisfies $E(c^+)=C^\boxplus, E(c^\cdot)=C^\boxtimes$ as well as
$E\lambda=\lambda, E\rho=\rho$.\end{proof}}}

{\rem{Let me emphasise that I can get away with such a strict structure, because
	the $2$-cells in $\M$ are just parts of the appropriate product categories
	with no additional compatibility condition among them. If one were to impose
	``weak invertibility'' on the matrices of $2$-cells for instance, I do not
	know, if we still get such a strict bipermutative structure.}}

\section{Transposition and Involutions} In commutative rings we are 
well aware of the formula: \[(AB)^t = B^tA^t.\] In preparation 
for involutions on module bicategories I want to isolate
how this formula behaves with genuine bipermutative 
categories as coefficients.

{\defn{For any bicategory $\C$ consider the $1$-opposed bicategory 
$\C^{op_1}$, which has the same objects, $1$-cells, $2$-cells, but 
opposed composition of $1$-cells, which I denote by $\circ$, while I 
do not denote the composition of $1$-cells in $\C$, just as usual for ordinary 
matrix multiplication. The associator is then the inverse of the 
original associator: \[\xymatrix{A\circ (B\circ C) = (CB)A 
\ar[r]^{\alpha^{-1}} &C(BA) = (A\circ B)\circ C.}\] For a bimonoidal
$1$-category $\R$ we also consider the $\mu$-opposed category 
$\R^\mu$ with the same objects and morphisms, opposed multiplication,
and hence exchanged distributors.}}

{\prop{\label{transstrfun}
For bimonoidal coefficients transposition is a strict normal \ref{strictn}
functor:\[(\cdot)^t\colon \MM(\R^\mu) \rightarrow \MM(\R)^{op_1}.\]
\begin{proof}We calculate on $1$-cells: \[\begin{aligned}
(A^t\circ_1 B^t)_{ij} &= (B^tA^t)_{ij} \\
&= \sum_k B^t_{ik}A^t_{kj}\\&= \sum_k A^t_{kj}\circ B^t_{ik}\\
&= \sum_k A_{jk}\circ B_{ki} = (AB)_{ji} = (AB)^t_{ij}. \end{aligned}\]
So transposition strictly respects composition of $1$-cells and
strictly respects identities.\end{proof}}}

{\rem{For bipermutative coefficients we could use the multiplicative twist
to suppress the $\mu$-opposition. However to make the book keeping of oppositions
more transparent I do not use that.}}

I want to consider involutions on the coefficient category $\R$ as 
considered by Richter in \cite[Definition 3.1]{Ri2010}.

{\defn{An anti-involution on a bipermutative category $\R$ is given by
a self-inverse strictly symmetric monoidal functor 
$T\colon (\R,+)\rightarrow (\R,+)$ with respect to $(\R,+,0,c^\R_+)$ together
with a natural isomorphism: \[t\colon T(a)T(b)\rightarrow T(ba).\]
Satisfying:\begin{itemize} \item $(T,t)$ strictly respects the unit, 
i.e., $T(1) = 1$ and \[t=id\colon T(a)1=1T(a)=T(1)T(a) \rightarrow 
T(a),\] \item $t$ is associative with respect to multiplication
\[\xymatrix{ T(a)T(b)T(c) \ar[r]^{t T(c)}\ar[d]^{T(a)t} & T(ba)T(c) 
\ar[d]^t\\ T(a)T(cb) \ar[r]^t & T(cba).}\] \item $(T,t)$ is symmetric 
with respect to multiplication: \[\xymatrix{ T(a) T(b) \ar[r]^t \ar[d]^c 
& T(ba) \ar[d]^{T(c)}\\ T(b) T(a) \ar[r]^t & T(ab).}\] \item the 
involution commutes with the distributors: \[\xymatrix{
T(a)T(b)+T(a)T(c)\ar[r]^-{\lambda} \ar[d]^{t+t} & T(a)T(b+c)\ar[d]^t\\
T(ba)+T(ca) \ar[r]^{T(\rho)} & T((b+c)a),}\]and\[\xymatrix{
T(a)T(c)+T(b)T(c)\ar[r]^-{\rho} \ar[d]^{t+t} & T(a+b)T(c)\ar[d]^t\\
T(ca)+T(cb) \ar[r]^{T(\lambda)} & T(c(a+b)).}\]\end{itemize}}}

{\rem{It is quite obvious that in the bipermutative case, the way I 
consider it in this thesis, one of the compatibilities with 
distributors implies the other, but the exposition is more transparent
this way.}}

{\rem{Richter proceeds in \cite{Ri2010} to define an induced 
involution on the bar construction of the monoidal categories 
$GL_n\R$. I define this involution as induced on matrix
bicategories.}}

{\prop{Let $(F,\varphi)\colon \mathcal{R}\rightarrow \mathcal{A}$ be a
strictly additive functor of strictly bimonoidal categories, i.e.
\[F(0)=0, F(r+s)=F(r)+F(s),\] furthermore let $F$ be strictly unital 
$F(1)=1$, then \begin{itemize} \item a lax transformation $\varphi\colon F(a)F(b)
\rightarrow F(ab)$ promotes to a lax normal 
functor \[\mathcal{M}F\colon \mathcal{M}(\mathcal{R})\rightarrow 
\mathcal{M}(\mathcal{A}),\] \item if furthermore 
$\varphi\colon F(a)F(b)\rightarrow F(ab)$ is a natural isomorphism, so
$F$ is strongly multiplicative, then $(F,\varphi)$ 
promotes to a pseudofunctor \[\mathcal{M}F\colon 
\mathcal{M}(\mathcal{R})\rightarrow \mathcal{M}(\mathcal{A}).\]
\end{itemize} \begin{proof} Again the interesting point is what 
happens on $1$-cells: \[\begin{aligned} 
(\mathcal{M}FA\cdot\mathcal{M}FB)_{ij}&=\sum_k FA_{ik}\cdot FB_{kj}\\
&\Rightarrow^\varphi \sum_k F(A_{ik}B_{kj})=F(\sum_k A_{ik}B_{kj})\\
&= F(AB_{ij}) = \mathcal{M}F(AB)_{ij},\end{aligned}\] obviously the 
functor $(\mathcal{M}F,\varphi)$ then is just as good as the 
constraint $\varphi$ of $F$.\end{proof}}}

{\lem{An anti-involution on a bimonoidal category $\R$ is a strictly
additive, strongly multiplicative functor from $\R$ to 
its multiplicative opposition $\R^\mu$
\[T\colon \R\rightarrow \R^\mu.\] Consequently an anti-involution 
induces a pseudofunctor of module bicategories:
\[\mathcal{M}T\colon \M \rightarrow \mathcal{M}(\R^\mu).\]
$\hfill \Box$}}

This is as far as I can come in the bicategory setting without 
appealing to classifying spaces, so let me summarise what the 
involution induces on module bicategories. \label{invStop}

{\lem{Composing transposition and an anti-involution on coefficients 
gives a pseudofunctor \[\mathcal{M}T\circ (\cdot )^t\colon 
\M \rightarrow \M^{op_1}.\]}}

\subsection{Involution and Tensor-products}\label{invundtensor}
One aim of this chapter on module bicategories is to get a 
combinatorial insight on how the involution on the coefficient 
category and the $E_\infty$-structure on its module bicategory 
interact. Fortunately this is easily described on the level of 
bipermutative bicategories.

It is obvious that the induced involution strictly respects direct
sum:
{\lem{For a bimonoidal category $\R$ with involution $(T,t)$ the 
induced involution on module bicategories is strictly additive and 
symmetric:
\[\mathcal{M}T(A\boxplus B) = \mathcal{M}TA\boxplus \mathcal{M}TB,\]
and \[\mathcal{M}T(c^+_{m,n})=c^+_{m,n}.\]}}

The tensor product structure, if defined, is also easily seen to be
compatible with the coordinatewise involution:

{\lem{For a bipermutative category $\R$ with involution $(T,t)$ we 
have a strictly natural isomorphism of functors:
\[t\colon \boxtimes \circ{\mathcal{M}T\times\mathcal{M}T}
 \Rightarrow \mathcal{M}T \circ\boxtimes,\]
each considered as functors $\M\times\M\rightarrow 
\mathcal{M}(\R^\mu)$.
\begin{proof}
This is a simple calculation, again consider $1$-cells:
\[\begin{aligned}
(\mathcal{M}T(A)\boxtimes \mathcal{M}T(B))_{(i_1-1)|B|+j_1,
	(i_2-1)|B|+j_2}
&=\mathcal{M}T(A)_{i_1,i_2}\circ \mathcal{M}T(B)_{j_1,j_2}\\
&=T(A_{i_1,i_2})\circ T(B_{j_1,j_2})\\
&\Rightarrow^t T(A_{i_1,i_2}B_{j_1,j_2})\\
&=\mathcal{M}T((A\boxtimes B)_{(i_1-1)|B|+j_1,(i_2-1)|B|+j_2})\\
&=\mathcal{M}T(A\boxtimes B)_{(i_1-1)|B|+j_1,(i_2-1)|B|+j_2}.
\end{aligned}\]\end{proof}}}

So we can summarise: {\thm{For a bipermutative category $\R$ with 
involution $(T,t)$ the coordinatewise involution on the 
bicategory of matrices $\M$ induces a strong bipermutative functor:
\[\mathcal{M}T\colon (\M,\boxplus,\boxtimes)\rightarrow 
	(\mathcal{M}(\R^\mu),\boxplus,\boxtimes),\]
in the sense that it is strictly additive, and strongly multiplicative
with respect to $\boxtimes$. \begin{proof} I only need to elaborate on
the multiplicative symmetry of $\mathcal{M}T$, which is a consequence 
of the compatibility on coefficients: \[\xymatrix{T(a)T(b) \ar[r]^t 
\ar[d]^c  & T(ba) \ar[d]^{Tc}\\T(b)T(a) \ar[r]^t & T(ab).}\]
\end{proof}}}

In particular the fact that $\mathcal{M}T$ is strictly additive with respect to
$\boxplus$ implies that it induces a map of $\Gamma$-spaces, i.e., an
infinite loop map on classifying spaces as follows:
\[B\mathcal{M}T\colon B\M\rightarrow B\mathcal{M}(R^\mu),\]
cf. \ref{plusdeloop}. In what follows I want to define an internal 
involution on $B\M$, and refine Osorno's delooping of Theorem \ref{plusdeloop} 
to one that allows us to induce a multiplicative structure more easily.
Thus as the last compatibility, which is directly visible on the
level of bicategories of matrices, we see that transposition and
tensor-product strictly commute.

{\prop{\label{transmultopp}
	For any bipermutative category $\R$ transposition induces a strictly
	additive and strictly multiplicative strict normal \ref{strictn} functor on its
	bicategory of matrices $\M$:
	\[(\cdot)^t\colon (\MM(\R^\mu),\boxplus,\boxtimes^\mu)\rightarrow 
	(\MM(\R)^{op_1},\boxplus,\boxtimes^{op}),\]
	where we consider the opposite multiplication on $\M^{op_1}$, i.e.
	fully reversed $A\boxtimes^{op}B=B\boxtimes A.$
	\begin{proof}The fact that transposition is a strict normal functor is 
	Proposition	\ref{transstrfun}. The strict additivity is obvious as 
	$(A\boxplus B)^t = A^t\boxplus B^t$. For the multiplicativity consider 
	the following sequence of equations:
	\[\begin{aligned}(A\boxtimes^\mu B)^t_{\omega(i_1,i_2),\omega(j_1,j_2)} &=
	(A\boxtimes^\mu B)_{\omega(j_1,j_2),\omega(i_1,i_2)} \\
	&=A_{j_1,i_1}\circ B_{j_2,i_2}\\&=B^t_{i_2,j_2}A^t_{i_1,j_1}\\
	&=(B^t\boxtimes A^t)_{\omega(i_2,i_1),\omega(j_2,j_1)}
	&=(A^t\boxtimes^{op} B^t)_{\omega(i_1,i_2),\omega(j_1,j_2)}.\end{aligned}\]
	Thus transposition and the coordinatised tensor-product \ref{tensor}
	commute up to one exchange of factors, 
	yielding the claimed compatibility.\end{proof}}}

Thus we see that by strict additivity of transposition we get an
infinite loop map of classifying spaces as:
\[B\mathcal{M}T\circ (\cdot)^t\colon B\M\rightarrow B\mathcal{M}(R)^{op_1}.\]

\section{Basics on Nerves of Bicategories} On page 2 of \cite{CCG2010} 
one can see various constructions of nerves, thus
classifying spaces for bicategories, all homotopy equivalent after 
realisation. In previous versions of this thesis I considered the
``Segal Nerve'' as for instance in \cite[p.21, Definition 5.2]{CCG2010}. I
finally noticed that in this bisimplicial set associated to a 
bicategory one of the simplicial directions only
consists of homotopy equivalences \cite[Theorem 6.2.]{CCG2010}. Hence I can
restrict to one simplicial direction, simplifying the
exposition.

{\defn{The nerve of a bicategory $\C$ (with only isomorphism $2$-cells) 
is the simplicial set with $n$-simplices pseudofunctors: 
\[N\C_n:=\mathbf{NorHom}([n],\C).\]}}

Let me be more explicit about this, I consider the ordered set 
$[n]=\{0<1<\ldots<n-1<n\}$ as a $1$-category, which is a bicategory
with only identity $2$-cells. Then a pseudofunctor \[(F,\varphi)\colon
[n]\rightarrow \C\] is the same thing as a collection of objects $F_i 
\in Ob\C$, and for each pair $0\leq i<j\leq n$ a choice of $1$-cell 
$A_{i<j}\colon F_i\rightarrow F_j \in Ob\C(F_i,F_j)$ (where normality
corresponds to the fixed choice $A_{i\leq i}=id_{F_i}$), and for each
triple $i<j<k$ a $2$-cell $\varphi_{i<j<k} \colon A_{jk}A_{ij} 
\rightarrow A_{ik},$ assembling to the compositor $\varphi$ of $F$, 
which is hence associative in the appropriate sense.

Compare page 22 of \cite{LP2008}, where there is also a 
condition on identities I do not need, because I only consider normal 
functors.

So for a bicategory $\C$ (possibly enriched) we get a simplicial set:
\[N\C\colon \Delta^{op}\rightarrow Set.\]I want to elaborate 
on the simplicial operators, let \[\Phi\colon [n] \rightarrow 
[m]\] be a monotone map: The effect on an $n$-simplex $F\colon [n]
\rightarrow \C$ is then given as:\[\Phi^*F(i):= F\circ\Phi(i),
\] on $1$-cells we have: \[\Phi^*A_{ij} := A_{\Phi(i),\Phi(j)}\colon 
F_{\Phi(i)}\rightarrow F_{\Phi(j)},\] which is the identity, if 
$\Phi(i) = \Phi(j)$ according to the normality condition on $F$. 
Finally on compositors we get: \[\Phi^*\varphi_{i<j<k} := 
\varphi_{\Phi(i),\Phi(j),\Phi(k)},\] which is the identity if any two
of the three indices coincide. This is coherent, because I only 
consider bicategories with strict identity $1$-cells.

{\rem{The terminology varies, which is
partly due to the fact that there are at least $10$ reasonable ways
to define a nerve for bicategories (cf. the diagram 
\cite[p. 2]{CCG2010}). This particular construction is called the 
``unitary geometric nerve'' in \cite{CCG2010}, where more generally
all lax functors are considered. These coincide with pseudofunctors
for bicategories with just isomorphism $2$-cells. In particular the
warning after Theorem 6.5 in \cite{CCG2010} does not apply for 
bicategories with all $2$-cells isomorphisms.}}

We are used to the fact that natural transformations of functors on 
$1$-categories induce homotopies. For bicategories the same argument
yields that an arbitrary pseudonatural transformation induces a homotopy.
The observation is not original, but in the presence of $10$ different
nerve constructions I want to exhibit this fact specifically for the one
I use.
{\prop{A pseudonatural transformation $\eta$ of pseudofunctors \[F,G
	\colon \C\Rightarrow \D\] is equivalent to a pseudofunctor $\C\times 
	I\rightarrow \D,$ hence induces a map: \[N(\C\times I)\cong N\C\times 
	NI\rightarrow N\D.\] \begin{proof} This is plainly the universal 
	property of the product in bicategories, i.e., $Fun(\A,\C\times\D) =
	Fun(\A,\C)\times Fun(\A,\D),$ where $Fun$ can be any of the 
	classes of functors between bicategories. Thus in particular 
	for $\A=[n]$ and $Fun$ the class of normal pseudofunctors we get 
	the claimed natural isomorphism.\end{proof}}}

\subsection{Opposition of a Bicategory and its Nerve} As I alluded to
at the end of section \ref{invStop}, I want to induce an involution on
the nerve of a bicategory. For that I need one last preparation.

{\defn{Let the reversal functor \[r\colon \Delta \rightarrow \Delta\] 
be given as the identity on objects and on morphisms 
$\Phi\colon [n]\rightarrow [m]$ define: \[r(\Phi)(i) := m- \Phi(n-i).\]
Given a simplicial object in any category 
$X\colon \Delta^{op}\rightarrow \C$, set $X^{op} := X\circ r$, 
analogously for cosimplicial objects.}}

The special point special about $\mathit{Top}$ as a target category is the 
cosimplicial object that defines geometric realisation. (The analogous
isomorphism in chain complexes $Ch$ is given by only a sign depending
on the chain degree.)

{\lem{Let $\Delta^\bullet\colon \Delta\rightarrow \mathit{Top}$ be the 
cosimplicial space defined as usual:
\[\Delta^n = \{(t_0,\ldots,t_n)\in I^{n+1}|\sum t_i=1\},\]
with cosimplicial operators: \[\delta^i(t_0,\ldots,t_{n-1}) 
= (t_0,\ldots,t_{i-1},0,t_i,\ldots,t_{n-1})\] and 
\[\sigma^i(t_0,\ldots,t_n) 
= (t_0,\ldots,t_{i-1},t_i+t_{i+1},t_{i+2},\ldots,t_n).\]
Then we have an isomorphism of cosimplicial topological spaces:
\[\Gamma\colon \Delta^\bullet \rightarrow \Delta^\bullet\circ r.\]
\begin{proof} Define $\Gamma(t_0,\ldots,t_n) = (t_n,\ldots,t_0)$, this
is obviously a degreewise homeomorphism, and the identities:
\[\Gamma\circ \delta^i = \delta^{n-i}\circ\Gamma\] \[\Gamma\circ 
\sigma^i = \sigma^{n-i}\circ\Gamma\] give that $\Gamma$ is an 
isomorphism of cosimplicial objects.\end{proof}}}

{\lem{\label{simpoppTop}Let $X\colon \Delta^{op}\rightarrow \mathit{Top}$ be a 
simplicial space (in particular sets with discrete topology), then 
consider geometric realisation as a coend: \[X\otimes_\Delta 
\Delta^\bullet = |X|.\] Then we have the identity \[X\otimes 
(\Delta^\bullet\circ r) = (X\circ r)\otimes\Delta^\bullet,\] and hence a 
natural homeomorphism of realisations: \[X\otimes \Gamma\colon |X|
\rightarrow |X\circ r|.\] \begin{proof} The identity \[X\otimes 
(\Delta^\bullet\circ r) = (X\circ r)\otimes\Delta^\bullet\] can be seen 
as follows. Both objects are quotients of the object \[\coprod_n X_n
\times \Delta^n,\] because $r$ does not change anything on the 
simplices. The identification according to $X\otimes (\Delta^\bullet
\circ r)$ then is: \[[x,\delta^{n-i} t] = [d_ix,t],\] and in $(X\circ 
r)\otimes\Delta^\bullet$ it is: \[[d_{n-i}x,t] = [x,\delta^{i} t].\] 
So the idenfications are just listed in a different order, but the 
equivalence relation we divide out is the same.

As a consequence the natural homeomorphism spells out: \[\begin{aligned} 
X\otimes \Gamma\colon &|X|\rightarrow |X\circ r|\\ &[x,(t_0,\ldots,t_n)] 
\mapsto [x,(t_n,\ldots,t_0)].\end{aligned}\]\end{proof}}}

With the simplicial considerations in place I define the classifying
space of a bicategory as follows:

{\defn{Given a bicategory $\C$ consider its nerve, which is a simplicial 
set as defined before:\[N\C\colon \Delta^{op} \rightarrow Set.\]
The geometric realisation of this simplicial set then defines the 
classifying space of $\C$: \[B\C := |N\C|.\]}}

We can understand opposing $1$-cells as the opposition of simplicial objects 
by precomposition with the reversal functor $r\colon\Delta\rightarrow \Delta.$

{\lem{\label{catoppnerv} The nerve of the bicategory $\C^{op_1}$ with 
reversed composition of $1$-cells is isomorphic to the $r$-reversed 
simplicial set $NC\circ r$:\[NC\circ r\cong N(C^{op_1}).\]
\begin{proof} This is immediate from the definition. The core point is
that opposing functors $[n]\rightarrow \C$ does change the direction
of $1$-cells, but does not change the direction of $2$-cells, just their
indexing.\end{proof}}}

Hence we find that the homeomorphism $B\C \cong B\C^{op}$ extends
to bicategories:

{\lem{The isomorphism $\Gamma\colon \Delta^\bullet\rightarrow 
\Delta^\bullet\circ r$ extends to a natural homeomorphism: 
\[\Gamma\colon B\C \rightarrow B{\C^{op_1}}.\] }}

So the homeomorphism interprets a sequence of $n$ $1$-cells in $\C$
as an $n$-sequence of the opposed $1$-cells.

{\defn{\label{indinv} For $\R$ a bimonoidal category with involution 
$T$ define the induced involution on its module bicategory as 
follows:\[\xymatrix{B\M \ar[r]^{B\mathcal{M}(T)} & 
        B\mathcal{M}(\mathcal{R}^\mu) \ar[r]^{B(\cdot)^t}  
        & B\mathcal{M}(\R)^{op_1} \ar[r]^{\Gamma} & B\M.}\]}}

Recall that transposition and the involution are 
covariant with respect to the $2$-cells, so the $1$-cells of the functors are 
opposed twice, but the $2$-cells are never opposed, so the constraints 
of $F,G,H$ (for $n\geq 2$) do not change their direction.

{\rem{Chasing through the definitions and taking into account the 
homeomorphism \[B\M \cong \coprod_n |BGL_n\R|,\] where $BGL_n\R$ is 
the bar construction on the monoidal category $GL_n\R$ as defined in 
\cite[Definition 3.8]{BDR2004}, it is easy to
see that this is precisely the same involution as defined in 
\cite{Ri2010}.}}

\section{Examples for Nerves of Bicategories}
There is an integral class in degree $3,$ from which I can bootstrap my 
calculations of the involution on $V(1)_*K(ku)$. I can describe it
easily as induced from a functor $\Sigma^2\S^1\rightarrow
\mathcal{M}(\V_\CC)$, thus induced by a map $\S^3\rightarrow K(\Z,3)
\rightarrow K(ku)$. For this I want to prepare some preliminaries on
the nerve of bicategories. For this section recall that we
can understand the totally ordered set $[n]=[0<1<\ldots<n]$ as a 
$1$-category, thus as a bicategory with discrete morphism categories.

{\prop{Let $\C$ be an arbitrary bicategory, and $F\colon 
	[n]\rightarrow \C$ a strong normal functor. Then $F$ 
	is uniquely	determined by its restriction to all $2$-faces, i.e., 
	by all restrictions \[[2]\rightarrow [n] \rightarrow 
	\C.\] Such a system of pseudofunctors $[2]\rightarrow \C$ 
	determines a (unique) pseudofunctor	$[n]\rightarrow \C$ if 
	and only if each four compatible $2$-faces can be extended over 
	$[3]$ \[\xymatrix{(\partial\Delta^3=)\coprod_4 [2]
	\ar[d]\ar[rr] && \C\\[3],\ar[urr]}\]where the boundary 
	can obviously not be made into a (bi)category, but we can 
	still express it as functors on the disjoint union subject 
	to the appropriate compatibility on $1$-cells. \begin{proof}
	This follows by inspecting the definition of pseudofunctor 
	carefully. The data given by functors $[2]\rightarrow\C$
	precisely gives the compositor $2$-cells, and the condition 
	on extending a functor on the boundary of $[3]$ to all of 
	$[3]$ is precisely the associativity condition on 
	compositors \ref{pseudo}.\end{proof}}}

The following result is classical and implicit in Section 5 of \cite{Str2},
where Street even more generally considers nerves of $n$-categories for
each $n$. However the exposition is quite dated, so I want to phrase
the specific result I need in the context I set up here.

{\prop{The bicategory $\Sigma^2A$ with $A$ an abelian (possibly 
topological) group with one object $*$, one $1$-cell $\id_*$ and
$\Sigma^2A(\id_*,\id_*)=A$ yields as classifying space a double 
delooping of $A$, i.e., there is a homotopy equivalence 
\[\Omega^2|N\Sigma^2A|= A.\] Thus define $B^2A= |N\Sigma^2A|.$

In particular, if $A$ is a discrete group we get $B^2A=K(A,2)$, and
for $A=\S^1$ we have $B^2\S^1=K(\Z,3)$, so $\Sigma^2\S^1$ is a 
bicategory modelling a $K(\Z,3)$.\begin{proof} We see immediately 
from the definition $N\Sigma^2A_0 = N\Sigma^2A_1 = \{*\},$ as well 
as $N\Sigma^2A_2=A$. The functors $r_{a,b,a+b}\colon[0<1<2<3]
\rightarrow\C$ with $r_{a,b,a+b}(012)=a, r_{a,b,a+b}(023)=b, 
r_{a,b,a+b}(123)=a+b, r_{a,b,a+b}(013)=\id_{\id_*}$ introduce the
relations of the Bar complex, thus we get that $N\Sigma^2A$ is a
model for the double delooping as claimed.\end{proof}}}

\subsection*{The Prototypical Class in $H(\MM\V_\CC)=K(ku)$}

{\defn{Consider the topologically enriched $1$-category $SX$ for $X$ 
an arbitrary topological
space defined as \[\xymatrix{0 \ar[r]^X&1.}\] It is a category for
arbitrary $X$, because no non-trivial compositions need to be
defined. The classifying space is the suspension of
$X$, hence in particular we can realise spheres by $X=\S^n$, giving
$BSX = \S^{n+1}$. Call it the \textbf{directed suspension}.\label{dirsusp}}}

{\ex{Consider the categories $\V_\CC$ and $\MM_\CC$ and the 
	directed suspension of the topological circle $S\S^1$. 
	The functor $u\colon S\S^1\rightarrow \V_\CC\subset \MM_\CC$
	with $u(0)=u(1)=1$ and the identity on morphisms realises
	the Bott class on classifying spaces
	\[\S^2\rightarrow \coprod_nBGL_n\CC \rightarrow 
	\Omega B\left(\coprod_nBGL_n\CC\right) \simeq BU\times \Z,\]
	since the Bott class can be represented as
	\[\Sigma\S^1=\Sigma U(1)\rightarrow BU(1)\simeq \CC P^\infty\rightarrow BU_\otimes.\]

	By the fact that the objects $0$ and $1$ are sent to the
	same object we get a factorisation over the one-point suspension
	$\Sigma\S^1,$ because $\S^1$ is an associative
	monoid:\[\xymatrix{S\S^1 \ar[dr]\ar[r]& \V_\CC\\
	&\Sigma\S^1,\ar[u]}\] which on 
	classifying spaces realises:\[\xymatrix{\S^2 \ar[r]\ar[dr] & 
	\coprod_nBGL_n\CC \ar@{-->}[r] & BU\times \Z\\& B\S^1 \simeq 
	K(\Z,2).\ar[u]}\]}}

We can suspend these categories to bicategories analogously. 
Consider a $1$-category $\C$, and define its directed suspension 
bicategory $S\C$ by \[\xymatrix{0\ar[r]^\C & 1,}\] where
again we do not need a composition for $1$-cells, hence objects
of $\C$. It realises the (unreduced) suspension on
classifying spaces, i.e., $BS\C = \Sigma B\C.$

In particular we get the following example.
{\ex{We can suspend the category $S\S^1$ to the bicategory
$SS\S^1=S^2\S^1$ realising $\S^3$ on classifying spaces. Thus we get a
directed suspension of the ``Bott functor'' above
\[Su\colon S^2\S^1 \rightarrow S\V_\CC.\]}}

{In general the directed suspension of a bipermutative category includes
into the bicategory of matrices $j\colon S\R\rightarrow \M$ by 
the inclusion: on objects $j_0(0)=j_0(1)=2$, on $1$-cells:
$j_1(r)= \left(\begin{array}{cc}1&r\\&1 \end{array}\right)$,
and the identification $\M(j_1r,j_1s) \cong \R(1,1)^{\times 2}
\times \R(0,0)\times \R(r,s),$ yields the inclusion 
$j_2\colon \R(r,s)\rightarrow \{\id_1\}^{\times 2}\times\{\id_0\}
\times \R(r,s) \subset \R(1,1)^{\times 2} \times \R(0,0)\times 
\R(r,s).$}

{\ex{For the suspended Bott functor we get 
\[S^2\S^1\rightarrow S\V_\CC\rightarrow \MM(\V_\CC),\] 
with the additional factorisation over the double one-point
suspension $\Sigma^2\S^1,$ which is a bicategory because 
$\Sigma\S^1$ is a monoidal category, because $\S^1$ is an 
abelian group. So we get\[\xymatrix{S^2\S^1\ar[dr]\ar[r]&
\MM(\V_\CC)\\& \Sigma^2\S^1\ar[u],}\] and thus on classifying 
spaces:\[\xymatrix{\S^3\ar[dr]\ar[r] & B\MM(\V_\CC) \ar@{-->}[r] 
& \Omega B \left(\coprod BGL_n\V_\CC\right)\\& B^2\S^1\simeq 
K(\Z,3).\ar[u]}\]

This is the class representing the \emph{Dirac Monopole} in $\RR^3$ 
considered as a $2$ vector bundle as described in \cite{ADR}.}}

We can simplify the discussion of the suspended Bott class
as given in the following section. Since it is given by a class
over the multiplicative unit $1$, its $1\times 1$-matrix is 
weakly invertible, and we do not need to extend to $2\times 2$-matrices.

\subsection{The Involution on the Monopole $\S^3\rightarrow \MM(\V_\CC)$}
The examples above determine a class $a\in\pi_3H\MM(\V_\CC)=\pi_3K(ku),$
the Dirac Monopole (cf. \cite{ADR}), thus we can understand the involution on it.

{\lem{\label{funcmono}
	Consider the class $\bar a\in\pi_3K(ku)=K_3(ku)$ represented by the
	functor:
	\[S_1u\colon S^2\S^1 \rightarrow \MM(\V_\CC)\]
	with $S_1$ the functor which assigns matrix rank $1$ to the 
	objects	$j(0)=j(1)=1$, and considers the $1$-cells of $S^2\S^1$ both
	as the $1\times 1$-matrix $(1)$ in $\MM(\V_\CC)_1$
	It obviously commutes with complex conjugation on 
	$2$-cells, while transposition has no effect. 
	So the diagram:
	\[ \xymatrix{S^2\S^1\ar@{=}[r]\ar[d]^{S_1u} &S^2\S^1\ar[r]^{\overline{(\cdot)_2}}
		\ar[d]^{S_1u}& S^2\S^1\ar[d]^{S_1u}\\
		\MM(\V_\CC)\ar[r]^{(\cdot)^t}&\MM(\V_\CC)^{op_1}\ar[r]^{\overline{(\cdot)_2}}
		&\MM(\V_\CC)^{op_1}}	\]
	strictly commutes.}}

Additionally observe that the functor $S_1$ is oblivious to opposition
of $1$-cells, because in a directed suspension this only amounts to relabelling source
and target object. In summary we find: 
{\thm{On the class $a\in\pi_3K(ku)$ the internalised involution induced by $\MM(\V_\CC)$
	is represented as the composite: \[\xymatrix{|NS^2\S^1|\ar[r]^{|N\overline{(\cdot)}_2|} 
	& |NS^2\S^1| \ar[r]^-\cong & |N(S^2\S^1)^{op_1}|=|\widetilde{N(S^2\S^1)}| \ar[r]^-\Gamma
	& |N(S^2\S^1)|.}\] In particular, the outer maps induce multiplication by $-1$ on $a$, 
	thus the involution induces the identity $a\mapsto a$.\label{invonMono}
	\begin{proof}As above we see that the double directed suspension of $\S^1$ realises 
	to $\Sigma^2\S^1=\S^3$. The conjugation represents
	a reflection along an equator, thus has degree $-1$. 

	For $\Gamma$ consider non-degenerate simplices of maximal degree.
	These are precisely given by functors $[0<1<2]\rightarrow S^2\S^1$ assigning
	for example $01$ to the initial $1$-cell, $12$ to the identity $1$-cell, 
	$02$ to the terminal $1$-cell, and choosing any $x\in\S^1$ as compositor $2$-cell.
	Abusively call such a functor $x$ as well. Then the maximal cells are parametrised
	as $[x,(t_0,t_1,t_2)]$. Here $\Gamma$ acts as: $[x,(t_0,t_1,t_2)]\mapsto[x,(t_2,t_1,t_0)].$
	In particular it has degree given by the sign of the transposition $(02),$ which is
	thus $-1$.\end{proof}}}
