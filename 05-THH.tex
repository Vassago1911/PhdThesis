\chapter{$THH$ and the Trace Map}\label{THHundtr}
Since my construction of the delooping of a bimonoidal bicategory 
naturally yields a symmetric spectrum and I do not want to switch 
contexts too much, I refer the reader to \cite{ShTHH} for Topological 
Hochschild Homology in symmetric spectra, as well as
\cite{AnR} for a careful study of the algebraic 
structure present for nice input spectra. I also want to mention 
the nicely model independent paper \cite{BFV}, which states 
in general that $THH$ of an $E_n$-ring spectrum is an $E_{n-1}$-ring 
spectrum by careful analysis of the involved operads. They do not 
need to fix their context of a model category of spectra, since any 
category tensored over topological spaces or simplicial sets will do, 
and they all are.

The generalisation of Hochschild homology to ring spectra by way of ``functors
with smash products'' is due to B\"okstedt and part of a paper, which is
notorious for its preprint-stage \cite{B1}. It is however highly influential, in particular,
since B\"okstedt subsequently fully calculated $THH(\Z/p)$ as well as 
$THH(\Z)$ in \cite{B2}, he provided the foundations for many subsequent 
calculations of topological Hochschild homology.

\section{$THH$ with Coefficients}
The algebra $V(1)_*THH(ku)$ that Christian Ausoni describes in 
\cite{AuTHH} is quite unwieldy --- in particular it is of good
use to have descriptions of easier objects with clear relations
to $THH(ku)$ the way he describes it in \cite{AuTHH}. 
By introducing the appropriate analog of logarithmic structures 
on ring spectra, Sagave and Schlichtkrull provided a localisation 
sequence, which makes the extension $\ell\rightarrow ku$ tamely 
ramified in a well-defined way \cite{SaSch14}. In particular, they 
describe the algebra $V(1)_*THH(ku)$ as a square-zero extension 
of the $V(1)$-homotopy of $THH^{log}(ku)$. However, since I 
introduce $THH$ purely for analysis of $V(1)_*K(ku)$ by trace 
methods along the lines of \cite{AuTHH,AuKku}, the introduction of 
logarithmic structures and the resulting cofibre sequence on 
topological Hochschild homology seems a drastic detour, which 
I do not want to present here.

Even before it was known that there are symmetric monoidal categories
of spectra, people have studied $THH$ in that then hypothetical 
category \cite{MS} or like B\"okstedt in auxiliary categories, which
turned out to be equivalent to the model categories of spectra \cite{MMSS}. 
In particular, given the construction of Hochschild homology 
for discrete algebras as presented in \cite{Loday} for instance, 
the generalisation to spectra is straightforward. 

{\defn{\label{THHblubbAM}\cite[Definition IX.2.1]{EKMM}
Given an associative $\S$-algebra $A$ with product $\mu\colon 
A\wedge A\rightarrow A$ and unit $\eta\colon\S \rightarrow A$, and an
$A$-bimodule $M$ with actions $M\wedge A \rightarrow M$ and
$A\wedge M\rightarrow M$, which I denote by $\mu$ as well, let 
$THH_\bullet(A,M)$ be the following simplicial spectrum: In degree $n$ 
we have the $(n+1)$-fold smash product:\[THH_n(A,M):=M\wedge A^{\wedge 
n},\]with face maps\[d_i\colon M\wedge A^{\wedge n}\rightarrow M\wedge
A^{\wedge n-1}\]defined as\[d_i=\begin{cases}id^{\wedge i-1}\wedge\mu
\wedge id^{\wedge n-i}&0\leq i\leq n\\(\mu\wedge id^{n-1})\circ t & 
i=n+1,\end{cases}\]for $t$ the symmetry of the smash product that 
exchanges factors as follows:\[\xymatrix{M\wedge A_1\wedge A_2\wedge
\ldots\wedge A_n \ar[r]^-t& A_n\wedge M\wedge A_1\wedge A_2\wedge
\ldots\wedge A_{n-1}.}\] The degeneracies are given by insertion of 
units at all places after the module $M$: $s_i\colon M\wedge A^{\wedge
n}\rightarrow M\wedge A^{\wedge n+1}$ for $0\leq i\leq n$:\[s_i=id^{i
+1}\wedge \eta \wedge id^{n-i},\] where I have notationally suppressed
the unit isomorphism $A\cong \S\wedge A.$ Call this the simplicial 
Hochschild spectrum of $A$ with coefficients in $M$.

For $M$ the algebra itself we set $THH_\bullet(A):=THH_\bullet(A,A)$ and call the
resulting simplicial spectrum the simplicial Hochschild spectrum of $A$.}}

This construction can be defined for arbitrary $\S$-algebras $A$ and $A$-bimodules
$M$. However, for it to be of topological significance, in particular for 
$THH_\bullet(A,M)$ to be homotopy invariant, we impose a technical condition.

{\rem{If the unit of the algebra $\S\rightarrow A$ is a cofibration in the
    model structure on a chosen model category of spectra, and $A$ and $M$
    are cofibrant $\S$-modules, the simplicial spectrum $THH_\bullet(A,M)$ 
    is \emph{proper} \cite[Theorem VII.6.7]{EKMM}. This means that for each simplicial
    degree $n$ the inclusion \[sTHH_n(A,M)\rightarrow THH_n(A,M)\] of $sTHH_n(A,M)$ the
    image of all degeneracies with target degree $n$ is a cofibration \cite[p. 182]{EKMM}.

    This in particular implies that a weak equivalence $M\rightarrow M'$ of
    $A$-bimodules and a weak equivalence $A\rightarrow A'$ of algebras, with $M'$ and
    $A'$ again cofibrant $\S$-modules with the unit $\S\rightarrow A'$ a cofibration,
    gives rise to a levelwise weak equivalence $THH_\bullet(A,M)\rightarrow 
    THH_\bullet(A',M')$, and since both spectra are proper this induces an
    equivalence on the realisations as well.\label{cofibTHH}}}

For a nicely short exposition of this compare to section 7 of \cite{BLPRZ}.

{\defn{Let $A$ be an associative $\S$-algebra, which is a cofibrant
$\S$-module, and for which the unit $\S\rightarrow A$ is a cofibration. Let
furthermore $M$ be an $A$-bimodule, which is a cofibrant $\S$-module. The
\emph{topological Hochschild homology} of $A$ with coefficients in $M$ 
is defined as the realisation of the (proper) simplicial
spectrum $THH_\bullet(A,M)$. To give this unambiguous meaning, understand
this as the coend
\[THH(A,M):=|THH_\bullet(A,M)|=\int^\Delta THH_q(A,M)\wedge (\Delta_q)_+,\]
where we use the tensored structure of a model category of spectra
over topological spaces to form $\_\wedge (\Delta_q)_+$ and then
form the coend.

Analogously, with $A=M$ define $THH(A):=|THH_\bullet(A)|=|THH_\bullet(A,A)|.$}}

We care how multiplicative opposition affects this construction
and find that it opposes the simplicial structure as expected.
{\rem{For an associative $\S$-algebra $A$ denote its multiplicative opposition $A^\mu$.
    In particular, I do not want to use the notation $A^{op}$, since this notation
    implies the wrong idea for the Eilenberg-MacLane-spectrum of a bipermutative
    (bi)category $H\C$: No opposition of morphisms is involved, only the symmetry
    of the smash product in spectra.}}

{\prop{(cf.\cite[5.2.1]{Loday}, \cite[Chapter 5]{L2011}) 
For an $\S$-algebra $A$
and an $A$-bimodule $M$ we see that simplicial
opposition on the Hochschild spectrum is isomorphic to multiplicative opposition, i.e.
\[\iota\colon THH_\bullet(A^\mu,M^\mu)\cong \widetilde{THH}_\bullet(A,M)\]
where on the left we oppose the multiplication on $A$ and exchange
the left- and right-action on $M$, while on the right we reverse
the simplicial direction as in \ref{catoppnerv}, i.e., $\widetilde{THH}_\bullet(A,M)
= THH_\bullet(A,M)\circ r$ for $r\colon \Delta\rightarrow \Delta$ the reversion
functor. \begin{proof}
The relevant isomorphism in discrete algebra reads 
$m\otimes a_1\otimes\ldots\otimes a_n\mapsto  m\otimes a_n\otimes \ldots\otimes a_1.$ 
This can be generalised to spectra by the appropriate twists of the smash product.\end{proof}}}

The lemma \ref{simpoppTop} literally applies, because we used the 
tensored structure over $\mathit{Top}$. This gives the following isomorphism
of spectra: {\prop{We have the identification given by reversing
the simplex coordinates in realisations:
\[\Gamma\colon|\widetilde{THH(A)}|\rightarrow |THH(A)|.\]}}

Given these two structural maps, we can easily induce an involution 
on $THH$ of a ring spectrum with anti-involution.
{\defn{\label{THHiota} Let $(A,\mu,\tau)$ be an associative $\S$-algebra with 
anti-involution $\tau$, i.e., a self-inverse $\S$-algebra map:
$\tau\colon (A,\mu)\rightarrow (A,\mu^{opp})$. Then we call the following 
sequence of maps: \[\xymatrix{THH(A)\ar[r]^\tau & THH(A^\mu)
\ar[r]^\iota & \widetilde{THH(A)} \ar[r]^\Gamma & THH(A)}\] the induced 
involution of $\tau$ on $THH(A).$}}

\section{The B\"okstedt Spectral Sequence}
The B\"okstedt spectral sequence, calculating topological from algebraic
Hochschild homology, is the essential tool B\"okstedt uses in \cite{B2}
to calculate $THH(\Z)$ and $THH(\Z/p)$. For a published reference see 
\cite[Theorem IX.2.9]{EKMM}.

The induced maps of $\tau$ and $\iota$ are simplicial by the results
before, so for the spectral sequence associated to the simplicial 
filtration, which yields the B\"okstedt spectral sequence under 
flatness assumptions, we can deduce the following result. 
For technical convenience assume we have arranged for $THH_\bullet(A,M)$
to be proper by the conditions mentioned above in Remark \ref{cofibTHH}.
{\thm{Let $h$ be a generalised homology theory, then the simplicial 
filtration of $|THH(A)|$ yields a spectral sequence \[E^1_{*,n}=
C^{cell}_*(THH_n(A),THH_{n-1}(A),h_*)\Rightarrow h_*(THH(A)).\]If $h$ 
satisfies the K\"unneth-formula on $A$, i.e., $h_*(A\wedge A)\cong 
h_*(A)\otimes_{h_*(\S)} h_*(A)$, then we can identify the $E^2$-term 
with the algebraic Hochschild-homology of $h_*(A)$, i.e.: \[E^2\cong 
HH(h_*(A)).\]

The induced involution given above is compatible with the simplicial
filtration, thus induces a map of spectral sequences.
In particular we find on $E^2$-terms: \[\xymatrix{HH(h_*A)
\ar[r]^-\tau & HH(h_*(A^\mu)) = HH((h_*A)^\mu) \ar[r]^-{\iota} & 
\widetilde{HH}(h_*A)\ar[r]^{\Gamma} & HH(h_*A).}\]}}

{\rem{\label{Gammasign}
Do note that $\tau$ and $\iota$ can be induced on the simplicial
level, while $\Gamma$ is a map of chain complexes given by 
introducing the adequate sign associated to the map 
\[\Delta^n/\partial\Delta^n \rightarrow \Delta^n/\partial\Delta^n\]
with $[t_0,t_1,\ldots,t_n]\mapsto [t_n,t_{n-1},\ldots,t_0],$
which is given by the sign of the permutation that fully inverts
the set $\{0,1,\ldots,n\}$, i.e.,\[(0~ n)(1~ n-1)\ldots (\floor*{\frac{n}2}
\ceil*{\frac{n}2}),\] which is $(-1)^{\frac{n(n+1)}2}.$}}

The B\"okstedt spectral sequence generalises to the case with 
coefficients in a bimodule as well by the same filtration:
{\thm{Let $h$ be a generalised homology theory, $A$ an associative
$\S$-algebra, $M$ an $A$-bimodule; additionally assume that $h$ 
satisfies the K\"unneth-isomorphisms $h_*(A\wedge M) = h_*(A)
\otimes_{h_*(\S)}h_*(M),$ $h_*(M\wedge A)=h_*(M)\otimes_{h_*(\S)}h_*(A),$ and
$h_*(A\wedge A)=h_*(A)\otimes_{h_*(\S)}h_*(A)$, then we have a B\"okstedt
spectral sequence of the form: \[HH(h_*(A),h_*(M))\Rightarrow h_*(THH(A,M)).
\] If additionally for $M_*=h_*(M)$ and $A_*=h_*(A)$ the algebra
$A_*$ is projective over $k=h_*\S$ then we can understand the $E^2$-term
as a derived functor: \[Tor^{A_*\otimes_k{A_*}^{op}}(M_*,A_*).\] }}

{\rem{I have suppressed convergence discussions in spectral sequences,
because the homology theories and spectra I consider only 
give modules and algebras with non-negative grading, thus we have 
strongly convergent first quadrant spectral sequences.}}

\section{The Multiplicative Structure of the Involution}
\subsection*{Structural Example: The Product for the Commutative Case}
Most sources discuss the product on Hochschild homology for the chain
complex associated to the simplicial module formed by the Hochschild
complex (cf. \cite[Lemma 1.6.11]{Loday} or \cite[Theorem VIII.8.8]{McL}). 
However this introduces the complications associated to dealing with
the sum of shuffles in the Eilenberg-Zilber map, so I prefer the simplicial 
structure of the product in algebraic modules as a template for spectra.

{\rem{For $A$ a commutative $k$-algebra let $M,N$ be $A$-modules and 
    consider the simplicial modules $C_n(M,A)=M\otimes A^{\otimes n}$ and
    $C_n(N,A)= N\otimes A^{\otimes n}$. In general we have the pairing
    $M\otimes A^{\otimes n}\otimes N\otimes A^{\otimes n} \rightarrow 
    M\otimes N\otimes (A\otimes A)^{\otimes n},$ which is given by 
    reordering the tensor factors according to the two-rail rail 
    fence cipher, i.e., by pairing $M$ with $N$ and the $i$th $A$-factor
    of the first $A^{\otimes n}$ with the $i$th $A$-factor in the second
    $A^{\otimes n}$. This assembles to a simplicial isomorphism:
    \[C_\bullet(M,A)\otimes C_\bullet(N,A)\rightarrow C_\bullet(M\otimes N,
    A\otimes A).\]

    Hochschild homology is a functor in the module variable for arbitrary maps
    linear over the appropriate algebra. Furthermore it is a functor in the
    algebra argument for algebra maps. For commutative $A$ we have that
    $\mu\colon A\otimes A \rightarrow A$ is an $A$-linear map of $A$-modules 
    as well as a map of $k$-algebras, thus for $M=N=A$ we get a map
    \[C_\bullet(A,A)\otimes C_\bullet(A,A)\rightarrow C_\bullet(A\otimes A,
    A\otimes A) \rightarrow C_\bullet(A,A).\]

    Analogously, we can define for $A$ a commutative $\S$-algebra a pairing
    of the simplicial spectrum $THH_\bullet A$ with itself as
    \[THH_\bullet(A)\wedge THH_\bullet(A)\rightarrow THH_\bullet(A\wedge A)
    \rightarrow THH_\bullet(A).\]

    Since this map is simplicial, it is obviously compatible with the filtration
    giving the B\"okstedt spectral sequence, and thus introduces the structure
    of a spectral sequence of differential graded algebras.}}

{\rem{To use \cite{BGT} I need to diverge into spectral categories, i.e., categories
    enriched over a symmetric monoidal model category of spectra: 
    The above discussion applies to spectral categories as follows. Consider 
    two small spectral categories $\C,\D$, and set their smash product to be 
    the category with objects $Ob\C\times Ob\D$ with smash-product on morphism spectra 
    $\C\wedge \D((c_1,d_1),(c_2,d_2))=\C(c_1,c_2)\wedge\D(d_1,d_2).$ 

    For a small spectral category with cofibrant morphism spectra $\C$ define
    its simplicial Hochschild spectrum as (compare \cite[p. 73]{BGT2013}):
    \[THH_q(\C):=\bigvee \C(c_{q-1},c_q)\wedge \C(c_{q-2},c_{q-1})\wedge 
    \ldots\wedge \C(c_0,c_1)\wedge \C(c_q,c_0).\]
    Take special note of the last factor, making it a ``circle of degree $q$'' in
    $\C$. The analogous face- and degeneracy-maps make it a simplicial spectrum,
    call its realisation the topological Hochschild homology of $\C$.

    The indicated rail-fence shuffle above induces an isomorphism of simplicial 
    spectra: \[THH_\bullet(\C)\wedge THH_\bullet(\D)\rightarrow THH(\C\wedge
    \D),\] which by coherence of the symmetry of $\wedge$ in spectra is strongly 
    symmetric monoidal in the sense that the following diagram strictly 
    commutes: \[\xymatrix{THH_\bullet(\C)\wedge THH_\bullet(\D)\ar[r]
    \ar[d]_{c_\wedge} & THH_\bullet(\C\wedge\D)\ar[d]^{c_\wedge}\\
    THH_\bullet(\D)\wedge THH_\bullet(\C)\ar[r] & THH_\bullet(\D\wedge\C).}\]

    Since realisation and smash product commute by naturality of the product
    isomorphism $|X\times Y|\cong |X|\times|Y|$ in compactly generated Hausdorff 
    spaces for $X,Y$ simplicial spaces, this descends to the same 
    multifunctoriality after realisation.}}

{\prop{Consider the category of small spectral categories $Cat_{Sp}$ enriched
    over a chosen symmetric monoidal model category of spectra. By \cite{Tab}
    we know that by cofibrantly replacing a small spectral category in the
    model structure on $Cat_{Sp}$ we obtain a weakly equivalent small 
    spectral category with cofibrant morphism spectra. 

    On the category of small spectral categories with cofibrant morphism 
    spectra consider topological Hochschild homology $THH_\bullet(\cdot).$
    This is a strong symmetric monoidal and simplicially enriched functor
    \[THH_\bullet(\cdot)\colon Cat_{Sp}\rightarrow sSp.\] In particular, 
    it is a multifunctor by considering the multicategory-structures
    induced by the symmetric monoidal products on $Cat_{Sp}$ and $sSp$.

    By passing to realisations we get a strong symmetric monoidal and
    simplicially enriched functor, hence a multifunctor: \[THH(\cdot)\colon
    Cat_{Sp}\rightarrow Sp.\]

    \begin{proof}The structure of a multifunctor on $THH$ is an elaboration on 
    the transformation $THH(\C)\wedge THH(\D)\rightarrow THH(\C\wedge\D),$ which
    is strictly symmetric, strictly unital, and coherently associative, hence 
    yields a multifunctor of the described type.\end{proof}}}

{\rem{I strongly recommend the preprint of Bj\o rn Ian Dundas establishing multifunctoriality
    of topological cyclic homology as well, and moreover proving that the trace maps are
    also natural transformations for these multifunctor-structures \cite{D2}.}}

In absence of an explicit symmetric monoidal product on symmetric monoidal
categories (cf. however \cite{GGN} for a monoidal structure on the $\infty$-category
of symmetric monoidal categories) the authors in \cite{EM} describe a
multicategory-structure on permutative categories instead, which has bipermutative
categories as its $E_\infty$-monoids, and establish that the Eilenberg-MacLane spectrum
of permutative categories (which \cite{EM} call ``$K$-theory'') can be
given the structure of a multifunctor. Thus by the natural equivalence $K(H\R)=
H(\M)$ established in \cite{BDRR2011} we can consider an
induced multifunctor-structure on $H(\MM_{\_})= H\circ \MM_{\_}.$
In particular, in absence of a comparison of the multiplicative structure
I describe in chapter \ref{multbidel}, I refer to the $E_\infty$-structures induced
by the multifunctor-structure as \textbf{the} $E_\infty$-structure. 

I summarise the appropriate identifications of \cite{BDRR2011,BGT} in the following 
theorem -- to summarise the known results.

{\thm{The algebraic $K$-theory space $B\M$ of a permutative category $\R$ is naturally 
    equivalent to the algebraic $K$-theory space of its associated Eilenberg-MacLane
    spectrum $K(H\R)$ \cite{BDRR2011}. 

    The algebraic $K$-theory functor from small spectral categories to 
    spectra can be given the structure of a (symmetric) multifunctor in an 
    essentially unique way \cite[Theorem 1.5]{BGT}.

    There is an essentially unique natural transformation
    of (symmetric) multifunctors $K\Rightarrow THH,$ which by \cite[Theorem 6.3]{BGT}
    is the \emph{trace map} from algebraic $K$-theory to topological Hochschild
    homology \cite[Theorem 1.9]{BGT}. Thus in particular we get a unique natural 
    multiplicative map $K(H\R)\rightarrow THH(H\R)$ for Eilenberg-MacLane spectra.}}

{\rem{As seen in \ref{conjun} I have to concede that I do not know if the uniqueness
    of \cite{BGT} forces the $E_\infty$-structure of chapter \ref{multbidel} on $H(\M)$ 
    to agree with the canonical structure on $K(H\R)$ asserted by
    \cite{BGT}. If, however, one were able to prove that the Eilenberg-MacLane spectrum
    functor admits an essentially unique multifunctor-structure, or more modestly
    to produce $E_\infty$-equivalences $K(H\R)\rightarrow H\M$ and 
    $H\M\rightarrow K(H\R)$, then either of these would imply that $H\M$ has 
    the unique multifunctor-structure given by composition of
    the unique structure on $K\colon Cat_{Sp}\rightarrow Sp$ and the conjecturally
    unique structure on $H\colon PermCat\rightarrow Sp$, while the second approach
    obviously directly gives the claimed equivalence.}}

Since I introduce the trace map $K\rightarrow THH$ by its multiplicative
universality as proven in \cite{BGT}, I want to emphasise the equivalence of 
$E_\infty$ and commutative structures in most model categories of spectra.

{\rem{\label{commeinf}
    In orthogonal and symmetric spectra with their positive stable model
    structure as well as in the category of $\S$-modules we have that commutative
    ring spectra model all $E_\infty$-ring spectra -- \cite[Lemma 15.5]{MMSS} 
    and also \cite[Theorem 5.1, Chapter III]{EKMM}. However, since I already use the
    reference \cite{EM} often in preceding chapters, I follow the setup of their
    Theorem 1.4; in particular I restrict to the case of symmetric and orthogonal
    spectra in the positive stable model structure for this remark.

    Considering the multicategory-structure on symmetric (or orthogonal) spectra 
    induced by the smash-product, it is meaningful to speak of multifunctor-categories
    $MFun(\mathbb{M},Sp^\Sigma),$ where we consider a simplicially enriched
    multicategory $\mathbb{M}$ and symmetric spectra with their natural simplicial
    mapping spaces. Given an enriched multifunctor $f\colon \mathbb{M}'\rightarrow
    \mathbb{M}$ we have an induced restriction functor: \[f^*\colon 
    MFun(\mathbb{M}',Sp^\Sigma)\rightarrow MFun(\mathbb{M},Sp^\Sigma)\]
    by precomposition with $f$, which by \cite[Theorem 1.4]{EM} is the right adjoint
    in a Quillen adjunction -- with the left-adjoint given by extension in the 
    appropriate manner (cf. p.56 of \cite{EM}).

    If $f$ is an equivalence of simplicially enriched multicategories, i.e., $\pi_0f$ is
    an equivalence of ordinary $1$-categories, and for each set of objects we have a
    weak equivalence of simplicial sets $M(a_1,\ldots,a_n;b)\rightarrow M'(fa_1,\ldots,fa_n;fb),$
    then Theorem 1.4 of \cite{EM} furthermore yields that this adjunction is a Quillen equivalence 
    (compare also \cite{Bergn1}). To my knowledge it has not been established that these
    equivalences of multicategories are weak equivalences in a model structure on 
    small multicategories, which preferably would extend the Bergner model structure 
    on simplicially enriched categories.

    When we consider the Barratt-Eccles operad $E\Sigma_*$ as a one-point 
    multicategory and map it to the terminal multicategory $Com$, we have
    an underlying isomorphism of (one-point) $1$-categories. Since each 
    multimorphism-category
    of $E\Sigma_*$ is equivalent to the one-point category we get a weak equivalence 
    of its nerve to a point as well, giving a Quillen equivalence:
    \[\xymatrix{\mathbb{P}\colon MFun(E\Sigma_*,Sp^\Sigma)\ar@<0.5ex>[r] & MFun(Com,Sp^\Sigma)
    \ar@<0.5ex>[l] \colon U,}\] where I consider the functors given by extension and 
    restriction as a prolongation functor $\mathbb{P}$ and a forgetful functor $U$. In particular
    for a cofibrant $E_\infty$-ring spectrum $A$ in $Sp^\Sigma$ we have a natural $E_\infty$-map
    $\eta\colon A\rightarrow UP(A)$ to a stably equivalent commutative symmetric ring 
    spectrum.}}

\subsection{The Opposite $E_\infty$-Structure}
Following Section 9 of \cite{EM} define the following map of operads.
{\defn{For each $k\geq 0$ set $r_k\colon \{1,\ldots,k\}\rightarrow \{1,\ldots,k\}$ to be $r_k(j)=k+1-j$, i.e.,
the map consisting of only transpositions $(1~ k)$, $(2~ k-1)$ until the centre. In other words $r_k$ fully
reverses the set $\{1,\ldots,k\}$.}}
{\lem{The symmetric sequences $(\Sigma_n)_{n\in\mathbb{N}}$ and $(E\Sigma_n)_{n\in\mathbb{N}}$ of 
categories, where $\Sigma_n$ is the discrete category on objects $\sigma\in\Sigma_n$, while $E\Sigma_n$
is the translation category of $\Sigma_n$ with objects $\sigma\in\Sigma_n$ and a unique morphism between
each pair of objects, form the associative and the Barratt-Eccles-operad.

Both maps $op\colon \Sigma_*\rightarrow \Sigma_*$, $op\colon E\Sigma_*\rightarrow E\Sigma_*$ defined
as $op(\sigma)=r_k\circ \sigma$ for $\sigma\in\Sigma_k$, and extended to a covariant functor in the unique
way on $E\Sigma_*$, are maps of operads. Specifically, $op$ respects units, is equivariant with respect to the
obvious right $\Sigma_*$-action by functors, and preserves multicomposition, i.e., block sum followed by
permutation of blocks.
\begin{proof}
The claim is explicit after Definition 9.1.11 of \cite{EM}, however, the proof is ``left to the reader''. Since
in particular the compatibility with multicompositions can be confusing, I want to elaborate on that.

Unitality of $op$ is obvious, since $r_1=id_{\{1\}}$. Equivariance is obvious as well, since we defined
$op$ by a left-action and the equivariance-condition involves the right-action of $\Sigma_n$ on itself.

For compatibility with multicompositions I first reduce the condition we have
to show drastically:
For $\Theta$ the multicomposition on $\Sigma_*$ and $E\Sigma_*$ we can
reduce an expression $\Theta(r_n\circ \sigma_n; r_{k_1}\circ \sigma_{k_1},\ldots, r_{k_n}\circ \sigma_{k_n})$
by using the right-$\Sigma_*$-action on the resulting product:
$\Theta(r_n\circ \sigma_n; r_{k_1}\circ \sigma_{k_1},\ldots, r_{k_n}\circ \sigma_{k_n})
= \Theta(r_n\circ\sigma_n; r_{k_1},\ldots,r_{k_n}).(\sigma_{k_1}\oplus\ldots\oplus\sigma_{k_n}).$
Thus we can, without loss of generality, consider a multiproduct $\Theta(r_n\circ\sigma_n; r_{k_1},\ldots,r_{k_n})$.
But by the ``inner'' equivariance condition on an operad, we can replace $\sigma_n$ on the left by
the appropriate permutation on the right, giving an expression $\Theta(r_n;r_{l_1},\ldots,r_{l_n})$.
As the final reduction, note that by multiassociativity of the multicomposition $\Theta$ it suffices
to show $\Theta(r_2;r_n,r_m)=r_{n+m},$ which yields the higher compatibilities by an easy induction.

Note that $\Theta(r_2;\sigma_n,\sigma_m)=\chi^+_{n,m}\circ(\sigma_n\oplus \sigma_m)$, for $\chi^+_{n,m}$
the symmetry of the sum in $\mathrm{Fin}$ and $\sigma_i\in\Sigma_i$. Thus we have to show 
$\chi^+_{n,m}\circ(r_n\oplus r_m)=r_{n+m}.$ This is an easy calculation, recall 
\[\chi^+_{n,m}(i)=\begin{cases}i+m,~ & 1\leq i\leq n,\\ i-n,~ & n+1\leq i\leq n+m,\end{cases}\]
and for convenience \[(r_n\oplus r_m)(i)=\begin{cases}r_n(i),~&1\leq i\leq n,\\r_m(i-n)+n,~&n+1\leq i\leq n+m.\end{cases}\]
The block sum trivially preserves the condition on $i$, thus the composite $\varphi:=\chi^+_{n,m}\circ(r_n\oplus r_m)$
is given by
\[\varphi(i)=\chi^+_{n,m}\circ(r_n\oplus r_m)(i)=\begin{cases}r_n(i)+m,~&1\leq i\leq n\\r_m(i-n),~&n+1\leq i\leq n+m.\end{cases}\]
Clearly we have $\varphi(1)=r_n(1)+m = n+m$ and $\varphi(n+m) = r_m(m) = 1$, so $\varphi=(1 ~n+m)\circ \bar\varphi$ and
$\bar\varphi$ is of the form $\Theta(r_2;r_{n-1},r_{m-1})$, so we are done by induction.\end{proof}}}

In particular we can define what the opposite algebra for an associative and an $E_\infty$-algebra are,
when we restrict to $E\Sigma_*$ as the $E_\infty$-operad.
{\defn{An associative algebra in spectra determines and is determined by a 
multifunctor $\Sigma_*\rightarrow Sp$, which sends the unique object of the
multicategory $\Sigma_*$ to the algebra. In particular, we define its opposite
algebra as the composition $\Sigma_*\rightarrow \Sigma_*\rightarrow Sp$ with 
the first map being the opposition $op$ above. This multifunctor determines
and is determined by the algebra with opposed multiplication, which is associative
if and only if the original multiplication is associative.

Analogously an $E_\infty$-algebra in spectra is a multifunctor $E\Sigma_*\rightarrow Sp,$
which is simplicially enriched with respect to the usual simplicial structure on spectra
and $E\Sigma_*$ made simplicial by arity-wise application of the nerve. Since $op$ is
in particular a map of sets in each arity, it extends uniquely to a functor $E(op)\colon
E\Sigma_*\rightarrow E\Sigma_*$ in each arity, thus to a simplicially enriched multifunctor.
The opposite $E_\infty$-algebra is given by precomposition with $E(op)$.

In particular the underlying associative algebra of an $E\Sigma_*$-algebra is given by
restriction along $\Sigma_*\rightarrow E\Sigma_*$ the inclusion of objects, and the underlying
associative algebra of the opposed $E_\infty$-structure is the opposed associative algebra.}}

We can thus define what an anti-involution on an $E\Sigma_*$-spectrum is.
{\defn{An anti-involution $\tau\colon A\rightarrow A$ for $A$ an $E\Sigma_*$-algebra is an
$E\Sigma_*$-map with respect to the $E\Sigma_*$-structure on the source, and the opposed
$E\Sigma_*$-structure on the target, with $\tau^2=\id$.}}

{\rem{Recall the Quillen-equivalence of $E_\infty$-ring spectra and commutative ring
spectra for instance in the positive stable model structure on symmetric spectra \cite{EM,MMSS},
or the model structure on $\S$-modules as exhibited in \cite{EKMM}. With the notation as in
Remark \ref{commeinf} we find that for $A$ a 
cofibrant $E\Sigma_*$-algebra the prolonged involution $\mathbb{P}\tau\colon \mathbb{P}A\rightarrow\mathbb{P}A^\mu$
is a map of commutative spectra. Since $\mathbb{P}(A^\mu)=\mathbb{P}(A)^\mu=\mathbb{P}A$ by
strict commutativity, we find that $\mathbb{P}\tau$ is a self-inverse algebra map of
commutative algebras, which by $\mathbb{P}(A)^\mu=\mathbb{P}A$ becomes an endomorphism.}}

\subsection{Induced Multiplications on $THH$ and the Involution}
I introduced the trace by its multiplicative universality as proven in \cite{BGT}, thus
we need to see that the induced involution of \ref{THHiota} opposes multiplication to find
that the trace map commutes with the involutions on $K$-theory and topological Hochschild 
homology.

The identification $N\C^{op}=\widetilde{N\C}$ extends to the cyclic nerve defining
$THH$ as we see above, and is strictly symmetric monoidal.
{\prop{The natural isomorphism
    \[\iota\colon THH(A^\mu,M^\mu)\rightarrow \widetilde{THH(A,M)}\]
    is strictly symmetric monoidal with respect to the smash product.
    \begin{proof}
    The rail-fence isomorphism indicated above and $\iota$
    are instances of the symmetry of the smash product, 
    thus the coherence of the smash-symmetry yields the claim.\end{proof}}}

The natural homeomorphism $\Gamma$ is symmetric monoidal as well.
{\prop{The natural homeomorphism $\Gamma\colon |\cdot|\Rightarrow |\widetilde{~\cdot~}|$ is
    symmetric monoidal with respect to cartesian as well as smash-product.
    \begin{proof} This is a bit easier to see by considering the symmetric
    monoidal structure of realisation oplax, i.e., $|X\times Y|\cong |X|\times |Y|$. The
    natural map in this case is given by realisation of $pr_X$ and $pr_Y$ respectively.
    In particular it is induced on simplicial objects. The natural transformation
    $\Gamma$, however, operates on the realisation coordinates, thus the transformations
    strictly commute.\end{proof}}}

These propositions assemble to the following:
{\thm{For an $E_\infty$-ring spectrum $A$ with anti-involution $T\colon (A,\mu)
\rightarrow (A,\mu^{opp})$, consider the internal involution induced on $THH$ by
\[\xymatrix{  THH(A)\ar[r]^-T & THH(A^\mu)\ar[r]^\iota & \widetilde{THH(A)} 
\ar[r]^\Gamma &THH(A).}\] This is a natural $E_\infty$-map with respect to the
induced $E_\infty$-structure on the source and the opposed $E_\infty$-structure
on the target $THH(A)$.
\begin{proof} This is just assembling the last three propositions, where we have
analysed each of the maps individually. In particular, the multiplicative opposition
induced from $T$ is not changed by $\iota$ and $\Gamma,$ thus follows the claim.\end{proof}}}

We can reuse \cite{BGT} to establish that the trace map commutes with the induced
involutions. {\thm{The unique natural transformation of (simplicially) enriched 
    multifunctors $tr\colon K\Rightarrow THH$ commutes with the involution on $K$ induced
    as in \ref{indinvSP} and induced on $THH$ as above.\label{trinv}\begin{proof} In the diagram
    \[\xymatrix{K(A) \ar[d]^{T_*}\ar[r] & THH(A)\\ K(A)\ar[r] & THH(A)\ar[u]^{T_*}}\]
    both the upper horizontal map as well as the map given by composing
    the induced involutions with the trace give an $E_\infty$-map $K(A)\rightarrow 
    THH(A)$. The first $E_\infty$-structure is directly asserted by \cite{BGT}, 
    the second follows from the fact, that the $E_\infty$-structure
    is opposed twice by the respective involutions. Thus by uniqueness of the
    multiplicative trace \cite{BGT} we get that the threefold composite describes
    the trace as well.\end{proof}}}

{\rem{With just a conjectural identification of the $E_\infty$-structures 
on $H(\MM(H\R))$ and $H(\M)$ the reference to \ref{indinvSP} in the theorem
is more informal than I intended. Formally, one could, however, set up exactly the same
procedure I describe in chapter \ref{multbidel} to establish that the involution
opposes the $E_\infty$-structure on $H(\MM(H\R))$ as well. This amounts
to mostly rewriting \cite{EM} the way I describe in \ref{pcatneu} but 
explicitly establishing the multiplicative structure for topologically
enriched permutative categories the way I do for bicategories in chapter
\ref{multbidel}. This is not an immediate specialisation of chapter \ref{multbidel},
because I implicitly assume discrete sets of $1$-cells in this thesis, 
but I expect no essential difficulties in generalising chapter \ref{multbidel} 
to bicategories with morphism categories internal to topological spaces.}}

{\rem{I think, the calculations in chapter \ref{calc} are more easily readable if I
declare for the reader how I think about the three maps involved in inducing
the involution on $THH,$ and the analogous sequence on $K$-theory \ref{indinvSP}, i.e.
the classifying spaces of the bicategory of matrices:
\[\xymatrix{THH(A)\ar[r]^-T & THH(A^\mu)\ar[r]^\iota & \widetilde{THH(A)} 
\ar[r]^\Gamma &THH(A),}\]
\[\xymatrix{B\M \ar[r]^{T} & 
        B\mathcal{M}(\mathcal{R}^\mu) \ar[r]^{(\cdot)^t}  
        & B\mathcal{M}(\R)^{op_1} \ar[r]^{\Gamma} & B\M.}\]
In both cases we directly induce a multiplicatively opposing map using
the involution. The transposition on matrices and the map $\iota$ 
allow to identify the simplices in the
nerve of the multiplicative opposition with simplices in the simplicially
opposed nerve. Finally $\Gamma$, in both cases, modifies the simplices
in the realisation by a map of degree $\pm 1$ only depending on the
simplicial degree of the simplex, which internalises the involution.

In summary: $\Gamma\circ\iota$ as well as $\Gamma \circ (\cdot)^t$ consist
of degrees and a preferred identification of simplices, thus are usually
easily analysed.}}

\section{A Useful Subspectrum of $THH$}
In section 3 of \cite{MS} the authors identify a simplicial subspectrum of
$THH_\bullet(A)$ which is naturally included for any $A$. Since then the
functor homology interpretations of Hochschild homology (cf. for instance
\cite{Loday,PRi}) via the Loday functor $\mathcal{L}(A,M)\colon \Fi\rightarrow 
k\mathrm{-Mod}$ have been established, providing a nicely clean, natural
interpretation of this subspectrum. I want to elaborate on this in this
section.

This section deserves an emphasised \emph{special 
acknowledgement}:
Since Stephanie Ziegenhagen and I have been close ever since our own
modest beginnings in Algebraic Topology, I have also observed her
conception of her thesis \cite{Z} in quite some detail. If I had not
been a test-case for numerous ``functor co*homology''-talks provided 
in her own trademark-clarity, I am quite sure I would never have 
understood and probably not even bothered to understand that context.
Thus the essential clarifications in this section rest firmly on her
shoulders.

\subsection*{Non-Commutative Sets}
The existence of the subspectrum identified in \cite[Section 3]{MS} 
does not depend on any additional structure on an associative ring 
spectrum $A$. In particular, it is not relevant if $A$ happens to be 
commutative or not. 

To properly identify the simplicial topological 
Hochschild homology spectrum for an associative algebra object 
however, one obviously needs to keep track of the order with respect 
to which one multiplies. The category of non-commutative sets $\Fia$ 
does just that. I follow the exposition in the sections 1.2-1.4 in 
\cite{PRi}, but instead immediately consider pointed sets, 
called $\Gamma(as)$ in \cite{PRi}.

{\defn{(cf. \cite[Section 1.2]{PRi})\label{fiass}
    The category $\Fia$ has objects pointed finite sets $n_+=\{*,1,\ldots,n\}$,
    and morphisms pointed maps $f\colon n_+\rightarrow m_+$ with chosen total
    orderings on the fibres $f^{-1}j$ for every $j\in m_+$ (including the
    basepoint). For maps composable in finite pointed sets, i.e., $f\colon n_+\rightarrow m_+$, 
    $g\colon m_+\rightarrow l_+$ the underlying map is the composite in finite
    sets $gf$ with total orderings on the fibres given as indicated by:
    \[(gf)^{-1}i=f^{-1}g^{-1}i=\coprod_{j\in g^{-1}i} f^{-1}j.\]
    Explicitly, the ordering of elements $j\in g^{-1}i$ provides an ordering
    of the summands, while each summand is ordered with the order chosen for $f$.}}

{\lem{\cite[Lemma 1.1]{PRi}\label{dec}
    Any morphism $f\colon n_+\rightarrow m_+$ in $\Fia$ has a unique decomposition 
    $\Delta_f\circ \sigma_f,$ where $\Delta_f\colon n_+\rightarrow m_+$ is
    order-preserving and pointed, and $\sigma_f\colon n_+\rightarrow n_+$ is 
    a bijection, usually not pointed.
    \begin{proof}The proof of \cite{PRi} is for the unpointed case, so I want
    to retrace the decomposition for pointed maps. Given a map $f\colon n_+
    \rightarrow m_+$ we find a unique order-preserving map $\Delta_f$ isomorphic 
    to it over $m_+$, i.e., with 
    \[\xymatrix{n_+\ar[r]^f \ar[d]_{\sigma_f}& m_+\\ n_+.\ar[ur]_{\Delta_f}}\]
    More explicitly $\Delta_f$ is the unique order-preserving map with the same
    fibre-cardinalities as $f$, i.e., $|\Delta_f^{-1}\{i\}|=|f^{-1}\{i\}|$ for 
    every $i\in m_+$. In particular there is a unique bijection 
    $n_+\rightarrow n_+$ which makes the fibres of $f$ into intervals in 
    $n_+$ while order preserving on the fibres. The total ordering chosen 
    on the fibres of $f$ then fixes a unique fibre-wise 
    bijection $n_+\rightarrow n_+$ of the order induced by $f$ to the order
    induced by the total order on $n_+$. The composite is the unique bijection
    $\sigma_f$.

    In particular we find that $\sigma_f$ is pointed if and only if the base-point
    is minimal in the chosen order of $f^{-1}\{*\}.$\end{proof}}}

{\rem{Since $\sigma_f$ is not pointed in general the
    decomposition is not internal to pointed sets. However, the bijection 
    necessarily still satisfies $\sigma_f(*)\in\Delta_f^{-1}(*),$ because 
    the composite is a pointed map.}}

This category makes it possible to define a Loday functor 
$\mathcal{L}(A,M)\colon \Fia\rightarrow Sp$ for an associative $\S$-algebra
$A$ and $A$-bimodule $M$. I want to specifically elaborate on the
dependence of $\mathcal{L}$ on the symmetric monoidal structure with
respect to which it is defined. Hence I consider a general symmetric
monoidal category $(\C,\otimes,\mathbbm{1},c_\otimes)$ an associative
$\otimes$-algebra $A$ in $\C$ and an $A$-bimodule $M$, keeping the
designations usual for monoids and bimodules in $(Sp,\wedge,\S,c_{\wedge}).$

{\defn{(cf. \cite[Section 1.3]{PRi})\label{Lod}
    The Loday functor $\mathcal{L}(A,M)\colon \Fia\rightarrow \C$ is given on
    objects as $n_+\mapsto M\otimes A^{\otimes n}$. For a morphism $f\colon 
    n_+\rightarrow m_+$ consider its unique decomposition $f=\sigma\circ \delta$.
    Then $\sigma$ describes a unique symmetry of the monoidal structure:
    \[c_\sigma\colon M\otimes A^{\otimes n} \rightarrow 
    A^{\otimes |f^{-1}\{*\}< *|}\otimes M \otimes A^{\otimes |f^{-1}\{*\}> *|}
    \otimes A^{\otimes |f^{-1}\{1\}|}\otimes\ldots \otimes A^{\otimes |f^{-1}\{m\}|}.\]
    From this reordered object we consider the map
    \[A^{\otimes |f^{-1}\{*\}< *|}\otimes M \otimes A^{\otimes |f^{-1}\{*\}> *|}
    \otimes A^{\otimes |f^{-1}\{1\}|}\otimes\ldots \otimes A^{\otimes |f^{-1}\{m\}|}
    \rightarrow M\otimes A^{\otimes m},\]
    composed of the left- and right-action of $A$ on $M$, i.e., 
    $A^{\otimes |f^{-1}\{*\}< *|}\otimes M \otimes A^{\otimes |f^{-1}\{*\}> *|}
    \rightarrow M$ and $|f^{-1}i|$-fold
    products $A^{\otimes |f^{-1}\{i\}|}\rightarrow A$. Call this map $\delta_*,$
    then define $f_*=\delta_*\circ c_\sigma.$ This makes $\mathcal{L}(A,M)$ a
    functor by uniqueness of the decomposition \ref{dec}.}}

To relate this functor to the simplicial topological Hochschild homology spectrum
I need to elaborate on the simplicial circle a bit more. More specifically, we need to
know that it is in fact a simplicial associative pointed set. 

{\prop{(cf. \cite[Section 1.4]{PRi})\label{ptdassS1}
    Recall the simplicial set $\Delta_1=\Delta(\_,[1])$ with its boundary 
    $\partial\Delta^1$ identified as the constant maps $f\colon [n]\rightarrow [1].$
    The quotient $\S^1=\Delta_1/\partial\Delta_1$ is a pointed simplicial set by 
    \ref{pointedS1}, which can be promoted to a pointed associative simplicial
    set $\S^1_{As}\colon \Delta^{op}\rightarrow \Fia$.}}

Given an associative $\S^1$, we can interpret topological Hochschild homology
for associative algebras. Recall the spectral
Loday functor for an associative $\S$-algebra $A$ and $A$-bimodule $M$:
\[\mathcal{L}(A,M)\colon \Fia\rightarrow Sp.\]
More generally for an associative algebra and bimodule in a 
symmetric monoidal category $(\C,\otimes,\mathbbm{1},c_\otimes)$ with
the assignment on objects: $\mathcal{L}(A,M)(n_+)=M\otimes A^{\otimes n}$.
The total order on the fibres precisely makes this a well-defined functor
for associative (as opposed to commutative) objects. We can easily identify
the simplicial Hochschild spectrum of $A$ with coefficients in $M$ 
as the composite of the Loday functor $\mathcal{L}(A,M)\colon \Fia\rightarrow Sp$
with the pointed associative circle \ref{ptdassS1}: $\S^1_{As}\colon \Delta^{op}
\rightarrow \Fia$.

{\prop{(cf. \cite[p. 213]{Loday}) The composite:
    \[\xymatrix{\Delta^{op}\ar[r]^-{\S^1_{As}}&\Fia\ar[rr]^-{{\mathcal{L}}(A,M)}&&Sp}\]
    is strictly equal to the simplicial Hochschild spectrum of $A$ with
    coefficients in $M$ as defined in \ref{THHblubbAM}. In particular, the
    identification is natural in maps of algebras and bimodules.
    \begin{proof}Objectwise the identification is clear, the example above 
    should convince the reader that I fixed the choices of orderings and 
    basepoints in \ref{ptdassS1} just so that one can identify the face 
    maps and degeneracies in the standard complex easily with the maps 
    induced on the pointed associative circle.\end{proof}}}

With these considerations in place we can find that the subspectrum
identified in \cite[Section 3]{MS} is inherent to the Loday-functor.

{\prop{\cite[Section 3]{MS}
    Let $A$ be an associative algebra, and $M$ an $A$-bimodule with
     a map of $A$-bimodules
    $A\rightarrow M.$ In particular both have a fixed unit map from 
    the sphere spectrum $\S\rightarrow A \rightarrow M.$

    Then we have two factorisations of the identity at $M$:
    \[\xymatrix{M\cong M\wedge\S\ar[rr]\ar[d]&& M\wedge A \ar[d]\\
    M\vee A \ar[r]& M\vee M=(M\wedge\S)\vee(\S\wedge M)\ar[r]& M,}\] 
    with the analogous factorisations 
    holding for the left-module action and the algebra structure of $A$. 

    Thus in particular for $A$ an $\S$-algebra, and $M$ an $A$-algebra, 
    we can define the Loday-functor with respect to coproducts
    $\L^\vee(A,M)(n_+)=M\vee A^{\vee n}$. By universal property of the
    coproduct to describe a map $\L^\vee(A,M)(n_+)=M\vee A^{\vee n}
    \rightarrow M\wedge A^{\wedge n}=\L(A,M)(n_+),$ we need to describe
    it on each summand. Thus for $M$ consider $M\cong M\wedge \S^{\wedge n}
    \rightarrow M\wedge A^{\wedge n},$ while the $i$th $A$-summand is
    mapped to the $i$th smash factor by the analogous description.

    The factorisation above gives that this is a natural transformation 
    \[\L^\vee(A,M)\Rightarrow \L(A,M),\]
    which is moreover natural in the algebra $A$ and the $A$-algebra $M$
    appropriately.}}

{\prop{(cf. \cite[Lemma 3.3]{MS}) The coproduct Loday-functor on an 
    $\S$-algebra $A$, and an $A$-algebra $M$, evaluated on the 
    associative circle is naturally isomorphic to the simplicial 
    spectrum $M\vee ((\S^1)\wedge A)$, for $\S^1=\Delta_1/\partial\Delta_1$ 
    the simplicial circle. 

    In particular, the geometric realisation of the natural transformation 
    above yields a natural map: \[|\L^\vee(A,M)(\S^1_{As})|=|M\vee 
    ((\S^1)\wedge A)| = M\vee \Sigma A \rightarrow THH(A,M).\]}}

I used the description of topological Hochschild homology as a Loday
functor on the associative circle to facilitate the following identification:
{\thm{The Loday functor evaluated on the opposite associative 
    circle $\S^1_{As}\circ r\colon \Delta^{op}\rightarrow \Fia$ 
    is naturally isomorphic to the Loday functor on
    the opposite algebra and opposed bimodule:
    \[\L(A,M)(\S^1_{As}\circ r)\cong \L(A^\mu,M^\mu)(\S^1_{As}).\]
    In particular, for $M$ an $A$-algebra, the natural transformation
    \[\L^\vee(A,M)\Rightarrow \L(A,M),\]
    commutes with this isomorphism, as does its geometric realisation.}}

We can draw the following corollary, which I use repeatedly in the following chapter.
To not confuse opposition of module actions with opposition of multiplications I only
assume an appropriate map $A\rightarrow M,$ which we need for the map $\L^\vee(A,M)\rightarrow \L(A,M)$.
{\thm{Given an associative $\S$-algebra $A$ with anti-involution 
    $T\colon A\rightarrow A^\mu$, and an left- and right-$A$-linear map 
    $A\rightarrow M$ we have the following commutative diagram:
    \[\xymatrix{ M\vee \Sigma A\ar[d]\ar[r]&M\vee \Sigma A\ar[d]\ar[r]
    \rruppertwocell<10>^{1\vee(-1)}{\omit}& M\vee \widetilde{\Sigma} A\ar[d]\ar[r] 
    & M\vee \Sigma A\ar[d]\\  THH(A,M)\ar[r]^{(T,\id_M)}&THH(A^\mu,M^\mu)\ar[r]^\iota 
    &|\widetilde{THH(A,M)}|\ar[r]^\Gamma &THH(A,M).}\]

    More explicitly: The bimodule includes at simplicial degree $0$, thus each of the lower
    three maps restricts to the identity. The suspended algebra is the realisation of the
    simplicial spectrum $\S^1\wedge A,$ hence includes at degree $1$, so that $\iota$ and
    $\Gamma$ together induce a sign.}}

{\rem{I want to place the appropriate emphasis on this subspectrum. Despite the fact that these
    are the ``obvious'' classes in $THH(A,M)$, they are usually not trivial. Instead one can
    usually use the suspension $\Sigma A\rightarrow THH(A,M)$, inducing a map 
    $h_*A\rightarrow h_{*+1}THH(A,M)$, and multiplicative structures on $THH$ to exhaust 
    the classes of interest. Good examples of this are \cite{MS} and \cite{AuTHH}, the second
    of which we study in detail in the next chapter.}}
