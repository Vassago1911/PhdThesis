\documentclass[12pt,bibliography=totoc,final]{scrbook} %11pt+=oneside
\usepackage{a4} \linespread{1.2} \usepackage[all,2cell]{xy} \UseAllTwocells
\usepackage{stmaryrd}\usepackage{bbm} %defines \mathbbm, thus \mathbbm{1}
\usepackage{amsmath,amssymb,amsfonts,amsthm} \usepackage[english]{babel} \usepackage[final]{hyperref} \usepackage{mathtools}
%\usepackage{fancyhdr} \usepackage{lscape}

\newtheorem{thm}{Theorem}[section] \newtheorem{lem}[thm]{Lemma} \newtheorem{prop}[thm]{Proposition} \newtheorem{cor}[thm]{Corollary} \theoremstyle{definition}
\newtheorem{Remi}[thm]{Reminder} \newtheorem{rem}[thm]{Remark} \newtheorem{defn}[thm]{Definition} \newtheorem{ex}[thm]{Example} \newtheorem{conj}[thm]{Conjecture}
\def\div{\mathrm{div}} \def\mod{\mathrm{mod}} \def\Ob{\mathrm{Ob}} \def\A{\mathcal{A}} \def\C{\mathcal{C}} \def\D{\mathcal{D}} \def\V{\mathcal{V}} \def\K{\mathcal{K}}
\def\l{\ell} \def\L{\mathcal{L}} \def\R{\mathcal{R}} \def\RR{\mathbb{R}} \def\M{\mathcal{M}(\mathcal{R})} \def\MM{\mathcal{M}} \def\CC{\mathbb{C}} \def\fC{\mathfrak{C}}
\def\P{\mathbb{P}} \def\S{\mathbb{S}} \def\H{\mathbb{H}} \def\E{\mathbb{E}} \def\F{\mathbb{F}} \def\Aix{A_+\downarrow\mathrm{Fin}_+} \def\Bix{B_+\downarrow\mathrm{Fin}_+}
\def\Fi{\mathrm{Fin}_+} \def\Fia{\mathrm{Fin^{As}_{+}}} \def\Ep{\mathrm{Epi}_+} \def\barF{\overline{\mathbb{F}}}
\def\cp{\boxempty} \DeclareMathOperator{\id}{id} \DeclareMathOperator{\im}{im} \DeclareMathOperator{\sk}{sk} \DeclareMathOperator{\cosk}{cosk} 
\DeclareMathOperator{\colim}{colim} \DeclareMathOperator{\const}{const} \DeclareMathOperator{\hocolim}{hocolim} \DeclareMathOperator{\holim}{holim}
\renewcommand{\S}{\ensuremath{\mathbb{S}}} \newcommand{\Z}{\ensuremath{\mathbb{Z}}} \newcommand{\N}{\ensuremath{\mathbb{N}}} 
\newcommand{\SymMonCat}{\mathcal{S}\mathrm{ym}\mathcal{M}\mathrm{on}\mathcal{C}\mathrm{at}}
\DeclarePairedDelimiter\ceil{\lceil}{\rceil} \DeclarePairedDelimiter\floor{\lfloor}{\rfloor}
\def\arxiv#1{\href{http://arxiv.org/abs/#1}{\texttt{arxiv:#1}}} \def\arxivold#1{\href{http://arxiv.org/abs/math/#1}{\texttt{arxiv:#1}}}


\newcommand*{\titleGP}{\begingroup % Create the command for including the title page in the document
\thispagestyle{empty}\clearpage\thispagestyle{empty}\clearpage

\begin{center} % Center all text
\vspace*{\baselineskip} % White space at the top of the page

\rule{\textwidth}{1.6pt}\vspace*{-\baselineskip}\vspace*{2pt} % Thick horizontal line
\rule{\textwidth}{0.4pt}\\[\baselineskip] % Thin horizontal line

{\LARGE MULTIPLICATIVE STRUCTURES\\ AND INVOLUTIONS ON \\[0.3\baselineskip] ALGEBRAIC $K$--THEORY}\\[\baselineskip] % Title

\rule{\textwidth}{0.4pt}\vspace*{-\baselineskip}\vspace{3.2pt} % Thin horizontal line
\rule{\textwidth}{1.6pt}\\[\baselineskip] % Thick horizontal line

{{\scshape % Small caps
Dissertation with the aim of achieving a doctoral degree
at the \\ Faculty of Mathematics, Informatics and Natural Sciences,
Department of Mathematics \\% Tagline(s) or further description
of Universit{\"a}t Hamburg \vfill}
Submitted by: Marc Lange \\ 
{2015} in Hamburg}

\end{center}
\endgroup}

\date{\the\day.\the\month.\the\year}

\begin{document}
\pagestyle{empty}\titleGP \null\newpage \null\vfill Als Dissertation angenommen vom Fachbereich \\
Mathematik der Universit\"at Hamburg\\[2em] aufgrund der Gutachten von Prof. Dr. Birgit Richter\\
\phantom{aufgrund der Gutachten} und Prof. Dr. Bj\o rn Ian Dundas.\\[2em]

Hamburg, den 24.7.2015 \\
Datum der Disputation: 26.6.2015 \newpage\clearpage \par\vspace*{.35\textheight}{\centering To Cedric\par} \newpage
\chapter*{Introduction}

Algebraic $K$-Theory encompasses a variety of mathematical disciplines, natural
settings, questions and tools, and thus also motivations. As an algebraic topologist
I primarily use the motivation by chromatic homotopy theory. I refer the interested
reader to \cite{Rog14} for a general introduction to the red-shift conjecture, while 
focussing on the aspects which centrally motivate my thesis here.

Since algebraic topology formed as an independent mathematical discipline, a major
aspect of it is the study of invariants associated to topological spaces. The most prominent
and easily defined of these invariants are the homotopy groups associated to a topological
space. The higher homotopy groups, i.e., for $n\geq2$, 
give a countable family of abelian groups associated to a space. These
however suffer from the defect of being very hard to compute. Even more drastically,
it is usually a very hard question to establish, if a homotopy group of some fixed degree
is trivial or non-trivial for some given space, unless elementary arguments
force its triviality.

Much easier to compute, but harder to define are (singular) homology groups, defined by 
a variety of constructions up to the 1950s, proved to coincide by an axiomatic approach by 
Eilenberg and Steenrod in 1945. In the 1950s people found that there are more
constructions satisfying all except one of the axioms, thus establishing topological
$K$-theory and bordism as ``extraordinary'' homology theories. 

The Atiyah-Hirzebruch spectral sequence in principle makes it possible to compute
any homology theory on any space, given only its singular homology with $\Z$-coefficients
as input. However, chromatic homotopy theory establishes
that singular homology is the least complex of all homology theories. In particular,
it gives a conceptual reason for the difficulties one encounters, when one tries to actually
fully calculate the Atiyah-Hirzebruch spectral sequence for specific spaces. Cobordism is
at the other extreme, having chromatic complexity $\infty$ in a conceptually satisfying sense. 

In particular one could hope for an iterative approach to understanding invariants on a 
topological space by starting with singular homology on this space, thus at complexity
$0$, and then iterating from complexity $n$ to $n+1$. Chromatic
red-shift as described for instance in \cite{Rog14} is the conjecture that one way to
produce such intermediary theories of increasing complexity is given by 
iterating algebraic $K$-theory on the associated spectra. 

Suslin has established in 1984 that for any separably closed field $F$ its algebraic 
$K$-theory completed at primes $p$ (other than its characteristic) is equivalent to
topological complex $K$-theory. Thus the singular homology with coefficients in $F$
being a theory of complexity $0$ is transformed into a theory of complexity $1$.
By the results of Christian Ausoni, specifically the ones in \cite{AuKku} we know
that $K(ku_p)$ is the spectrum associated to a cohomology theory of complexity $2$.
In general chromatic red-shift predicts that this is part of a pattern, saying that
algebraic $K$-theory raises chromatic complexity by one.

This thesis is specifically concerned with the involutive structures present on these
spectra. Singular homology arises as the Eilenberg-MacLane spectrum of a discrete
rig category, while \cite{Ri2010} shows that on algebraic $K$-theory objects we always
have an involution induced by transposing and inverting matrices. On complex $K$-theory
this involution specialises to the natural involution induced by complex conjugation.
One chromatic step higher this involution describes the operation on $2$ vector bundles
\cite{BDR2004}, which conjugates each transitional vector bundle. On $K$-theory of
complex $K$-theory we can describe this as the involution induced by transposition
and inversion in both iterations of $K$-theory.

In Chapter 1 I recall the most prominent combinatorial models for connective spectra given
by ringlike categories, specifically bimonoidal and bipermutative categories, while in
addition recalling the results of \cite{EM} delooping their classifying spaces. I rewrite
the construction of \cite{EM} in such a way that I can easily generalise it in Chapter 3 to
bicategories.

In Chapter 2 I establish a multiplicative structure on a combinatorial model for modules
over a bipermutative category as already studied additively by Ang\'elica Osorno in
\cite{Os}. I tie in the multiplicative structure with her additive structure in a manner
that makes this module bicategory a ringlike object again.

In Chapter 3 -- the technical core of this thesis -- I set up a multiplicative delooping
analogous to \cite{EM} for permutative bicategories, which generalises the one of \cite{Os},
while allowing a multiplication to be induced by the tensor product defined in chapter 2. 

In Chapter 4 I offer a few partial results, essentially summaries of known results in
various papers, pertaining to the uniqueness of such structures. The delooping of a
module category of a permutative category is sufficiently unique to fix the spectrum
by minimal data as observed by May and Thomason in 1978 \cite{MT1978}. However, I was
unfortunately unable to prove multiplicative uniqueness of this delooping, which would
as a corollary imply that the multiplicative structure of chapter 3 is the same as the
one obtained by iterating the construction of \cite{EM} twice. I do outline two
arguments by which one might approach this conjectural uniqueness.

Chapters 5 and 6 are concerned with the motivating calculational example $K(ku)$. In particular,
since the calculations of \cite{AuKku} are done by trace methods, i.e., by computations along
the natural map $K\rightarrow THH$, I recall the definition of topological Hochschild homology
in chapter 5. Fixing conventions along the way, specifically how an involution on a ring spectrum
induces one on its topological Hochschild homology, we find that the trace is compatible with
the involutions defined in chapters 2 and 5. 

Compare this to the introduction of Dundas \cite{D98}, where he states that the construction of the
trace map in the context of \cite{D98} 
is compatible with involutions induced by the appropriate functors. Thus the result in
chapter 5 establishes that we have internalised the involution in chapter 2 on $K$-theory
and in chapter 5 on topological Hochschild homology in a compatible way.

Finally in Chapter 6 I retrace the calculations of \cite{AuKku} to the extent that I can
establish the effects of the involution on $K(ku)$ on classes, which are not in the kernel
of the trace map.

\section*{Acknowledgements}
Obviously such a thesis cannot be written in a social vacuum, so I want to take this opportunity
to thank a few people.

First and foremost obviously my advisor, Birgit Richter: In addition to creating a safety, from
which I could explore the multiplicative delooping in chapters 2 and 3, she quite regularly helped
me with indispensable advice when I became disoriented mathematically or socially.

I gratefully acknowledge the funding and environment of the Research Training Group 1670. In addition
to the fact that I was able to focus exclusively on my research in the first three years of my
postgraduate studies, being a part of a big group of PhD students made many experiences seem less 
lonely, which would have been frustrating otherwise. Furthermore being entrusted with the position
of spokesperson for the PhD students I gained valuable insight in how to obtain a
group opinion efficiently and also had an in-depth look at the process of application talks, which
made them much more transparent, hence in particular less frightening to me.

It was plain fortunate that in 2012 at the Arolla conference I had the opportunity to meet with and
talk to Robert Bruner, who explained to me the Adams spectral sequence and how to effectively calculate
with it so pleasantly and patiently that I still remember each minute of it fondly.
Also I want to thank Christian Ausoni for giving me a nice reader's guide to his papers, which helped
a lot in reading them correctly and efficiently, thus kickstarting my knowledge of $K(ku)$.

This section could never be complete without mentioning Stephanie Ziegenhagen. I agree it
was a perfect match, when our paths crossed in Algebraic Topology I, which is now 8 years ago. Since
then we have been through so many things together, and experienced a lot of things so integrally 
together that I could never imagine them without you. As this is probably the last thesis, in which
we address acknowledgements to each other, I want to close the circle with: \emph{May the force of the
universal property always save us from coordinates.} :)

Finally I want to extend special thanks to people, who read the draft in various stages of completion:
My parents for catching numerous errors and typos, Birgit Richter in particular for pruning the 
nonsense I had written, Fabian Kirchner for his careful check of the mathematical prose, and finally
Stephanie Ziegenhagen for her fail-safe instinct to find undefined and confusing parts, and attach
the criticism down to the word or letter that causes the confusion.
 \tableofcontents\thispagestyle{empty} \addtocontents{toc}{\protect\thispagestyle{empty}}  %unterdrueckt Seitenzahlen im Toc

\chapter{Permutative Categories and Connective Spectra}
\pagestyle{headings} 
This chapter is just a summary of known techniques to combinatorially
model (connective) spectra by permutative categories. In particular
until \ref{pcatneu} there is nothing original in this chapter. 
The examples are borrowed from \cite[pp. 160-167]{MayEinf} and 
\cite[pp. 337+338]{Ri2010}. However I deviate quite a bit
from May's notation. Furthermore I want to warn the reader that I am 
close to an erroneous sequence of lemmas in \cite{MayEinf} 
(VI.2.3, VI.2.6, VI.4.4) (cf. \cite[p. 321]{May2009}). 

The claimed result in \cite{MayEinf} can be stated informally as: 
Bipermutative categories yield maximally homotopy commutative ring-spectra. 
The error is combinatorial, in how the multiplication and 
addition ought to interact, governed by the notion of an 
``operad pair''. But the claimed ``operad pair'' in \cite{MayEinf} is
not an operad pair as defined there. The result still holds \cite{May2009} 
(and its accompanying papers), \cite{EM}, but the techniques employed differ 
quite a bit from the planned proof in \cite{MayEinf}.

\section{Delooping Permutative Categories}
Permutative categories seen through a modern eye are a categorified
version of abelian groups with just as much strictness as generality
would allow - compare the classical strictification result \ref{strict1}.
{\defn{A \textbf{permutative category} $(\A,+,0,c_+)$ is a category $\A$ 
together with a functor $+\colon \A\times\A\rightarrow \A$, a strict 
additive unit $0\in\A$, and a twist natural transformation (for $T\colon \A\times\A\rightarrow
\A\times\A$ the exchange of factors):
\[c_+\colon (+\circ T)\Rightarrow +,\] satisfying the following 
conditions:\begin{enumerate}\item $+$ is strictly associative:
\[+\circ (+\times id)=+\circ (id\times +),\]
\item The unit $0$ is a strict unit for $+$:\[0+\_=\_+0=id_\A,\]
\item The twist is trivial at $0$: For every object $a\in\A$ we have the
identity \[c_+=id: a=0+a\rightarrow a+0=a,\]
which is natural in $a$.
\item The twist is its own inverse: For every two objects $a,b\in \A$ we 
have the commutative diagram:
\[\xymatrix{a+b\ar[dr]_{c_+} \ar@=[rr]&& a+b \\& b+a.\ar[ur]_{c_+}}\]
\item The twist is associative: For each triple of objects $a,b,c\in\A$
we have the commutative diagrams:
\[\begin{array}{cc}\xymatrix{a+b+c\ar[r]^{c_+}\ar[dr]_{id+c_+} & c+a+b\\
& a+c+b\ar[u]_{c_++id},}~~~
\xymatrix{a+b+c\ar[r]^{c_+}\ar[dr]_{c_++id} & b+c+a\\
& b+a+c\ar[u]_{id+c_+}.}\end{array}\]\end{enumerate}}}

I have no need in this thesis for the most general symmetric monoidal 
categories given for instance by module categories. Nonetheless the 
following statement shows that the structure of permutative 
categories is sufficiently general:

{\thm{\label{strict1} For any symmetric monoidal category 
$\C$ there is a symmetrically monoidally 
equivalent permutative category $Str(\C)$ with a natural
equivalence $Str(\C)\rightarrow \C$.}}

There are many ways to obtain this result. A brute force way is to
consider words in objects, have the empty word be the strict unit, and
add in morphisms accordingly 
\cite[Prop VI.3.2,cf. pp.155-157]{MayEinf}. This proof has as a 
corollary that a small symmetric monoidal category of cardinality 
$\aleph$ yields a permutative category of size smaller than 
$\aleph^\omega$ (for $\omega=|\N|$). In particular the theorem stays 
true with ``small'' added in everywhere.

The high-tech way to show this result is given by the Yoneda Lemma
in bicategories \cite[2.3]{Lei}. One considers a monoidal category 
$\C$ as a one-point bicategory $\Sigma\C$, embeds it into the equivalent 
one-point 2-category given by the essential image of the Yoneda 
embedding \[Y\colon \Sigma\C \rightarrow Fun(\Sigma\C^{op},Cat_1)\]
and the symmetry just comes along. Then sizes are limited by the
bicategorical Yoneda Lemma.

Regarding enrichments we find that a
topological, simplicial, etc. symmetric monoidal category strictifies to a 
permutative category of the same kind.

\subsection{Bimonoidal Categories} Since this thesis is about 
multiplicative structures, I want to introduce the types of 
multiplication on permutative categories right away. Furthermore, I 
prove a convenient lemma, so that I do not have to bore the 
reader with pages of coherence diagrams. The following two concepts 
are directly copied from \cite{EM} for instance, although a look into
\cite{MayEinf} shows that at least the $E_\infty$-version (i.e., 
bipermutative categories) was known to be a fruitful concept for much 
longer. Like the concept of permutative categories these concepts
are strictified versions of general ringlike objects in $1$-categories.
Laplaza has shown in 1972 \cite{Lap} that the analogous strictification
result to \ref{strict1} above holds for these structures. Thus the
following definitions represent no loss of generality.

{\defn{\label{bim1}A \textbf{ring category} $(\R,+,\cdot,0,1,c_+)$ is given by a 
permutative structure $(\R,+,0,c_+)$ and a strictly associative and 
strictly unital monoidal structure $(\R,\cdot,1)$, which interact by 
two natural isomorphisms: \[\lambda \colon ab+ab'
\rightarrow a(b+b'),~~ \rho \colon ab+a'b\rightarrow (a+a')b,\]
 such that the following properties hold:
 \begin{enumerate} \item $0$ is a strict zero for multiplication $\cdot$:
\[0\cdot a = a\cdot 0 = 0~~~\forall a\in \R,\]
\item $+$-associativity of distributors:
\[\lambda\circ(\lambda+ \id)=\lambda\circ(\id+ \lambda),~~
\rho\circ(\rho+ \id)=(\id+ \rho)\circ \rho,\]
\item additive symmetry of distributors:
\[(c_+ \cdot \id)\circ\lambda = \lambda\circ c_+,~~
(\id \cdot c_+)\circ\rho = \rho\circ c_+,\]
\item $\cdot $-associativity of distributors:
\[\lambda = (\id\cdot \lambda)\circ\lambda,~~
\rho = (\rho \cdot \id)\circ\rho,\]
\item middle $\cdot$-associativity of distributors:
\[(\id\cdot \rho)\circ\lambda=(\lambda\cdot \id)\circ\rho,\]
\item mixed associativity of distributors:
\[\lambda\circ(\rho+\rho)= \rho\circ(\lambda+\lambda)
	\circ(\id+c_++\id).\]\end{enumerate}}}

It is nice to have a multiplicative structure on any given object, but
it is genuinely hard to produce ring categories, which are not also 
commutative up to some degree or an infinity of degrees. In particular,
I do not investigate plain ring categories in this thesis. So I define 
the $E_\infty$-multiplicative version next, also directly following 
\cite{MayEinf,EM}, but explicitly with no strictness assumptions on
either distributor. This type of category is the central object of study
in the first three chapters.

{\defn{A \textbf{bipermutative category} $(\R,+,\cdot,0,1,c_+,c_\cdot)$ is a 
ring category, where the multiplicative category $(\R,\cdot,1)$ is 
also permutative with twist $c_\cdot$, and where the distributors are 
interrelated via the multiplicative twist as follows:
\[\xymatrix{ab+ab'\ar[r]^{\lambda} \ar[d]^{c_\cdot+c_\cdot} & a(b+b') 
						\ar[d]^{c_\cdot}\\
ba+b'a \ar[r]^{\rho} & (b+b')a.}\]}}

{\rem{\label{IsoIsoIso} Do note that although I am following 
\cite{EM} as well, I want the distributivity transformations to be 
isomorphisms! This is essential in defining the bicategory of matrices
in Chapter \ref{modulbicat}. 

I do not fix either distributor to be the identity intentionally: 
Since I want to investigate multiplicative structures 
interacting with involutions, having both distributors general 
isomorphisms not forced to be identities, makes it meaningful to speak 
of the multiplicatively opposite bimonoidal/bipermutative category.}}

\subsection{Bipermutative Structures on Finite Sets}
As an illustration that bipermutative categories are natural 
things to consider, I give a lemma which applies in a variety of 
cases, where the category is a skeletal version of some category 
with coproducts and products (or tensor products). I have denoted the
following lemma analogous to the second monoidal structure being the
product. However, I want to explicitly emphasise that I do not assume
$\pi$ to be a product-functor or the unit $*\in\C$ to be terminal.

{\lem{\label{bipermallthethings} Let $\C$ be a small category with 
coproducts, which is furthermore a permutative closed category with
monoidal structure \[\pi\colon \C\times\C\rightarrow \C,\] 
which is strictly associative and has unit $*\in\C$. Assume $\pi$ has a
right adjoint: \[Hom(-,-)\colon \C^{op}\times\C\rightarrow 
\C.\] Assume furthermore a chosen functor representing coproducts:
\[\sqcup\colon \C\times \C \rightarrow \C,\] which is strictly 
associative, and a chosen representative $\emptyset$ for the
initial object, thus the unit for $\sqcup$.

Then this can be endowed with unique natural transformations 
$c_\sqcup,c_\pi,\lambda,\rho$, such that the resulting tuple 
$(\C,\sqcup,\pi, \emptyset,*,c_\sqcup,c_\pi)$ is a bipermutative 
category.

\begin{proof} The proof is just a repeated application of universal 
properties. Because $\pi(-,c)$ has a right adjoint, it commutes with 
coproducts, in particular we have $\pi(\emptyset,c)=\emptyset, ~~
\forall c\in\C$. Consider the additive symmetry: For $T\colon \C\times
\C\rightarrow \C\times \C$ the symmetry of the product on categories
both $\sqcup$ and $\sqcup\circ T$ represent a coproduct-functor on $\C$, 
hence by the universal property of the coproduct, we get a unique natural 
transformation \[c_\sqcup\colon \sqcup\circ T\Rightarrow \sqcup.\]
By uniqueness of the natural isomorphism $c\sqcup d\rightarrow c\sqcup d$, for
every pair $c,d\in\C$, we also get that $c_\sqcup$ is a 
symmetry: \[id_{\sqcup}=c_\sqcup^2\colon \sqcup = 
\sqcup\circ T^2\Rightarrow \sqcup.\]
The other natural isomorphisms are constructed much the same way.

For the interaction of the natural isomorphisms consider for
instance the relation $\lambda=c_\pi\circ\rho\circ(c_\pi\sqcup c_\pi)$.
Both are natural isomorphisms between the functors:
\[\sqcup\circ (\pi\times\pi)\circ (id\times T\times id)
\circ (\Delta\times id) \Rightarrow \pi\circ (id\times \sqcup),\]
hence again by uniqueness of those natural isomorphisms we have 
equality. Every other diagram commutes by the same reasoning.
\end{proof}}}

This lemma illustrates well, why ring categories which are just
associative but admit no multiplicative symmetries, are a bit
harder to come by. The easy (closed) monoidal constructions usually
come from symmetric universal properties: product, tensor product, 
smash product, etc.. What follows are the prototypical examples 
which provide the structural morphisms in my examples of interest.

{\ex{\label{Fin} In everything ringlike that follows, the category 
with objects the non-negative integers $\{\mathbf{n}=\{1,\ldots,n\}|~~
n\in\N_0\}$ and with 
morphisms the symmetric groups $\Sigma_*(\textbf{n},\textbf{n}) = 
\Sigma_n$, i.e., $\Sigma_* = \coprod_n \Sigma_n$, features 
prominently. In particular categories and bicategories of the 
form ``free modules over $\R$'' have their 
structural natural transformations given by permutations.

Define the following functors: \[+,\cdot\colon \Sigma_*\times\Sigma_*
\rightarrow \Sigma_*\] on objects: \[\textbf{n}+\textbf{m} := \{1,\ldots,n+m\}, 
\textbf{n}\cdot \textbf{m} := \{1,\ldots,nm\},\]
and more interestingly on morphisms: \[(f+g)(i):=\begin{cases} f(i), ~~
& i\leq n,\\ g(i-n)+n, ~~& i\geq n+1, \end{cases}~~~ \mathrm{for~~} 
f\in \Sigma_n, g\in\Sigma_m,\] and  \[(fg)((i-1)m+j) := (f(i)-1)m+g(j) 
~~~ (i=1,\ldots,n; j=1,\ldots m).\] Note that this implicitly fixes a 
choice of bijections $n\times m\rightarrow nm$, in this case given by 
$(i,j) \mapsto (i-1)m + j$. Two easy calculations show that $+$ and 
$\cdot$ defined this way are strictly associative. To use Lemma
\ref{bipermallthethings} we need to exhibit $+$ as representing 
coproducts, so consider the embedding \[\Sigma_* \rightarrow \mathrm{Fin}.\]
That is, we embed the skeletal category of finite sets and
bijections into the skeletal category of finite sets and all maps.
Together with the canonical injections $\textbf{n}\rightarrow \textbf{n}+\textbf{m}$
and $\textbf{m}\rightarrow \textbf{n}+\textbf{m}$, which are part of the
category $\mathrm{Fin}$, we have that $+$ represents coproducts.
Hence $\mathrm{Fin}$ is a bipermutative category by Lemma \ref{bipermallthethings}.
The structural maps we get are all isomorphisms, so restricting to
$\Sigma_*$ again makes $\Sigma_*$ a bipermutative category. Furthermore
we can restrict to its subcategory on all objects with just epimorphisms
and get the bipermutative category $\mathrm{Epi}$. Also we can restrict
to the subcategory on all objects with just injections to get the bipermutative
category $\mathrm{Inj}$.

The induced additive symmetry $c_+\colon n+m\rightarrow m+n$ is given 
by: \[c_+(i) :=\begin{cases}i+m, ~~& i\leq n,\\ i-n, ~~&n+1\leq i,
	\end{cases}\] and we have the multiplicative symmetry $c_\cdot
\colon nm \rightarrow mn$ given by:\[c_\cdot((i-1)m+j) = (j-1)n + i.\]

Recall that the distributivity transformations of bipermutative 
categories determine each other \[\xymatrix{ab+ac\ar[r]^{\lambda} 
\ar[d]_{c+c} & a(b+c) \ar[d]^c\\ ba+ca\ar[r]^\rho & (b+c)a.}\]
May shows in full generality \cite[p. 155, Proposition 3.5]{MayEinf} that one
can always strictify one distributivity to be the identity. For finite
sets this corresponds to ordering a product of finite sets either 
lexicographically or anti-lexicographically. This way we find for 
general bipermutative categories two one-sidedly strict cases:
\[\lambda = id \Rightarrow \rho_{b,c;a} = c_{a,b+c}\circ 
(c_{b,a}+c_{c,a}),\] \[\rho = id \Rightarrow \lambda_{a;b,c} 
= c_{b+c,a}\circ (c_{a,b}+c_{a,c}).\] The choice of bijection for 
products considered above forces $\lambda=id$ and a non-trivial right 
distributivity, so the first case. This makes $\Sigma_*$ into a 
bipermutative category. The opposite choice with $\rho=id$ is given
for instance in \cite[p. 161, Example 5.1]{MayEinf}.}}

{\ex{\label{Finp}We can just repeat the argument above to find 
the pointed analogues of the above categories, hence we get 
$\mathrm{Inj}_+,\Ep,\Fi$. For definiteness let me reemphasise 
the bipermutative structure on these categories:

The coproduct (for $\mathrm{Inj}_+$) is the pointed sum, hence 
we can define a strictly associative functor representing it by:
\[ \mathbf{n}_++\mathbf{m}_+:= \mathbf{n}_+\vee \mathbf{m}_+ = 
(\mathbf{n}+\mathbf{m})_+,\] with the obvious extension to morphisms.

Fixing a choice of associative bijections $\bar\omega_{n,m}\colon
\mathbf{n}\times \mathbf{m}\rightarrow \mathbf{nm}$ induces associative 
bijections for the smash product by: \[\mathbf{n}_+\wedge \mathbf{m}_+ 
= (\mathbf{n}\times \mathbf{m})_+ \rightarrow (\mathbf{nm})_+ = \mathbf{nm}_+.\]

Hence the symmetries and distributors are given by adjunction of 
basepoints to the relevant morphisms above. In other words, given 
the bijections $\omega$, I fix the pointed structures such that
\[(\cdot)_+ \colon \mathrm{Fin}\rightarrow \mathrm{Fin}_+\]
becomes a strictly bipermutative functor with respect to disjoint
union and pointed sum, and cartesian product and smash product.}}

\subsection{Bicategories - Notation for this thesis}
In \ref{pcatneu} I use functors between bicategories with non-trivial
coherence 2-cells, so I fix the notations and conventions for 
bicategories here.
{\defn{A small \textbf{bicategory} with strict identities $\C$ is given by 
a set of objects $\C_0=Ob\C$, a set of 1-categories $\C_1=Mor\C$, 
and the following maps:\begin{itemize}
\item source and target \[s,t\colon \C_1\rightarrow \C_0,\]
where we call objects $a,b\in\C_1$ with $t(a)=s(b)$ composable,
\item identity objects \[\id_\_\colon \C_0\rightarrow \C_1,\]
with $s\circ \id_\_ = t\circ \id_\_=id_{\C_0},$
\item a composition functor \[\boxempty\colon \C_1\times_{\C_0}\C_1
\rightarrow \C_1,\]where the pullback is to be understood as 
\emph{composable pairs} in the sense described above,
\item a natural associativity isomorphism
\[\alpha\colon (\_\boxempty\_)\circ(\_\boxempty\_\times id_{\C_1})
\Rightarrow (\_\boxempty\_)\circ(id_{\C_1}\times\_\boxempty\_).\]
\end{itemize}These satisfy:\begin{itemize}
\item The identity 1-cells are strict units, i.e., we have strict equalities
of functors ($\C_0\times\C_1\rightarrow \C_1$ or $\C_1\times\C_0
\rightarrow \C_1$ respectively):\[(\_\cp\_)\circ (\id_{\_}\times 
id_{\C_1}) =(\_\cp\_)\circ (id_{\C_1}\times \id_{\_})=pr_{\C_1},\]
where $pr_{\C_1}$ denotes the respective projection onto the $\C_1$-factor.
\item The transformation $\alpha$ is the identity, if any factor is
$\id_\_$. \item The transformation $\alpha$ satisfies Mac Lane's 
pentagon identity, i.e., we have a unique associator on fourfold 
$\cp$-composites, hence by induction on composites of arbitrary 
length.\end{itemize}}}

{\rem{Most of the definition is standard apart from the fact that for
this thesis I can get away with strict identities, so I took them as 
part of the definition.

For clarity: I stick to the convention that the associator always
transforms expressions with left-biased bracketing $(ab)c$ into
expressions with right-biased bracketing $a(bc)$.

Further notation: I will refer to morphisms in (the disjoint union of)
the categories $\C_1$ as 2-cells, and refer to them globally as $\C_2$.
Furthermore I denote the component of $\C_1$, which is $(s,t)$-over 
objects $a,b\in\C_0$ as $\C_1(a,b)=:\C(a,b),$ i.e., the full subcategories 
of $\C_1$ with objects: \[\C(a,b)_0 = \{f\in\C_1| s(f)=a \mathrm{~and~} 
t(f)=b\}.\] Moreover I will always refer to the objects of the bicategory
as objects, the objects of the morphism categories as 1-cells, thus in particular
I refrain from calling 1-cells ``objects'' of their respective morphism categories.

I only stick to the notation $\cp$ here to stress the 
difference between the composition functor and composition in $\C_1$. 
In the example of interest $\M$ the 1-cells are given by matrices, and
hence the composition functor $\cp$ is matrix multiplication, which I
denote by $\cdot$ or drop from notation altogether, while the 
composition of 2-cells is just composition in a product $1$-category,
so I refer to that as $\circ$.
Also just for this section I denote identity $1$-cells by $\id_a$ to 
stress that they are usually in fact natural basepoints for the
categories $\C(a,a)$, but usually are not defined as maps of the 
type $x\mapsto x$. Again in $\M$ these are the identity matrices, but
the way these matrices represent ``linear maps of modules'' is at the
very least not obvious. After this section I trust the context is
sufficient to infer which type of identities I refer to.}}

{\ex{A category enriched in categories, i.e., a 2-category, is a 
bicategory with $\alpha=id$. In particular, every 1-category is a
bicategory with discrete morphism categories and hence trivial
associator as well.

\label{sigmasymmmon}
Any monoidal category $(\C,\otimes,1)$ can be 
understood as a one-point bicategory $\Sigma\C$: set 
$\Sigma\C_0=\{*\}$ and $\Sigma\C_1
=\Sigma\C(*,*)=\C$. Composition is given by $\otimes$,
the associator hence by the associator for $\otimes$.

Conversely the endomorphism category of any object $a\in\A_0$ in a
bicategory $\A$ yields the monoidal category $\A(a,a)$, occasionally
denoted by $\Omega_a\A$. In particular we have the trivial equality
\[\Omega_*\Sigma(\C,\otimes)=(\C,\otimes),\]
and for each object $a\in\A_0$ the strict inclusion functor
\[\eta_a\colon \Sigma\Omega_a\A=\Sigma\A(a,a)\rightarrow \A\]
with $\eta_a(*)=a.$}}

Given two bicategories there are adequate 1-cells between them, but
the designations in the literature vary quite a bit - specifically 
compare the classical ``Introduction to Bicategories'' of B\'enabou 
\cite{Bena} with the more recent overview in \cite{CCG2010}. On
pages 9 and 10 of \cite{CCG2010} the authors provide an excellent 
dictionary of the common terms for morphisms. A more detailed
exposition can be found in Ross Street's ``Categorical Structures'' 
\cite{Str}, in particular section 9. Furthermore its references are
a nice guide to the literature up to 1993.

I fix my notation here, and only define the types of functors
that appear in this thesis.

{\defn{\label{pseudo} For two bicategories $\C,\D$ a 
\textbf{pseudofunctor} $F\colon \C\rightarrow \D$ consists of the 
following maps: \begin{itemize} \item A map on objects $F_0\colon\C_0
\rightarrow\D_0.$\item For each pair of objects $a,b\in\C_0$ a functor
\[F_1\colon \C(a,b)\rightarrow\C(F_0a,F_0b),\] that is pointed at 
identities, i.e., for every $a\in\C_0$: \[F_1(\id_a)=\id_{F_0a}.\] 
\item For each triple of objects $a,b,c\in\C_0$ a natural isomorphism, 
which I refer to as \emph{compositor 2-cell}, \[F_2\colon (\_ 
\cp_{\D} \_)(F_1\times F_1)\Rightarrow F_1\times (\_ \cp_{\C} \_),\] 
which is trivial at identities, i.e., $F_2=id_{pr_{\D_1}}$ at: 
\[pr_{\D_1}= (\_ \cp_{\D}\_)\circ
(F_1\times F_1)\circ(id_{\C_1}\times\id_\_)\Rightarrow F_1\circ(
\_\cp_{\C}\_)\circ(id_{\C_1}\times\id_\_)=pr_{\D_1},\] and similarly
for the other argument.\end{itemize}The compositor satisfies 
associativity, i.e., for every composable triple $f,g,h$ of $1$-cells 
in $\C_1$ the diagram \[\xymatrix{(F_1f\cp_{\D} F_1g)\cp_{\D} F_1h 
\ar[d]_{\alpha_\D} \ar[rr]^{F_2\cp_{\D}id_{\D_1}} && F_1(f\cp_{\C}g)
\cp_{\D}F_1h\ar[r]^{F_2} &F_1((f\cp_{\C}g)\cp_{\C}h) 
\ar[d]^{F_1(\alpha_\C)}\\F_1f\cp_{\D}(F_1g\cp_{\D}F_1h) 
\ar[rr]_{id_{\D_1}\cp_{\D}F_2}&&F_1f\cp_{\D}(F_1(g\cp_{\C}h)) 
\ar[r]_{F_2}&F_1(f\cp_{\C}(g\cp_{\C}h))}\]commutes.

A pseudofunctor is called strict, if furthermore $F_2=id$ as
morphisms in $\D_1$, i.e., $F_1(fg)=F_1fF_1g$ and $F_1\alpha=\alpha$ 
and for every pair of composable 1-cells $f,g\in\C_1$ we have \[F_2=
id_{F_1(fg)}\colon F_1f\cp_\D F_1g \Rightarrow F_1(f\cp_\C g).\]}}

{\rem{Occassionally -- cf. \cite[pp. 9--10]{CCG2010} --
one refers to pseudofunctors as \emph{strong normal} functors, where strong
refers to the fact that the involved $2$-cell is an isomorphism and
normality refers to strictly fixing identity $1$-cells, which I carry
through this thesis as a permanently standing assumption.

For emphasis I occassionally refer to functors, which strictly respect
identity $1$-cells and composition of $1$-cells, as \emph{strict normal}
functors. \label{strictn}}}

{\rem{I only study bicategories arising from 
``finitely generated free module''-constructions, and thus by 
standard assumptions for $K$-theory only consider isomorphism 
subcategories of permutative $1$-categories. For a clarifying mathematical
reason to restrict to isomorphisms see in particular 
\cite[Proposition 8.14]{GGN}. Since I can restrict to exclusively
isomorphism $2$-cells, I have no need for less rigid functors between 
bicategories.}}

Pseudofunctors satisfy a preservation property on $1$-cells, 
which (op)lax functors of bicategories do not satisfy in general.
{\prop{A pseudofunctor $F\colon \C\rightarrow\D$
sends equivalence $1$-cells of $\C$ to equivalences in $\D$. In 
particular, isomorphism $1$-cells in $\C$ are sent to equivalences
in $\D$.\begin{proof} Let $f\colon a\rightarrow b$ be an equivalence
in $\C$, i.e., there is a $1$-cell $g\colon b\rightarrow a$ and 
isomorphism $2$-cells $\varepsilon\colon gf\Rightarrow \id_a,$
$\eta\colon \id_b \Rightarrow fg$. Then $Ff$ is an equivalence
inverted by $Fg$ by the following diagrams:
\[\begin{array}{lcr}{\xymatrix{\rrlowertwocell<-7>_{\id}{<-3.5>~F_2}
\rruppertwocell<1>^{F(gf)}{<1.5>~~F(\varepsilon)}
\rrcompositemap<7>_{Fg}^{Ff}{\omit}&&}}
&\mathrm{~and~}&{{\xymatrix{\rruppertwocell<9>^{\id}{<2>~~~F_2^{-1}}
\rrlowertwocell<2>_{<-0.25>F(fg)}{<-2.5>~~F(\eta)}
\rrcompositemap<-7>_{Ff}^{Fg}{\omit}&&.}}}\\\end{array}\]
Hence we need the compositor two-cell and its inverse, and find that
$Ff$ is an equivalence. In particular, if $F$ is not a strict functor,
$F$ sends isomorphisms to equivalences with $F_2$ as a non-trivial
isomorphism to the identity.

(Furthermore note that I have implicitly used the assumption that
$F$ is normal in the diagrams by $\id=F\id$.)\end{proof}}}

{\defn{Let $F,G$ be two pseudofunctors of bicategories:
\[(F_0,F_1,F_2),(G_0,G_1,G_2)\colon \C\rightarrow\D.\] Then a 
(strong) \textbf{pseudonatural transformation} $\sigma\colon F\Rightarrow G$ 
consists of chosen $1$-cells \[\sigma^0\colon \C_0\rightarrow \D_1\]
with $s(\sigma^0(a))=F_0a$ and $t(\sigma^0(a))=G_0a,$ as well as 
coherence isomorphism $2$-cells chosen for every pair $a,b\in\C_0$ 
and every 1-cell $f\in\C(a,b)$: \[\xymatrix{F_0a
\drtwocell<\omit>{~~\sigma^1_f} \ar[r]^{F_1f} \ar[d]_{\sigma^0a}& 
F_0b\ar[d]^{\sigma^0b}\\ G_0a \ar[r]_{G_1f}& G_0b,}\] which are 
appropriately natural and are compatible with the compositors, i.e.
we have for all objects $a,b,c\in\C_0$ and all $1$-cells $f,g\in\C_1$: 
\[\begin{array}{lcccr}{\xymatrix{F_0a\ar[d]_{\sigma_0}
\rruppertwocell<10>^{F_1(gf)}{<-3>\;\;\;\;F_2^{-1}} \ar[r]|{F_1f} 
\drtwocell<\omit>{~~\sigma^1_f} &F_0b\ar[d]|{\sigma_0}\ar[r]|{F_1g} 
\drtwocell<\omit>{~~\sigma^1_g}& F_0c\ar[d]^{\sigma_0} \\ G_0a
\rrlowertwocell<-10>_{G_1(gf)}{<3>\;\;\;G_2}\ar[r]|{G_1f} &G_0b
\ar[r]|{G_1g}&G_0c}} && = && {\xymatrix{F_0a \ar[d]_{\sigma_0}
\ar[r]^{F_1(gf)} \drtwocell<\omit>{~~\sigma^1_{gf}} & F_0c
\ar[d]^{\sigma_0}\\ G_0a \ar[r]_{G_1(gf)} & G_0c.}}\end{array}\]
Furthermore for bicategories with strict units we want 
$\sigma^1_{\id_a}=id_{\sigma^0_a}$ for each object $a\in\C_0$.
If in addition the two-cells $\sigma^1$ are identities we call
$\sigma$ a strict natural transformation.}}

{\rem{Since I restrict attention to bicategories with only isomorphism
two-cells or at least functors and transformations with isomorphism
two-cells, I do not need to introduce the concepts of lax and oplax
natural transformations.}}

It is classical for $1$-categories that a natural transformation
$\eta\colon F\Rightarrow G$ is the same thing as a functor 
$\C\times I\rightarrow \D$ for $I = [0<1]$ the interval
category. The analogous statement for bicategories holds true as well, which
I want to isolate into a proposition for emphasis and reference.

{\prop{A pseudonatural transformation $\eta$ of functors $F,G\colon 
	\C\rightarrow\D$ consists of the same data as a pseudofunctor
	$\eta\colon \C\times I\rightarrow \D,$ while the coherence
	of $2$-cells is equivalent to the pseudonaturality of the
	transformation.}}

Given sufficient experience with $1$-categories one would expect that 
I introduced all types of morphisms, but it should not be surprising 
that the extra-level of morphisms in bicategories (i.e., two-cells)
introduces a higher type of morphisms between ``natural 
transformations'', called modifications.

{\defn{\label{modis}
Given two pseudonatural transformations $\sigma,\tau$ between
the same two pseudofunctors $F,G$, a \textbf{modification} $\xi\colon \sigma
\Rightarrow \tau$ consists of a choice of isomorphism two-cells 
\[\xi\colon \C_0\rightarrow \D_2,\] making the following two 
diagrams of two-cells equal:
\[\xymatrix{~~~\ar[rr]^{Ff}\ddtwocell_{<2>{\tau^0a}}^{<2>\sigma^0a}<6>{\xi_a} 
\ddrrtwocell<\omit>{\sigma^1}& &\ar[dd]^<<<<{\sigma^0b}&\ar[rr]^{Ff}
\ddrrtwocell<\omit>{\tau^1~~~~~~~}\ar[dd]_>>>>>>{\tau^0a}&&~~~ 
\ddtwocell^{<2>{\sigma^0b}}_{<2>{\tau^0b}}<6>{_\xi_b}\\&&~~~~\ar@{{}={}}[r]&~~~~&\\
~~~\ar[rr]_{Gf}&&&\ar[rr]_{Gf}&&~~~.}\]}}

{\rem{Do note that on the level of $2$-cells the diagrams above are
only commutative squares resembling naturality in the context of 
$1$-categories.}}

With the relevant morphisms in place I can introduce 
equivalences of bicategories. {\defn{A pseudofunctor of 
bicategories $F\colon \C\rightarrow \D$ is an \textbf{equivalence of 
bicategories}, if there is a pseudofunctor $G\colon\D\rightarrow\C$
and there are two pseudonatural equivalences $\eta\colon FG\Rightarrow 
\id_\D$ and $\zeta\colon GF\Rightarrow \id_\C$.}}

The following proposition is particularly useful in chapter \ref{multbidel}, 
nonetheless its appropriate context is abstract nonsense about 
bicategories, so this section. I repeat the proof in particular to convince
the reader that it holds for bicategories with enriched morphism categories.
{\prop{\label{meinCzuZF}A pseudonatural transformation of pseudofunctors of 
small bicategories is an equivalence if and only if all its 
$1$-cells are equivalences.\begin{proof} It is clear that an inverse
equivalence establishes each component $1$-cell as an equivalence in
the target category. So we have to establish that having all $1$-cells 
equivalences is sufficient.

Let $\eta\colon F\Rightarrow G$ be a pseudonatural transformation
comprised of $1$-cells $\eta^1\colon \C_0\rightarrow \D_1$, and for 
each pair of objects $a,b$ in $\C$ a natural transformation:
\[\xymatrix{\C(a,b)\rrtwocell<5>^{{\eta^1_b}_*\circ F_1}_{{\eta^1_a}^*
\circ G_1}{~~~\eta^2_{a,b}}&& \D(Fa,Gb).}\]

By assumption each $1$-cell $\eta^1_a$ is an equivalence, so by the
axiom of choice choose for each $a$ an inverse equivalence $\zeta^1_a$
and isomorphism two-cells $\sigma\colon \zeta^1\eta^1 \rightarrow \id$
and $\tau\colon \id \rightarrow \eta^1\zeta^1$.

With these choices in place we can make $\zeta$ into a pseudonatural
transformation by choosing its two-cells as indicated by the 
following diagram:\[\xymatrix{\zeta GA= \zeta GA \id \ar[r]^{\tau} &
(\zeta GA)\cdot(\eta\zeta)\ar[d]^\alpha& (\zeta\eta)(FA\zeta) 
\ar[r]^\sigma & \id FA\zeta=FA\zeta.\\&\zeta ((GA\eta)\zeta) 
\ar[r]^{\eta^2} &\zeta ((\eta FA)\zeta) \ar[u]^\alpha & }\]

Since $\tau$ and $\sigma$ are chosen objectwise, we get a natural 
transformation. Since each arrow is an isomorphism $2$-cell, the 
$1$-cells $\zeta$ along with the two-cells indicated above compromise 
a pseudonatural transformation. It is compatible with the 
compositors of $G$ and $F$ because $\eta$ is, and hence $\zeta$ is an 
inverse equivalence to $\eta$. The relevant modifications are by 
construction given by $\tau$ and $\sigma$.\end{proof}}}

{\rem{For this proposition bicategories
are much nicer than the stricter theory of $2$-categories. 
Even if the pseudo-natural transformation strictly satisfies 
naturality, its inverse equivalence might have non-trivial $2$-cells.}}

{\rem{Do note that despite the fact this proposition is the analogue
to the $1$-category statement that a natural transformation is a 
natural isomorphism if and only if each component is an isomorphism,
its truth is (ZF-)axiomatically equivalent to the statement that a 
functor is an equivalence of categories if and only if it is 
essentially surjective and fully faithful. Hence it is
stronger because of the missing uniqueness for the inverse 
$1$-cells.}}

The following proposition is an immediate generalisation from
the context of $1$-categories. I want to exhibit
that the proof works in the context I define above. 
Thus it reassures us that the definitions are consistently chosen.

{\prop{A pseudofunctor $F=(F_0,F_1,F_2)$ of (small) bicategories is an 
equivalence of bicategories if and only if $F_0$ is surjective up to 
equivalence and each functor $F_1$ is an equivalence of 
$1$-categories.\begin{proof}Given a pseudofunctor $F\colon\C
\rightarrow\D$ that satisfies the conditions, we can by the axiom of 
choice find a map $G_0\colon Ob\D=\D_0\rightarrow \C_0$ such that 
there is an equivalence $\eta_d\colon F(G_0(d))\rightarrow d$ for 
each $d\in\D_0$. Fix that equivalence and an inverse $\kappa^d$ 
together with the isomorphisms $\eta_d\kappa^d\cong \id$ and $\kappa^d\eta_d
\cong \id$ for each $d\in\D_0$, it is the system of $1$-cells for the 
natural equivalence $FG \simeq \id_\D$ we need.

By assumption we have for each pair $d_1,d_2$ an equivalence of 
categories \[F_1\colon \C(G_0d_1,G_0d_2)\rightarrow \D(FG_0d_1,
FG_0d_2).\] Fix an inverse equivalence for each such pair $G^{d_1,d_2}$,
then we make $G$ into a functor of bicategories by the following
assignment on morphism categories:
\[\xymatrix{ \D(d_1,d_2) \ar[r]^{\eta_{d_1}^*} 
& \D(FG_0d_1,d_2)\ar[r]^{\kappa^{d_2}_*} 
& \D(FG_0d_1,FG_0d_2)\ar[r]^{G^{d_1,d_2}} & \C(G_0d_1,G_0d_2).}\]
Without loss of generality make $G$ into a functor pointed
at the identity $1$-cells.

This is an inverse equivalence to $F$ by construction; it is a 
pseudofunctor with compositor $2$-cell given by the isomorphisms 
$\eta\kappa\cong\id$ chosen above and with the natural equivalence
on one side given by $\eta$ with inverse $\kappa$ and on the other
by $G\eta$ and $G\kappa$.\end{proof}}}

It is reassuring to know that bicategories can still be strictified
to 2-categories. (This is wrong for tricategories!)

{\lem{(cf. \cite{Lei}) Each bicategory is equivalent to a 2-category.}}

{\rem{\label{nota}From here I drop the properly emphasised but 
clumsy notation, and denote $\cp$ in a bicategory by $\cdot$ or do not
denote it at all, while composition of $2$-cells is denoted by 
$\circ$, as it is usually the composition in some product of $1$-categories
in my examples.

For functors I refer to $(F_0,F_1)$ generically as $F$ and I denote 
the compositor in uppercase greek letters $\Phi$, thus referring to a
pseudofunctor $(F,\Phi)$ for instance.

I stick to the following conventions for elements in 
a general bicategory: objects are denoted by lowercase 
latin letters $a,b,c,\ldots\in\C_0$, which in $\M$ are natural 
numbers, but I do not want to restrict to that case unnecessarily. 
In $\M$ the $1$-cells are matrices, hence I denote $1$-cells by 
uppercase latin letters $A,B,C,\ldots\in\C_1$, and finally $2$-cells
by lowercase greek letters $\varphi,\psi,\ldots$ which are morphisms 
in products of the coefficient category $\R$ for $\M$.}}

{\rem{I refer to bicategories $\C$ as \emph{enriched in} topological 
spaces, simplicial sets, categories,\ldots if the morphism categories
$\C_1$ are enriched in these monoidal categories.}}

{\ex{A category enriched in topologically or simplicially enriched
categories is a bicategory enriched in topological spaces or simplicial 
sets respectively. A topological/simplicial monoidal category gives 
rise to a one-point bicategory enriched in topological 
spaces/simplicial sets. Do note that the propositions before 
work in the enriched cases as well, i.e. for enriched bicategories, and
enriched pseudofunctors, since the axiom of choice was only 
involved objectwise.}}

The rest of this chapter -- apart from the section \ref{kumodels} -
can safely be skipped by the reader familiar
with the delooping in \cite{EM}. For ease of reference I rewrite
their delooping in the following sections, so that the delooping in
bicategories \ref{multbidel} can be read in close analogy with the
case in 1-categories.

\subsection{A Delooping of Permutative Categories} 
\label{pcatneu}  Permutative categories 
provide a classical useful tool to model connective 
spectra, hence are valuable in stable homotopy theory. Even 
more than that Thomason proved \cite{Th1} that ``Symmetric monoidal
categories model all connective spectra''. Thomason was also driven
by the desire to provide a nice model for a smash product of spectra:
``[I]n June 1993 [\ldots] I used this alternate model of stable 
homotopy to give the first known construction of a smash product 
which is associative and commutative up to coherent natural 
isomorphism in the model category.''
Since \cite{MMSS} showed ``all'' models for symmetric monoidal 
categories of spectra yield (Quillen-)equivalent results, Thomason's
construction of a smash product has lost attention.

I elaborate on the construction $\C^+$ in \cite{Th2} in excessive detail, 
so I can refer back to its details for the analogous construction in 
bicategories \ref{multbidel}. Warning on notation: The notational 
conventions for bicategories described in \ref{nota} do 
not apply here, because they would clash with the natural interpretations.

The following results are each found in section 4 of \cite{Th2}. In particular,
I repeat his results and definitions in order to fix the notations I mimic
for bicategories in chapter \ref{multbidel}.

{\defn{\label{fastocat}Let $(\C,+,0,c_+)$ be a permutative category and 
$f\colon n_+\rightarrow m_+$ a map of pointed sets $n_+=\{0,1,\ldots,n\}$.
Define the following functor: \[\begin{aligned} f_*\colon& 
  \C^{\times n} &\rightarrow& \C^{\times m}\\
&(c_1,\ldots,c_n) &\mapsto& (\sum_{i\in f^{-1}j} c_i)_{j=1,\ldots,m},
\end{aligned}\] where the empty sum is defined to be the zero object (and 
its identity). Note that we have to
use the induced ordering $f^{-1}j\subset (n,\leq)$ and sum the $c_i$ 
accordingly.}}

{\rem{Note in particular that this gives a left-action of the 
symmetric groups on the respective powers: 
$\sigma_*(c_1,\ldots,c_n)_j = \sum_{i\in \sigma^{-1}j}c_i 
= c_{\sigma^{-1}j}.$}}

The fact that we have to choose an ordering on the fibres of $f$ is 
precisely what breaks the strictness of the functor $(\cdot)_*\Fi\rightarrow 
Cat$, which on morphisms is the assignment $f\mapsto f_*$ according to
the above definition. I isolate this fact into the following lemma.

{\lem{\label{symmundstrong}Given pointed maps $f\colon n_+\rightarrow m_+$ and $g
\colon m_+\rightarrow l_+$ there is a natural isomorphism of functors 
$\varphi_{g,f}\colon(gf)_*\Rightarrow g_*f_*$. \begin{proof} We can 
consider this componentwise, so without loss of generality let $g
\colon m_+\rightarrow 1_+ = \{0,1\}$ the unique map with $g^{-1}0=
\{0\}$. Then the summation according to $g_*f_*$ looks as follows:
\[(g_*f_*)(c)=\sum_{i=1}^m (f_*c)_i = 
\sum_{i=1}^m\sum_{j\in f^{-1}i}c_j,\]whereas the summation of $(gf)_*$
is according to the linear order on $n$ given by \[(gf)_*(c)=
\sum_{i\in (gf)^{-1}1}c_i = \sum_{i=1}^n c_i.\] Then there is a unique
 additive symmetry giving the isomorphism: \[\xymatrix@1{ ((gf)_*c)_j
=\sum_{k\in (gf)^{-1}j}c_k\ar[r]^-{c^+_{g,f}} 
&\sum_{i\in g^{-1}j}\sum_{k\in f^{-1}i}c_k=\sum_{i\in g^{-1}j}(f_*c)_i 
= g_*(f_*(c)),}\] which yields a natural isomorphism of functors: 
\[(gf)_*\Rightarrow g_*f_*.\]\end{proof}}}

{\rem{\label{troublemaker} Take special note of the following 
composites, which reappear in the delooping constructions and the 
distributivity axioms of bimonoidal categories. Define \[f\colon 
4_+\rightarrow 2_+: f(1)=f(3)=1, f(2)=f(4)=2\] and \[g\colon 4_+
\rightarrow 2_+: g(1)=g(2)=1, g(3)=g(4)=2\]. We have the unique map 
$q\colon 2_+\rightarrow 1_+$ with $q^{-1}0=\{0\}$ for $n_+=\{0,1,\ldots,n\}$ pointed at $0$.
We also have a unique map $c\colon
4_+\rightarrow 1_+$ with $c^{-1}0=\{0\}$. Then we have $qf=qg=c$, and
hence for a permutative category a unique isomorphism:
\[q_*g_*=c_*=(qf)_*\Rightarrow q_*f_*, \] which is given by the 
symmetry: \[1+c^++1\colon a+b+c+d \rightarrow a+c+b+d.\] In more 
detail: The compositor for $q$ and $g$ is the identity: 
$\varphi_{q,g}=id$, and for $q$ and $f$ is the symmetry 
$\varphi_{q,f}=1+c^++1$.

More generally: Since the action of the symmetric groups on the
respective powers of $\C$ is strict, we get $\varphi_{\sigma_1,
\sigma_2}=id$, for each $n\in \N$ and $\sigma_1,\sigma_2\in \Sigma_n$.
Consider the composite of a permutation $\sigma\in\Sigma_n$ with the 
unique map $q\colon n_+\rightarrow 1_+$, with $q^{-1}0=\{0\}$. Then 
we get: $q_*(c_1,\ldots,c_n)=\sum_{i=1}^nc_i,$ and
$q_*\sigma_*(c_1,\ldots,c_n)=\sum_{i=1}^nc_{\sigma^{-1}i},$ hence
$\varphi_{q,\sigma}=c^+_{\sigma}$, for $c^+_\sigma$ the unique natural 
additive symmetry in $\C$ between these sums.}}

We can define a ``classifying'' pseudofunctor into the $2$-category 
$Cat$ for a permutative category.


{\prop{\label{BCfunctor} Given a permutative category $(\C,+,0,c_+)$, the assignment
\[\begin{array}{rcl}B_{\C}\colon Fin_+ &\rightarrow& Cat \\  n_+ 
&\mapsto &\C^{\times n}, \\  f\colon n_+\rightarrow m_+ &\mapsto 
&f_*\colon \C^{\times n}\rightarrow \C^{\times m} \end{array}\] 
defines a pseudofunctor of $2$-categories, where we consider 
$Fin_+$ as a $2$-category with discrete morphism categories.
\label{THEpseudofunctor} 
\begin{proof} We have to prove the coherence: \[\xymatrix{ (hgf)_* 
\ar[r]^{\varphi_{hg,f}} \ar[d]^{\varphi_{h,gf}} & (hg)_*f_* 
\ar[d]^{\varphi_{h,g}id}\\ h_*(gf)_* \ar[r]^{id\varphi_{g,f}} 
& h_*g_*f_*,}\] but we know by \cite{Lap} that the additive symmetry 
defining the transformation \[(hgf)_*\Rightarrow h_*g_*f_*\] is 
uniquely determined by the ordering of summands the composite 
$h_*g_*f_*$ induces, so the diagram commutes. Furthermore we 
obviously have $id_*=id$, so $B_\C$ is a normal functor.\end{proof}}}

{\rem{Observe that giving a covariant pseudofunctor
$\mathrm{Epi}\rightarrow Cat$, which on objects is the assignment $n\mapsto 
\C^n$, already defines a symmetric monoidal product on $\C$, which is 
strictly associative and includes a symmetry but does not have a unit or 
unitors. In order to take care of the zero object we need a strong
normal functor $\mathrm{Fin}\rightarrow Cat$. Choosing pointed sets as an index
category induces projections in the additive Grothendieck construction below
associated to the maps: \[\rho^i\colon n_+\rightarrow 1_+\] with
\[\rho^i(j)= \begin{cases}0 & j\neq i\\ 1 & j=i,\end{cases}\]
which is easily identified as: $(\rho^i)_*=pr_i\colon \C^{\times n} 
\rightarrow \C.$}}

This association of a monoidal category to a functor is sufficiently
natural for the following lemma to hold:
{\lem{A (pointed) functor $F\colon (\C,+)\rightarrow (\D,+)$ together 
with a natural transformation $\lambda\colon F(\_)+F(\_)\Rightarrow 
F(\_+\_)$ is strongly symmetrically monoidal if and only if the
induced map of pseudofunctors $B_\C\Rightarrow B_\D$ is a 
pseudonatural transformation. For $\C$ and $\D$ permutative
$F$ is a strictly additive functor if and only if the induced
map is a strict natural transformation $B_\C\Rightarrow B_\D$.

Additionally a natural transformation of symmetric monoidal functors
is symmetrically monoidal if and only if it induces a modification
of the respective induced transformations.
\begin{proof}
That a strong symmetric functor defines a pseudonatural transformation
is given in \cite[Paragraph 4.1.4]{Th2}. The converse follows from the observation
that we can reconstruct $F$ by restriction to
$U$-level $1$: $F=B_F|_{\{1_+\}}\colon \C\cong B_\C(1_+)\rightarrow 
B_\D(1_+)\cong \D.$ 
The pseudonaturality $2$-cell of $B_F$ for the unique map 
$q\colon 2_+\rightarrow 1_+$ with $q^{-1}(0)=0$, gives the
diagram:
\[\xymatrix{B_\C(2_+)\ar[r]^{B^1_F}\ar[d]_q
	\drtwocell<\omit>{B^2_F}& B_\D(2_+)\ar[d]^q\\
	B_\C(1_+)\ar[r]^{B^1_F}& B_\D(1_+),}\] 
which identifies $\lambda$ as the pseudonaturality $2$-cell $B^2_F$.
It is coherently associative and symmetric because of the appropriate
pseudonaturality diagrams in higher degrees.

Comparing the diagram above with the diagram in definition \ref{modis}
yields the properly analogous claim for monoidal natural transformations. 
\end{proof}}}

I do not give the full generality of Grothendieck constructions, but 
only define the resulting category of the Grothendieck construction 
on $B_\C$ with respect to the permutative category $(\C,+,0,c_+)$. It
destills the complexity of the functor into an ordinary 1-category, 
so I do not refer to bicategories again until \ref{modulbicat}. For
a compatible general exposition of the Grothendieck construction as
considered by Grothendieck compare pages 47--49 in \cite{Bena}.

{\defn{\label{c+}Define the category $\C^+$ as follows: Its objects 
are \[Ob\C^+ := \coprod_n \C^{\times n},\] its morphisms:
\[\C^+((c_1,\ldots,c_n),(d_1,\ldots,d_m)) =
\coprod_{f\in \Fi(n_+,m_+)}\C^m(f_*c,d).\]
The identities are given by the identities in $\Fi$ and $\C^m$, the 
composition is given as follows: $c=(c_1,\ldots,c_n),
d=(d_1,\ldots,d_m),e=(e_1,\ldots,e_l)$: \[\xymatrix{
\C^+(d,e)\times \C^+(c,d) \ar@{=}[d]\\
\coprod_{g,f}\C^l(g_*d,e)\times \C^m(f_*c,d) 
	\ar[d]^{\coprod id\times g_*}\\
\coprod_{g,f}\C^l(g_*d,e)\times \C^l(g_*f_*c,g_*d)
	\ar[d]^{comp_{\C^l}}\\
\coprod\C^l(g_*f_*c,e) \ar[d]^{\varphi_{g,f}^*}\\
\coprod\C^l((gf)_*c,e) \subset \C^+(c,e).}\]
It is associative precisely because $\varphi$ is, and if 
$(\C,+,0,c_+)$ carries an enrichment such that $+$ is an enriched 
functor, then $\C^+$ is enriched over the same category.

Call this the \textbf{additive Grothendieck construction} on a
permutative category $(\C,+)$.}}

{\rem{The construction $\C^+$ is already
given by Thomason in \cite[Definition 2.1.2]{Th2}. Let me summarise the 
idea of the construction. Given a monoidal product on a category: 
$\otimes\colon \C\times\C\rightarrow \C$, we can define for each map 
$f\colon n_+\rightarrow m_+$ of finite sets a functor $\C^{\times n}
\rightarrow \C^{\times m}$ (in the same direction). This assembles 
into a pseudofunctor $\Fi\rightarrow Cat$. The Grothendieck 
construction associated to it is $\C^+$. I find it useful to
describe the delooping constructions of \cite{EM,Os,Seg} in this
one syntax.

Note that by the description of the composition given above we have
an enriched additive Grothendieck construction for an enriched permutative
category - specifically if the monoidal functor $+\colon \C\times\C\rightarrow\C$
is enriched, then the enrichment carries over to $\C^+$.}}

{\ex{Consider an ordinary category, that is one enriched in sets, 
then we can write morphisms between tuples as pairs \[(f, (a_1,\ldots,
a_m))\colon (c_1,\ldots,c_n)\rightarrow (d_1,\ldots,d_m),\] where 
$f\colon n_+\rightarrow m_+$ and $a_i \colon \sum_{j\in f^{-1}i}c_j
\rightarrow d_i,$ where we understand the empty sum as the zero object.

Then composition looks as follows (with $f\colon n\rightarrow m, 
g\colon m\rightarrow l$):\[(g,(b_1,\ldots,b_l))\circ(f,(a_1,
\ldots,a_m))=(gf,(b_1,\ldots,b_l)\circ g_*(a_1,\ldots,a_m)\circ
\varphi_{g,f}).\]

I consider the two extreme cases: The composites of symmetries are 
strict. So we have trivial compositors here, and thus (with $\bar a^i
= (a^i_1,\ldots,a^i_n)$ ~$n$-tuples of morphisms in $\C$):
\[(\sigma,\bar a^3)\circ (\tau,\bar a^2)\circ (\rho,\bar a^1)
= (\sigma\tau\rho, \bar a^3\circ \sigma.\bar a^2\circ 
		(\sigma\tau).\bar a^1).\]
The other extreme case is given 
for the composite of any symmetry $\sigma\in \Sigma_n$ with a map 
$q\colon n_+\rightarrow 1_+$ with $q^{-1}0=\{0\}$, there is a unique
additive symmetry $c^+_\sigma$ in $\C$: \[c^+_\sigma\colon 
q_*(c_1,\ldots,c_n)=\sum^n_{i=1}c_i\rightarrow 
q_*(\sigma.(c_1,\ldots,c_n))=\sum^n_{i=1}c_{\sigma^{-1}i}.\]
Thus we get: \[\begin{array}{ll} (q,a)\circ(\sigma,(b_1,\ldots,b_n)) 
      &= (q\sigma, a\circ q_*(b_1,\ldots,b_n)
				\circ \varphi_{\sigma,q})\\
      &= (q, a\circ (\sum_i b_i) \circ c^+_{\sigma})\\
\multicolumn{2}{c}{\text{and by the naturality of the additive twist we can identify this with:}}\\ 
      &= (q, a\circ c^+_{\sigma}\circ (\sum_ib_{\sigma^{-1}i}))\\
      &= (q, a\circ c^+_\sigma \circ q_*(\sigma.(b_1,\ldots,b_n)))\\
      &= (q, a\circ c^+_\sigma) \circ (id, \sigma.(b_1,\ldots,b_n)).
\end{array}\] In particular this Grothendieck construction 
incorporates structural morphisms for permutations between $n$-tuples 
that project down to the ordinary additive symmetry in $\C$.}}

{\rem{\label{disccomp}Note that in particular the category $\C^+$ has structural 
morphisms $(f,id)\colon c\rightarrow f_*c$, for $c=(c_1,\ldots,c_n)$ 
and $f\colon n_+\rightarrow m_+$. We call these morphisms the 
\emph{discrete component} of morphisms $c\rightarrow d$. This 
is a forgetful functor $U\colon\C^+\rightarrow\Fi$, 
which is important for the delooping construction.}}

{\ex{Building on the previous example let us reconsider the maps from 
\ref{troublemaker}. We set\[f\colon 4_+\rightarrow2_+: f(1)=f(3)=1,f(2)
=f(4)=2,\]\[g\colon4_+\rightarrow2_+: g(1)=g(2)=1,g(3)=g(4)=2,\]\[q
\colon2_+\rightarrow1_+: q(2)=q(1)=1,\]and write $c\colon4_+\rightarrow
1_+$ for the unique map with $c^{-1}0=0$. Then the compositor 
$\varphi_{q,f}$ is given by $1+c_++1$, whereas the compositor 
$\varphi_{q,g}$ is the identity. So we find:\[\begin{aligned}(q,a)
\circ(g,(b_1,b_2))&=(qg,a\circ q_*(b_1,b_2))\\&=(c,a\circ(b_1+b_2))\\
(q,a)\circ(f,(b_1,b_2))&=(qf,a\circ(b_1+b_2))\\&=(c,a\circ(b_1+b_2)
\circ(1+c_++1)).\label{fuerdiehohenReds}\end{aligned}\]

Hence the following diagram represents a commutative square in $\C^+$:
\[\xymatrix{(c_1,c_2,c_3,c_4) \ar[d]^{(g,\id)}\ar[rr]^{(f,\id)} 
&& (c_1+c_3,c_2+c_4) \ar[d]^{(q,\id)}\\
(c_1+c_2,c_3+c_4) \ar[r]^{(q,\id)} & (c_1+c_3+c_2+c_4)
&\ar[l]_{\varphi_{q,f}} (c_1+c_2+c_3+c_4),}\] because $\varphi$ is 
part of the composition law.}}

The additive Grothendieck construction is naturally associated 
to the pseudofunctors $B_\bullet$, thus leading to the 
following naturality:
{\lem{A strong symmetric monoidal functor $(F,\mu)\colon
(\C,+)\rightarrow (\D,+)$ induces a canonical functor $F^+$ on
additive Grothendieck constructions as follows: It assigns tuples
componentwise $F^+(c_1,\ldots,c_n)=(Fc_1,\ldots,Fc_n)$.
For morphisms consider the case with morphism sets. 
The map \[(f,(\varphi_1,\ldots,\varphi_m))\colon (c_1,\ldots,c_n)
\rightarrow (d_1,\ldots,d_m)\] is sent to:
\[\xymatrix{(Fc_1,\ldots,Fc_n)\ar[d]^{(f,id)}\\
(\sum_{j\in f^{-1}i}Fc_j)_i\ar[r]^{(id,(\mu)_i)}&
(F(\sum_{j\in f^{-1}i}c_j))_i \ar[d]^{(id,(F(\varphi_1),
\ldots,F(\varphi_m)))}\\& (Fd_1,\ldots,Fd_m). }\]
This is a functor precisely because $(F,\mu)$ is symmetrically 
monoidal.

A monoidal natural transformation $\eta\colon (F,\mu)\Rightarrow 
(G,\nu)$ induces a natural transformation $\eta^+\colon F^+\Rightarrow
G^+$. \begin{proof}I only comment on the natural transformation. Again 
consider the case with morphism sets, then we set $\eta^+$ as tuples
with the appropriate instances of $\eta$ and no discrete component.
This is natural in morphisms $(f,(\mu)_i)$ trivially, because $\eta$
is monoidal, and it is natural in morphisms $(id,(F(\varphi_1),\ldots,
F(\varphi_m)))$ because it is a product of natural transformations
$\eta\colon F\Rightarrow G$.

Do note that it is not meaningful for a natural transformation to be
symmetrically monoidal; there is no additional compatibility for 
$\eta$ to be satisfied.\end{proof}}}

Given any symmetric monoidal category $\C$ we can restrict to its
subcategory of isomorphisms $\C^{iso}$, which is a symmetric monoidal
subcategory of $\C$, so this gives an inclusion on their Grothendieck
constructions. \label{isorest}{\cor{There is a natural inclusion 
$I\colon(\C^{iso})^+\rightarrow \C^+$.}}

The following definition describes the index categories relevant for
the delooping. Specifically, we consider the comma category of finite
pointed sets under $A_+$, for $A_+$ not necessarily an object of $\Fi$.

{\defn{\label{commaindex}For an arbitrary finite pointed set $A_+$ 
define the index category $A_+\downarrow 
\Fi$ as follows: Objects are pointed maps $p\colon A_+
\rightarrow n_+$ and morphisms $f\colon p\rightarrow q$ are 
commutative triangles under $A_+$:~~~~~~ \xymatrix{A_+ \ar[r]^p \ar[dr]_q
& n_+ \ar[d]^f\\ &m_+.}}}

{\defn{Call the morphisms $\rho^i$ with $\rho^i(j)=*$ for $j\neq i$ and $\rho^i(i)=1$. 
They fit into the diagram:\[\xymatrix{ A_+ \ar[r]^f \ar[dr]_{\chi_{f^{-1}i}} 
& n_+ \ar[d]^{\rho^i}\\ & 1_+,}\] for each $f$ and each non-empty
preimage $f^{-1}i\neq \emptyset.$}}

{\rem{Note that the functors $(\rho^i)_*\colon \C^n\rightarrow \C$ are
strictly equal to the projection onto the $i$th factor.}}

{\rem{We have a target functor, i.e., $T\colon A_+\downarrow
\Fi\rightarrow \Fi$. \label{target1}

For $f\colon A_+\rightarrow B_+$ a pointed map
we have a (covariant) functor $f^*\colon \Bix\rightarrow\Aix$, which
is a functor over $T$, i.e., the diagram
\[\xymatrix{\Bix \ar[dr]_T\ar[r]^{f^*} & \Aix\ar[d]^T\\&\Fi}\]
commutes.}}

{\rem{The categories $\Aix$ have a ``full'', actually discrete, 
subcategory on objects the characteristic maps $\chi_a\colon A_+\rightarrow 1_+$
with $\chi_a(x)=+$ for $x\neq a$ and $\chi_a(a)=1$. This yields a
functor $j_A\colon A^\delta \rightarrow \Aix,$ which includes the
unpointed set $A$ as a discrete subcategory $A^\delta$ into $\Aix$,
by identifying each element $a$ with its characteristic map $\chi_a$.}}

It is not strictly necessary to define the delooping construction 
for $n=1$ (\ref{ca-n!}), but the essential mathematics happen here.
The cases for $n>1$ are then reductions to this case.

{\defn{Given a permutative category $(\C,+)$ and a finite pointed set
$A_+$ define its \textbf{first delooping category} $\C(A_+,1)$ as the 
category of functors lifting $T\colon \Aix\rightarrow \Fi$ through 
$U\colon (\C^{iso})^+\rightarrow \Fi$, i.e., the dashed arrows in the
diagram: \[\xymatrix{	& (\C^{iso})^+\ar[d]^U\\
\Aix \ar@{-->}[ur] \ar[r]_T & \Fi,}\]
where $T$ is the target functor given in \ref{target1}, and $U$ is the
functor sending each morphism to its discrete component as in \ref{disccomp}.
Its morphisms are the natural transformations of the functors pushed
forward with $I\colon (\C^{iso})^+\rightarrow \C^+,$ i.e., natural 
transformations but with arbitrary components, not just isomorphisms.}}

{\rem{The delooping of a permutative category given in 
\cite{EM} can be understood as a categorified version of the 
usual classifying space construction for abelian groups 
(cf. \cite[pp.87+88 and Theorem 23.2]{MaySoat}). 
For the functoriality of the delooping construction in finite
pointed sets and arbitrary maps it is more convenient to consider
all maps of finite pointed sets as structural morphisms. But one should
think of the structural $\Ep$-morphisms as fundamental, whereas 
non-surjective morphisms just keep track of zeroes.}}

{\rem{By contravariant functoriality of the indexing categories
over $T$ we get that $\C(\_,1)$ defines a covariant 
functor $\Fi\rightarrow Cat$. Furthermore restriction along
$j_A\colon A^\delta \rightarrow \Aix$ 
is a functor $R\colon \C(A_+,1)\rightarrow \C^{\times A}$.}}

{\prop{Every functor $F\colon \Aix\rightarrow \C^+$ lifting $T$ through $U$
	is isomorphic to a unique \emph{strict representative}. i.e., a functor,
	which assigns to commutative triangles of $\Aix$ only morphisms with 
    discrete components and additive symmetries and appropriate 
    identities in the second component.
	In particular any two functors restricting to the same $A$-tuple along 
	$R$ are naturally isomorphic. \begin{proof} To build the 
	\emph{strict representative} $F^{st}$ proceed as follows: Choose a bijection 
	$\sigma_A\colon A_+\rightarrow |A|_+$ and set $F^{st}(\sigma_A)
	=(F(\chi_a\colon A_+\rightarrow 1_+))_{a\in A}$. Any other 
	object of $\Aix$ has a unique morphism from $\sigma_A$. 
	So for $p\in \Aix$ set $F^{st}(p)=(p\sigma_A^{-1})_*(F^{st}(\sigma_A))
	\in\C^{\times|T(p)|}\subset\C^+$. I drop the $\sigma_A^{-1}$ from the notation
	immediately, since it is only there to make coherent choices for all
	maps of finite sets at once. For a commutative triangle under $A_+$:
	\[\xymatrix{A_+\ar[r]^{p}\ar[dr]_{qp} & n_+\ar[d]^q\\&m_+}\]	we need 
	a morphism $F^{st}(p)=p_*(F^{st}(\sigma_A)) \rightarrow 
	q_*p_*(F^{st}(\sigma_A))\rightarrow (qp)_*(F^{st}(\sigma_A))=F^{st}(qp),$ 
	which we can choose to be $(q,\varphi^{q,p})$; in particular it only 
	has the claimed components. Furthermore it obviously projects down 
	to $q$ in $\Fi$ by the forgetful functor $U\colon \C^+\rightarrow \Fi,$ 
	so it is a lift of $T$ through $U$. Since the second component is always 
	a symmetry we also trivially have a functor $F^{st}\colon\Aix\rightarrow 
	(\C^{iso})^+$, hence an object of $\C(A_+,1)$.

	For the isomorphism first consider the following diagram:
	\[\xymatrix{A_+\ar[r]^{\sigma_A}\ar[dr]_{\chi_a}& |A|_+.\ar[d]^{\rho^a} 
	\\ & 1_+}\] By definition $F$ sends $\rho^a$ to a morphism ($U$-)over 
	$\rho^a$, hence of the form $(\rho^a,f_a)$ with $f_a\colon 
	(\rho^a)_*F\sigma_A=F(\sigma_A)_a\rightarrow F(\chi_a)$ an isomorphism
	in $\C$. These assemble to an isomorphism $F\sigma_A\rightarrow 
	(F\chi_a)_{a\in A}$ in $\C^{\times A}$.

	We can uniquely write each $Fq\colon Fp\rightarrow F(qp)$ as
	\[\xymatrix{Fp\ar[rr]^-{(q,\id)} && q_*Fp \ar[rr]^-{(\id,
	F^{\C}(q))} && F(qp),}\] with $F^\C$ a morphism in a product 
	category $\C^k$ for $q\colon A_+\rightarrow k_+$.

	In particular we get a canonical morphism:
	\[\xymatrix{Fp \ar[rr]^-{(F^{\C}(p\sigma_A^{-1}))^{-1}} && p_*F\sigma_A
	\ar[rrr]^-{(\id,p_*((F\chi_a)_{a\in A}))} &&& p_*F^{st}\sigma_A = F^{st}(p),}\]
	which assembles to a natural transformation $F\Rightarrow F^{st}$
	whose components are isomorphisms by the assumption on $F$.	So we 
	have a canonical isomorphism for each functor to its strict
	representative, and the strict representative only depends on the
	restriction of $F$ along $j_A$.\end{proof}}}

This immediately has the following corollary, which is useful for constructing
such lifting functors more easily.
{\cor{Each functor $F$ lifting $T$ through $U$ is uniquely determined up to natural
	isomorphism by its restriction to $A_+\downarrow \Ep$. In particular we
	can assume without loss of generality that for each $p\colon A_+\rightarrow k_+$
	with $k>|A|$ the object $Fp$ is given by padding with zeroes from a
	bijection $F\sigma_A$, and the morphisms accordingly only have identities
	in zero-components.}}

{\rem{\label{warumEpi}Restricting to epimorphisms is extremely convenient. 
Given any finite set $A_+$ we know a priori that the index-categories
$A_+\downarrow Epi_+$ are finite, i.e., have finitely many objects.

Since for finite $A_+$ there is always a non-unique maximal surjection, i.e.
a bijection $A_+ \rightarrow |A|_+$, any other object of $A_+ 
\downarrow \Fi$ can be written relative to a chosen bijection. 
In particular for $n>|A|$ there is an injection $|A|_+ \rightarrow n_+,$ 
which is unique, if we choose the injection strictly monotonous 
and with minimal maximal element.}}

Let me reemphasise the uniqueness clause of the strict representative to
the canonical delooping statement:
{\prop{For $(\C,+)$ a permutative category the delooping category $\C(A_+,1)$
	is naturally equivalent to a product category by restriction along 
	$j_A$. I.e., we have an equivalence:\[\C(A_+,1)\simeq \C^A.\]\begin{proof}
	The construction of the strict representative given above can also be
	used to promote each $A$-tuple to a lifting functor, which gives the
	inverse equivalence to restriction along $j_A$. 
	The natural isomorphism on the left was given above, on $\C^A$ these
	functors strictly compose to the identity.	\end{proof}}}

{\rem{It is true, but inessential and uninstructive to prove, that the 
	delooping category $\C(A_+,1)$ is actually naturally \textbf{isomorphic}
	to the Segal construction on a permutative category $\C$ as defined in 
	\cite[Construction 4.1, Theorem 4.2]{EM}.

	The isomorphism $\C(A_+,1)\rightarrow 
	\C^{Seg}$ is given by sending a	lifting functor $F$ to its 
	restriction on characteristic functions for subsets 
	$\chi_S\colon A_+\rightarrow 1_+$. The additors $\rho_{S,T}$ are given 
	by the maps associated to the factorisation in $\Aix$: \[\xymatrix{A_+
	\ar[r]^{\chi_{(S,T)}} \ar[dr]_{\chi_{S\cup T}}& 2_+\ar[d] \\ &1,}\]
	for $S,T$ disjoint subsets of $A$.	

	The associativity of the additors given in the diagrams of 
	\cite[Construction 4.1]{EM} follows from the fact that the map
	$\chi_{(S,T,U)}\colon A_+\rightarrow 3_+$ in particular
	has maps in $\Aix$ to $\chi_{(S\cup T,U)}$ and $\chi_{(S,T\cup U)},$
	which both map to $\chi_{S\cup T\cup U}$, giving a commutative
	square in $\Aix$, thus one in $\C^+$ for each lifting functor.}}

\subsection{The Construction $\C(A_+,n)$ for Permutative $1$-Categories}
\label{ca-n!} I want to describe the delooping
construction in \cite[Construction 4.4]{EM} in the same way that I 
just described the Segal construction $\C(A_+,1)$. To that 
end I consider the Segal construction with $n$-fold products 
of finite sets and maps flattened into the additive Grothendieck 
construction $\C^+$, such that the case before is $n=1$.

As in \ref{Finp} fix a smash product functor on $\Fi$.
Then we can define a symmetric monoidal structure on $\C^+$ when 
given a bimonoidal structure on $\C$.

{\prop{
	Consider the additive Grothendieck construction $\C^+$ for
	a bimonoidal category $(\C,+,\cdot)$, then
	we have an induced monoidal structure on $\C^+$,
	which makes the forgetful functor $U\colon \C^+ \rightarrow \Fi$
	strictly monoidal with respect to the induced
	multiplication on $\C^+$ and the smash-product functor on
	$\Fi$. If moreover the multiplication of $(\C,+,\cdot)$ makes
	$\C$ a bipermutative category, the induced monoidal structure
	is symmetric, and the functor $U$ is strictly symmetric
	monoidal.
	\begin{proof}
	Given a smash product functor on $\Fi$ we have fixed pointed 
	bijections $(n\times m)_+=n_+\wedge m_+\rightarrow nm_+,$
	hence also $\omega_{n,m}\colon n\times m\rightarrow nm,$ which 
	are associative. For two objects $c=(c_1,\ldots,c_n),
	d=(d_1,\ldots,d_m)$ in $\C^+$
	set their product to be: \[c\boxtimes d = (c_id_j)_{\omega(i,j)},\]
	where I have written points in the indexing set 
	$\{1,\ldots,nm\}$ as images $\omega(i,j)$.

	The essential subtlety is the fact that this can be made into
	a functor. Consider the following two objects in $\C^+$:
	\[(f\wedge g)_*(c\boxtimes \bar c)_{\omega(i,j)}
	=\sum_{\omega(k,l)\in(f\times g)^{-1}(\omega(i,j))}c_k\bar c_l,\]
	and analogously: \[((f_*c)\boxtimes(g_*\bar c))_{\omega(i,j)}
	= (f_*c)_i\cdot(g_*\bar c)_j = \left(\sum_{k\in f^{-1}i}c_k\right)
	\left(\sum_{l\in g^{-1}j}\bar c_l\right).\]

	By \ref{bim1} and \cite{Lap} we find that there is a unique composite
	of distributors and additive symmetries comparing these objects. For
	instance by first reducing summands on the left, then on the right,
	we get: \[(f\wedge g)_*(c\boxtimes \bar c) = \sum_{k,l}c_k\bar c_l
	\rightarrow \sum_k\left(c_k\left(\sum_l\bar c_l\right)\right)\]
	\[~~~~~~~
	\rightarrow \left(\sum_{k\in f^{-1}i}c_k\right)
	\left(\sum_{l\in g^{-1}j}\bar c_l\right) = f_*c\boxtimes g_*\bar c.\]
	Hence there is a unique structural morphism $D^{f,g}$ determined by
	the summations $f$ and $g$ induce. Thus for two
	maps in $\C^+$ in the case of hom-sets with structure:
	\[(f,(a_1,\ldots,a_{m_1}))\colon c=(c_1,\ldots,c_{n_1})
	\rightarrow (d_1,\ldots,d_{m_1})=d,\]\[(g,(b_1,\ldots,b_{m_2}))\colon 
	\bar c=(\bar c_1,\ldots,\bar c_{n_2})\rightarrow 
	(\bar d_1,\ldots,\bar d_{m_2})=\bar d,\]
	we set their product to be the composite:
	\[\xymatrix{c\boxtimes \bar c = (c_i\bar c_j) \ar[r]^-{(f\wedge g)_*}&
	\left(\sum c_k\bar c_l\right)\ar[r]^-{D^{f,g}}&	\left(\sum c_k
	\right)\left(\sum\bar c_l\right) \ar[r]^-{a_ib_j}&(d_i\bar d_j).}\]
	Analogously define the product for general enriched bimonoidal categories as
    follows: Consider $D^{f,g}\circ (f\wedge g)_*$ on the $(f,g)$-component
	of the morphisms on the product of the additive Grothendieck construction
	on $\C$, i.e., $(\C^+\times\C^+)((c,\bar c),(d,\bar d))=
	\coprod_{(p,q)}\C^{|\bar c|}(p_*c,\bar c)\times \C^{|\bar d|}(q_*d,\bar d)$,
	and postcompose with the monoidal product $\cdot$ of $\C$, which in its
	$\omega(i,j)$th component pairs the $i$th factor in the first product 
	with the $j$th factor in the second product.

	This assignment evidently sends identities to identities, and it
	respects composites, because $\cdot$ is part of a bimonoidal/bipermutative
	structure on $\C$, hence we can always interchange distributors $D^{f,g}$
	and tuples of genuine $\C$-morphism $(a_ib_j)$ by summing up and reducing
	the appropriate components.
	In summary we obtain a functor:
	\[\boxtimes\colon \C^+\times\C^+\rightarrow \C^+.\]

	Since we have chosen $\wedge$ to be a strictly unital functor 
	on $\Fi$ and $\cdot$ strictly unital on $\C$, the $1$-tuple 
	$(1)\in\C^+$ with entry the multiplicative unit of $\C$ is a 
	strict unit for $\boxtimes$. Since $\wedge$ is strictly associative
	and $\cdot$ is strictly associative, $\boxtimes$ is strictly associative
	as well.

	Finally for $\cdot$ not just a monoidal, but a braided or symmetric
	monoidal structure with symmetry $c^\cdot$, consider the symmetry 
	in $\Fi$ for $\wedge$, and call	it $\chi$. Then a multiplicative 
	symmetry for $\boxtimes$ on $\C^+$ is given	by:
	\[\xymatrix{c\boxtimes d = (c_id_j)_{\omega(i,j)} \ar[r]^-\chi
	& (c_id_j)_{\omega(j,i)} \ar[r]^-{c^\cdot} & (d_jc_i)_{\omega(j,i)}
	=d\boxtimes c.}\]
	It squares to the identity if $c^\cdot$ does, and satisfies the braiding
	coherence diagrams for triple products that $c^\cdot$ satisfies. Hence
	yields a braided/symmetric monoidal structure, if $(C,\cdot)$ is
	braided/symmetric monoidal and bimonoidal as $(C,+,\cdot)$.	We have
	evidently constructed the symmetric monoidal structure just so that
	$U$ becomes a strictly (braided/symmetric) monoidal functor.
	\end{proof}}}

I do not intend to get back to this multiplicative structure until chapter
\ref{multbidel}, but wanted to explicitly state it for $1$-categories.
It emphasises that the multiplicative structure exhibited in 
chapter \ref{multbidel} can be built easily here as well.

To define the higher delooping categories $\C(A_+,n)$ we need to consider
target functors for the product categories $(\Aix)^n$.
{\prop{\label{T-n}
	There is a forgetful functor $T_n\colon (\Aix)^n\rightarrow \Fi$ for each
	$n\geq 1$, which is faithful away from the basepoint, and can moreover 
	be chosen to be associative, i.e., the diagram \[\xymatrix{(\Aix)^n\times
	(\Aix)^m\times(\Aix)^l \ar[rr]^-{T_{n+m}\times \id}\ar[d]_{\id\times 
	T_{m+l}}&&\Fi\times (\Aix)^l \ar[d]^{\id\times T_l}\\(\Aix)^n\times\Fi 
	\ar[d]_{T_n\times \id}&& \Fi\times\Fi \ar[d]^{\wedge}\\\Fi\times\Fi 
	\ar[rr]^{\wedge} && \Fi,}\]	commutes. Moreover the functors $T_n$ can 
	be chosen symmetric, i.e., the functors
	\[\xymatrix{(\Aix)^n\times (\Aix)^m \ar[rr]^-{T_n\times T_m} &&
	\Fi\times \Fi\ar[r]^-{\wedge} & \Fi	}\]	and	\[\xymatrix{(\Aix)^{n+m}
	 \ar[r]^-{tw^\times_{n,m}}& (\Aix)^{m+n} \ar[r]^-{T_m\times T_n} 
	& \Fi\times\Fi\ar[r]^-\wedge&\Fi}\]	are naturally isomorphic by exchanging 
	priority of the smash components, i.e., $\chi^\wedge$ on $\Fi$.
	\begin{proof}Simply set $T_n$ to be the following functor:\[\xymatrix{
	(\Aix)^n\ar[r]^-{(T)^n} & \Fi^n	\ar[r]^{\wedge} & \Fi,}\]where $T$ is 
	the target functor of $\Aix$, and $\wedge$ is the 
	$n$-fold smash, which is defined because $\wedge$ is strictly 
	associative. Then the $T_n$ inherit associativity and symmetry as claimed, 
	and are just as faithful as $T$ and $\wedge$. Hence for maps with 
	$f^{-1}+=\{+\}$	we get injectivity on hom-sets.\end{proof}}}

These functors should give the reader a reasonable hunch how I define
$\C(A_+,n)$ such that $\C(A_+,1)$ considered above trivially becomes the
case $n=1$.
{\defn{The \textbf{higher delooping category} of a permutative category $\C(A_+,n)$ for
	$n\in\mathbb{N}$ and $A_+$ a finite pointed set, is given as the
	category of functors lifting $T_n$ through $U$, i.e., the dashed arrows
	in the diagram:
	\[\xymatrix{& (\C^{iso})^+\ar[d]^U \\ (\Aix)^n \ar@{-->}[ur]\ar[r]^-{T_n} 
	& \Fi.}\]Its morphisms are the natural transformations of functors pushed forward
	with the inclusion $(\C^{iso})^+\rightarrow \C$, i.e., natural transformations
	with arbitrary components, not just isomorphisms.}}

{\ex{\label{kommKub} Consider again (\ref{troublemaker}) the prototypical 
surjections $f,g\colon 4_+\rightarrow 2_+$, with $f(1)=f(3)=1$,$f(2)=f(4)=2$, 
$g(1)=g(2)=1$,$g(3)=g(4)=2$,and $q\colon 2_+\rightarrow 1_+$ again. 
Choosing the bijection $\omega$ for the smash product
gives the following commutative cube: \[\xymatrix{ (2\times 2)_+ 
\ar[rr]^<<<<<{id\times q}\ar[dd]_<<<<<{q\times id} \ar[dr]^\omega && 
(2\times 1)_+\ar[dr]^\omega \ar[dd]^<<<<<{q\times id}\\ &4_+
\ar[rr]^<<<<<{f}\ar[dd]_<<<<<{g} && 2_+\ar[dd]^>>>>>q\\ (1\times 2)_+ 
\ar[rr]^<<<<<{id\times q} \ar[dr]_\omega && (1\times 1)_+ 
\ar[dr]^\omega\\ &2_+\ar[rr]_>>>>>q&&1_+.}\]

Flattening this cube makes a pentagon with the additional arrow given
by $1+c^++1$, which appears for instance in axiom (5) of 
\cite[Construction 4.4]{EM}.}}

{\prop{\label{symmSpekt!!} Each permutation $\sigma\in\Sigma_n$ 
induces a functor \[\sigma\colon (A_+\downarrow \Fi)^{\times n} 
\rightarrow (A_+\downarrow \Fi)^{\times n}.\] Precomposing with this
permutation of the components and post-composing a functor with
the symmetry $\chi$ of the smash-product on $\Fi$, induces a natural
$\Sigma_n$-action on $\C(A_+,n)$ .
\begin{proof}The statement concerning the symmetry warrants some 
explanation. By \ref{T-n} we know that we can choose the target
functors $T_n$ as $(\Aix)^n\rightarrow \Fi^n\rightarrow \Fi$, hence
the symmetry isomorphism from $T_n$ to $\sigma^*T_n$, for 
$\sigma\in\Sigma_n$ can be pushed forward to $\Fi$ by the appropriate
symmetry $\chi_{\sigma}^{\wedge}$ of the smash-product on $\Fi$. Then
by pushing forward the permuted functor in $\C^+$ with the same symmetry
we get a functor lifting $T_n$ again.\end{proof}}}

{\rem{To state the following proposition conveniently I introduce a standard
    simplifying assumption. For $\C$ a permutative category we can without 
    loss of generality assume that $0\in\C$ is an isolated object, i.e., it
    has at most non-trivial endomorphisms, but no maps in $\C$ from or to
    different objects. This makes $\C\setminus\{0\}\sqcup\{0\}$ a decomposition
    of $\C$ by full subcategories. 

    For $\C$ with isolated zero $0$ as above we can define the smash product of $\C$
    with a finite pointed set $A_+=A\sqcup\{*\}$ as:
    \[A_+\wedge \C := End(0)\coprod_{x\in A} (\C\setminus\{0\}).\]

    The assumption is without loss of generality since we can attach to each
    category $\C$ a disjoint basepoint $\C_+:=\C\sqcup\{*\}$. For $\C$ permutative
    we extend the monoidal functor by letting $*$ act as a strict unit, thus in 
    particular the object $0\in\C$ is not neutral in $\C_+$. On classifying spaces
    we get $|N\C_+|=|N\C|\sqcup \{*\}$, i.e., we only added a disjoint basepoint
    to the classifying space as well.

    Finally note that for $\C$ permutative with an isolated additive unit $0$, any
    finite pointed set $A_+$ and any natural number $n$, the
    higher delooping categories have an isolated zero as well:
    The category $\C(A_+,n)$ has a basepoint given by the 
    functor $$O\colon (A_+\downarrow Epi_+)^{\times n}\rightarrow \C^+$$ 
    with $O((f_1,\ldots,f_n)\colon (A_+,\ldots,A_+) \rightarrow (k^1_+,
    \ldots,k^n_+))=0$ the $k^1\cdot\ldots\cdot k^n$-tuple consisting 
    only of the additive unit, and each morphism is sent to its appropriate 
    discrete component with $\C^{\times \bullet}$-components only 
    identities. If the additive unit in $\C$ is isolated, then this zero functor $O$ is 
    an isolated basepoint of $\C(A_+,n)$ as well, and we can identify the smash as:
    \[A_+\wedge \C(A_+,n) 
    =End(O) \sqcup \coprod_{x\in A} (\C(A_+,n) \setminus\{O\})\times 
    \{x\}.\] Take note that the disjoint union is over all non-basepoints 
    in $A_+$, hence all elements of $A$. The fact that $O$ is isolated 
    ensures that each $\C(A_+,n)\setminus\{O\}$ forms a (sub)category.}}

The extension functors are a bit obscured by the fact that I chose to 
reduce the arguments in the delooping-construction of \cite{EM}
to $n$ equal inputs $A_+$ only, but it coalesces nicely. 

{\prop{We have a natural inclusion of categories: \[e\colon A_+
\wedge\C(A_+,n)\rightarrow \C(A_+,1+n).\] Furthermore this inclusion
is $\Sigma_n$-equivariant, where on $\C(A_+,1+n)$ the 
action is given by restriction along the inclusion $\Sigma_n=\Sigma_1
\times\Sigma_n\rightarrow \Sigma_{1+n}$, i.e., letting $n$-permutations 
act on the indices $\{2,\ldots,n+1\}$.
\begin{proof} 
As above we see if the additive unit in $\C$ is isolated, then 
the zero functor $O$ is an isolated basepoint of $\C(A_+,n)$, 
and we can identify the smash as: \[A_+\wedge \C(A_+,n) 
=End(O) \sqcup \coprod_{x\in A} (\C(A_+,n) \setminus\{O\})\times \{x\}.\]
Hence I can describe the extension functor on each component 
seperately. We set $e(O)=O$ and send an endomorphism of 
$O\in\C(A_+,n)$ to the endomorphism of $O\in \C(A_+,1+n)$, which we obtain 
by appropriately extending with identities.

More interestingly consider a summand $\C(A_+,n)\setminus\{0\}\times 
\{x\}$. Then we have to define $e(F,x)(p_1,\ldots,p_{n+1})$ for each 
$(n+1)$-tuple of maps $p_i\colon A_+\rightarrow k_i$. We set: 
\[e(F,x)(p_1,\ldots,p_{n+1})=\begin{cases} 0 &~~p_{1}\neq (\rho^x
\colon A_+\rightarrow 1_+),\\F(p_2,\ldots,p_{n+1}) &~~p_{1}=\rho^x.
\end{cases}\] Accordingly, $e(F,x)$ is the identity on the additive unit
for all $(n+1)$-tuples of morphisms, which do not have $id_{1_+}=
id_{\rho^x}$ as its first component. On the tuples, where the 
last component is $id_{1_+}$ and target and source have last 
component $\rho^x$ we can use $F$ on the first $n$ maps.

This is obviously a functor, natural in $\C$ and $A_+$.
Equivariance follows by our choice of singling out the first component. 
Changing the marked component changes the inclusion $\Sigma_n\rightarrow 
\Sigma_{1+n}$ but still yields equivariance as claimed.\end{proof}}}

Again we have the result making $\C(A_+,n)$ a delooping of $\C$.
{\thm{For $(\C,+)$ a permutative category we have a natural equivalence
	of categories \[\C(A_+,n)\simeq Set(A,\C(A_+,n-1)),\]
	with the map $\C(A_+,n)\rightarrow Set(A,\C(A_+,n-1))$ given by
	restriction of functors along $(\id_{n-1},j_A)\colon
	(\Aix)^{n-1}\times A^\delta\rightarrow (\Aix)^n.$
	Inductively we get a natural equivalence:
		\[\C(A_+,n)\simeq \C^{A^{\times n}}.\]\begin{proof}
	The start of the induction is the case $n=1$ displayed above. The same
	argument with \emph{strict representatives} can be made to prove the
	equivalence above by making a functor in $\C(A_+,n)$ only consist of
	discrete components and additive symmetries one component at a time.
	\end{proof}}}

I give the construction of the Eilenberg-Mac Lane spectrum based on
this delooping in chapter \ref{multbidel} in the maximal generality
I need it. The case of permutative $1$-categories then follows by 
considering them as permutative bicategories with discrete morphism
categories. The maximal generality in this thesis is motivated by the 
principal example of interest $K(ku)$. The next chapter is thus 
concerned with the module bicategory of a bimonoidal category. 
For the spectrum $ku$ I want to fix the relevant models and maps next.

\section{Models for $ku$}\label{kumodels}
Assuming that the delooping given by $\C(A_+,n)$ yields an $E_\infty$ 
symmetric ring spectrum $H\C$, which I prove in chapter \ref{multbidel}, 
we get models for connective $K$-theory with an $E_\infty$-multiplication 
by considering nicely explicit bipermutative categories.

{\ex{\label{modultensor}
	Given a commutative ring $k$, consider its (skeletal)
	category of finitely generated free modules $\MM_k$ on 
	objects: \[\Ob\MM_k = \Ob~\mathrm{Fin} = \{\mathbf{n}| n\in\mathbb{N}\}.\] 
	Consider the unpointed sets $\mathbf{n}$ as ranks of finitely generated
	free modules over $k$. To establish its 
	bipermutative structure first consider the morphism sets in a bigger category
    $\MM^L$ (compare \ref{Fin}):
	\[\MM^L_k(\mathbf{n},\mathbf{m}):= 
	\mathit{Hom}_k(k\{1,\ldots,n\},k\{1,\ldots,m\})\]
	of all $k$-linear maps of free modules on the unpointed sets $\mathbf{n},
	\mathbf{m}$.

	Fixing the direct sum functor as the linear extension of disjoint
	union gives a strictly associative coproduct-functor for $\MM^L$:
	\[k\{1,\ldots,n\}\oplus k\{1,\ldots,m\}:=k\{\mathbf{n+m}\}
	=k\{1,\ldots,n,n+1,\ldots,n+m\},\] with the obvious extension to 
	morphisms by linearly extending the description of $\mathrm{Fin}$ 
	on basis elements.

	The product functor chosen on finite sets (specifically by fixing
	associative bijections $\omega=\omega_{n,m}\colon \mathbf{n}\times 
	\mathbf{m} \rightarrow \mathbf{nm}$) 
	extends to the tensor-product of free modules: \[k\{1,\ldots,n\}\otimes 
	k\{1,\ldots,m\} := k\{1,\ldots,nm\},\] where we define the 
	tensor-product of linear maps represented as quadratic matrices
	$f\in M_n(k), g\in M_m(k)$ as follows:
	\[(f\otimes g)(e_{\omega{(i,j)}}):= \sum_{\omega(s,t)\in nm} 
	f_{si}g_{tj}e_{\omega(s,t)}.\] For the entries of the representing
	matrix for $f\otimes g$ with respect to the ordering on $\mathbf{nm}$ fixed
	by $\omega\colon \mathbf{n}\times\mathbf{m}\rightarrow \mathbf{nm}$ we get:
	\[(f\otimes g)_{\omega(i_1,j_1),\omega(i_2,j_2)}
	=f_{i_1,i_2}\cdot g_{j_1,j_2},\] which is a strictly associative
	representation of the tensor-product. We know it is left-adjoint
	to the $\mathit{Hom}_k$-functor, hence Lemma \ref{bipermallthethings} applies, and
	we get a bipermutative structure on the  $\MM^L_k$.

	Analogous to finite sets we can restrict to 
	injections, surjections and isomorphisms. The case of 
	isomorphisms is the one of primary interest in this thesis, so I 
	define \emph{the} module category $\MM_k$ of a commutative ring
    as:\[\MM_k(n,m)=\begin{cases}GL_n(k) &~~n=m\\ 
	\emptyset &~~n\neq m.\end{cases}\]}}

For $k=\RR,\CC$ we have a topological version of the above example.
{\ex{The same constructions as in the example above describe continuous
     functors with respect to the topologies on $GL_n\RR$ and $GL_n\CC$
     as subspaces of $M_n\RR\cong \RR^{n^2}$ and $M_n\CC\cong \CC^{n^2}$.
     Call the category with objects the natural numbers and morphism
     spaces $GL_nk$ considered as a topologically enriched category $\MM_k^c,$
     where the upper index is a reminder for continuity. Call the 
     analogous discrete category with morphism sets $GL_nk$ and
     their discrete topology $\MM_k^\delta$.

     For real and complex coefficients we can restrict 
     to the respective compact subgroups:
		\[O_n\rightarrow GL_n(\mathbb{R}), \mathrm{~and~}~U_n \rightarrow 
		GL_n(\mathbb{C}).\]
     These inclusions define subcategories of the topological as well as the
     discrete module categories.
     Denote the topological subcategories by $\V^c_k\subset \MM_k^c$ and 
     analogously the discrete subcategories by $\V^\delta_k\subset \MM_k^\delta$.
     Since the symmetries and distributors are unitary morphisms as well, 
     the canonical inclusion functors are bipermutative:
	    \[\V_\mathbb{R}\rightarrow \MM_\mathbb{R},\]
	    \[\V_\mathbb{C}\rightarrow \MM_\mathbb{C},\]
     for the topological as well as the discrete versions.

     A continuous inverse is given by the Gram-Schmidt process, which I
     denote by $r\colon \MM_k\rightarrow \V_k.$ This is compatible
     with direct sum by considering the sum as orthogonal. 
     It can be promoted to a homeomorphism
     as follows (cf. \cite[pp.33-35]{MT}): We have a natural map \[g_n\colon O_nk
     \times H^+_nk\rightarrow GL_nk\] with 
	$g_n(U,B)=UB$ for $O_nk$ the orthogonal group for $k=\RR$ and the 
	unitary group for $k=\CC$, and $H^+_nk$ the space of symmetric/hermitian 
	positive definite matrices. For each $n$ the map $g_n$ is a 
	homeomorphism with inverse:\[h_n\colon GL_nk \rightarrow O_nk\times H^+_nk\] 
	given by: \[h_n(A)= (A\sqrt{(A^*A)}^{-1}, \sqrt{(A^*A)}).\]

	The map $g_n$ is compatible with direct sums. For tensor-products
	consider the following diagram:	\[\xymatrix{
	O_n\times H_n^+\times O_m\times H_m^+\ar[r]^-{g_n\times g_m}\ar[d] 
	& GL_n\times GL_m\ar[dd]^\otimes\\ O_n\times O_m\times H_n^+\times H_m^+
	\ar[d]_{\otimes\times\otimes}\\
	O_{nm}\times H^+_{nm} \ar[r]_{g_{nm}} & GL_{nm}, }	\]
	which needs to commute for $g$ to induce a strictly multiplicative functor 
	$\V_k\rightarrow \MM_k$. Fixing associative bijections $\omega$ 
	write:$(A\otimes B)_{ij}=A_{i_1j_1}\cdot B_{i_2j_2}.$

	Then we have:
	\[\begin{aligned} 
	(g_n(U,B)\otimes g_m(V,C))_{ij}&= g_n(U,B)_{i_1j_1}g_m(V,C)_{i_2j_2}\\
	&=(UB)_{i_1j_1}(VC)_{i_2j_2}=\sum_{k,l} U_{i_1k}B_{kj_1}V_{i_2l}C_{lj_2},
	\end{aligned}\]	while the other side is given by: \[\begin{aligned}
	g_{nm}(U\otimes V,B\otimes C)_{ij}	&= ((U\otimes V)(B\otimes C))_{ij}\\
	&= \sum_p (U\otimes V)_{ip}(B\otimes C)_{pj}= 
	\sum_p U_{i_1p_1}V_{i_2p_2}B_{p_1j_1}C_{p_2j_2}.	\end{aligned}\]
	The terms thus agree by commutativity of addition and multiplication 
	in $k$, so we get a bipermutative inclusion $I\colon \V_k\rightarrow 
	\MM_k$ as well as a bipermutative retraction $R\colon \MM_k\rightarrow 
	\V_k$ given by $g$. Specifically we get topologically enriched  
	functors $R\colon \MM_k^c\rightarrow \V_k^c,$ i.e., functors, which
	are continuous on the morphism spaces, and the same assignments define
	functors on the discrete versions $R\colon \MM_k^\delta\rightarrow 
	\V_k^\delta$.}}

{\rem{The tensor-product structure on categories of the form $\MM_k$ has
    a more natural interpretation: Choose an euclidean/hermitian scalar 
    product in each dimension $n$ for $k=\RR,\CC$ or any 
    non-degenerate bilinear form for an arbitrary field $k$ 
    - say $\langle\cdot,\cdot\rangle$. By basic 
	linear algebra we know that any bilinear form $b$ yields a uniquely 
	determined linear map $f$ such that $\langle f\cdot,\cdot \rangle = 
	b(\cdot,\cdot).$ If moreover the bilinear form $b$ is non-degenerate
	as well, then $f$ is an isomorphism. Hence fixing (any) non-degenerate 
	bilinear form $\langle\cdot,\cdot\rangle$ we get an isomorphism:
    \[Bil_+(V)\cong GL(V)\]
	of non-degenerate bilinear-forms on $V$ and linear automorphisms of $V$,
	which is a homeomorphism when meaningful.

	In light of this consider the following tensor product of bilinear forms:
    For two bilinear forms $b^V\colon V\otimes V\rightarrow k$ and $b^W\colon W\otimes W
	\rightarrow k$, define their tensor product $b^{V\otimes W}\colon (V\otimes W)^{\otimes 2}
	\rightarrow k$ on generators:
	\[b^{V\otimes W}(v_1\otimes w_1,v_2\otimes w_2) = b^V(v_1,v_2)\cdot 
	b^W(w_1,w_2),\]	and extend bilinearly.
	This is non-degenerate if both forms above are non-degenerate. It is
	symmetric/hermitian if both forms are symmetric/hermitian. 

	On representing matrices we get the following: \[b^{V\otimes W}(e_i\otimes 
	e_j,e_k\otimes e_l) = b^V(e_i,e_k)b^W(e_j,e_l) = b^V_{ik}b^W_{jl},\]
	hence precisely the coefficients given in \ref{modultensor} for the
	tensor-product of the linear maps associated to the representing
	matrices.}}

Finally I summarise the canonical functors between the examples:
{\ex{For each commutative ring $k$ we have a canonical inclusion functor
	\[k[\cdot]\colon \mathrm{Fin} \rightarrow \MM^L_k\]
	given by sending each finite set to its associated free module, and
	each map to its linear extension. This is a strictly additive as
	well as multiplicative functor, moreover it strictly respects the
	symmetries and distributors. We can obviously restrict to 
	$\mathrm{Inj}$, $\mathrm{Epi}$, $\Sigma$ and restrict the codomain to 
	the appropriate linear maps, i.e., monomorphisms, epimorphisms, 
	or isomorphisms.
	
	We also have the analogous canonical inclusion functor given by
	reduced free modules:
	\[\tilde k[\cdot]\colon \Fi\rightarrow \MM^L_k\]
	sending each finite set to the associated free module with the 
	basepoint divided out. It is again strictly symmetric monoidal
	with respect to pointed sum and smash product, and respects the
	distributivity transformations as well.

	Finally for $k=\mathbb{R},\mathbb{C}$ these functors even have image
    in the categories $\V_k$, since the described maps are obviously orthogonal/unitary.}}

{\defn{Since we want to model connective complex $K$-theory we consider
$k=\CC$ in the example above, and find that $\V^c_\CC$ also becomes a
topological bipermutative category in that case. The delooping of 
$\V^c_\CC$ is the prototypical model for connective complex $K$-theory. 
In particular I denote by $ku:=H\V^c_\CC$ the delooping spectrum of the
topological category of finitely generated complex vector spaces with
morphism spaces the unitary isomorphisms.}}

\subsection*{Preliminaries on Discrete Models for $ku$}\label{ltolzeta}
For each odd prime $p$ there are in addition ``discrete models'' for $ku$, 
which are a suitable replacement when studying its $H\F_p$-homology.

Fix a prime $p$ for which we want an $H\F_p$-approximation
of $ku$, i.e., a spectrum $E$ with a map $E\rightarrow ku,$ which
induces an isomorphism on $H\F_p$-homology. Following Quillen \cite{Q1971}
we want to approximate $ku$ by algebraic $K$-theory of the algebraic
closure $\bar\F_p$ of the finite field with $p$ elements. In fact it
is sufficient to restrict to a subfield of $\bar\F_p$ which contains
the appropriate roots of unity, which we construct here.

For that we need to choose a prime that is a generator 
of $(\Z/p^2)^\times$. Observe that necessarily the existence of just 
one generator implies that $p$ is odd, because for $p=2$ we have 
$(\Z/2^k)^\times\cong\{\pm 1\}\times\Z/2^{k-2}\Z,$ where for $k\geq 3$ 
the second factor is always generated by the powers of $5$ 
\cite[Art. 91, p. 89 - Latin edn.]{Gauss}. 
For $p$ odd the group of units has a 
decomposition $(\Z/p^k)^\times \cong \Z/(p-1)\times \Z/p^{k-1},$ 
and is thus a product of two cyclic groups of coprime order 
\cite[Art. 84, p. 82 - Latin edn.]{Gauss}. Given an integer $g$ reducing
to a multiplicative generator of the units of $\Z/p$ we know by 
Fermat's little theorem $g^{p-1}=1~ \mathrm{mod}~ p$, so 
$g^{p-1}=1+lp$ for some $l\in\Z$. Then we have in $\Z/p^2$: 
\[(g+p)^{p-1}=g^{p-1}+(p-1)g^{p-2}p \mathrm{~mod~}p^2\neq g^{p-1}
\mathrm{~mod~}p^2,\] so at least one of the integers $\{g,g+p\}$ 
satisfies \[g^{p-1}\neq 1 \mathrm{~mod~}p^2~~~ \mathrm{or}~~~(g+p)^{p-1}\neq 1\mathrm{~mod~} 
p^2.\] An integer of $\{g,g+p\}$ satisfying the above inequality 
(multiplicatively) generates the units of $\Z/p^k$ for each $k\geq 2$, 
in particular it generates the units of the $p$-adic integers $\Z_p$ 
topologically.

We use Dirichlet's Theorem on arithmetic progressions
in the following form:
{\thm[Dirichlet]{For a natural number $n\geq 2$ and a unit 
$a\in(\Z/n)^\times$ consider the class of primes 
$P_a=\{~p\in\N~|~p \mathrm{~prime~ and~~} p = a \mathrm{~mod~} n \}$.
Then each class $P_a$ has ``logarithmic density'' $\frac1{\varphi(n)}$ 
in the set of all primes, for $\varphi(n)$ the number of units in
$\Z/n$.}}

{\rem{I do not need the concept of logarithmic density again, so I 
only give a vague description: The intuition 
is that it is an adapted way to measure subsets of countable sets 
(such as the set of all prime numbers), such that the measure is $0$ 
for finite subsets.}}

One proof of the theorem by complex analysis involves the Dirichlet 
$L$-series associated to a homomorphism $(\Z/n)^\times \rightarrow 
\CC^\times.$ For the trivial homomorphism which sends everything 
to $1\in\CC$ the $L$-series has a singularity in $1$. This forces 
the $L$-series of every non-trivial character to be bounded, 
but non-zero, in $1$. This gives the following comparison of 
divergence around $s=1$: 
\[\sum_{p\equiv a \mod n}p^{-s}=\frac1{\varphi(a)}\log\frac1{s-1}\pm C.\] 
In words: The sum over all primes, which are in $P_a$, 
taken with the exponent $-s$ diverges like $\log\frac1{s-1}$ 
in $1$ (up to a constant $C\in \RR$). In particular there are 
infinitely many such primes.

We can use the theorem in particular to specialise to a 
generator $a\in\Z/p^2$, and find a prime $q$ with $q=a~\mod ~p^2,$ 
which generates the units of each $\Z/p^k$ by the considerations 
before. 

{\ex{I want
to exhibit valid choices for all primes below $100$. I organised the 
table by smallest multiplicative generator $q$ for $\Z/p^2$:\\
\phantom{aaaaaaaaaaaaaaaaaaaaaaaaaaaa} \begin{tabular}{c|r}$q$&$p$\\
\hline$2$&$3,5,11,13,19,29,37,53,59,61,67,83,$\\$3$&$7,17,31,43,79,
89,$\\$5$&$23,47,73,97,$\\$7$&$41,71.$\end{tabular}
\phantom{blubb}\\}}

We want to approximate $ku$ by algebraic $K$-theory applied to 
a suitable tower of field extensions.
Start with the prime field with $q$ elements $\F_q$. Since $q$ 
generates the units of $\Z/p^2$ it generates $\Z/p^\times$
as well. So the cyclotomic polynomial of degree $p-1$ \[\varphi_{p}(X)
=\sum_{i=0}^{p-1}X^i\] is irreducible over $\F_q$. Thus we have the 
extension of fields \[\l_0:=\F_q\rightarrow \F_q[X]/\varphi_p\cong 
\F_{q^{p-1}}=\l_0(\zeta_{p})=:k_0\] for $\zeta_p$ some chosen primitive 
$p$th root of unity.
Since $q$ and $p$ are trivially coprime by the assumptions, each 
element in $\F_q$ has a $p$th root in $\l_0$. However, the units of
$k_0$ have order $q^{p-1}-1$. Because $q$ is a unit in $\Z/p$, we get
\[q^{p-1}-1=0\mathrm{~mod~} p,\]
so taking the $p$th power is not injective, hence not surjective.
So there is an element $a\in k_0$, which does not have a $p$th root.
There is an obvious candidate: $\zeta_p$. Because of our assumption on
$q$ and $p$ we have $q^{p-1}-1=0\mathrm{~mod~} p$, but 
$q^{p-1}-1\neq0\mathrm{~mod~} p^2,$ because $p-1$ is strictly smaller
than the order of the units of $\Z/p^2$, so the units of $\l_0(\zeta_p)$ 
decompose as:\[\l_0(\zeta_p)^\times \cong
\Z/p\langle\zeta_p\rangle\times \Z/s,\] for some $s$ coprime to $p$. 
In particular we find that the kernel of $(\cdot)^p$ is contained 
in the $\Z/p$-summand, while the image is contained in the $\Z/s$-factor,
hence $\zeta_p$ does not have a $p$th root in $k_0=\l_0(\zeta_p)$.
Since $p$ is odd, we get that $f_m(X)=X^{p^m}-\zeta_p$ is irreducible
for each $m\geq 1$ if and only if $\zeta_p$ has no $p$th root in $k_0$,
which we just established. So inductively call $\alpha_i = 
\sqrt[p]{\alpha_{i-1}}$ with $\alpha_0 = \zeta_p$. More explicitly
for each $i\geq 1$ we choose a primitive $p^i$th root of $\zeta_p$, and
call it $\alpha_i$.

\label{myfields}
For exposition let me choose a presentation. We can write $k_i$ for 
$i\geq1$ as:\[\F_q[X,Y_i]/\left(Y_i^{p^i}-X,\sum_{k=0}^{p-1}
X^k\right)\cong\F_q[Y_i]/\left(\sum_{k=0}^{p-1}Y^{p^ik}
=\varphi_p(Y_i^{p^i})\right).\] Then the field $\l_i\subset k_i$ is 
given as the fixed-set under the Galois-action of 
the factor $\Z/p-1$, which stems from the cyclotomic extension.

In the fields $k_i$ we trivially have the inclusions $k_i\rightarrow k_{i+1}$
with $Y_i\mapsto Y^p_{i+1},$ i.e., identifying $Y_{i+1}$ as a $p$th root
of $Y_i$. This yields the following diagram
\[\xymatrix{ k_0\ar[r] & k_1\ar[r] &\ldots \ar[r]& k_i \ar[r]&\ldots& \\
\l_0 \ar[u]\ar[r] & \l_1\ar[r]\ar[u] &\ldots\ar[r] & \l_i\ar[r]\ar[u] &\ldots,&}\] 
where the horizontal arrows signify $\Z/p$-Galois-extensions, while
the vertical arrows are $\Z/(p-1)$-Galois-extensions. Hence on colimits
$K=\colim_i k_i$ and $L=\colim_i \l_i$ we get the $\Z/(p-1)$-extension:
$L\rightarrow K,$ which by the presentations given above is
of the form \[L\rightarrow K=L[X]/\varphi_p =L(\zeta_p).\]

{\rem{Let me emphasise that the prime $p$ defining the degree
of the extension with Galois group $\Z/p-1$ is structurally important, while
$q$ only serves to ensure the existence of this extension and its particular
choice is irrelevant to the construction.}}

{\ex{Building on the previous example the extension
of $L$ by a primitive $p$th root of unity $\zeta_p$, as above
$L\rightarrow L(\zeta_p)$,
induces a map of bipermutative categories $\V_L\rightarrow
\V_{L(\zeta_p)}.$ Hence the delooping of these bipermutative 
categories provides a map \[H(\V_L)\rightarrow H(\V_{L(\zeta_p)}),\]
which is a map of $E_\infty$ symmetric ring spectra. Again referring 
to chapter \ref{multbidel} we see that these spectra are models for the 
algebraic $K$-theory of their respective fields and hence we understand 
this map as:\[K(L)\rightarrow K(L(\zeta_p)).\] Consider the Galois 
group of the extension $L\rightarrow L(\zeta_p)$: 
$G=Gal(L(\zeta_p)/L).$ For any homology theory $h_*$ with $p-1=|G|
=|\Z/p-1|$ a unit in its coefficients the map $K(L)\rightarrow K(L(\zeta_p))$
induces an isomorphism on $h$-homology groups:
\[h_*K(L)=(h_*K(L(\zeta_p)))^{G}.\] In particular for $h = H\F_p$ we 
get an equivalence of $p$-completed spectra: \[K(L)^\wedge_p\simeq
(K(L(\zeta_p))^\wedge_p)^{hG}.\]}}

\subsection*{Comparison of the Models}
One essential insight that led Quillen to the definition of algebraic
$K$-theory \cite{Q1971,Q1972,Q1973} was the fact that he could compute
the full Algebraic $K$-theory of all finite fields by comparison to
fibres of Adams operations on $BU$, so in essence by comparison to
$ku$. I want to exhibit the map involved, which is established by the
Brauer lift, but I shall defer the proofs to the relevant sources.

Again by the construction in \ref{myfields} we can easily fix a
homomorphism:\[\mu\colon L(\zeta_p)^\times \subset \langle\zeta_p
\rangle \times\bigoplus_{l ~\mathrm{prime} \neq p}\Z/l^\infty 
\rightarrow \mathbb{C}^\times\] by setting $\mu(\zeta_p) =
\exp(\frac{2\pi i}{p}).$ For a summand indexed by a prime $l$ choose a 
primitive $l$th root of unity in $\CC^\times$ as 
$\exp(\frac{2\pi i}{l})$ as well as the primitive $l^j$th roots of 
unity $\exp(\frac{2\pi i}{l^j})=\zeta_{l,j}$. This yields coherent 
homomorphisms for each $l$ \[\mu_l\colon \Z/l^j\rightarrow \CC^\times, 
\mathrm{~~hence~on~the~colimit~~} \mu_l\colon\Z/l^\infty \rightarrow 
\CC^\times.\]

With these choices fixed we can use the following theorem (cf.
\cite[p. 283, Theorem 5.3.4]{Ros1994}):
{\thm{Let $G$ be a finite group, and let $\rho\colon G\rightarrow 
GL_n\barF_q$ be a finite-dimensional representation of $G$ over the
algebraic closure of $\F_q$. Let $\{\xi^i_g~|~i=1,\ldots,n\}$ be the
eigenvalues of $\rho_g$ with multiplicities, so that the trace of
$\rho_g$ is given as: $tr(\rho_g)=\sum_i \xi^i_g.$

The function $f^\rho\colon G\rightarrow \CC$ with 
$f^\rho(g)=\sum_i\mu(\xi_g)$ is a class-function, hence by basic 
complex representation theory (cf. \cite[Part I, Chapters 1-3]{Serre}) 
a linear combination of characters of complex $G$-representations. 
Call $f^\rho$ the Brauer character of $\rho$.

In fact we have integral coefficients, so $f$ uniquely determines 
a complex virtual representation of $G$, called the 
\emph{Brauer lift} of $\rho$ - denote it $F(\rho)$. Furthermore 
the Brauer lift is additive, i.e., for a short
exact sequence of $\barF_q[G]$-modules:
\[0\rightarrow U\rightarrow V\rightarrow W\rightarrow 0\]
the lifts satisfy $F(V)=F(U)+F(W)$.}}

{\rem{Observe that the Brauer lift consists of
representations of $G$ that have eigenvalues on the circle 
$\S^1\subset \CC$, because the group has finite order.}}

I want to exhibit the idea of how this induces a comparison map, 
but I gloss over quite a few details, which are in part explained 
in \cite[pp. 284--285]{Ros1994} and much more in the original 
\cite{Q1971,Q1972}.

We want to consider the homomorphism $i_n\colon GL_n(L(\zeta_p))
\rightarrow GL_n(\barF_q)$. Since $L(\zeta_p)$ is a colimit of finite
fields the general linear groups $GL_n(l_i)$ are finite groups, which
yield $GL_n(L(\zeta_p))$ as their colimit. The Brauer Lift is
evidently stable in the colimit over the fields $l_i$, since the 
eigenvalues of the matrices do not change. 

We can determine the virtual dimension of the Brauer Lift by the 
trace of the identity: We get $\xi^i(id_n)=1$ for $1\leq i \leq n$, 
so $\sum_i\mu(\xi^i)=\sum_i\mu(1) = \sum_i1=n$.
This is obviously not stable in $n$; thus subtract the 
trivial $GL_n(\CC)$ representation of $GL_nl_i$,
and consider the lift of $i_n$ minus $n$. Obviously the 
trivial complex representation of $GL_nl_i$ of 
dimension $n$ is a lift for the trivial $GL_nl_i$-representation over
$\barF_q$, so we get $F(i_n-n)=F(i_n)-n$ and hence a stable class
of a virtual representation of dimension zero giving a map 
\[BGL(L(\zeta_p))\rightarrow BGL^\delta(\CC)\rightarrow BGL(\CC)\simeq BU.\]
This map induces homology isomorphisms with $\F_m$-coefficients for
any prime $m$ other than $q$, thus induces an equivalence of 
completed spaces at each prime $m\neq q$:
\[BGL(L(\zeta_p))^{\wedge}_m\rightarrow BU^{\wedge}_m,\]
given as theorems 1.6 and 4.7 by Quillen in \cite{Q1971}.

If the Brauer Lift happened to be not just a virtual representation
but indeed a genuine homomorphism $\Phi\colon GL(L(\zeta_p))
\rightarrow GL(\CC)$ we could try to restrict to the $GL_n$ again,
and induce a map of bipermutative categories
$\V_{L(\zeta_p)}\rightarrow \V_\CC,$ which would give an infinite loop
map, i.e., a map of spectra $K(L(\zeta_p))\rightarrow K(\CC)=ku$, and
furthermore of $E_\infty$-ring spectra. But calculating in low 
dimensions for $GL_2(\F_q)$ shows that the Brauer Lift has a genuine
negative component. I suspect this approach could be repaired
with a ``ring-complete'' version of $\V_\CC$ as given by \cite{BDRR2013},
but the result has been established long before that by other methods.

The additivity directly yields that the Brauer Lift is an 
$E_\infty$-map with respect to the $E_\infty$-structure on 
$BGL(L(\zeta_p))^+$ and $BU^+$ induced by direct sum of matrices. 
Furthermore in ``$E_\infty$ Ring Spaces and $E_\infty$ Ring Spectra'' 
May shows \cite[pp. 212-222]{MayEinf} that the induced map is also an 
$E_\infty$-map for the $E_\infty$-structure induced by tensor-products. 
With all this in place we have an equivalence of 
$E_\infty$ ring spectra at $p$:
\[K(L(\zeta_p))^\wedge_p\rightarrow K(\CC)=ku^\wedge_p.\]
The equivalence of spectra is given in 
\cite[pp. 217+218, Corollary VIII.2.7, Theorem VIII.2.8]{MayEinf}, the compatibility
with the $E_\infty$ structures on these spectra is 
\cite[pp. 219-222, Theorem VIII.2.11]{MayEinf}. \label{einfapprox}

\subsection*{The Involutions}
In this thesis I want to investigate the induced involution on the 
algebraic $K$-theory of $ku$ as well, so as the final comment on the
models I establish which involution is induced on $K(l(\zeta_p))$ by
the Brauer lift.

{\prop{For any multiplicative embedding $\mu\colon l(\zeta_p)^\times
\rightarrow\CC^\times$ we have the following relation for involutions:
\[\mu\circ(\cdot)^{-1}=(\cdot)^{-1}\circ\mu=\overline{(\cdot)}
\circ\mu.\]
That is, we have on $\CC^\times$ that multiplicative inversion $(\cdot)^{-1}$
and complex conjugation $\overline{(\cdot)}$ coincide on the image of the
embedding, and the embedding is a monoid homomorphism, thus compatible with
multiplicative inversion.
\begin{proof} Since we have $\mu(1)=1$ it commutes with inverting 
elements, which is a homomorphism because the involved groups are 
commutative. But since the order of every element of 
$l(\zeta_p)^\times$ is finite, we know that $\mu(l(\zeta_p)^\times)
\subset \S^1$. Hence inverting and complex conjugation coincide.
\end{proof}}}

{\prop{For any representation of a finite group $\rho\colon G
\rightarrow GL_n\barF_q$ and the induced representation
of the group $G^{op}$ with opposed multiplication given by
$\rho\circ(\cdot)^{-1}\colon G^{op}\rightarrow GL_n\barF_q$ we have 
the following relation for the Brauer characters:
\[\overline{f^{\rho}} = f^{\rho\circ(\cdot)^{-1}}.\]
\begin{proof} For $g\in G$ calculate the Brauer character with 
$\xi^i_g$ again the eigenvalues of $\rho_g$ with multiplicities:
\[\overline{\sum_i\mu\xi_g^i}=\sum_i\overline{\mu\xi_g^i} 
=\sum_i(\mu\xi_g^i)^{-1}=\sum_i\mu((\xi_g^i)^{-1})
=\sum_i\mu(\xi_{g^{-1}}^i),\] hence follows the claim.\end{proof}}}

Finally, we would like to induce this Brauer character by some virtual
representation which only explicitly depends on $\rho$ and starts
from the same group $G$ instead of the one with opposed 
multiplication, but we do not want to cancel out the inversion. 
Note that the target is a general linear group. For $G=GL_nR$ the group 
hence comes equipped with a second isomorphism from $G$ to $G^{op}$ 
given by transposition.

{\thm{For any representation of a finite group $\rho\colon 
G\rightarrow GL_n\barF_q$ let $\rho^\dagger$ be the representation 
induced by considering the composition:
\[\xymatrix{G\ar[r]^{(\cdot)^{-1}} & G^{op} \ar[r]^\rho 
& GL_n\barF_q^{op}\ar[r]^{(\cdot)^t}&GL_n\barF_q}\]
Then their Brauer characters satisfy:
\[f^{\rho^\dagger}=\overline{f^\rho},\]
and hence by uniqueness of the associated (virtual) representation
we find
\[F(f^{\rho^\dagger})=\overline{(\cdot)}\circ F(f^\rho).\] 
\begin{proof}Obviously transposing matrices does not change the 
eigenvalues involved in the definition of the Brauer character, so the 
preceding proposition directly yields the claimed result.\end{proof}}}

For ease of reference I summarise Quillen's approximation
\cite{Q1971} by the Brauer lift with respect to its multiplicative
and involutive structure in one theorem.
{\thm{The Brauer lift at any prime $p\geq 3$ is a map of $E_\infty$ ring 
	spectra	$K(L(\zeta_p))=H(\MM(L(\zeta_p)))\rightarrow H(\MM(\CC))=ku,$
	which is an equivalence of $E_\infty$-ring spectra after completion
	at $p$: $K(L(\zeta_p))^\wedge_p\rightarrow ku_p^\wedge.$

	Furthermore the involution on $ku$ given by complex conjugation is
	approximated by $(\cdot)^t\circ(\cdot)^{-1}$ on $L(\zeta_p)$, in 
	particular the involution as induced on $ku$ by \ref{indinvSP}
	from complex conjugation cancels out to give the approximation of
	$E_\infty$-ring spectra with involution:
	\[(K(L(\zeta_p))^\wedge_p,\id)\rightarrow (ku_p^\wedge,
	\overline{(\cdot)}_*).\]}}
\label{Invandmultappr}
 \chapter{Bicategories of Matrices}\label{modulbicat}
\section{Osorno's Delooping of $\M$}
Given a permutative category that has a compatible associative multiplication,
one can define its module bicategory \cite{Os}. Furthermore, Osorno
provides a delooping of this bicategory by considering block sums of
matrices and organising these into a $\Gamma$-structure on the module
bicategory. This leads to an associated spectrum given a permutative
bicategory. 

I extend Osorno's result in a multiplicative manner. 
This means I define the bicategory-analogue of bipermutative
categories in this chapter and adapt Osorno's delooping in a 
manner that it has an induced multiplicative structure in the 
next chapter, leading to an $E_\infty$ symmetric ring spectrum.

Given a bimonoidal category $(\R,\oplus,\otimes)$ one can define its
bicategory of matrices as follows:
{\defn{The \textbf{bicategory of matrices} $\M$ associated to a bimonoidal
category $\R$ is given as follows. It has as objects the natural numbers 
$n\in\N_0$, and its morphism categories are:\[\M(n,m):=\begin{cases}
GL_n\R, ~~~&n=m,\\ \emptyset,  & \mathrm{else,}\end{cases}\] with 
$GL_n\R$ the categories of weakly invertible $n\times n$-matrices 
over the bimonoidal coefficients $\R$ (cf. \cite{Os,Ri2010,BDR2004}}).}

For this category to have an associator it is vital that the 
distributivity morphisms of $\R$ are \emph{isomorphisms} 
(cf. equation (4) on p.323 of \cite{Ri2010})! 

In what follows I need that for each bimonoidal category the 
bipermutative category $\Sigma_*$ is part of the bicategory $\M$ 
in a well-behaved way:
{\prop{\label{symmMatr}Consider the map \[\begin{aligned}E_{\bullet}
\colon& \Sigma_n&\rightarrow &~GL_n\R & \sigma \mapsto E_\sigma,\end{aligned}\] 
with $(E_\sigma)_{ij}:=\delta_{i,\sigma j}$. This map satisfies 
$E_{\sigma \tau} = E_{\sigma}E_{\tau}$. It is a faithful 
functor between monoidal categories.

Furthermore there is an action on general matrices: \[(E_\sigma A 
E_\tau)_{ij}=A_{\sigma^{-1}i,\tau j},\] hence in particular: 
\[(E_{\sigma^{-1}}AE_{\sigma})_{ij}=A_{\sigma i, \sigma j} 
\label{indexPerm}.\]}}

Structurally more satisfactory we get the following embedding.
{\prop{\label{strictAss}The category of finite sets as described in 
Example \ref{Fin} includes into the bicategory of modules for each coefficient
category $\R$: \[\Sigma_* \rightarrow \M.\] More explicitly: Consider 
$\Sigma_*$ as a bicategory with discrete morphism categories, then for
each coefficient category $\R$ we get a strict normal functor 
$E_\bullet\colon \Sigma_* \rightarrow \M,$ i.e., $E_\bullet$ strictly
respects identities and compositions \ref{strictn}. \begin{proof} I only give the 
indication of why this is true in my setup. The essential 
point is the strictness of $0$ and $1$ in the coefficient category as 
units, as well as the strict equality $0\cdot a = 0$.\end{proof}}}

This has the following extremely convenient corollary.
{\cor{For each (small) coefficient category $\R$ its module bicategory 
$\M$ has a sub-bicategory which is the faithful image of $E_\bullet$, 
in particular, this sub-bicategory is a $2$-category, so the 
associator restricts to the identity there.}}

{\rem{With these results it is just a minor abuse of notation to 
identify permutations with their images in $\M$, hence I write 
$\sigma = E_\sigma$. In particular the identity matrix of an object
$n\in\M$ is in the image of $E$ and I write $E_n=E_{id_n}$ for the
unit matrix.}}

In \cite{Os} Osorno established that the direct sum of matrices equips 
this bicategory with a well-behaved permutative structure 
\cite[Theorem 4.7]{Os}, and the main result of the paper 
\cite[Theorem 3.6]{Os} states that this can be delooped just as the
classical case \cite{Seg}.

{\thm[{\cite[Theorem 4.7]{Os}}]{\label{OsPlus} The bicategory of matrices $\M$ 
is strictly symmetric monoidal with respect to the \textbf{block sum} of 
matrices: \[\begin{aligned}\boxplus\colon&\M\times\M &\rightarrow&~\M\\
& (n,m)&\mapsto & ~n+m\\&(A,B)&\mapsto &\left(\begin{array}{c|c}A&0\\
		    \hline0&B\end{array}\right).\end{aligned}\]
The symmetry is just the one given by the functor $E_\bullet$ 
defined in Proposition \ref{symmMatr} from $\Sigma_*$ (cf. Example \ref{Fin}), 
i.e.: \[\Sigma_{n+m}\ni c^+_{n,m}=c_+\colon n\boxplus m \rightarrow m\boxplus n .\]}}

This symmetric monoidal structure exhibits the classifying space of
$\M$ as an infinite loop space.

{\thm[{\cite[Theorem 3.6]{Os}}]{\label{plusdeloop} Let $\mathcal{M}$ be a strict 
symmetric monoidal bicategory. Then there is a special $\Gamma$-bicategory 
$\widehat{\mathcal{M}}$ such that:\[\widehat{\mathcal{M}}(1)\cong 
\mathcal{M}.\] Therefore the classifying space 
$|N\mathcal{M}|$ is an infinite loop space upon group completion.}}

I elaborate on the permutative structure and the delooping further in \ref{Mult} 
once my multiplicative matters are in place. In 
particular my main result of this chapter is the following.
{\thm{Given a bipermutative coefficient category $(\R,\oplus,\otimes)$
there are two permutative structures $\boxplus,\boxtimes$ on its 
module bicategory $\M$ that can be arranged into a bipermutative 
bicategory.}}

This bipermutative structure can then be fitted onto the delooping of
$\M$, such that the result is an $E_\infty$-ring spectrum. This is
the content of the next chapter \ref{multbidel}.

{\thm{There is an $E_\infty$ symmetric ring spectrum $H\M$, which is weakly 
equivalent to the spectrum of Osorno's $\Gamma$-space 
$|N\widehat{\M}|$, with multiplication induced by the 
multiplicative structure of $\M$.}}

{\rem{The embeddings of the symmetric groups are compatible with 
direct sum of matrices in the nicest possible way: \[\xymatrix{
\Sigma_n\times \Sigma_m\ar[r]^\sqcup \ar[d] & \Sigma_{m+n}\ar[d]\\
GL_n\R\times GL_m\R \ar[r]^-\boxplus & GL_{n+m}\R.}\] So we have
$\sigma\boxplus\tau=\sigma\sqcup \tau,$ i.e., considering 
permutations as matrices yields that their direct sum is the
disjoint union.}}

\section{Definition - Symmetric Monoidal and Permutative Bicategories}
In what follows I need two types of symmetric monoidal structures.
One is the $E_\infty$-structure which we deloop, thus the one thought
of as additive. The other one gives the induced $E_\infty$-multiplication
on the delooping.

For convenience I use the shorthand $Ob\C=\C_0$. For $1$-cells, 
i.e., objects of morphism categories, when I do not want to refer 
to their source and target I use $Mor\C=\C_1$.

By weakening the definition of $2$-categories to bicategories one has an assortment
of ways how monoidality can be defined for a bicategory. Apart from the weakenings
of unit axioms, one can
impose associativity up to isomorphism, and varying degrees of symmetry. For 
bicategories there is one more degree of symmetry in addition to ``associative,
braided,'' and ``symmetric'', which is called ``sylleptic''. For a detailed discussion
of these notions, as well as a guide to the 7 sources, which incrementally built the
notion of ``monoidal bicategory'' in a fully weakened version, I defer to the PhD thesis
of Christopher Schommer-Pries \cite{SP09}. The original fully weakened definition of 
braided monoidal bicategories goes back to Kapranov and Voevodsky in \cite{KV} as
braided Gray monoids.

{\defn{\label{permbic} A \textbf{permutative bicategory} $(\C,+,0,c_+)$ is a 
bicategory with strict identities, a strict normal \ref{strictn} functor \[+\colon 
\C\times\C \rightarrow \C,\] a chosen additive unit $0\in\C_0$ and a 
strict natural transformation\[c_+\colon +\circ tw\Rightarrow +\] 
for $tw\colon \C\times\C\rightarrow \C\times\C$ the strict isomorphism,
which exchanges factors. 

These satisfy the following identities: \begin{itemize} \item Adding 
$0$ is strictly equal to the identity functor on $\C$: \[0+\_=\_+0=
id_\C.\]\item Addition is strictly associative, i.e., we have an 
equality of functors \[(\_+\_)\circ((\_+\_)\times id) =(\_+\_)\circ(id
\times (\_+\_)),\] \end{itemize} giving a 
well-defined strict normal $n$-fold sum functor for each $n\geq 0$:
\[\sum_n\colon \C^{\times n}\rightarrow \C.\]
Additionally the additive twist has to make the following 
diagrams strictly commutative for every $a,b,c\in\C_0$:
\begin{itemize}\item[]\[\begin{array}{cc} {\xymatrix{a+b+c \ar[r]^{
c_++id_c}\ar[dr]_{c_+} & b+a+c\ar[d]^{id_b+c_+}\\&b+c+a}}&{\xymatrix{
a+b+c\ar[r]^{id_a+c_+}\ar[dr]_{c_+}& a+c+b\ar[d]^{c_++id_b}\\&c+a+b}}
\end{array}\]\item[] \[\xymatrix{a+b \ar@{=}[rr]\ar[dr]_{c_+}&&a+b\\
&b+a,\ar[ur]_{c_+} & }\]\end{itemize} giving a 
unique (strict) natural transformation for every $n\in\N_0$ and 
$\sigma\in\Sigma_n$: \[c_\sigma\colon \sum_n\circ\left({\sigma\colon 
\C^{\times n} \rightarrow \C^{\times n}}\right)\Rightarrow \sum_n\] 
built from composites of $c_+$.\label{uniquecplussigma}}}

{\rem{This is a maximally strictified version of the definition of ``strict symmetric monoidal''
	Ang\'elica Osorno uses in \cite[Definition 3.1]{Os}. She considers more generally a monoidal
	product, which is just a pseudofunctor, as well as a symmetry, which is just pseudonatural.
	Thus in addition to the assumption of strict identities, the strict functoriality of $+$ and
	the strict naturality are stronger conditions to impose.}}

{\defn{\label{mdlbic} A \textbf{symmetric monoidal bicategory} 
$(\C,\cdot,1,c_\cdot)$ consists of a bicategory with strict units 
$\C$, a pseudofunctor \[(\cdot,\Phi)\colon \C\times\C
\rightarrow \C,\] a chosen unit-object $1\in\C_0$, a strong 
pseudonatural transformation \[(c^1_\cdot,c^2_\cdot)\colon \cdot\circ 
tw\Rightarrow \cdot,\] satisfying the following identities:
\begin{itemize}\item Multiplying with $1$ is strictly equal to the 
identity functor on $\C$: \[1\cdot \_=\_\cdot 1=id_\C.\] 
\item Multiplication is strictly associative, i.e., we have a 
strict equality of functors (and their compositors): 
\[(\_\cdot\_)\circ((\_\cdot\_)\times id)=(\_\cdot\_) \circ(id \times 
(\_\cdot\_)),\] giving a well-defined pseudofunctor:
\[(\prod_n,\Phi^{\prod_n})\colon \C^{\times n}\rightarrow \C.\]
In addition the multiplicative twist makes the following
diagrams strictly commute for every $a,b,c\in\C_0$: 
\[\begin{array}{ccc} {\xymatrix{abc \ar[r]^{c_\cdot\cdot id_c}
\ar[dr]_{c_\cdot} & bac\ar[d]^{id_b\cdot c_\cdot}\\&bca}}&{\xymatrix{
abc\ar[r]^{id_a\cdot c_\cdot}\ar[dr]_{c_\cdot}& acb\ar[d]^{c_\cdot
\cdot id_b}\\&cab}}& {\xymatrix{ab \ar@{=}[r]
\ar[dr]_{c_\cdot}  & ab \\  &ba,\ar[u]_{c_\cdot}}}\end{array}\]
which means in more detail that the functors as well as their 
compositors coincide. In particular ${c^1}_\cdot$ squares to the 
identity transformation with identity two-cell, hence ${c^2}_\cdot$
has to square to the identity as well. Again this implies that we have
a unique strong pseudonatural transformation for every $n\in\N_0,
\sigma\in\Sigma_n$:\[c_\sigma\colon \prod_n\circ\left({\sigma\colon 
\C^{\times n} \rightarrow \C^{\times n}}\right)\Rightarrow \prod_n\]
built from composites of $(c^1_\cdot,c^2_\cdot)$.\end{itemize}}}

{\rem{The above notion is precisely the notion of ``strict symmetric monoidal'' Osorno
	gives in \cite{Os} apart from my standing assumption on strict identity $1$-cells
	in the underlying bicategory. Since I consider no other symmetric monoidal structures
	on bicategories than the ones given by block sum and tensor-product on $\M$, I choose
	to drop the attribute ``strict'', since the essential difference to ``permutative'' is
	the \textbf{non-strictness} of the monoidal functor $(\cdot,\Phi)$.}}

{\rem{Do note that the condition that $c^1$ squares to the identity
implies that it is an isomorphism $1$-cell, not just an equivalence as
one might guess.}}

{\rem{In chapter \ref{multbidel} I can much more easily generalise the 
Grothendieck construction as I defined it in \ref{c+}, since the 
additive structure of $\M$ is even a bit stricter than Osorno 
axiomatised, making her delooping apply to more general monoidal
bicategories than mine does. 

I chose the symbols before incorporating the intuition
that I think of permutative structures as additive structures, which we deloop,
while symmetric monoidal structures can potentially give superimposed
multiplications on the delooping.}} 

\section{The Multiplicative Structure on $\M$} \label{Mult} Just as 
in the classical case of commutative rings one should expect the 
module category of a bipermutative category to have a multiplicative 
structure analogous to the tensor product of modules. Since for 
combinatorial reasons I decided to restrict to a coordinatised version
of modules, given by ranks and matrices, the tensor product has to be
one of matrices as well. 

Let me reiterate that the distributivity morphisms
for a bipermutative category are isomorphisms in this
thesis! (Compare Remark \ref{IsoIsoIso}.)

 Given a bipermutative coefficient category 
$(\R,+,\cdot)$ we want to define a tensor product on its 
bicategory of matrices $\M$. Choose an associative 
bijection $\omega_{n,m}\colon \mathbf{n}\times \mathbf{m}\rightarrow \mathbf{nm}$, 
defining a strictly associative monoidal product on $\mathrm{Fin}$,
which represents the cartesian product of finite sets - cf. Example \ref{Fin}.  
For definiteness I set $\omega_{n,m}(i,j):= (i-1)\cdot m + j$ with 
inverse $\theta_{n,m}(i)=(((i-1) ~\mathrm{div} ~m)+1, ((i-1) ~\mathrm{mod}~ m +1),$
where $i~\mathrm{div} ~m := \floor{\frac{i}{m}}$ is the integer part
of division of $i$ by $m$, while $i~\mathrm{mod}~ m$ is the remainder
$r$ for $i=qm+r$ the Euclidean division of $i$ by $m$. 

Recall that this is consistent with the associative smash product 
on $\Fi$ described in Example \ref{Finp}.

{\defn{\label{tensor}
Given a choice of associative bijections $\omega\colon \mathbf{n}
\times \mathbf{m}\rightarrow \mathbf{nm}$ define the tensor product as 
follows: \[\begin{array}{rlll}\boxtimes\colon&\M&\times&\M
\longrightarrow\M\\&(n,m)&\mapsto&nm\\&(A,B)&\mapsto&(A\boxtimes B)_{
\omega(i_1,j_1),\omega(i_2,j_2)}:=A_{i_1,i_2}\cdot B_{j_1,j_2}.\end{array}\] 
The same description applies to the tensor product of 2-cells.}}

The rest of the section is devoted to proving that $\M$ equipped with
this monoidal structure satisfies the axioms given in \ref{mdlbic}.

{\rem{Obviously my choice of $\omega$ is dictated by the choices I
fixed in \ref{Fin} so that I can establish $E_\bullet$ as a 
bipermutative functor.}}

{\ex{For clarity consider the following small example. Let \[A=\left(
\begin{array}{cc} A_{11}&A_{12}\\A_{21}&A_{22}\end{array}\right)\] and
\[B=\left(\begin{array}{cc} B_{11}&B_{12}\\B_{21}&B_{22}\end{array}
\right),\] then with the bijections chosen above we 
have: \[A\boxtimes B=\left(\begin{array}{cc|cc}
A_{11}B_{11} & A_{11}B_{12} &A_{12}B_{11} &A_{12}B_{12} \\
A_{11}B_{21} & A_{11}B_{22} &A_{12}B_{21} &A_{12}B_{22} \\\hline
A_{21}B_{11} & A_{21}B_{12} &A_{22}B_{11} &A_{22}B_{12} \\
A_{21}B_{21} & A_{21}B_{22} &A_{22}B_{21} &A_{22}B_{22} \\
\end{array}\right).\]}}

{\rem{Following this example it is easy to see that the tensor product
as defined in \ref{tensor} respects weakly invertible matrices with
coefficients in a bipermutative category. In particular, if we 
considered a tensor-product of matrices given by columnwise or 
linewise ordering (as opposed to blockwise), any entry $a_{ij}=0$ 
would produce a full zero column or line, hence definitely not a 
(weakly) invertible matrix.}}

{\rem{As I indicated before the structure induced by direct sum of
matrices on $\M$ yields a permutative structure, but the 
tensor-product does not. The fact that $(\M,\boxplus)$ is permutative 
is shown in the proof of \cite[Theorem 4.7]{Os}. The tensor-product already fails
at the first strictness of a permutative bicategory, because $\boxtimes$ 
is not a strict functor, i.e., it only respects composition up to an 
isomorphism two-cell, which is strict if and only if the multiplicative 
symmetry of the coefficients is trivial, hence only for $\R$ an ordinary 
ring (or rig as in the case of $\N$).}}

\subsection{The Matrix Tensor Product is Symmetric Monoidal}
This subsection is devoted to proving that the tensor product equips
$\M$ with a symmetric monoidal structure in all detail. It can be 
skipped safely by the reader without losing any essential information.
I am sure that an elegant short proof by exploiting the functor 
$E_\bullet\colon \Sigma_*\rightarrow\M$ can be devised, but I want to
exhibit the additional strictness the tensor-product on $\M$ satisfies
in explicit detail.

{\lem{\label{tensoristFunktor} The assignment $\boxtimes$ of 
definition \ref{tensor} is a pseudofunctor of bicategories.

In addition we have the following natural identities on $1$-cells:
\[A\boxtimes B = (A\boxtimes id)(id\boxtimes B)\] as well as strict 
compositors:
\[(A^1A^2)\boxtimes id = (A^1\boxtimes id)(A^2\boxtimes id),\]
\[id\boxtimes (B^1B^2) = (id \boxtimes B^1)(id\boxtimes B^2).\]

\begin{proof}Normality is obvious: the identity matrices $E_n$ and 
$E_m$ are sent to \[E_n\boxtimes E_m = E_{nm},\] by strictness of $0$ 
and $1$ in $\R$. The interesting aspect is the compositor 
\[\Phi^\boxtimes\colon (A^1\circ A^2)\boxtimes (B^1\circ B^2)
\Rightarrow (A^1\boxtimes B^1)\circ (A^2\boxtimes B^2).\]
Composition of 1-cells in $\M$ is given by matrix multiplication, 
hence
\[(A^1\circ A^2)\boxtimes(B^1\circ B^2)_{\omega(i_1,j_1),\omega(i_2,j_2)}\]
\[\begin{aligned}&=(A^1\circ A^2)_{i_1,i_2}(B^1\circ B^2)_{j_1,j_2}\\
             &=\left(\sum_k A^1_{i_1,k}A^2_{k,i_2}\right)
\left(\sum_l B^1_{j_1,l}B^2_{l,j_2}\right)\\    
             &\Rightarrow^{\rho^{-1}} \sum_k A^1_{i_1,k}A^2_{k,i_2}
\left(\sum_lB^1_{j_1,l}B^2_{l,j_2}\right)\\
             &\Rightarrow^{\sum_k\lambda^{-1}} 
	\sum_{(k,l)}A^1_{i_1,k}A^2_{k,i_2}B^1_{j_1,l}B^2_{l,j_2}\\
             &\Rightarrow^{\sum_{(k,l)}id\cdot c^{\R}_\otimes\cdot id} 
\sum_{(k,l)} A^1_{i_1,k}B^1_{j_1,l}A^2_{k,i_2}B^2_{l,j_2}\\
             &= \sum_{(k,l)}(A^1\boxtimes B^1)_{\omega(i_1,j_1),\omega(k,l)}
(A^2\boxtimes B^2)_{\omega(k,l),\omega(i_2,j_2)}\\
             &= (A^1\boxtimes B^1)
\circ(A^2\boxtimes B^2)_{\omega(i_1,j_1),\omega(i_2,j_2)}.\end{aligned}\]
So define $\Phi^\boxtimes := (id\cdot c_\otimes\cdot id)\circ\lambda^{
-1}\circ \rho^{-1}$ in the manner described above for each component 
(with summations suppressed because of the appropriate coherences in 
the coefficient category). It is natural, because the involved 
morphisms are natural in $\R$. It is obvious if either $A^1A^2=id$ or 
$B^1B^2=id$ then the involved natural isomorphisms are forced to be 
identities, hence follow the strict identities claimed above.

To see that $\Phi$ is associative I refer the reader to
\cite{Lap}: Given a morphism \[((A^1\circ A^2)\circ A^3)\boxtimes 
((B^1\circ B^2)\circ B^3) \Rightarrow (A^1\circ (A^2\circ A^3))
\boxtimes (B^1\circ (B^2\circ B^3))\] comprised only of structural 
(iso)morphisms of the bipermutative category $\R$ there is a unique 
structural morphism between the given source and target. Since the 
associator of $\M$ is given by structural morphisms of $\R$ and the 
compositor $\Phi^\boxtimes$ of $\boxtimes$ is given by structural 
morphisms of $\R$ as well, this gives that $\Phi^\boxtimes$ is 
associative in the appropriate manner (cf. \cite[p. 4]{Lei}).
\end{proof}}}

{\rem{Since this is the first proof of this type let me emphasise that 
	it is sufficient to consider the compatibilities on $1$-cells, because
	the $2$-cells are any type of $n\times n$-matrix with no additional
	condition. So the calculations on $1$-cells are ``always'' strictly
	natural with respect to $2$-cells.}}

{\rem{The additional strict identities show that the compositor of 
$\boxtimes$ is a result of the natural isomorphism: 
$(id\boxtimes B)(A\boxtimes id)\Rightarrow 
(A\boxtimes id)(id\boxtimes B) = A\boxtimes B.$}}

{\rem{Consistently with the Deligne conjecture for Algebraic 
$K$-theory we see that we need at least a braiding (i.e., an 
$E_2$-structure) on the coefficient category $\R$ to define 
an $E_1=A_\infty$-multiplication on its module category. 
Cf. for instance \cite[Example 3.9]{Bar} and 
\cite[Remarks after C.6.3.5.17]{Lu2}. For a more thorough
survey of the Deligne conjecture on Hochschild cohomology as well as
a survey of its proofs see Section 16 of \cite{MSm}.}}

{\lem{The functor $\boxtimes$ is strictly associative. \begin{proof}
For this proof I fix the specific associative bijections from the beginning
of this section. For $A\in GL_n\R, B\in GL_m\R, C\in GL_l \R$ we have:
\[((A\boxtimes B)\boxtimes C)_{(i_1-1)ml+(j_1-1)l+k_1,
(i_2-1)ml+(j_2-1)l+k_2}\]
\[\begin{aligned}
&= (A\boxtimes B)_{(i_1-1)m+j_1,(i_2-1)m+j_2}\cdot C_{k_1,k_2} \\
&=  A_{i_1,i_2}\cdot B_{j_1,j_2}\cdot C_{k_1,k_2}\\
&=  A_{i_1,i_2}\cdot (B\boxtimes C)_{(j_1-1)l+k_1,(j_2-1)l+k_2}\\
&= (A\boxtimes (B\boxtimes C))_{(i_1-1)ml + (j_1-1)l + k_1,
(i_2-1)ml + (j_2-1)l +k_2}. \end{aligned}\]\end{proof}}}

{\lem{The object $1$ with its identities is a strict unit for 
$\boxtimes$. \begin{proof} We have:
\[(A\boxtimes 1)_{(i_1-1)1+j_1,(i_2-1)1+j_2} = A_{i_1,i_2}\]
for $i_1,i_2=1,\ldots, |A|$ and $j_1=j_2=1$, analogously 
$1\boxtimes A = A$.\end{proof}}}

The following statement can also be thought of as a convention: Just 
as the empty matrix is a strictly neutral element for 
$\boxplus$, it is a strict zero for $\boxtimes$.

{\lem{The object $0$ with its identity considered as the empty matrix 
(of objects and morphisms respectively) is a strict zero for 
$\boxtimes$.}}

I needed some commutativity to show that $\boxtimes$ is a 
functor, it should be much less surprising that it is necessary for
commutativity of $\boxtimes$.

{\lem{The bicategory of matrices $\M$ over a bipermutative coefficient category
$\R$ is symmetric monoidal with respect to $\boxtimes$. \begin{proof}
At this point I borrow the bipermutative structure from $\Sigma_*$ 
(cf. \ref{symmMatr}), let $A\in GL_n\R$, $B\in GL_m \R$, and consider:
\[(c_{n,m}(A\boxtimes B))_{(i_1-1)n + j_1,(i_2-1)m+j_2}
= (A\boxtimes B)_{(j_1-1)m + i_1,(i_2-1)m+j_2} 
= A_{j_1,i_2}B_{i_1,j_2},\]
\[((B\boxtimes A)c_{n,m})_{(i_1-1)n + j_1,(i_2-1)m+j_2}
= (B\boxtimes A)_{(i_1-1)n+j_1,(j_2-1)n+i_2} 
= B_{i_1,j_2}A_{j_1,i_2},\]
these can obviously be transformed into each other by the 
multiplicative twist of $\R$, so the symmetry has as one-cells 
$c_{n,m}\colon nm\rightarrow mn$ and two-cells 
$(C^\boxtimes)_{ij} = c^\R_\cdot~~\forall i,j$.\end{proof}}}

{\ex{Consider this again on $2\times 2$-matrices, i.e., a diagram:
\[\xymatrix{2\cdot 2\drtwocell<\omit>{~c^\R}\ar[r]^{A\boxtimes B} \ar[d]_{c_{2,2}}
	&2\cdot 2\ar[d]^{c_{2,2}}\\
2\cdot 2\ar[r]_{B\boxtimes A} & 2\cdot 2.}\]
Use the identification $c_{2,2}=(23)$ to calculate:
\[\begin{aligned}
(B\boxtimes A)E_{(23)}
&=\left({\begin{array}{cccc}
B_{11}A_{11} & B_{11}A_{12} &B_{12}A_{11} &B_{12}A_{12} \\
B_{11}A_{21} & B_{11}A_{22} &B_{12}A_{21} &B_{12}A_{22} \\
B_{21}A_{11} & B_{21}A_{12} &B_{22}A_{11} &B_{22}A_{12} \\
B_{21}A_{21} & B_{21}A_{22} &B_{22}A_{21} &B_{22}A_{22} \\
\end{array}}\right)\left({\begin{array}{cccc}
1&&&\\
&&1&\\
&1&&\\
&&&1
\end{array}}\right)\\
&=\left({\begin{array}{cccc}
B_{11}A_{11} &B_{12}A_{11} & B_{11}A_{12} &B_{12}A_{12} \\
B_{11}A_{21} &B_{12}A_{21} & B_{11}A_{22} &B_{12}A_{22} \\
B_{21}A_{11} &B_{22}A_{11} & B_{21}A_{12} &B_{22}A_{12} \\
B_{21}A_{21} &B_{22}A_{21} & B_{21}A_{22} &B_{22}A_{22} \\
\end{array}}\right)
\end{aligned}\]
and the other side:
\[\begin{aligned}
E_{(23)}(A\boxtimes B)
&=\left({\begin{array}{cccc}
1&&&\\
&&1&\\
&1&&\\
&&&1
\end{array}}\right)\left({\begin{array}{cccc}
A_{11}B_{11} & A_{11}B_{12} &A_{12}B_{11} &A_{12}B_{12} \\
A_{11}B_{21} & A_{11}B_{22} &A_{12}B_{21} &A_{12}B_{22} \\
A_{21}B_{11} & A_{21}B_{12} &A_{22}B_{11} &A_{22}B_{12} \\
A_{21}B_{21} & A_{21}B_{22} &A_{22}B_{21} &A_{22}B_{22} \\
\end{array}}\right)\\
&=\left({\begin{array}{cccc}
A_{11}B_{11} & A_{11}B_{12} &A_{12}B_{11} &A_{12}B_{12} \\
A_{21}B_{11} & A_{21}B_{12} &A_{22}B_{11} &A_{22}B_{12} \\
A_{11}B_{21} & A_{11}B_{22} &A_{12}B_{21} &A_{12}B_{22} \\
A_{21}B_{21} & A_{21}B_{22} &A_{22}B_{21} &A_{22}B_{22} \\
\end{array}}\right),
\end{aligned}\]
thus we have the $2$-cell given by the multiplicative twist of $\R$ in
each component:
\[\xymatrix{\left({\begin{array}{cccc}
B_{11}A_{11} &B_{12}A_{11} & B_{11}A_{12} &B_{12}A_{12} \\
B_{11}A_{21} &B_{12}A_{21} & B_{11}A_{22} &B_{12}A_{22} \\
B_{21}A_{11} &B_{22}A_{11} & B_{21}A_{12} &B_{22}A_{12} \\
B_{21}A_{21} &B_{22}A_{21} & B_{21}A_{22} &B_{22}A_{22} \\
\end{array}}\right)\ar[d]_{c^\R}\\
 \left({\begin{array}{cccc}
A_{11}B_{11} & A_{11}B_{12} &A_{12}B_{11} &A_{12}B_{12} \\
A_{21}B_{11} & A_{21}B_{12} &A_{22}B_{11} &A_{22}B_{12} \\
A_{11}B_{21} & A_{11}B_{22} &A_{12}B_{21} &A_{12}B_{22} \\
A_{21}B_{21} & A_{21}B_{22} &A_{22}B_{21} &A_{22}B_{22} \\
\end{array}}\right).}\]}}

Let me summarise these results into one big lemma:
{\lem{The module bicategory $\M$ of a bipermutative category $\R$ is 
strictly symmetric monoidal with respect to the tensor product of 
matrices $\boxtimes$, i.e., we have:
\begin{itemize}
\item $(\boxtimes,\Phi^\boxtimes)$ is a pseudofunctor:
\[\boxtimes\colon \M\times\M\rightarrow\M,\]
\item $\boxtimes$ is strictly associative, i.e., 
\[\boxtimes\circ(\boxtimes\times id) 
= \boxtimes\circ (id\times\boxtimes),\]
\item $\boxtimes$ has a strict unit $1$, i.e.,
\[\boxtimes\circ (id\times 1) 
= \boxtimes \circ (1\times id) = id_{\M},\]
\item $\boxtimes$ has a strong symmetry transformation 
$(c^\Sigma, c^\R_\cdot)$.
\end{itemize}

Additionally the symmetry satisfies the following coherences strictly:
\[\begin{array}{cc}
{\xymatrix{l\cdot m\cdot n \ar[dr]_{c^\Sigma_{l,mn}} 
\ar[r]^{c^\Sigma_{l,m}\boxtimes 1_n}
 & m \cdot l\cdot n \ar[d]^{1_m\boxtimes c^\Sigma_{l,n}}\\
 & m\cdot n\cdot l}}
&
{\xymatrix{l\cdot m\cdot n \ar[dr]_{c^\Sigma_{lm,n}} 
\ar[r]^{1_l\boxtimes c^\Sigma_{m,n}}
 & l\cdot n\cdot m \ar[d]^{c^\Sigma_{n,l}\boxtimes 1_m}\\
 & n\cdot l\cdot m.}}\end{array}\]
Furthermore the symmetry is its own inverse:
\[{\xymatrix{
n\cdot m\ar@{=}[rr] \ar[dr]_{c^\Sigma_{n,m}} && n\cdot m \\
 & m\cdot n.\ar[ur]_{c^\Sigma_{m,n}}}}\]

\begin{proof} Each of the properties that do not follow from the 
previous lemmas is just promoted to $\M$ from $\Sigma_*$ by the 
functor $E_\bullet$, so there is nothing new to prove.\end{proof}}}

In summary I have proved that $(\M, \boxtimes, 1, c_\cdot)$
is a strict symmetric monoidal bicategory in the sense also used by 
\cite[Definition 3.1]{Os}. Since the tensor product of matrices satisfies
these strict axioms, it is sufficient for me to consider this type of 
symmetric monoidal bicategory, although it is very probable that 
this class does not cover all equivalence classes of the most general
type of bicategories with a symmetric monoidal structure one could 
devise.

With the tensor-product in place I can state the second strict 
monoidality the functor $E$ satisfies, which quite trivially follows
from the fact that I chose the same bijection for the tensor-product
as I did for the product in $\Sigma_*$.
{\prop{For each bipermutative coefficient category $\R$ the inclusion
\[E\colon \Sigma_* \rightarrow \M\] is strictly symmetric monoidal 
with respect to $\times$ on $\Sigma_*$ and $\boxtimes$ on $\M$.}}

{\rem{With the symmetric monoidal structures on $\M$ settled the remark
	that everything works enriched as well is obligatory. The calculations
	before extend to $2$-cells, because they are defined merely as part of
	the appropriate product-categories with no additional conditions, thus
	reordering $2$-cells as indicated by the $1$-cells is compatible with
	the enrichment.}}

\section{The Bimonoidal Structure on $\M$}
Osorno has proved that $(\M,\boxplus, 0, c_+)$ is a permutative
bicategory (see Theorem \ref{OsPlus}), and in Section \ref{Mult} I elaborate on the 
fact that $(\M,\boxtimes, 1, c_\cdot)$ is a second symmetric monoidal
bicategory structure on $\M$. One would want these to 
interact in a manner analogous to bipermutative $1$-categories. 
This section is devoted to making the analogy precise, and establishing
$\M$ as a bipermutative bicategory.

In \ref{Fin} the choice of a strictly associative functor representing
the product made the left-distributor strict in $\Sigma_*$, i.e., we 
have $\lambda=id$. Here I used the same bijection that fixes this for 
the tensor-product structure in \ref{tensor}. It should be intuitive 
that this makes $E_\bullet$ into a well-behaved bipermutative functor. I
elaborate on that after the appropriate definition for bicategories.

The distributors of the bipermutative structure on $\Sigma_*$
promote to natural transformations in $\M$ without using two-cells.

{\prop{We have strict equalities of one-cells for $A\in GL_n\R$, 
$B\in GL_m\R$, $C\in GL_l\R$: \[(A\boxplus B)\boxtimes C = A\boxtimes C
\boxplus A\boxtimes C\] and \[A\boxtimes(B\boxplus C)c^\Sigma_{m+l,n}
(c^\Sigma_{n,m}\boxplus c^\Sigma_{n,l}) = c^\Sigma_{m+l,n}(
c^\Sigma_{n,m}\boxplus c^\Sigma_{n,l})((A\boxtimes B) \boxplus 
(A\boxtimes C)).\] \begin{proof} I only comment on the strictness, 
which is a result of the fact that the multiplicative twist enters 
twice as a two-cell, hence cancels out.\end{proof}}}

{\ex{Let me elaborate on $l=n=2, m=1$, so we get:
\[c_{m+l,n}=c_{3,2},\] which is \[c_{3,2}((i-1)3+j) = (j-1)2+i\]
i.e., $c_{3,2}$ is the cycle $(2453)$. We have\[c_{2,1}+c_{2,2}=id+
(23)=(45),\]hence\[c_{3,2}(c_{2,1}+c_{2,2})=(2453)(45)=(432).\] 

On one side we find:\[\begin{aligned}&A\boxtimes(B\boxplus C)E_{(432)}
\\&={\left(\begin{array}{cccccc}A_{11}b &&&A_{12}b&&\\
& A_{11}C_{11}&A_{11}C_{12}&&A_{12}C_{11}&A_{12}C_{12}\\
& A_{11}C_{21}&A_{11}C_{22}&&A_{12}C_{21}&A_{12}C_{22}\\
A_{21}b &&&A_{22}b&&\\
& A_{21}C_{11}&A_{21}C_{12}&&A_{22}C_{11}&A_{22}C_{12}\\
& A_{21}C_{21}&A_{21}C_{22}&&A_{22}C_{21}&A_{22}C_{22}\\\end{array}
\right)}{\left(\begin{array}{cccccc}
1&&&&&\\&&1&&&\\&&&1&&\\&1&&&&\\&&&&1&\\&&&&&1\end{array}\right)}\\
&={\left(\begin{array}{cccccc}A_{11}b &A_{12}b&&&&\\
&& A_{11}C_{11}&A_{11}C_{12}&A_{12}C_{11}&A_{12}C_{12}\\
&& A_{11}C_{21}&A_{11}C_{22}&A_{12}C_{21}&A_{12}C_{22}\\
A_{21}b &A_{22}b&&&&\\
&& A_{21}C_{11}&A_{21}C_{12}&A_{22}C_{11}&A_{22}C_{12}\\
&& A_{21}C_{21}&A_{21}C_{22}&A_{22}C_{21}&A_{22}C_{22}\\
\end{array}\right).}\end{aligned}\] On the other side we have:
\[\begin{aligned} &E_{(432)}(A\boxtimes B\boxplus A\boxtimes C)\\
&={\left(\begin{array}{cccccc}1&&&&&\\&&1&&&\\&&&1&&\\&1&&&&\\&&&&1&\\
&&&&&1\end{array}\right)}{\left(\begin{array}{cccccc}
A_{11}b &A_{12}b&&&&\\ 
A_{21}b &A_{22}b&&&&\\
&& A_{11}C_{11}&A_{11}C_{12}&A_{12}C_{11}&A_{12}C_{12}\\
&& A_{11}C_{21}&A_{11}C_{22}&A_{12}C_{21}&A_{12}C_{22}\\
&& A_{21}C_{11}&A_{21}C_{12}&A_{22}C_{11}&A_{22}C_{12}\\
&& A_{21}C_{21}&A_{21}C_{22}&A_{22}C_{21}&A_{22}C_{22}\\
\end{array}\right)}\\&={\left(\begin{array}{cccccc}A_{11}b&A_{12}b&&&&
\\&& A_{11}C_{11}&A_{11}C_{12}&A_{12}C_{11}&A_{12}C_{12}\\
&& A_{11}C_{21}&A_{11}C_{22}&A_{12}C_{21}&A_{12}C_{22}\\
A_{21}b &A_{22}b&&&&\\
&& A_{21}C_{11}&A_{21}C_{12}&A_{22}C_{11}&A_{22}C_{12}\\
&& A_{21}C_{21}&A_{21}C_{22}&A_{22}C_{21}&A_{22}C_{22}\\\end{array}
\right)}=A\boxtimes(B\boxplus C)E_{(432)}.\end{aligned}\]}}

Because of this strictness I define what a bipermutative bicategory is
in close analogy with $1$-categories.

{\defn{\label{bipbic}A \textbf{bipermutative bicategory} $\R$ is a bicategory 
(with strict identities) with two monoidal structures $\boxplus, 
\boxtimes$, an additive symmetry $c_\boxplus$, making $(\R,\boxplus)$ 
into a permutative bicategory (Definition \ref{permbic}), a multiplicative symmetry 
$c_\boxtimes$, making $(\R,\boxtimes)$ into a symmetric monoidal 
bicategory (Definition \ref{mdlbic}), and strictly natural distributivity 
isomorphisms (strictly invertible 1-cells):
\[\lambda\colon a\boxtimes b\boxplus a\boxtimes b' \rightarrow 
a\boxtimes (b\boxplus b'),\] \[\rho\colon a\boxtimes b\boxplus 
a'\boxtimes b \rightarrow (a\boxplus a')\boxtimes b,\]
satisfying the following strict identities of 1-cells:
\begin{enumerate} \item strict zero: \[0\boxtimes a=a\boxtimes 0 =0~~
\forall a\in\R,\] \item $\boxplus$-associativity of distributors:
\[\lambda(\lambda\boxplus id) = \lambda(id\boxplus \lambda),\]
\[\rho(\rho\boxplus id) = \rho(id\boxplus \rho),\]
\item additive symmetry of distributors:
\[(c_\boxplus\boxtimes id)\lambda = \lambda\circ c_\boxplus,\]
\[(id\boxtimes c_+) \rho = \rho c_+,\]
\item $\boxtimes$-associativity of distributors:
\[\lambda = \lambda \circ (\lambda \boxtimes id),\]
\[\rho = \rho\circ (id\boxtimes \rho),\]
\item middle associativity of distributors:
\[\lambda\circ (id\boxtimes \rho)=\rho\circ (\lambda\boxtimes id),\]
\item mixed associativity of distributors:
\[\lambda(\rho\boxplus \rho) =\rho (\lambda\boxplus\lambda)
 (1\boxtimes c_\boxplus \boxtimes 1),\]
\item multiplicative symmetry of distributors:
\[c_\boxtimes \circ \lambda = \rho \circ 
	(c_\boxtimes \boxplus c_\boxtimes).\]\end{enumerate}}}

{\rem{Let me emphasise that I have modelled this definition of 
bipermutative bicategory in such a way that the only thing left to 
show given the two symmetric monoidal structures $\boxplus$ and 
$\boxtimes$ on $\M$ is: There are distributors $\lambda$ and 
$\rho$, they are strict natural transformations and they
satisfy the coherences above. There is no additional data in the form
of coherence $2$-cells involved.}}

{\defn{Define the distributivity $1$-cells for $\M$ as follows:
\[\lambda:=\id=E_\lambda,~~\mathrm{and~~} \rho:=E_{\rho^\Sigma}
=c^\Sigma_{m+l,n}(c^\Sigma_{n,m}\boxplus c^\Sigma_{n,l}),\] with 
identities as 2-cells.}}

With these distributivity $1$-cells I can easily prove the 
following theorem, which I use to summarise all explicit details about
the bipermutative structure of $\M$, because most of it is part of the
lemmas already proven above.

{\thm{For $\R$ a bipermutative $1$-category (possibly enriched over the
symmetric monoidal categories $\mathit{Top}, \mathit{Cat}, 
\mathit{sSet}$), the following is a bipermutative bicategory $\M$ 
(with $2$-cells in the same enrichment):\begin{itemize}
\item $Ob\M = \N_0,$ \item $\M(n,m)=\begin{cases}GL_n\R, &~n=m,\\ 
                     		    \emptyset, & ~n\neq m,\end{cases}$
\item $(A\boxplus B)_{i,j}=\begin{cases}A_{i,j}, &~1\leq i,j\leq |A|,\\ 
                       B_{i-|A|,j-|A|}, & ~1\leq i-|A|,j-|A|\leq |B|,\\
                                        0,\end{cases}$
\item $\boxplus$ is a strict normal \ref{strictn} functor, 
i.e., $(A_1A_2\boxplus B_1B_2)=(A_1\boxplus B_1)(A_2\boxplus B_2)$
and $\id_n\boxplus \id_m = \id_{n+m},$
\item $(A\boxtimes B)_{(i_1-1)|B|+j_1,(i_2-1)|B|+j_2} 
      := A_{i_1,i_2}B_{j_1,j_2},$
\item $\boxtimes$ is a pseudofunctor, i.e., 
$(\id_n\boxtimes \id_m) = \id_{nm}$ and there is a natural isomorphism 
$2$-cell $(A_1\boxtimes B_1)(A_2\boxtimes B_2) \Rightarrow 
(A_1A_2)\boxtimes (B_1B_2)$ given by the adequate composition of 
(both) $\R$-distributors and its multiplicative symmetry $c^\R$ 
(cf. \ref{tensoristFunktor}), \item the matrix $E_{c^+}$ for the 
additive twist in $\Sigma_*$: \[c^+_{n,m}(i)=\begin{cases} 
i+m, &~i\leq n,\\i-n, &~i\geq n+1, \end{cases}\] yields the additive 
twist with $(C^+)_{i,j}=\delta_{i,c^+_{n,m}(j)},$ which is a strict
natural transformation $\boxplus \circ tw \Rightarrow \boxplus$, i.e., 
a pseudonatural transformation with coherence $2$-cells identities,
\item the bijections $c_{n,m}((i-1)m+j) = i + (j-1)n$ yield the 
multiplicative twist with $C_{i,j}=\delta_{i,c_{n,m}(j)}$, which is 
a strong pseudonatural transformation with $2$-cell given by the 
$c^\R$ in each component.\end{itemize} the distributors are given as:
\begin{itemize} \item $\lambda = \id\colon nm+nl \rightarrow n(m+l),$
and \item $\rho = c_{n,m+l}(c_{m,n}+c_{l,n})\colon mn+ln \rightarrow 
(m+l)n,$\end{itemize} and satisfy the coherences of \ref{bipbic}.
\begin{proof} The only thing left to prove is the fact that the 
distributors satisfy the claimed coherences. For that consider the 
functor \[E\colon \Sigma_*\rightarrow \M\] again. I already 
established that it is strictly symmetric monoidal with respect to $\boxplus$,
but given $\boxtimes$ as in \ref{tensor} and $\times$ as in \ref{Fin}
it is obvious that $E$ is also strictly symmetric monoidal with respect to 
these structures. Take particular note that the coherence $2$-cell of
$c_\boxtimes$ does not feature here because the multiplicative 
symmetry of $\R$ is forced to be the identity for the product 
$0\cdot 0 = 0\cdot 1=0$ and $1\cdot 1$ by the axioms of symmetric 
monoidal categories for the first and third case and the additional 
zero-axiom for bipermutative categories.

The distributors are defined as part of the image of $E$ explicitly,
so obviously we have $E_\lambda = \lambda$ and $E_\rho=\rho$, but $E$
is a strict functor of bicategories (and even $2$-categories, if we
restrict our attention to its image), so all coherences these 
distributors satisfy in $\Sigma_*$ directly promote to $\M$ for 
arbitrary bipermutative coefficients $\R$. 

In particular, $E$ is a strict functor of bipermutative bicategories, 
which is strictly additive, strictly multiplicative, and strictly
satisfies $E(c^+)=C^\boxplus, E(c^\cdot)=C^\boxtimes$ as well as
$E\lambda=\lambda, E\rho=\rho$.\end{proof}}}

{\rem{Let me emphasise that I can get away with such a strict structure, because
	the $2$-cells in $\M$ are just parts of the appropriate product categories
	with no additional compatibility condition among them. If one were to impose
	``weak invertibility'' on the matrices of $2$-cells for instance, I do not
	know, if we still get such a strict bipermutative structure.}}

\section{Transposition and Involutions} In commutative rings we are 
well aware of the formula: \[(AB)^t = B^tA^t.\] In preparation 
for involutions on module bicategories I want to isolate
how this formula behaves with genuine bipermutative 
categories as coefficients.

{\defn{For any bicategory $\C$ consider the $1$-opposed bicategory 
$\C^{op_1}$, which has the same objects, $1$-cells, $2$-cells, but 
opposed composition of $1$-cells, which I denote by $\circ$, while I 
do not denote the composition of $1$-cells in $\C$, just as usual for ordinary 
matrix multiplication. The associator is then the inverse of the 
original associator: \[\xymatrix{A\circ (B\circ C) = (CB)A 
\ar[r]^{\alpha^{-1}} &C(BA) = (A\circ B)\circ C.}\] For a bimonoidal
$1$-category $\R$ we also consider the $\mu$-opposed category 
$\R^\mu$ with the same objects and morphisms, opposed multiplication,
and hence exchanged distributors.}}

{\prop{\label{transstrfun}
For bimonoidal coefficients transposition is a strict normal \ref{strictn}
functor:\[(\cdot)^t\colon \MM(\R^\mu) \rightarrow \MM(\R)^{op_1}.\]
\begin{proof}We calculate on $1$-cells: \[\begin{aligned}
(A^t\circ_1 B^t)_{ij} &= (B^tA^t)_{ij} \\
&= \sum_k B^t_{ik}A^t_{kj}\\&= \sum_k A^t_{kj}\circ B^t_{ik}\\
&= \sum_k A_{jk}\circ B_{ki} = (AB)_{ji} = (AB)^t_{ij}. \end{aligned}\]
So transposition strictly respects composition of $1$-cells and
strictly respects identities.\end{proof}}}

{\rem{For bipermutative coefficients we could use the multiplicative twist
to suppress the $\mu$-opposition. However to make the book keeping of oppositions
more transparent I do not use that.}}

I want to consider involutions on the coefficient category $\R$ as 
considered by Richter in \cite[Definition 3.1]{Ri2010}.

{\defn{An anti-involution on a bipermutative category $\R$ is given by
a self-inverse strictly symmetric monoidal functor 
$T\colon (\R,+)\rightarrow (\R,+)$ with respect to $(\R,+,0,c^\R_+)$ together
with a natural isomorphism: \[t\colon T(a)T(b)\rightarrow T(ba).\]
Satisfying:\begin{itemize} \item $(T,t)$ strictly respects the unit, 
i.e., $T(1) = 1$ and \[t=id\colon T(a)1=1T(a)=T(1)T(a) \rightarrow 
T(a),\] \item $t$ is associative with respect to multiplication
\[\xymatrix{ T(a)T(b)T(c) \ar[r]^{t T(c)}\ar[d]^{T(a)t} & T(ba)T(c) 
\ar[d]^t\\ T(a)T(cb) \ar[r]^t & T(cba).}\] \item $(T,t)$ is symmetric 
with respect to multiplication: \[\xymatrix{ T(a) T(b) \ar[r]^t \ar[d]^c 
& T(ba) \ar[d]^{T(c)}\\ T(b) T(a) \ar[r]^t & T(ab).}\] \item the 
involution commutes with the distributors: \[\xymatrix{
T(a)T(b)+T(a)T(c)\ar[r]^-{\lambda} \ar[d]^{t+t} & T(a)T(b+c)\ar[d]^t\\
T(ba)+T(ca) \ar[r]^{T(\rho)} & T((b+c)a),}\]and\[\xymatrix{
T(a)T(c)+T(b)T(c)\ar[r]^-{\rho} \ar[d]^{t+t} & T(a+b)T(c)\ar[d]^t\\
T(ca)+T(cb) \ar[r]^{T(\lambda)} & T(c(a+b)).}\]\end{itemize}}}

{\rem{It is quite obvious that in the bipermutative case, the way I 
consider it in this thesis, one of the compatibilities with 
distributors implies the other, but the exposition is more transparent
this way.}}

{\rem{Richter proceeds in \cite{Ri2010} to define an induced 
involution on the bar construction of the monoidal categories 
$GL_n\R$. I define this involution as induced on matrix
bicategories.}}

{\prop{Let $(F,\varphi)\colon \mathcal{R}\rightarrow \mathcal{A}$ be a
strictly additive functor of strictly bimonoidal categories, i.e.
\[F(0)=0, F(r+s)=F(r)+F(s),\] furthermore let $F$ be strictly unital 
$F(1)=1$, then \begin{itemize} \item a lax transformation $\varphi\colon F(a)F(b)
\rightarrow F(ab)$ promotes to a lax normal 
functor \[\mathcal{M}F\colon \mathcal{M}(\mathcal{R})\rightarrow 
\mathcal{M}(\mathcal{A}),\] \item if furthermore 
$\varphi\colon F(a)F(b)\rightarrow F(ab)$ is a natural isomorphism, so
$F$ is strongly multiplicative, then $(F,\varphi)$ 
promotes to a pseudofunctor \[\mathcal{M}F\colon 
\mathcal{M}(\mathcal{R})\rightarrow \mathcal{M}(\mathcal{A}).\]
\end{itemize} \begin{proof} Again the interesting point is what 
happens on $1$-cells: \[\begin{aligned} 
(\mathcal{M}FA\cdot\mathcal{M}FB)_{ij}&=\sum_k FA_{ik}\cdot FB_{kj}\\
&\Rightarrow^\varphi \sum_k F(A_{ik}B_{kj})=F(\sum_k A_{ik}B_{kj})\\
&= F(AB_{ij}) = \mathcal{M}F(AB)_{ij},\end{aligned}\] obviously the 
functor $(\mathcal{M}F,\varphi)$ then is just as good as the 
constraint $\varphi$ of $F$.\end{proof}}}

{\lem{An anti-involution on a bimonoidal category $\R$ is a strictly
additive, strongly multiplicative functor from $\R$ to 
its multiplicative opposition $\R^\mu$
\[T\colon \R\rightarrow \R^\mu.\] Consequently an anti-involution 
induces a pseudofunctor of module bicategories:
\[\mathcal{M}T\colon \M \rightarrow \mathcal{M}(\R^\mu).\]
$\hfill \Box$}}

This is as far as I can come in the bicategory setting without 
appealing to classifying spaces, so let me summarise what the 
involution induces on module bicategories. \label{invStop}

{\lem{Composing transposition and an anti-involution on coefficients 
gives a pseudofunctor \[\mathcal{M}T\circ (\cdot )^t\colon 
\M \rightarrow \M^{op_1}.\]}}

\subsection{Involution and Tensor-products}\label{invundtensor}
One aim of this chapter on module bicategories is to get a 
combinatorial insight on how the involution on the coefficient 
category and the $E_\infty$-structure on its module bicategory 
interact. Fortunately this is easily described on the level of 
bipermutative bicategories.

It is obvious that the induced involution strictly respects direct
sum:
{\lem{For a bimonoidal category $\R$ with involution $(T,t)$ the 
induced involution on module bicategories is strictly additive and 
symmetric:
\[\mathcal{M}T(A\boxplus B) = \mathcal{M}TA\boxplus \mathcal{M}TB,\]
and \[\mathcal{M}T(c^+_{m,n})=c^+_{m,n}.\]}}

The tensor product structure, if defined, is also easily seen to be
compatible with the coordinatewise involution:

{\lem{For a bipermutative category $\R$ with involution $(T,t)$ we 
have a strictly natural isomorphism of functors:
\[t\colon \boxtimes \circ{\mathcal{M}T\times\mathcal{M}T}
 \Rightarrow \mathcal{M}T \circ\boxtimes,\]
each considered as functors $\M\times\M\rightarrow 
\mathcal{M}(\R^\mu)$.
\begin{proof}
This is a simple calculation, again consider $1$-cells:
\[\begin{aligned}
(\mathcal{M}T(A)\boxtimes \mathcal{M}T(B))_{(i_1-1)|B|+j_1,
	(i_2-1)|B|+j_2}
&=\mathcal{M}T(A)_{i_1,i_2}\circ \mathcal{M}T(B)_{j_1,j_2}\\
&=T(A_{i_1,i_2})\circ T(B_{j_1,j_2})\\
&\Rightarrow^t T(A_{i_1,i_2}B_{j_1,j_2})\\
&=\mathcal{M}T((A\boxtimes B)_{(i_1-1)|B|+j_1,(i_2-1)|B|+j_2})\\
&=\mathcal{M}T(A\boxtimes B)_{(i_1-1)|B|+j_1,(i_2-1)|B|+j_2}.
\end{aligned}\]\end{proof}}}

So we can summarise: {\thm{For a bipermutative category $\R$ with 
involution $(T,t)$ the coordinatewise involution on the 
bicategory of matrices $\M$ induces a strong bipermutative functor:
\[\mathcal{M}T\colon (\M,\boxplus,\boxtimes)\rightarrow 
	(\mathcal{M}(\R^\mu),\boxplus,\boxtimes),\]
in the sense that it is strictly additive, and strongly multiplicative
with respect to $\boxtimes$. \begin{proof} I only need to elaborate on
the multiplicative symmetry of $\mathcal{M}T$, which is a consequence 
of the compatibility on coefficients: \[\xymatrix{T(a)T(b) \ar[r]^t 
\ar[d]^c  & T(ba) \ar[d]^{Tc}\\T(b)T(a) \ar[r]^t & T(ab).}\]
\end{proof}}}

In particular the fact that $\mathcal{M}T$ is strictly additive with respect to
$\boxplus$ implies that it induces a map of $\Gamma$-spaces, i.e., an
infinite loop map on classifying spaces as follows:
\[B\mathcal{M}T\colon B\M\rightarrow B\mathcal{M}(R^\mu),\]
cf. \ref{plusdeloop}. In what follows I want to define an internal 
involution on $B\M$, and refine Osorno's delooping of Theorem \ref{plusdeloop} 
to one that allows us to induce a multiplicative structure more easily.
Thus as the last compatibility, which is directly visible on the
level of bicategories of matrices, we see that transposition and
tensor-product strictly commute.

{\prop{\label{transmultopp}
	For any bipermutative category $\R$ transposition induces a strictly
	additive and strictly multiplicative strict normal \ref{strictn} functor on its
	bicategory of matrices $\M$:
	\[(\cdot)^t\colon (\MM(\R^\mu),\boxplus,\boxtimes^\mu)\rightarrow 
	(\MM(\R)^{op_1},\boxplus,\boxtimes^{op}),\]
	where we consider the opposite multiplication on $\M^{op_1}$, i.e.
	fully reversed $A\boxtimes^{op}B=B\boxtimes A.$
	\begin{proof}The fact that transposition is a strict normal functor is 
	Proposition	\ref{transstrfun}. The strict additivity is obvious as 
	$(A\boxplus B)^t = A^t\boxplus B^t$. For the multiplicativity consider 
	the following sequence of equations:
	\[\begin{aligned}(A\boxtimes^\mu B)^t_{\omega(i_1,i_2),\omega(j_1,j_2)} &=
	(A\boxtimes^\mu B)_{\omega(j_1,j_2),\omega(i_1,i_2)} \\
	&=A_{j_1,i_1}\circ B_{j_2,i_2}\\&=B^t_{i_2,j_2}A^t_{i_1,j_1}\\
	&=(B^t\boxtimes A^t)_{\omega(i_2,i_1),\omega(j_2,j_1)}
	&=(A^t\boxtimes^{op} B^t)_{\omega(i_1,i_2),\omega(j_1,j_2)}.\end{aligned}\]
	Thus transposition and the coordinatised tensor-product \ref{tensor}
	commute up to one exchange of factors, 
	yielding the claimed compatibility.\end{proof}}}

Thus we see that by strict additivity of transposition we get an
infinite loop map of classifying spaces as:
\[B\mathcal{M}T\circ (\cdot)^t\colon B\M\rightarrow B\mathcal{M}(R)^{op_1}.\]

\section{Basics on Nerves of Bicategories} On page 2 of \cite{CCG2010} 
one can see various constructions of nerves, thus
classifying spaces for bicategories, all homotopy equivalent after 
realisation. In previous versions of this thesis I considered the
``Segal Nerve'' as for instance in \cite[p.21, Definition 5.2]{CCG2010}. I
finally noticed that in this bisimplicial set associated to a 
bicategory one of the simplicial directions only
consists of homotopy equivalences \cite[Theorem 6.2.]{CCG2010}. Hence I can
restrict to one simplicial direction, simplifying the
exposition.

{\defn{The nerve of a bicategory $\C$ (with only isomorphism $2$-cells) 
is the simplicial set with $n$-simplices pseudofunctors: 
\[N\C_n:=\mathbf{NorHom}([n],\C).\]}}

Let me be more explicit about this, I consider the ordered set 
$[n]=\{0<1<\ldots<n-1<n\}$ as a $1$-category, which is a bicategory
with only identity $2$-cells. Then a pseudofunctor \[(F,\varphi)\colon
[n]\rightarrow \C\] is the same thing as a collection of objects $F_i 
\in Ob\C$, and for each pair $0\leq i<j\leq n$ a choice of $1$-cell 
$A_{i<j}\colon F_i\rightarrow F_j \in Ob\C(F_i,F_j)$ (where normality
corresponds to the fixed choice $A_{i\leq i}=id_{F_i}$), and for each
triple $i<j<k$ a $2$-cell $\varphi_{i<j<k} \colon A_{jk}A_{ij} 
\rightarrow A_{ik},$ assembling to the compositor $\varphi$ of $F$, 
which is hence associative in the appropriate sense.

Compare page 22 of \cite{LP2008}, where there is also a 
condition on identities I do not need, because I only consider normal 
functors.

So for a bicategory $\C$ (possibly enriched) we get a simplicial set:
\[N\C\colon \Delta^{op}\rightarrow Set.\]I want to elaborate 
on the simplicial operators, let \[\Phi\colon [n] \rightarrow 
[m]\] be a monotone map: The effect on an $n$-simplex $F\colon [n]
\rightarrow \C$ is then given as:\[\Phi^*F(i):= F\circ\Phi(i),
\] on $1$-cells we have: \[\Phi^*A_{ij} := A_{\Phi(i),\Phi(j)}\colon 
F_{\Phi(i)}\rightarrow F_{\Phi(j)},\] which is the identity, if 
$\Phi(i) = \Phi(j)$ according to the normality condition on $F$. 
Finally on compositors we get: \[\Phi^*\varphi_{i<j<k} := 
\varphi_{\Phi(i),\Phi(j),\Phi(k)},\] which is the identity if any two
of the three indices coincide. This is coherent, because I only 
consider bicategories with strict identity $1$-cells.

{\rem{The terminology varies, which is
partly due to the fact that there are at least $10$ reasonable ways
to define a nerve for bicategories (cf. the diagram 
\cite[p. 2]{CCG2010}). This particular construction is called the 
``unitary geometric nerve'' in \cite{CCG2010}, where more generally
all lax functors are considered. These coincide with pseudofunctors
for bicategories with just isomorphism $2$-cells. In particular the
warning after Theorem 6.5 in \cite{CCG2010} does not apply for 
bicategories with all $2$-cells isomorphisms.}}

We are used to the fact that natural transformations of functors on 
$1$-categories induce homotopies. For bicategories the same argument
yields that an arbitrary pseudonatural transformation induces a homotopy.
The observation is not original, but in the presence of $10$ different
nerve constructions I want to exhibit this fact specifically for the one
I use.
{\prop{A pseudonatural transformation $\eta$ of pseudofunctors \[F,G
	\colon \C\Rightarrow \D\] is equivalent to a pseudofunctor $\C\times 
	I\rightarrow \D,$ hence induces a map: \[N(\C\times I)\cong N\C\times 
	NI\rightarrow N\D.\] \begin{proof} This is plainly the universal 
	property of the product in bicategories, i.e., $Fun(\A,\C\times\D) =
	Fun(\A,\C)\times Fun(\A,\D),$ where $Fun$ can be any of the 
	classes of functors between bicategories. Thus in particular 
	for $\A=[n]$ and $Fun$ the class of normal pseudofunctors we get 
	the claimed natural isomorphism.\end{proof}}}

\subsection{Opposition of a Bicategory and its Nerve} As I alluded to
at the end of section \ref{invStop}, I want to induce an involution on
the nerve of a bicategory. For that I need one last preparation.

{\defn{Let the reversal functor \[r\colon \Delta \rightarrow \Delta\] 
be given as the identity on objects and on morphisms 
$\Phi\colon [n]\rightarrow [m]$ define: \[r(\Phi)(i) := m- \Phi(n-i).\]
Given a simplicial object in any category 
$X\colon \Delta^{op}\rightarrow \C$, set $X^{op} := X\circ r$, 
analogously for cosimplicial objects.}}

The special point special about $\mathit{Top}$ as a target category is the 
cosimplicial object that defines geometric realisation. (The analogous
isomorphism in chain complexes $Ch$ is given by only a sign depending
on the chain degree.)

{\lem{Let $\Delta^\bullet\colon \Delta\rightarrow \mathit{Top}$ be the 
cosimplicial space defined as usual:
\[\Delta^n = \{(t_0,\ldots,t_n)\in I^{n+1}|\sum t_i=1\},\]
with cosimplicial operators: \[\delta^i(t_0,\ldots,t_{n-1}) 
= (t_0,\ldots,t_{i-1},0,t_i,\ldots,t_{n-1})\] and 
\[\sigma^i(t_0,\ldots,t_n) 
= (t_0,\ldots,t_{i-1},t_i+t_{i+1},t_{i+2},\ldots,t_n).\]
Then we have an isomorphism of cosimplicial topological spaces:
\[\Gamma\colon \Delta^\bullet \rightarrow \Delta^\bullet\circ r.\]
\begin{proof} Define $\Gamma(t_0,\ldots,t_n) = (t_n,\ldots,t_0)$, this
is obviously a degreewise homeomorphism, and the identities:
\[\Gamma\circ \delta^i = \delta^{n-i}\circ\Gamma\] \[\Gamma\circ 
\sigma^i = \sigma^{n-i}\circ\Gamma\] give that $\Gamma$ is an 
isomorphism of cosimplicial objects.\end{proof}}}

{\lem{\label{simpoppTop}Let $X\colon \Delta^{op}\rightarrow \mathit{Top}$ be a 
simplicial space (in particular sets with discrete topology), then 
consider geometric realisation as a coend: \[X\otimes_\Delta 
\Delta^\bullet = |X|.\] Then we have the identity \[X\otimes 
(\Delta^\bullet\circ r) = (X\circ r)\otimes\Delta^\bullet,\] and hence a 
natural homeomorphism of realisations: \[X\otimes \Gamma\colon |X|
\rightarrow |X\circ r|.\] \begin{proof} The identity \[X\otimes 
(\Delta^\bullet\circ r) = (X\circ r)\otimes\Delta^\bullet\] can be seen 
as follows. Both objects are quotients of the object \[\coprod_n X_n
\times \Delta^n,\] because $r$ does not change anything on the 
simplices. The identification according to $X\otimes (\Delta^\bullet
\circ r)$ then is: \[[x,\delta^{n-i} t] = [d_ix,t],\] and in $(X\circ 
r)\otimes\Delta^\bullet$ it is: \[[d_{n-i}x,t] = [x,\delta^{i} t].\] 
So the idenfications are just listed in a different order, but the 
equivalence relation we divide out is the same.

As a consequence the natural homeomorphism spells out: \[\begin{aligned} 
X\otimes \Gamma\colon &|X|\rightarrow |X\circ r|\\ &[x,(t_0,\ldots,t_n)] 
\mapsto [x,(t_n,\ldots,t_0)].\end{aligned}\]\end{proof}}}

With the simplicial considerations in place I define the classifying
space of a bicategory as follows:

{\defn{Given a bicategory $\C$ consider its nerve, which is a simplicial 
set as defined before:\[N\C\colon \Delta^{op} \rightarrow Set.\]
The geometric realisation of this simplicial set then defines the 
classifying space of $\C$: \[B\C := |N\C|.\]}}

We can understand opposing $1$-cells as the opposition of simplicial objects 
by precomposition with the reversal functor $r\colon\Delta\rightarrow \Delta.$

{\lem{\label{catoppnerv} The nerve of the bicategory $\C^{op_1}$ with 
reversed composition of $1$-cells is isomorphic to the $r$-reversed 
simplicial set $NC\circ r$:\[NC\circ r\cong N(C^{op_1}).\]
\begin{proof} This is immediate from the definition. The core point is
that opposing functors $[n]\rightarrow \C$ does change the direction
of $1$-cells, but does not change the direction of $2$-cells, just their
indexing.\end{proof}}}

Hence we find that the homeomorphism $B\C \cong B\C^{op}$ extends
to bicategories:

{\lem{The isomorphism $\Gamma\colon \Delta^\bullet\rightarrow 
\Delta^\bullet\circ r$ extends to a natural homeomorphism: 
\[\Gamma\colon B\C \rightarrow B{\C^{op_1}}.\] }}

So the homeomorphism interprets a sequence of $n$ $1$-cells in $\C$
as an $n$-sequence of the opposed $1$-cells.

{\defn{\label{indinv} For $\R$ a bimonoidal category with involution 
$T$ define the induced involution on its module bicategory as 
follows:\[\xymatrix{B\M \ar[r]^{B\mathcal{M}(T)} & 
        B\mathcal{M}(\mathcal{R}^\mu) \ar[r]^{B(\cdot)^t}  
        & B\mathcal{M}(\R)^{op_1} \ar[r]^{\Gamma} & B\M.}\]}}

Recall that transposition and the involution are 
covariant with respect to the $2$-cells, so the $1$-cells of the functors are 
opposed twice, but the $2$-cells are never opposed, so the constraints 
of $F,G,H$ (for $n\geq 2$) do not change their direction.

{\rem{Chasing through the definitions and taking into account the 
homeomorphism \[B\M \cong \coprod_n |BGL_n\R|,\] where $BGL_n\R$ is 
the bar construction on the monoidal category $GL_n\R$ as defined in 
\cite[Definition 3.8]{BDR2004}, it is easy to
see that this is precisely the same involution as defined in 
\cite{Ri2010}.}}

\section{Examples for Nerves of Bicategories}
There is an integral class in degree $3,$ from which I can bootstrap my 
calculations of the involution on $V(1)_*K(ku)$. I can describe it
easily as induced from a functor $\Sigma^2\S^1\rightarrow
\mathcal{M}(\V_\CC)$, thus induced by a map $\S^3\rightarrow K(\Z,3)
\rightarrow K(ku)$. For this I want to prepare some preliminaries on
the nerve of bicategories. For this section recall that we
can understand the totally ordered set $[n]=[0<1<\ldots<n]$ as a 
$1$-category, thus as a bicategory with discrete morphism categories.

{\prop{Let $\C$ be an arbitrary bicategory, and $F\colon 
	[n]\rightarrow \C$ a strong normal functor. Then $F$ 
	is uniquely	determined by its restriction to all $2$-faces, i.e., 
	by all restrictions \[[2]\rightarrow [n] \rightarrow 
	\C.\] Such a system of pseudofunctors $[2]\rightarrow \C$ 
	determines a (unique) pseudofunctor	$[n]\rightarrow \C$ if 
	and only if each four compatible $2$-faces can be extended over 
	$[3]$ \[\xymatrix{(\partial\Delta^3=)\coprod_4 [2]
	\ar[d]\ar[rr] && \C\\[3],\ar[urr]}\]where the boundary 
	can obviously not be made into a (bi)category, but we can 
	still express it as functors on the disjoint union subject 
	to the appropriate compatibility on $1$-cells. \begin{proof}
	This follows by inspecting the definition of pseudofunctor 
	carefully. The data given by functors $[2]\rightarrow\C$
	precisely gives the compositor $2$-cells, and the condition 
	on extending a functor on the boundary of $[3]$ to all of 
	$[3]$ is precisely the associativity condition on 
	compositors \ref{pseudo}.\end{proof}}}

The following result is classical and implicit in Section 5 of \cite{Str2},
where Street even more generally considers nerves of $n$-categories for
each $n$. However the exposition is quite dated, so I want to phrase
the specific result I need in the context I set up here.

{\prop{The bicategory $\Sigma^2A$ with $A$ an abelian (possibly 
topological) group with one object $*$, one $1$-cell $\id_*$ and
$\Sigma^2A(\id_*,\id_*)=A$ yields as classifying space a double 
delooping of $A$, i.e., there is a homotopy equivalence 
\[\Omega^2|N\Sigma^2A|= A.\] Thus define $B^2A= |N\Sigma^2A|.$

In particular, if $A$ is a discrete group we get $B^2A=K(A,2)$, and
for $A=\S^1$ we have $B^2\S^1=K(\Z,3)$, so $\Sigma^2\S^1$ is a 
bicategory modelling a $K(\Z,3)$.\begin{proof} We see immediately 
from the definition $N\Sigma^2A_0 = N\Sigma^2A_1 = \{*\},$ as well 
as $N\Sigma^2A_2=A$. The functors $r_{a,b,a+b}\colon[0<1<2<3]
\rightarrow\C$ with $r_{a,b,a+b}(012)=a, r_{a,b,a+b}(023)=b, 
r_{a,b,a+b}(123)=a+b, r_{a,b,a+b}(013)=\id_{\id_*}$ introduce the
relations of the Bar complex, thus we get that $N\Sigma^2A$ is a
model for the double delooping as claimed.\end{proof}}}

\subsection*{The Prototypical Class in $H(\MM\V_\CC)=K(ku)$}

{\defn{Consider the topologically enriched $1$-category $SX$ for $X$ 
an arbitrary topological
space defined as \[\xymatrix{0 \ar[r]^X&1.}\] It is a category for
arbitrary $X$, because no non-trivial compositions need to be
defined. The classifying space is the suspension of
$X$, hence in particular we can realise spheres by $X=\S^n$, giving
$BSX = \S^{n+1}$. Call it the \textbf{directed suspension}.\label{dirsusp}}}

{\ex{Consider the categories $\V_\CC$ and $\MM_\CC$ and the 
	directed suspension of the topological circle $S\S^1$. 
	The functor $u\colon S\S^1\rightarrow \V_\CC\subset \MM_\CC$
	with $u(0)=u(1)=1$ and the identity on morphisms realises
	the Bott class on classifying spaces
	\[\S^2\rightarrow \coprod_nBGL_n\CC \rightarrow 
	\Omega B\left(\coprod_nBGL_n\CC\right) \simeq BU\times \Z,\]
	since the Bott class can be represented as
	\[\Sigma\S^1=\Sigma U(1)\rightarrow BU(1)\simeq \CC P^\infty\rightarrow BU_\otimes.\]

	By the fact that the objects $0$ and $1$ are sent to the
	same object we get a factorisation over the one-point suspension
	$\Sigma\S^1,$ because $\S^1$ is an associative
	monoid:\[\xymatrix{S\S^1 \ar[dr]\ar[r]& \V_\CC\\
	&\Sigma\S^1,\ar[u]}\] which on 
	classifying spaces realises:\[\xymatrix{\S^2 \ar[r]\ar[dr] & 
	\coprod_nBGL_n\CC \ar@{-->}[r] & BU\times \Z\\& B\S^1 \simeq 
	K(\Z,2).\ar[u]}\]}}

We can suspend these categories to bicategories analogously. 
Consider a $1$-category $\C$, and define its directed suspension 
bicategory $S\C$ by \[\xymatrix{0\ar[r]^\C & 1,}\] where
again we do not need a composition for $1$-cells, hence objects
of $\C$. It realises the (unreduced) suspension on
classifying spaces, i.e., $BS\C = \Sigma B\C.$

In particular we get the following example.
{\ex{We can suspend the category $S\S^1$ to the bicategory
$SS\S^1=S^2\S^1$ realising $\S^3$ on classifying spaces. Thus we get a
directed suspension of the ``Bott functor'' above
\[Su\colon S^2\S^1 \rightarrow S\V_\CC.\]}}

{In general the directed suspension of a bipermutative category includes
into the bicategory of matrices $j\colon S\R\rightarrow \M$ by 
the inclusion: on objects $j_0(0)=j_0(1)=2$, on $1$-cells:
$j_1(r)= \left(\begin{array}{cc}1&r\\&1 \end{array}\right)$,
and the identification $\M(j_1r,j_1s) \cong \R(1,1)^{\times 2}
\times \R(0,0)\times \R(r,s),$ yields the inclusion 
$j_2\colon \R(r,s)\rightarrow \{\id_1\}^{\times 2}\times\{\id_0\}
\times \R(r,s) \subset \R(1,1)^{\times 2} \times \R(0,0)\times 
\R(r,s).$}

{\ex{For the suspended Bott functor we get 
\[S^2\S^1\rightarrow S\V_\CC\rightarrow \MM(\V_\CC),\] 
with the additional factorisation over the double one-point
suspension $\Sigma^2\S^1,$ which is a bicategory because 
$\Sigma\S^1$ is a monoidal category, because $\S^1$ is an 
abelian group. So we get\[\xymatrix{S^2\S^1\ar[dr]\ar[r]&
\MM(\V_\CC)\\& \Sigma^2\S^1\ar[u],}\] and thus on classifying 
spaces:\[\xymatrix{\S^3\ar[dr]\ar[r] & B\MM(\V_\CC) \ar@{-->}[r] 
& \Omega B \left(\coprod BGL_n\V_\CC\right)\\& B^2\S^1\simeq 
K(\Z,3).\ar[u]}\]

This is the class representing the \emph{Dirac Monopole} in $\RR^3$ 
considered as a $2$ vector bundle as described in \cite{ADR}.}}

We can simplify the discussion of the suspended Bott class
as given in the following section. Since it is given by a class
over the multiplicative unit $1$, its $1\times 1$-matrix is 
weakly invertible, and we do not need to extend to $2\times 2$-matrices.

\subsection{The Involution on the Monopole $\S^3\rightarrow \MM(\V_\CC)$}
The examples above determine a class $a\in\pi_3H\MM(\V_\CC)=\pi_3K(ku),$
the Dirac Monopole (cf. \cite{ADR}), thus we can understand the involution on it.

{\lem{\label{funcmono}
	Consider the class $\bar a\in\pi_3K(ku)=K_3(ku)$ represented by the
	functor:
	\[S_1u\colon S^2\S^1 \rightarrow \MM(\V_\CC)\]
	with $S_1$ the functor which assigns matrix rank $1$ to the 
	objects	$j(0)=j(1)=1$, and considers the $1$-cells of $S^2\S^1$ both
	as the $1\times 1$-matrix $(1)$ in $\MM(\V_\CC)_1$
	It obviously commutes with complex conjugation on 
	$2$-cells, while transposition has no effect. 
	So the diagram:
	\[ \xymatrix{S^2\S^1\ar@{=}[r]\ar[d]^{S_1u} &S^2\S^1\ar[r]^{\overline{(\cdot)_2}}
		\ar[d]^{S_1u}& S^2\S^1\ar[d]^{S_1u}\\
		\MM(\V_\CC)\ar[r]^{(\cdot)^t}&\MM(\V_\CC)^{op_1}\ar[r]^{\overline{(\cdot)_2}}
		&\MM(\V_\CC)^{op_1}}	\]
	strictly commutes.}}

Additionally observe that the functor $S_1$ is oblivious to opposition
of $1$-cells, because in a directed suspension this only amounts to relabelling source
and target object. In summary we find: 
{\thm{On the class $a\in\pi_3K(ku)$ the internalised involution induced by $\MM(\V_\CC)$
	is represented as the composite: \[\xymatrix{|NS^2\S^1|\ar[r]^{|N\overline{(\cdot)}_2|} 
	& |NS^2\S^1| \ar[r]^-\cong & |N(S^2\S^1)^{op_1}|=|\widetilde{N(S^2\S^1)}| \ar[r]^-\Gamma
	& |N(S^2\S^1)|.}\] In particular, the outer maps induce multiplication by $-1$ on $a$, 
	thus the involution induces the identity $a\mapsto a$.\label{invonMono}
	\begin{proof}As above we see that the double directed suspension of $\S^1$ realises 
	to $\Sigma^2\S^1=\S^3$. The conjugation represents
	a reflection along an equator, thus has degree $-1$. 

	For $\Gamma$ consider non-degenerate simplices of maximal degree.
	These are precisely given by functors $[0<1<2]\rightarrow S^2\S^1$ assigning
	for example $01$ to the initial $1$-cell, $12$ to the identity $1$-cell, 
	$02$ to the terminal $1$-cell, and choosing any $x\in\S^1$ as compositor $2$-cell.
	Abusively call such a functor $x$ as well. Then the maximal cells are parametrised
	as $[x,(t_0,t_1,t_2)]$. Here $\Gamma$ acts as: $[x,(t_0,t_1,t_2)]\mapsto[x,(t_2,t_1,t_0)].$
	In particular it has degree given by the sign of the transposition $(02),$ which is
	thus $-1$.\end{proof}}}
 \chapter{Multiplicative Delooping of Bipermutative Bicategories}
\label{multbidel}
In the previous chapters I convinced the reader that bipermutative
bicategories exist, and that they occur when one wants to 
study algebraic $K$-theory of a bipermutative ($1$-)category. 
In particular, the primary example of this thesis $K(ku)$ can be 
described as the Eilenberg-MacLane-spectrum of the bicategory of 
finitely generated free modules of finite-dimensional complex vector 
spaces:\[K(ku)=H\mathcal{M}(\mathcal{V}_\mathbb{C}).\]
To tie this in with the calculations made by Christian Ausoni in the
papers \cite{AuTHH,AuQku,AuKku} we need a combinatorial handle on the 
$E_\infty$-structure on $K(ku)$ induced by the tensor-product on
$\mathcal{M}(\mathcal{V}_\mathbb{C})$. To this end I modify the
delooping given by Ang\'elica Osorno in \cite{Os} in a manner 
analogous to \cite{EM} (cf. in particular the paragraph after
Definition 4.3.) so the resulting construction allows an induced
multiplication by the multiplicative structure of a bipermutative
bicategory.
I do restrict to the case of $E_\infty$-structures, and 
also use a specific $E_\infty$-operad, the Barratt-Eccles-operad 
in a tentative multicategory of permutative bicategories. 

The reader should compare the delooping of this chapter to the
delooping in section 6 of \cite{GJOs}. The authors, however, are
driven by the desire to generalise \cite{Th1} to permutative 
bicategories, thus their emphasis is different from mine. This
makes the deloopings differ in a few ways, which I suspect are
inessential.

\section{The Additive Grothendieck Construction}
This section is where the work in \ref{pcatneu} to rewrite the 
delooping construction of \cite{EM} becomes fruitful. Since the 
additive symmetric monoidal structure on $\M$ as described by
\cite{Os} is sufficiently strict that the Grothendieck construction
\ref{c+} can be used for symmetric monoidal bicategories as well. So
we need to study
which functors are adequate for the analogous construction
of $\C(A_+,n)$ such that the delooping given by Osorno is the
case $n=1$ and such that we get pairings \ref{pairing} \[\C(A_+,n)\times \C(A_+,m)
\rightarrow \C(A_+,n+m),\] which induces an $E_\infty$-multiplication on 
the resulting spectrum \ref{mymult}.

I again use the shorthand $Ob\C=\C_0$ and for $1$-cells when I do not 
want to refer to their source and target I write $Mor\C=\C_1$.

{\rem{As indicated before $\boxplus$, i.e., block sum of matrices
turns $\M$ into a permutative bicategory. The tensor-product does not.}}

Symmetric monoidal bicategories of the strict type of permutative
bicategories allow for the same construction of an associated $\C^+$
as in \ref{c+}. 

{\lem{A permutative bicategory $\C$ has an associated pseudofunctor 
\[B_\C\colon \Fi\rightarrow Bicat,\] given by $B_\C(n_+)=
\C^{\times n}$ and $F_\C(f\colon n_+\rightarrow m_+) = f_*\colon 
\C^{\times n} \rightarrow \C^{\times m},$ where $f_*$ is the strict 
functor $f_*(c_1,\ldots,c_n)_j = \sum_{i\in f^{-1}j}c_i,$ with
compositors given by the additive symmetry: $\varphi^{f,g}\colon 
f_*\circ g_*\Rightarrow (fg)_*$ as a strict natural transformation of 
strict functors. \begin{proof} The proofs in \ref{pcatneu} transfer 
without any problems when I restrict the target to be the $2$-category
of bicategories with strict functors as $1$-cells and strictly natural
transformations as $2$-cells, which works because of the strictness of
permutative bicategories.\end{proof}}}

In particular the construction \ref{c+} translates literally:

{\defn{\label{bic+}Given a permutative bicategory $(\C,+,0,c_+)$ 
define its additive Grothendieck construction $\C^+$ as follows: It 
has objects: \[\C^+_0 = \coprod_{n\geq 0} \C^{\times n}\] and 
morphism categories: \[\C^+((c_1,\ldots,c_n),(d_1,\ldots,d_m))= 
\coprod_{f\in \Fi(n_+,m_+)} \C^{\times m}(f_*(c_1,\ldots,c_n),(d_1,
\ldots,d_m)),\] with composition functors given for a triple of
objects $a=(a_1,\ldots,a_n), b=(b_1,\ldots,b_m), c=(c_1,\ldots,c_l)$ as:
\[\xymatrix{\C^+(b,c)\times\C^+(a,b) \ar@{=}[d]\\ \coprod_{f,g} \C^{
\times l}(g_*b,c)\times\C^{\times m}(f_*a,b)\ar[d]^{\coprod id\times 
g_*}\\\coprod_{f,g}\C^{\times l}(g_*b,c)\times\C^{\times l}(g_*f_*a,
g_*b)\ar[d]^{\coprod comp_{\C^l}}\\\coprod_{f,g}\C^{\times l}(g_*f_*a,
c)\ar[d]^{\varphi^*}\\\coprod_{f,g}\C^{\times l}((gf)_*a,c)\subset
\C^+(a,c).}\]Identities are given by pairs $(id,(id))$ for 
$id\colon n_+\rightarrow n_+$ and $(id)\colon (a_1,\ldots,a_n)
\rightarrow (a_1,\ldots,a_n)$ the $n$-tuple of identities. In 
particular the identities are strict identities because the ones in 
$\C$ are strict.

The associator is given as follows: In each product bicategory $\C^l$ 
we have an associator given by the $l$-tuple with the appropriate 
instances of the $\C$-associator. For this paragraph call this 
$\alpha_l$. Then consider the following two ways of forming a 
three-fold composite for $a,b,c$ as above and $d=(d_1,\ldots,d_k)$:
\[\xymatrix{\C^+(c,d)\times \C^+(b,c)\times \C^+(a,b) \ar[r] \ar[d]
& \C^+(b,d)\times\C^+(a,b)\ar[d]\\\C^+(b,d)\times\C^+(a,b) \ar[r] & 
\C^+(a,d).}\] So we need a natural transformation of the two 
composition-functors: \[\C^+(c,d)\times \C^+(b,c)\times \C^+(a,b)
\Rightarrow \C^+(a,d),\] which is defined on the category $\C^+(a,d)=
\coprod_f \C^k(f_*a,d)$, hence it has components $\alpha_k$.}}

{\rem{Let me issue a warning here: I have no idea what happens, when
one tries to apply the same construction to less strict symmetric
monoidal categories - even of the type I defined in \ref{mdlbic}. I
strongly suspect, it involves a lot more care.

In particular this construction is not naturally set up with respect
to the context of bicategories, in that it is the Grothendieck 
construction of a functor into the $2$-category of bicategories
with strict functors and strict natural transformations as morphisms.
Not every strong functor of symmetric monoidal $1$-categories
can be made strict - consider for instance the strictification functor
$\varepsilon\colon (\C,+)^{st}\rightarrow (\C,+)$, which augments the 
permutative strictification of an arbitary symmetric monoidal category
 \ref{strict1}. Its suspension to bicategories \ref{sigmasymmmon} 
hence is an example of a strong normal functor/pseudofunctor, which is 
not equivalent to a strict functor.}}

{\rem{If the input bicategory $\C$ is in fact a $2$-category,
    i.e., a bicategory with associator-cells only identities, then
    $\C^+$ is a $2$-category as well.}}

{\rem{Do note that the subcategory of morphisms with only discrete
components is a strict $2$-category, because its composition is just
the one in $\Fi$.

Furthermore, just as in the $1$-categorical case, we have that each
morphism $1$-cell in $\C^+$ can be written uniquely as: \[\xymatrix{
{\mathbf{c}}=(c_1,\ldots,c_n) \ar[r]^-{(f,\id)}&
f_*\mathbf{c}\ar[rrr]^-{(\id_m, (A_1,\ldots,A_m))} &&& (d_1,\ldots,
d_m).}\]}

\section{A Multiplicative Delooping for Bicategories}
{\Remi{Because it features prominently in this chapter let me recall
the concept of an \emph{equivalence} in a bicategory. It is a
$1$-cell, say $A\colon a\rightarrow b$, which has a $1$-cell in
the other direction $B\colon b\rightarrow a$ such that $AB\cong \id_b$
and $BA\cong\id_a$. By fixing directions and only demanding morphisms
instead of isomorphism $2$-cells we arrive at the concept of 
\emph{adjoint} $1$-cells, but I do not need that here.}}

{\ex{Adjoint as well as equivalence $1$-cells in our primary example
of interest $\mathcal{M}(\MM_k)$ are just permutation matrices. To see 
this first note that we consider only isomorphism $2$-cells, because
we only have isomorphisms in $\MM_k$, hence being adjoint and being equivalent
is the same. Furthermore since $\MM_k$ is skeletal the existence of 
an isomorphism $AB\rightarrow \id_n$ already gives an equality of the $1$-cells,
i.e., matrices, $AB=\id_n$, analogously for $BA$. Hence $A$ and $B$ are strictly 
invertible matrices, both in $GL(\mathbb{N})$, thus permutation matrices.}}

The natural analogue for bicategories of \ref{isorest} is the 
following proposition:
{\prop{There is a natural inclusion $(\C^{eq_1})^+\rightarrow \C^+$,
where the bicategory $\C^{eq_1}$ is the bicategory with the same
objects as $\C$, and morphism categories on $1$-cells only the
equivalences and all $2$-cells.

Furthermore there is a natural inclusion $(\C^{eq_1,iso_2})^+
\rightarrow (\C^{eq_1})^+\rightarrow \C^+$ where we restrict to just
equivalence $1$-cells and isomorphism $2$-cells.}}

\subsection*{The Construction $\C(A_+,1)$ for Permutative Bicategories}
Many of the next considerations would probably work for bicategories 
with not just isomorphism $2$-cells, if one modifies the results 
appropriately. Most of the time this means changing equivalences into
adjunctions, which probably introduces a lot more thought about when
these $1$-cells compose appropriately. But since my emphasis is on 
delooping to a $K$-theory spectrum, in this chapter I only consider 
bicategories with $\C = \C^{iso_2},$ hence also $\C^+=(\C^{iso_2})^+$. 
This in particular makes every adjunction in $\C$ already an 
equivalence.

To arrange the delooping bicategories as functor bicategories
we need to understand the forgetful functor $\C^+\rightarrow \Fi$ 
again, which yields a stronger assertion for bicategories than
for $1$-categories.
{\prop{The forgetful functor $U\colon \C^+\rightarrow \Fi,$ which assigns
each tuple of objects in $\C^+$, i.e., $(c_1,\ldots,c_n)$ to its
finite pointed set $n_+$ and on morphisms
$(f,(A_1,\ldots,A_m))$ forgets down to  
the discrete component in finite pointed sets $f\colon n_+\rightarrow m_+$, is a 
strict functor of bicategories.\begin{proof}
This is trivially true, since the target is a $1$-category, thus 
does not support non-trivial compositor $2$-cells.\end{proof}}}

Recall the ``comma categories'' introduced in \ref{commaindex}: For
an arbitrary finite set $A$ with added disjoint basepoint $\{*\}\sqcup A=A_+$
consider the category of pointed maps under it, i.e., $\Aix$ 
with objects maps from $A_+$ to a natural number $n_+=\{0,1,
\ldots,n \}=\{*\}\sqcup\{1,\ldots,n\},$ and morphisms commutative 
triangles under $A_+$. 

Recall that the index categories also have forgetful functors 
$T\colon \Aix \rightarrow \Fi$, which send each object to its 
target and forgets the commutativity of triangles under $A_+$. 

This way I can define $\C(A_+,1)$ for permutative bicategories:
{\defn{The bicategory $\C(A_+,1)$ has objects strong normal functors
that lift $T$ through $U$ \[\xymatrix{&(\C^{eq_1,iso_2})^+\ar[d]^{U}\\
\Aix\ar@{-->}[ur]\ar[r]^T & \Fi,}\] i.e., send maps of finite sets 
under $A_+$ to equivalences in $\C^+$.

The category of morphisms between two such lifts $F,G$ is given by
the morphism category $\mathrm{Bicat}(J_*F,J_*G)$, with $J\colon
(\C^{eq_1})^+\rightarrow \C^+$ the natural inclusion. That is,
we consider the morphism category with strong pseudonatural 
transformations, i.e., with isomorphism $2$-cells but arbitrary 
$1$-cells of $\C$, and modifications between those comprised of 
isomorphism $2$-cells.}}

{\rem{Recall that a map of finite based sets $f\colon A_+\rightarrow
    B_+$ induces a map of the indexing categories in the opposite
    direction $f^*\colon \Bix\rightarrow\Aix$, which is a functor
    over $T\colon \Bix\rightarrow \Fi$. By restricting lifting 
    functors from $\Aix$ to $\Bix$ along $f^*$ we thus get lifting 
    functors from $\Bix$, so in summary a strict normal functor 
    $f_*\colon\C(A_+,1)\rightarrow \C(B_+,1)$ in the same direction 
    as $f$.}}

Since in what follows the subbicategory of $\C^+$ with just 
equivalence $1$-cells is the central object, I reduce the notation 
to ${\C^{eq}}^+$ to refer to the additive Grothendieck
construction on the subbicategory of equivalence $1$-cells and 
isomorphism $2$-cells of a permutative bicategory $(\C,+)$.

{\rem{Again consider in $\Aix$ the ``full'', actually
discrete, subcategory given by characteristic functions $\chi_a\colon
A_+ \rightarrow 1_+$ with $\chi_a(x)=*$ for $x\neq a$ and 
$\chi_a(a)=1$. This yields a natural inclusion:
\[\chi_\bullet\colon A^\delta \rightarrow \Aix.\]

On the other hand we have the subcategory of $\Aix$ given by objects
the bijections and morphisms between them. This gives an inclusion
of the translation category associated to the bijections of $A$,
or equivalently pointed bijections of $A_+$\[E\Sigma_A \rightarrow 
\Aix.\]This inclusion embeds the full subcategory of initial objects
of $\Aix$, since each map under $A_+$ can be factored uniquely through 
a bijection. In particular I do not refer to these initial objects 
as initial again, because the isomorphisms between them are 
prominent in the delooping.}}

The delooping is supposed to be a generalisation of the classical
delooping of (topological) abelian groups, so we should expect the
objects to be determined by their ``summands''. The following 
proposition should thus not be surprising, parallel
to the analogous statement in \ref{pcatneu}.

{\prop{Any pseudofunctor lifting $T$ through $U$ has a unique up to
equivalence \emph{strict representative}. More precisely: Any two 
functors with the same restrictions along $\chi_\bullet$ are 
naturally equivalent in $\C(A_+,1)$.\begin{proof}
Choose a total ordering on $A$, hence a bijection $\sigma^A\colon
A_+\rightarrow |A|_+$, and consider a lifting functor 
$F\colon \Aix\rightarrow (\C^{eq})^+.$ 

By the assumption that $F$ sends $1$-cells to equivalences we have
an equivalence in the product bicategory $\C^{\times |A|}$ of the
form $F\sigma_A\rightarrow (F\chi_a)_{a\in A_+}$ given by the
components associated to the diagrams in $\Aix$:
\[\xymatrix{A_+\ar[r]^{\sigma_A}\ar[dr]^{\chi_a} & |A|_+
\ar[d]^{\rho^a} \\	  & 1_+.}\]
So the equivalence is given by $(\id_{|A|},(F^\C\rho_a)_{a\in A})$ in 
$\C^{\times|A|}\subset \C^+$, for $F^\C$ the $\C$-$1$-cells of the 
equivalence without their discrete components $\rho^a$ in $\C^+$.
Choose an inverse to this equivalence in the product category, hence
$\zeta_a F^\C(\rho_a)\cong \id_{F\chi_a}$ with the analogous isomorphism
for the other composition of $\zeta_a$ with $F^\C(\rho_a)$.

Build the \emph{strict representative} as follows: $F^{st}(\sigma_A)
:= (F\chi_a)_{a\in A}$. Any other object of $\Aix$ has a unique 
morphism coming from $\sigma_A$, so for $p\in \Aix$ set 
$F^{st}(p):=(p\circ\sigma_A^{-1})_*(F^{st}(\sigma_A))\in\C^{\times
|Tp|}.$ Again drop $\sigma_A^{-1}$ from the notation for instance by
assuming $A_+$ totally ordered, thus a unique element of $\Fi$ itself.
For a commutative triangle under $A_+$:\[\xymatrix{ A_+\ar[r]^p
\ar[dr]^{qp}& n_+\ar[d]^q\\& m_+}\] we need to have a morphism
\[F^{st}(p)=p_*(F^{st}(\sigma_A))\rightarrow 
q_*p_*(F^{st}\sigma_A) \rightarrow (qp)_*(F^{st}(\sigma_A))=
F^{st}(qp),\] which by construction of $\C^+$ we can take to be 
$(q,\varphi^{q,p}),$ and this is obviously a morphism over $q$ in 
$\Ep$. So we have constructed $F^{st}$ as a lift of $T$ through $U$, 
which sends each commutative triangle in $\Aix$ to morphisms in $\C^+$ 
with just discrete components and additive symmetries. In particular
we can choose $F^{st}$ with identity $2$-cells, and thus have a 
strict normal functor, because the additive symmetries were assumed
to be strictly natural for permutative bicategories.

By the decomposition of $1$-cells in $\C^+$, we can uniquely write the
map $F(p\circ\sigma_A^{-1})\colon F(\sigma_A)\rightarrow F(p)$ as its
discrete component followed by a $1$-cell with discrete component the
identity \[\xymatrix{F\sigma_A \ar[r]^{(p,\id)} &p_*F\sigma_A 
\ar[rrr]^{(\id,F^\C(p\circ\sigma_A^{-1}))} &&& Fp}.\]
So we have in $\C^+$ with the equivalence $1$-cells $\zeta_a$ as 
chosen before: \[\xymatrix{F^{st}p = p_*((F\chi_a)_{a\in A}) 
\ar[rrr]^-{(\id,p_*((\zeta_a)_a))}&&& p_*F\sigma_A 
\ar[rr]^-{(\id,F^\C(p\circ\sigma_A^{-1}))}&& Fp,}\] which we can promote to 
a pseudonatural transformation by choosing as the naturality $2$-cells 
the inverses of the adequate compositor $2$-cells $F$. This 
transformation then trivially commutes with the strict compositor of 
$F^{st}$ and the ones of $F$, and has as $1$-cells equivalences by
construction. So we have established a pseudonatural equivalence 
$F^{st}\simeq F,$ which only depended on data coming from $F,$ 
while $F^{st}$ even only depended on the restriction of $F$ along 
$A^\delta\rightarrow \Aix$, hence is as unique as claimed.
\end{proof}}}

{\rem{At this point let me informally compare this construction to
the one displayed in the proof of Theorem 3.6 in \cite{Os}.
The objects as described there are strict functors $(\Aix)^{op}
\rightarrow \C^+,$ so precisely the chosen inverse equivalences 
$\zeta$ I just described. For notational convenience let me treat 
them as if the functors in \cite{Os} were written down as covariant 
functors $\Aix\rightarrow\C^+$.

The passage to the strict representative as I indicated above shows
that each pseudofunctor $\Aix \rightarrow {\C^{eq}}^+$ lifting $T$ 
through $U$ is naturally equivalent to one that is not just a strict
functor, but also just comprised of discrete components. So we can
include the functors described by Osorno into $\C(A_+,1)$ and find
that the strict representative is of the kind described in \cite{Os}, 
so we get a surjection up to equivalence, which by inspection of
the $1$- and $2$-cells described in that same proof is also 
an equivalence on the morphism categories. (Most specifically she
describes the construction on $\underline{\mathbf{n}},$ which is
$n_+$ in my convention, so a subbicategory of $\C(n_+,1)$.)

I chose the morphism categories in the delooping construction just so 
that this equivalence is true.}}
	
The passage to the strict representative is sufficiently natural
that the typical delooping result is an easy corollary:
{\cor{\label{dell} We have a natural equivalence of bicategories:
	\[(\cdot)^{st}\colon \C(A_+,1)\rightarrow \C^{A}.\]	
\begin{proof} We know that 
$\C^{A}$ is strictly equal to the bicategory of functors $A^\delta 
\rightarrow \C^+$ and we can restrict each functor to its components 
on $(\chi_a)_{a\in A}$, which is precisely $A^\delta$ as a full 
subcategory of $\Aix$. So the inclusion of the
product bicategory by the functor, which sends each tuple to a lifting
functor which is its own strict representative, is an inverse 
equivalence for $(\cdot)^{st}$. On the left we find that the natural 
equivalence to the identity is just the one described at the end of 
the proof before. On the right we have: Making a tuple into a strict 
functor and then restricting to its $\chi_a$-summands is strictly 
equal to the identity functor on $\C^A$.\end{proof}}}

{\rem{This is the initial step in the induction to prove the
analogous equivalence for the higher delooping bicategories 
$\C(A_+,n).$}}

\subsection{The Construction $\C(A_+,n)$ for Permutative Bicategories}
This section is where the rewriting of the delooping constructions 
of \cite{EM} and \cite{Os} in \ref{pcatneu} and the section before
comes to fruition, because this way it easily generalises to using 
$(\Aix)^{\times n}$ as the index category, and letting coherence 
$2$-cells take care of themselves by using pseudofunctors.

Recall from \ref{pcatneu} the target functors $T_n\colon (\Aix)^n
\rightarrow \C_+$ given in \ref{T-n}: We have the analogous 
definition of $\C(A_+,n)$ for bicategories.
{\defn{For a permutative bicategory $(\C,+)$ the delooping bicategory 
	$\C(A_+,n)$ has as objects strong normal functors $F$ lifting
	$T_n$ through $U$:	\[\xymatrix{& (\C^{eq})^+\ar[d]^U\\ 
    (\Aix)^n \ar@{-->}[ur]^F \ar[r]_-{T_n} & \Fi.}\]
	Its morphism bicategories are again the pseudonatural 
	transformations and modifications of the functors after including
	by $(\C^{eq})^+\rightarrow \C^+.$}}

{\rem{Do note that as in the case in \ref{pcatneu} the delooping bicategories
	$\C(A_+,n)$ have a canonical basepoint object given by the functor $O_n$,
	which has as objects the adequate zero-tuples of each degree and morphisms
	consisting of the adequate discrete components with $\id_0$ as its 
	second component.\label{thebasenull}}}

{\rem{Since for $n=1$ we do not have to choose bijections for the smash
    product in $\Fi$ the functor $T_1$ is strictly the same as $T$ in 
    the section before, in particular I described the same bicategory 
    of functors.
    
    It is consistent to set $\C(A_+,0)=\C$, since $\left(\Aix\right)^0=*$ is
    the one-point category.}}

{\rem{Apart from an opposition of the indexing category $\Aix$ this 
    definition would read ``strict normal'' functors in the delooping
    considered by \cite{Os}, which works well there because the 
    additive structure is strict enough. Since I want to induce a 
    multiplicative structure from a symmetric monoidal structure
    as given by $\boxtimes$ in \ref{multbidel}, which prominently
    features a pseudofunctor which is usually not strict, I need to
    consider more generally all strong normal pseudofunctors with
    possibly non-trivial compositor $2$-cells.}}

It is possible to give an explicit construction of the 
delooping bicategories of a permutative bicategory along the lines 
of \cite{EM} and \cite{Os}. However, since the tensor functor I 
describe in \ref{tensor} has a non-trivial isomorphism $2$-cell I 
cannot restrict to strict additors the way Osorno does in 
\cite[Proof of Theorem 3.6, p. 11]{Os}, but have to allow potentially
non-trivial isomorphism $2$-cells. This becomes unwieldy in the 
explicit construction, so I arranged the delooping by functor 
bicategories, analogous to the rewriting of \cite{EM} I present
in \ref{pcatneu} above. In particular the construction runs parallel 
to the $1$-categorical case, with equivalences inserted where there 
are isomorphisms for permutative $1$-categories.

{\rem{At this point an informal comparison to \cite{EM} is convenient.
	For the case of bicategories with discrete morphism categories (or
	actually topological spaces or simplicial sets, which are discrete 
	as $1$-categories, but possibly non-discrete as spaces,) we can 
	compare to Construction 4.4. on page 19 of \cite{EM}. The systems 
	described there are indexed over arbitrary product categories 
	$(A_1\downarrow\Ep)\times\ldots\times(A_n\downarrow\Ep)$. By 
	passing their based systems of subsets $S=(S_1,\ldots,S_n)$ to 
	their unbased components we can associate to each such subset a 
	characteristic map, and thus an object in $A_i\downarrow\Ep$. The 
	additors $\rho$ described there are then given under the condition
	that we can factor a characteristic map over $2_+$, and hence we 
	get an associated map for this $2_+\rightarrow 1_+$-component.

	Condition (1) is then the fact that the wedge at the basepoints of 
	the index-categories $A_i\downarrow \Ep$, which is given by tuples 
	of maps, which have any component mapping to $0_+$, smashes to 
	$0_+$ and is hence sent to $0_+$ by	$T_n$. Condition (2) is the 
	normality of the functor, i.e., strictly respecting identities. 
	Condition (3) expresses the fact that the choice of which subset 
	to map to which element in $2_+$ should not matter. Condition (4)
	is simply the functoriality, as in respecting composites strictly, 
	because the	context in \cite{EM} are enriched $1$-categories, and 
	last condition (5) is by construction of $\C^+$ and by the remark
	\ref{fuerdiehohenReds}, which still applies for the additive
	Grothendieck construction on even general bicategories, also just
	strictly respecting composites. Here $1+c_++1$ is the twist
	needed to express the following trivially commutative diagram
	in $\Ep^{\times 2}:$\[\xymatrix{2_+\wedge 2_+ \ar[r]\ar[d]& 1_+
	\wedge 2_+\ar[d]\\2_+\wedge 1_+ \ar[r] & 1_+\wedge 1_+,}\]by 
	flattening it with the bijections $\omega$ chosen before, such 
	that we have get indexing sets appropriate for summations.
	Morphisms are the appropriate stricter version of the ones 
	considered in \cite{Os} as well, so the same remarks apply.

	In particular do note that $\C(A_+,n)$ could easily be generalised
	to $n$ different indexing categories, but since the resulting
    spectrum is defined by inserting $\S^1$ for each $A_+$, I chose
    to reduce to the case with equal inputs. }}

As indicated at the end of the last section I prove the equivalence
of $\C(A_+,n)$ to the appropriate product bicategory by displaying
the inductive step in constructing the equivalence. For this let me
emphasise that for arbitrary bicategories (possibly enriched) we have
the following simple case of the exponential law, for $A,B$ sets 
considered as discrete categories, hence bicategories:
\[Fun(A,Fun(B,\C))\cong Fun(A\times B, \C)\cong \C^{A\times B}.\]
In particular I can reduce the index juggling quite a bit by proving
this form of the following theorem.
{\thm{For $(\C,+)$ a permutative bicategory we have the following 
	natural equivalence of bicategories:\[\C(A_+,n)\simeq 
	Set(A,\C(A_+,n-1)).\]Inductively we find the natural equivalence:
	\[\C(A_+,n)\simeq \C^{A^{\times n}}.\]\begin{proof}
	The functor $\C(A_+,n)\rightarrow Set(A,\C(A_+,n-1))$ is -- as
    in the case of permutative $1$-categories -- given by restricting
    along $(\Aix)^{n-1}\times A^\delta \rightarrow (\Aix)^n$ with one
    component the inclusion of $A$ as the full discrete subcategory
    of characteristic functions $\chi_a$ in $\Aix$.

    The same reasoning as for $\C(A_+,1)$ before yields for each
    functor in $\C(A_+,n)$ a strict representative, which in this
    case means strict with respect to one of the $\Aix$-factors, for
    instance the last one as described above. So the restriction 
    has an inverse functor given by extending an $A$-tuple of 
    functors in $\C(A_+,n-1),$ by sending the maps in the last 
    factor to the appropriate discrete components in $\C^+$, and 
    thus summing up the functors according to the chosen bijections.
    \end{proof}}}

Furthermore we have the following generalisations of the analogous
results in \ref{pcatneu}.
{\prop{For each $n\in\mathbb{N}$ we have a strictly natural strict 
    $\Sigma_n$-action on $\C(A_+,n),$ given by permuting the inputs 
    and pushing forward with the induced symmetry $\chi^\wedge$ of 
    $\Fi$ in $\C^+$.}}
{\prop{For each pointed finite set $A_+$ we have natural strict extension
    functors \[A_+\wedge \C(A_+,n)\rightarrow \C(A_+,1+n),\] which are
    $\Sigma_1\times\Sigma_n$-equivariant.}}

Both proofs essentially proceed as the case for $1$-categories, where
the strictness of the functors is a consequence of the fact that they
only use the discrete components in $\C^+$, which are part of the
included $2$-category on all objects but just discrete morphisms.

{\rem{The extension functors as well as the $\Sigma_n$-action of the
propositions above strictly respect the basepoint functors $O$. For the
extension this is obvious, since we extend functors by zeroes. For the
symmetric action observe that in particular the assignment on objects
gives constant tuples, which are hence invariant under permutations.}}

The delooping construction $\C(A_+,n)$ is directly comparable to
Osorno's delooping \cite{Os} by restricting the source of the lifting
functors. {\prop{For two finite pointed sets $A_+,B_+$ we have a functor
	\[\Aix\times\Bix\rightarrow (A_+\wedge B_+)\downarrow \Fi,\]which by
	the canonical identification $A_+\wedge B_+\cong (A\times B)_+$ is 
	an indexing category for the construction $\C(\_,1)$.\begin{proof}
	By choosing an identification $k_+\wedge l_+\cong kl_+$, i.e., a
	total ordering on binary products, for instance the lexicographic
	order, we get a smash product functor on $\Fi$. Thus we map a pair
	of morphisms $f\colon k_+\rightarrow l_+$, $g\colon m_+\rightarrow 
	n_+$ to $f\wedge g \colon km_+\rightarrow ln_+$.
	
	Analogously on objects, for a pair $p\colon A_+\rightarrow k_+$ and
	$q\colon B_+\rightarrow l_+$, we can consider their product
	$p\times q\colon A_+\times B_+\rightarrow k_+\times l_+$ composed
	with the canonical projection $k_+\times l_+\rightarrow kl_+$ to
	the smash product. This factors over $(A\times B)_+= A_+\wedge B_+$.
	
	Thus the above assignments define a functor, since the smash
	product on $\Fi$ is a functor.\end{proof}}}

{\cor{For any finite pointed set $A_+$ we have a natural functor
	\[(\Aix)^{\times n}\rightarrow {A_+}^{\wedge n}\downarrow \Fi.\]}}

{\prop{Restricting a functor $F\in\C((A_+)^{\wedge n},1)$ along the 
	functor	$S\colon (\Aix)^{\times n}\rightarrow {A_+}^{\wedge n}
	\downarrow \Fi$	gives an element of $\C(A_+,n)$. \begin{proof}
	Reconsider the definition of $T_n$ as in \ref{T-n}: For 
	$T\colon \Aix\rightarrow \Fi$ the forgetful functor assigning to
	each object $p\colon A_+\rightarrow n_+$ its target $n_+$ and to
	each commutative triangle $f\colon p\rightarrow fp$ the map $f$ we
	set $T_n$ to be \[(\Aix)^n\rightarrow \Fi^n\rightarrow \Fi\]
	with first map $(T)^n$ and second map the $n$-fold smash product.
	
	This factors as: \[\xymatrix{(\Aix)^n \ar[r]^-{(T)^n}\ar[d]_{S}&\Fi^n
	\ar[d]^{\wedge}\\(A_+)^{\wedge n}\downarrow\Fi \ar[r]^-{T} & \Fi.}\]
	
	Thus a functor $F\colon {A_+}^{\wedge n}\downarrow \Fi \rightarrow 
	\C^+$ satisfying $UF = T$ trivially satisfies $UFS = TS$, which by
	the commutative square above gives $UFS = TS = T_n$, so $FS$ is
	an element of $\C(A_+,n)$.\end{proof}}}

Finally to tie the delooping categories in with the delooping 
constructed in \cite{Os} we need the following equivalence:
{\thm{\label{AeqzuOs} Smashing the source category 
    $(\Aix)^n\rightarrow (A_+)^{\wedge n}\downarrow \Fi$ induces a 
    natural strict restriction functor \[S^*\colon \C((A^{\times n})_+,1)
    \rightarrow \C(A_+,n),\] which is a natural equivalence of 
    bicategories. \begin{proof} Recall the statement and proof
    of corollary \ref{dell}: Specifically for the delooping bicategories
    $\C((A^{\times n})_+,1)$ and $\C(A_+,n)$ we find that the natural
    inclusions ${(A^{\times n})_+}^\delta\rightarrow (\Aix)^n$ and
    ${(A^{\times n})_+}^\delta \rightarrow (A_+)^{\wedge n}\downarrow 
    \Fi$ -- each identifying ${(A^{\times n})_+}^\delta$ as a discrete
    subcategory of the respective indexing categories -- fit into a
    commutative triangle:
    \[\xymatrix{
    &{(A^{\times n})_+}^\delta \ar[dl]\ar[dr]&\\
    (\Aix)^n \ar[rr]^S && (A^{\times n})_+\downarrow \Fi.}\]
    Since restriction along the diagonal arrows of this triangle each
    give equivalences of bicategories by \ref{dell}, we find that
    $S^*$ also is an equivalence of bicategories.\end{proof}}}

\section{The Multiplicative Structure on $\C(A_+,n)$}
Let me reiterate that I only consider multiplicative structures as
induced on matrices for a bipermutative coefficient
category $\R$. Hence I restrict to the $E_\infty$-case and by 
choosing the Barratt-Eccles-operad as in 
\cite[p. 16, Theorem 3.7]{EM} I can avoid constructing a 
multi(bi)category-structure for permutative bicategories.

To understand how a bipermutative structure induces a multiplication
on the delooping bicategories $\C(A_+,n)$ I have to fix a 
multiplication on $\C^+$ and compatible target functors
$T_n$. The induced multiplication on $\C^+$
is a direct generalisation from the case of $1$-categories, so I repeat 
the proof to keep track of strictnesses and genuine $2$-cells.

{\thm{\label{bipzumult}
    Consider a bipermutative bicategory (see \ref{bipbic}) 
    $(\C,+,\cdot)$. Fix a smash product on $\Fi$, then we have a 
    symmetric monoidal structure $\boxtimes$ on $\C^+$ making the 
    forgetful functor strictly symmetric monoidal \[U\colon
    (\C^+,\boxtimes)\rightarrow (\Fi,\wedge).\] \begin{proof} 
    Do the same on objects as in \ref{pcatneu}: \[(c_1,\ldots,
    c_n)\boxtimes(d_1,\ldots,d_m)=(c_id_j)_{\omega(i,j)},\] where the 
    multiplication on the right is the multiplicative structure
    of the bipermutative bicategory $(\C,+,\cdot)$. Again the subtlety 
    is the definition on morphisms. For this consider first the following
    objects in $\C^+$: \[(f\times g)_*(c\boxtimes \bar c)_{\omega(i,j)}
    = \sum_{\omega(k,l)\in (f\times g)^{-1}(\omega(i,j))}c_k\bar c_l,\]
    and analogously:\[(f_*c\boxtimes g_*\bar c)_{\omega(i,j)}
    = (f_*c)_i\cdot (g_*\bar c)_j = \left(\sum_{k\in f^{-1}i}c_k\right)
    \left(\sum_{l\in g^{-1}j}\bar c_l\right).\]
    By \ref{bipbic} we find a unique structural map $D^{f,g}$ comprised
    of isomorphism $1$-cells (given for instance here by first all left 
    reductions, then all right reductions) \[(f\times g)_*(c\boxtimes
    \bar c)=\sum_{k,l}c_k\bar c_l \rightarrow \sum_k \left(c_k \left(\sum_l 
    \bar c_l\right)\right)\]
    \[~~~~~~~~\rightarrow\left(\sum_{k\in f^{-1}i}c_k\right)
    \left(\sum_{l\in g^{-1}j}\bar c_l\right)=f_*c\boxtimes g_*\bar c,\]
    which is given by composites of distributors, and uniquely 
    determined by the summations of $f$ and $g$. Hence for two maps in 
    $\C^+$: \[(f,(a_1,\ldots,a_{m_1}))\colon c=(c_1,\ldots,c_{n_1})
    \rightarrow d=(d_1,\ldots,d_{m_1}),\]
    \[(g,(b_1,\ldots,b_{m_2}))\colon \bar c = (\bar c_1,\ldots,
    \bar c_{n_2}) \rightarrow \bar d = (\bar d_1,\ldots,\bar d_{m_2}),\]
    we set their product to be the following composite:
    \[\xymatrix{c\boxtimes \bar c =(c_i\bar c_j)\ar[r]^-{(f\times g)_*} 
    & \left(\sum c_k\bar c_l\right) \ar[r]^-{D^{f,g}}
    & \left(\sum c_k\right)\left(\sum \bar c_l\right)
    \ar[r]^-{a_i\cdot b_j} & (d_i\cdot \bar d_j). }\]

    The map $\boxtimes$ respects identity $1$-cells strictly, because 
    $\cdot$ was assumed to be normal \ref{bipbic}. Since $f,g$ are part of 
    the data of morphisms in $\C^+$, the structural morphism 
    $(f\times g)_*$ is strictly natural. Since the distributors in 
    \ref{bipbic} are strict natural transformations, $D^{f,g}$ 
    strictly commutes with genuine morphisms of $\C$ as well. Thus 
    only the appropriate products of compositor $2$-cells for $\cdot$ 
    yield the compositor for $\boxtimes$ with no additional $2$-cells 
    introduced by either $(f\times g)_*$ or $D^{f,g}$. So we have a 
    strong normal functor \[\boxtimes\colon \C^+\times\C^+\rightarrow 
    \C^+.\]

    Since $1_+$ is a strict unit object for $\wedge$ on $\Fi$, the
    $1$-tuple $(1)\in \C^+$ with entry the multiplicative unit of $\C$
    yields a strict unit object in $\C^+$. 

    The functor $\boxtimes$ is strictly associative: visibly on objects
    precisely because the bijections for the smash-product in $\Fi$
    are chosen that way, and because the multiplication on $\C$ was
    assumed strictly associative. Because of the strict identity of
    functors for triple products that $\cdot$ on $\C$ satisfies by
    assumption, we get strict associativity for $\boxtimes$ as a 
    strict functor identity on $\C^+$.

    Finally the multiplicative symmetry transformation is given as
    follows. Let the symmetry in $\Fi$ with respect to $\wedge$ be 
    $\chi$, then the $1$-cell for the symmetry of $\boxtimes$ is the 
    composite: \[\xymatrix{c\boxtimes d=(c_id_j)_{\omega(i,j)}
    \ar[r]^-{\chi}&(c_id_j)_{\omega(j,i)} \ar[r]^-{c^\cdot_1}
    &(d_jc_i)_{\omega(j,i)}=d\boxtimes c.}\] Since the symmetry $\chi$ 
    introduces no $2$-cell, the $2$-cell for the symmetry of $\boxtimes$ 
    is thus given as the appropriate product of the symmetry $2$-cells 
    of $\cdot$ in $\C$.

    The symmetry squares to the identity strictly, since the symmetries
    of $(\Fi,\wedge)$ and $(\C,\cdot)$ do. It satisfies the two
    diagrams for triple products strictly for the same reason.

    For the final claim we only need to observe that the discrete 
    components of the functor $\boxtimes$ and its symmetry $c^\boxtimes$
    are modelled just so that the forgetful functor $U\colon \C^+
    \rightarrow \Fi$ is strictly symmetric monoidal. \end{proof}}}

{\rem{Since I have established that a bipermutative structure gives
    a functor over the smash-product functor $\wedge\colon \Fi
    \times\Fi\rightarrow\Fi$ I strongly conjecture that this could
    be used to make bimonoidal and bipermutative categories much more
    explicit in the context of $\infty$-categories. Compare this for
    instance to (p. 149) Definition 2.1.3.7 in \cite{Lu2} and more 
    directly to (p. 136) Definition 2.0.0.7, where Lurie defines a
    symmetric monoidal $\infty$-category just so that by adding in
    all morphisms of $\Fi$ into $\C^+$ the map $U\colon \C^+
    \rightarrow \Fi$ exhibits its nerve $N\C^+$ as a symmetric
    monoidal $\infty$-category, and by 2.1.3.7 $\boxtimes$ as a
    symmetric monoidal functor.}}

The multiplication on $\C^+$ induces a pairing of the delooping bicategories.
{\thm{\label{pairing}Given a bipermutative bicategory $(\C,+,\cdot)$ 
    the delooping bicategories $\C(A_+,n)$ have a pairing pseudofunctor: 
    \[\mu_{n,m} \colon\C(A_+,n)\times\C(A_+,m)\rightarrow \C(A_+,n+m),\] 
    which is strictly natural in $A_+$, and 
    $\Sigma_n\times\Sigma_m$-equivariant. 
        
    Furthermore the pairing strictly satisfies $\mu(O_n,\_)=\mu(\_,O_m)
    = O_{n+m}$, i.e., pairing with the zero-functor yields the constant
    map to the zero-functor $O_{n+m}\in\C(A_+,n+m)$.
    \begin{proof}
    By the propositions before we know that we can pair two lifting
    functors $F_n\colon (\Aix)^n\rightarrow \C^+$ and $G_m\colon
    (\Aix)^m\rightarrow \C^+$ by the symmetric monoidal structure
    on $\C^+$ to give $\boxtimes_*(F_n,G_m)\colon (\Aix)^{n+m}
    \rightarrow \C^+\times \C^+\rightarrow \C^+$, which is evidently
    compatible with the $\Sigma_n$-operation on $F_n$ and the $\Sigma_m$-
    operation on $G_m$ independently, thus with $\Sigma_n\times\Sigma_m$
    as a whole.
	Furthermore evidently $\boxtimes_*(O_n,\_)=\boxtimes_*(\_,O_m)=O_{n+m}$,
	since the zero-functor acts as a strict zero for $\boxtimes$, which
	is induced by $\cdot$ on the bipermutative bicategory.

    Since the symmetric monoidal structure of $\C^+$ is defined over the
    forgetful functor $U\colon \C^+\rightarrow \Fi$ such that it becomes
    strictly symmetric monoidal, the resulting functor lifts the map
    \[\xymatrix{(\Aix)^n\times(\Aix)^m\ar[r]^-{T_n\times T_m} &\Fi
    \times\Fi\ar[r]^\wedge & \Fi,}\] which by \ref{T-n} is the same 
    as $T_{n+m}$. Hence the resulting functor is in $\C(A_+,n+m)$.

    The same description applies to $1$- and $2$-cells, since I did not
    need to refer to $1$-equivalences to define the symmetric monoidal
    structure on $\C^+$. Hence we have a strict symmetric monoidal 
    inclusion $(\C^{eq})^+\rightarrow \C^+$, and can extend the product
    to $1$- and $2$-cells. 
    By applying the compositor of $\cdot$ appropriately componentwise
    we get the compositor $2$-cell for $\mu$.

    Strict naturality in the pointed set is a consequence of the fact 
    that a map of pointed finite sets $f\colon A_+\rightarrow B_+$ 
    induces a strict normal \ref{strictn} functor $\C(A_+,n)\rightarrow
    \C(B_+,n)$ by pulling the category $\Bix$ back along $f$ to $\Aix$ 
    and then pulling back functors along this pullback. In particular 
    it is restriction of the source category, hence the 
    multiplication is strictly natural in $A_+$.\end{proof}}}

{\rem{Do note that by convention $\C(A_+,0)=\C$, so for $(\C,+,\cdot)$
	bipermutative we trivially have	a map $\eta_0\colon 
	*\rightarrow \C(A_+,0)$ sending the object to $1$ and its identity.
	
	Furthermore by the extension $A_+\times \C(A_+,0)\rightarrow
	\C(A_+,1)$ we get a map $\eta_1\colon A_+ \cong A_+\times \{1\}
	\rightarrow \C(A_+,1)$, which hence sends a pair $(a,1)$ to the
	functor, which is the tuple $(1)$ at 
	$\rho^a\colon A_+\rightarrow 1_+$ with zeroes
	adequately added everywhere else.
	
	In particular for $(\C,+,\cdot)$ we can rewrite the extension maps
	$A_+\times \C(A_+,n)\rightarrow \C(A_+,1+n)$ as the multiplication 
	with $\eta_1$: \[A_+\times \C(A_+,n) \rightarrow \C(A_+,1)\times
	\C(A_+,n) \rightarrow \C(A_+,1+n).\]}}

The pairing inherits strict associativity from the strict associativity
of $(\C,\cdot)$.
{\prop{For $(\C,+,\cdot)$ a bipermutative bicategory the 
	pairing of \ref{pairing} is strictly associative, i.e.,
	\[\xymatrix{\C(A_+,l)\times \C(A_+,m)\times \C(A_+,n) \ar[r] \ar[d] 
		& \C(A_+,l)\times\C(A_+,m+n)\ar[d]\\ \C(A_+,l+m)\times\C(A_+,n)
		\ar[r] & \C(A_+,l+m+n)	}\]
	is strictly commutative for each pointed finite set $A_+$ and natural
	numbers $l,m,n$. \begin{proof} This follows from the strict
	associativity of the chosen bijections $\omega\colon n\times m 
	\rightarrow nm$ making the monoidal structure on $\C^+$ with 
	$((c_i),(d_j))\mapsto (c_i\cdot d_j)_{\omega(i,j)}$ strictly 
	associative, because $\cdot$ is strictly associative.\end{proof}}}

By rewriting the extension maps as above we get the following corollary:
{\cor{The pairings commute with extension maps strictly, i.e., there
is a unique pairing of the form
\[\xymatrix{\C(A_+,n)\wedge A_+ \wedge \C(A_+,m) \ar[r] & \C(A_+,n+1+m).}\]}}
	
{\defn{For $\sigma\in\Sigma_n$ define the map:
\[\mu_\sigma\colon \C(A_+,k_1)\times\ldots\times\C(A_+,k_n)
\rightarrow \C(A_+,\sum_i k_{\sigma^{-1}(i)})\]
as the composite of the symmetry in $(Bicat,\times)$
\[c^\times_{\sigma}\colon \prod_i \C(A_+,k_i)\rightarrow 
\prod_i \C(A_+,k_{\sigma^{-1}(i)})\]
followed by the $n$-fold pairing (uniquely determined by the proposition
above):\[\prod_i \C(A_+,k_{\sigma^{-1}(i)})\rightarrow
\C(A_+,\sum_ik_{\sigma^{-1}(i)}).\]}}

{\rem{By definition of $\mu_\bullet$ we have a functor
\[\Sigma_n\times\left(\coprod_{k_1+\ldots+k_n=N}\C(A_+,k_1)\times
\ldots \times \C(A_+,k_n)\right)\rightarrow \C(A_+,N)\]
with $\Sigma_n$ considered as a discrete category, which factors as
\[\Sigma_n\times_{\Sigma_n}\left(\coprod_{k_1+\ldots+k_n=N}\C(A_+,k_1)
\times\ldots\times \C(A_+,k_n)\right)\rightarrow \C(A_+,N).\]

Furthermore since we established that binary multiplication becomes the
constant map to the zero-functor, if one parameter is the zero-functor,
we find that each $\mu_\bullet$ becomes the constant map if one
parameter is restricted to the zero-functor.}}

Finally I want to state the $E_\infty$-commutativity in its binary form
for clarity before summarising the $E_\infty$-structure in \ref{mymult}.

{\prop{\label{minieinf} For a bipermutative bicategory $(\C,+,\cdot)$, a 
finite pointed set $A_+$ and two natural numbers $n,m$ the two pairings
$\mu_{\id},\mu_{(12)}\colon \C(A_+,n)\wedge \C(A_+,m)\rightarrow
\C(A_+,n+m)$ are pseudonaturally isomorphic.

Furthermore the pseudonatural isomorphisms inherent the coherence of the
$\cdot$-symmetry in that for each two $\sigma,\tau\in\Sigma_N$ there
is a unique composite pseudonatural isomorphism
$\mu_\sigma \Rightarrow \mu_\tau$ of pairings
\[\C(A_+,n_1)\wedge\ldots\wedge\C(A_+,n_N)\rightarrow \C(A_+,\sum_i n_i).\]
\begin{proof} The $1$-cells of the pseudonatural transformation consist 
of $(\chi_{n,m},c^1_\cdot)$ with $\chi_{n,m}$ the block permutation 
shifting the first $n$ elements of $n+m$ past the last $m$ elements, and
$c^1_\cdot$ the $1$-cell of the pseudonatural symmetry for $(\C,\cdot)$.

In particular by \ref{mdlbic} we already see that each $1$-cell is a 
strict isomorphism, which squares to the identity, thus we only need 
pseudonaturality. The pseudonaturality $2$-cell is given accordingly
by $(\id, c^2_\cdot)$.

Furthermore we see immediately that the coherence of $c_\cdot$ promotes
to the coherence claimed above.\end{proof}}}

{\Remi{For $1$-categories $\C,\D$ that natural 
transformations $\eta\colon F\Rightarrow G$ are in a natural one-to-one 
correspondence with functors $H\colon \C\times I\rightarrow \D$.

Specifically, the functors $F,G$ are restrictions of $H$ to the
objects $0,1\in I$ respectively, while the components
of the natural transformation $\eta$ are the arrows $H(c,0\rightarrow 1)
=\eta_c$. Naturality of $\eta$ is then equivalent to $H$ being a functor,
because $\C\times I$ is a product-category.}}

{\rem{For $\C,\D$ bicategories the above correspondence generalises to
pseudonatural transformations, which are in one-to-one correspondence
with pseudofunctors. However, since for pseudofunctors the compositor 
$2$-cells fill triangles, while the $2$-cell involved in the pseudonaturality
condition fills a square, we actually get (at least) two correspondences
by fixing one or the other triangle in the diagramme \label{quadtotrig}
\[\xymatrix{H(c,0) \ar[r]\ar[dr]\ar[d] & H(d,0)\ar[d]\\
H(c,1) \ar[r] & H(d,1)}\] to be filled with the compositor $2$-cell.

The same correspondence establishes that a pseudonatural
transformation is strictly natural, i.e., has only identity $2$-cells,
if and only if its associated pseudofunctor is a strict functor.}}

To introduce the specific $E_\infty$-coherences for the pairing on 
the delooping bicategories, recall the Barratt-Eccles operad in 
$1$-categories (cf. \cite[p. 15]{EM}).
{\defn{For any discrete set $M$ define its translation category $EM$ as 
follows: Its objects are the elements of $M$, its arrow set is 
$M\times M$, where $(s,t)\colon s\rightarrow t$ with composition
$(t,u)\circ (s,t) = (s,u)$ and identities $(u,u)$ for an object 
$u\in M$.
By definition each object is initial and terminal, hence the classifying
space of $EM$ is contractible for any $M$.

Moreover for $G$ a group we have a canonical action 
$EG\times G\rightarrow EG$ by the assignment $(s,t).g := (sg,tg)$.}}

We can by coherence of the pseudonatural isomorphisms in \ref{minieinf}
extend the map \[\Sigma_n\times_{\Sigma_n}\left(\coprod_{k_1+\ldots+k_n=N}
\C(A_+,k_1)\times \ldots\times \C(A_+,k_n)\right)\rightarrow \C(A_+,N)\]
over $E\Sigma_n$ as follows:
{\cor{The $n$-fold pairing
\[\mu\colon \Sigma_n\times_{\Sigma_n}\left(\coprod_{k_1+\ldots+k_n=N}
\C(A_+,k_1)\times \ldots \times \C(A_+,k_n)\right)
\rightarrow \C(A_+,N)\]
extends to a pairing
\[\mu^E\colon E\Sigma_n\times_{\Sigma_n}\left(\coprod\C(A_+,k_1)\times \ldots
\times \C(A_+,k_n)\right)\rightarrow \C(A_+,N).\]
\begin{proof}
Since we want to extend the assignment $\mu$, we can define $\mu^E$ at
each object of $E\Sigma_n$ by $\mu$. Locally, i.e., for each
arrow $(s,t)\in E\Sigma_n$, we set $\mu^E(\_,(s,t))$ to be the canonical
pseudonatural $1$-cells for $\mu_s\Rightarrow \mu_t$ as established
in \ref{minieinf}.

By analogy with \ref{quadtotrig} fill in the upper right triangle
with the pseudonaturality $2$-cell for the canonical pseudonatural
isomorphism $\mu_s\Rightarrow \mu_t$.

This assignment defines a normal functor (i.e., one pointed at 
identity $1$-cells) because $c^1_\cdot$ strictly squares to the identity
by \ref{mdlbic}. The compositor $2$-cells are coherent, because the
two diagrams:
\[\begin{array}{c}
\xymatrix{\mu^E(s,F_1,\ldots,F_n) \ar[r]\ar[d]\ar[dr] 
& \mu^E(s,G_1,\ldots,G_n) \ar[d]\\
\mu^E(t,F_1,\ldots,F_n) \ar[r]\ar[d]\ar[dr]&
\mu^E(t,G_1,\ldots,G_n) \ar[d]\\
\mu^E(u,F_1,\ldots,F_n) \ar[r]&
\mu^E(u,G_1,\ldots,G_n) }\\\\
\xymatrix{\mu^E(s,F_1,\ldots,F_n) \ar[r]\ar[dd]\ar[ddr] 
& \mu^E(s,G_1,\ldots,G_n) \ar[dd]\\\\
\mu^E(u,F_1,\ldots,F_n) \ar[r]&
\mu^E(u,G_1,\ldots,G_n) }\end{array}\]
with each upper right triangle filled by the pseudonaturality $2$-cells
express that there is a unique $\cdot$-symmetry from an $s$-permuted
input to an $u$-permuted input. In particular, the composite twist 
factored over a $t$-permuted input produces the same multiplicative
twist. Do note that all the other triangles are filled with identities,
including the ones expressing the equalities 
$E((t,u),\_)\boxempty E((s,t),\_)=E((s,u),\_)$, so that the above prism
with base a triangle degenerates to just three (potentially) 
non-trivial $2$-cells.\end{proof}}}

{\Remi{\label{esigmaop}Since the $\Sigma_*$-module $(E\Sigma_n)_n$ in 
fact is an operad, we have
a multiassociative, $\Sigma_*$-equivariant, and unital multicomposition:
\[E\Sigma_N\times E\Sigma_{k_1}\times \ldots\times E\Sigma_{k_N}
\rightarrow E\Sigma_{\sum_i k_i}.\]
I refer to this as \emph{block sum composition} as it is given by 
application of the functor $E$ to the multicomposition 
\[\Sigma_N\times \Sigma_{k_1}\times \ldots\times \Sigma_{k_N}
\rightarrow \Sigma_{\sum_i k_i},\]
which can be described as 
\[(\sigma, \tau_1,\ldots,\tau_N)
\mapsto \sigma\langle k_1,\ldots, k_N\rangle \circ (\tau_1\boxplus
\ldots\boxplus\tau_N)\]
for $\sigma\langle k_1,\ldots,k_N\rangle$ the permutation that permutes
the $N$ blocks of length $k_i$ by exchanging the blocks according to
$\sigma$, and $\boxplus$ a disjoint union functor on finite sets
as in \ref{Fin}.}}

{\prop{The $E\Sigma_*$-extensions of the pairings of 
delooping bicategories for a bipermutative bicategory $(\C,+,\cdot)$ of 
the above corollary make the following multiassociativity-diagram commute
\[\xymatrix{E\Sigma_N\times \left(\prod_i E\Sigma_{k_i}\right)
\times \left(\prod_j \C(A_+,l^i_j)\right)\ar[r]^-\cong\ar[d]
&E\Sigma_N\times \prod_i \left(E\Sigma_{k_i}\times \prod_j \C(A_+,l^i_j)
\right)\ar[d]\\
E\Sigma_{\sum_i k_i} \times \prod_i\prod_j \C(A_+,l^i_j)\ar[dr] & 
E\Sigma_N \times \prod_i \C(A_+,\sum_j l^i_j) \ar[d]\\
& \C(A_+,\sum_{i,j}l^i_j)}\]
for all natural numbers $N, k_i, l^i_j$.
\begin{proof} By the specific structure of $E\Sigma_*$ and the definition 
of the pairings $\mu_\sigma$ by equivariance, we can reduce to the case, 
where each object in $E\Sigma_*$ is the identity, which is just 
the strict associativity of the pairings. Since morphisms in the
$E\Sigma_*$ are uniquely determined by their source and target, and the
pseudonatural isomorphisms of \ref{minieinf} are coherent, this extends
to the morphisms as well.\end{proof}}}

I want to again suppress the operadic context for the $E_\infty$-structure 
and instead display what comprises the algebra structure of 
$\C(A_+,\bullet)$ over $E\Sigma_*$, including its coherences.

{\thm{\label{mymult} Given a bipermutative bicategory $(\C,+,\cdot)$ the resulting 
    pairing of delooping categories from the theorem above:
    \[\mu_{n,m}\colon \C(A_+,n)\times \C(A_+,m)\rightarrow \C(A_+,n+m)\]
    is $E_\infty$ in the following sense (cf. the definition of a commutative
    symmetric ring spectrum as in \cite[p. 9]{SchSym}, as well as
    \cite[pp. 66-68]{MayEinf}):
    \begin{itemize}
    \item It is strictly associative, i.e., we get a well-defined triple
    product for every $n,m,l\in\mathbb{N}$ as a strict identity of 
    strong normal functors: \[\mu_{n,m+l}\circ(\id\times\mu_{m,l})
    =\mu_{n+m,l} \circ(\mu_{n,m}\times \id).\]
    \item The functor $\{*\}\rightarrow 1\in\C\subset\C^+$ considered
    as an element of $\C(A_+,0)$ is a strict unit, turning $\mu_{0,n}
    =\mu_{n,0}=\id_{\C(A_+,n)}$ into a strict identity of functors.
    \item We have a natural central map $\iota_1\colon
    A_+\rightarrow \C(A_+,1)$ given by
    \[\iota_1(a)(p\colon A_+\rightarrow k_+) = \begin{cases}
    1, ~~& ~~ \mathrm{if~~} a\notin p^{-1}+,\\
    0, ~~& ~~ \mathrm{if~~} a\in p^{-1}+,\end{cases}\]
    with structural maps being given either by identities $0+0=0$ or
    $1+0=0+1=1$, hence strict identities. Centrality means that we
    have a strict equality of the functors
    \[A_+\times \C(A_+,n)\rightarrow \C(A_+,1)\times\C(A_+,n)
    \rightarrow \C(A_+,1+n)\]
    and
    \[A_+\times \C(A_+,n)\rightarrow \C(A_+,n)\times A_+
    \rightarrow \C(A_+,n)\times \C(A_+,1)\]
    \[\phantom{blaaaaaaaaaa}\rightarrow \C(A_+,n+1)
    \rightarrow \C(A_+,1+n),\]
    where the final arrow here is the action by $\chi_{1,n}$ which 
    shuffles the last input coordinate to first place without changing
    the order of the other inputs, and pushing forward the functor by
    $\chi$ so that it is a lift \ref{symmSpekt!!}. Do note that
    these maps describe the extension functors $A_+\wedge
    \C(A_+,n)\rightarrow \C(A_+,1+n).$
    \item For each two multiplications of $n$ inputs associated to
    two permutations $\sigma,\tau\in\Sigma_n$ there is a pseudonatural
    isomorphism:
    \[\xymatrix{\C(A_+,m_1)\times\ldots\times\C(A_+,m_n)
    \rrtwocell<7>^{\mu_\sigma}_{\mu_\tau}{~~C_{\overrightarrow{\sigma\tau}}} 
    && \C(A_+,m_1+\ldots+m_n). }\]
    \item The isomorphisms are coherent in that they compose 
    vertically as in $E\Sigma_n$: 
    $C_{\overrightarrow{\sigma\tau}}C_{\overrightarrow{\rho\sigma}}
    = C_{\overrightarrow{\rho\tau}}.$
    \item The isomorphisms have a block sum composition \ref{esigmaop}, 
    which is multiassociative and $\Sigma_*$-equivariant.
    \end{itemize} \begin{proof}
	All the claims are just summaries of the propositions above.
	Note that the coherence of the isomorphisms $C_\_$ follows from the
	coherence of the compositor $2$-cells for $\mu^E$, which itself
	follows from the coherence of the $\cdot$-symmetry for triple 
	products in $(\C,\cdot)$. \end{proof}}}

\section{The Symmetric Spectrum from the Delooping}
The passage from the delooping categories $\C(A_+,n)$ to the 
associated spectrum is fortunately very straight-forward. I fix a
pointed simplicial $\S^1$:

{\defn{\label{pointedS1} Consider the simplicial set 
	$\Delta_1=\Delta(\_,[1])$ and its simplicial subset of constant maps 
	$\partial\Delta_1$, then $\S^1:=\Delta_1/\partial\Delta_1$ is
	a pointed simplicial set.}}

Since the delooping bicategories $\C(A_+,n)$ are strict normal \ref{strictn} functors
in pointed finite sets we get the following:
{\thm{\label{insertS1}The delooping bicategories $\C(A_+,n)$ are strictly simplicial 
    bicategories by insertion of a simplicial set. In particular 
    $\C(\S^1,n)$ is a simplicial bicategory with $\Sigma_n$-action.

    By naturality of the extension functors and understanding $\S^1$
    as a discrete simplicial bicategory we get suspension maps as 
    strict functors of bicategories: \[\S^1\wedge\C(\S^1,n)\rightarrow
    \C(\S^1,1+n),\] which is $\Sigma_n$-equivariant, and assembles to 
    $\Sigma_m\times\Sigma_n$-equivariant maps \[\S^m\wedge\C(\S^1,n)
    \rightarrow \C(\S^1,m+n).\]

    For a bipermutative bicategory $\C$ the pairings from above assemble 
    into strictly associative pairings of simplicial bicategories:
    \[\C(\S^1,n)\wedge\C(\S^1,m)\rightarrow \C(\S^1,n+m),\] with two 
    strictly central unit maps \[\iota_0\colon \{*\}\rightarrow 
    \C(\S^1,0),~~~\iota_1\colon\S^1\rightarrow \C(\S^1,1), \]
    and each two permutations of higher multiplications connected by
    coherent simplicial pseudonatural isomorphisms.

    In summary: Set $HC_n=|N\C(\S^1,n)|,$ then the Eilenberg-Mac Lane
    spectrum $H$ to a permutative bicategory $\C$ inherits a natural 
    $E_\infty$-ring spectrum structure from a bipermutative structure
    on $\C$.}}

By applying the nerve and geometric realisation we get:
{\thm{\label{meinEinf}A bipermutative bicategory $\C$ yields an 
    $E_\infty$-symmetric ring spectrum, which is level-equivalent to 
    the spectrum defined in \cite{Os}. In particular it is 
    semi-stable, because it is equivalent to the symmetric spectrum of 
    a $\Gamma$-space. \begin{proof} By \ref{AeqzuOs} we know that 
    restriction of lifting functors along smashing $(\Aix)^n
    \rightarrow (A^n)_+\downarrow\Fi$ gives a natural equivalence 
    $\C((A_+)^{\wedge n},1)\rightarrow \C(A_+,n),$ which by inserting 
    $\S^1$ yields a level-equivalence: \[\C((\S^1)^{\wedge n},1)=
    \C(\S^n,1)\rightarrow \C(\S^1,n),\] so in particular on 
    realisation of nerves we get a map of symmetric spectra, 
    which is a level-equivalence. \end{proof}}}

\section{The Induced Involution}
We have already seen that an involution on bipermutative 
coefficients $\R$ induces a strictly additive functor on the
bicategories of matrices $\M\rightarrow \MM(\R^\mu)$. By transposition
we can remove the $\mu$-opposition, still strictly additively, but
at the expense of opposing $1$-cells and the tensor-product. 
Finally the comparison $B\C\cong B\C^{op_1}$ is what we need to 
study with respect to additivity and multiplicativity. 

{\lem{Given a bicategory $\C$ and its $1$-opposition, 
    i.e., with respect to $1$-cells, the homeomorphism 
    $\Gamma$ of their nerves is strictly natural with respect
    to functors $F\colon \C\rightarrow \D$. So we have a 
    commutative diagram:
    \[\xymatrix{|N\C|\ar[r]^\Gamma\ar[d]_{|NF|} & |N\C^{op}|
    \ar[d]^{|NF|}\\|N\D|\ar[r]^\Gamma & |N\D^{op}|.}\]
    $\hfill\Box$}}

In particular the homeomorphism is compatible with the pairings
established above.

{\cor{The pairing of delooping bicategories commutes with opposition:
    \[\xymatrix{|N\C(A_+,n)|\times|N\C(A_+,m)|\ar[d]
    \ar[r]^-{\Gamma\times\Gamma}
    &|N\C^{op_1}(A_+,n)|\times|N\C^{op_1}(A_+,m)|\ar[d]\\
    |N(\C(A_+,n)\times\C(A_+,m))|\ar[d]\ar[r]^{\Gamma} 
    &|N(\C^{op_1}(A_+,n)\times\C^{op_1}(A_+,m))|\ar[d]\\
    |N(\C(A_+,n+m))|\ar[r]^\Gamma
    &|N(\C^{op_1}(A_+,n+m))|.}\]
    \begin{proof}
    The only thing left to emphasise is that the homeomorphism 
    $|X|\times|Y|\rightarrow |X\times Y|$ is natural as well (given
    the compactly generated topology on the product).
    \end{proof}}}

{\rem{A minor warning is in order about the notation $\C^{op_1}(A_+,n)$.
    Since $(\C^{op_1})^+\neq (\C^+)^{op_1}$ this is potentially 
    ambiguous, thus I intend to mean the delooping bicategory of
    $\C^{op_1}$. }}

Since the induced involution is defined in \cite{Ri2010} and analogously
in \ref{indinv} as the composite:
\[\xymatrix{B\M \ar[r]^{B\mathcal{M}(T)} & 
        B\mathcal{M}(\mathcal{R}^\mu) \ar[r]^{B(\cdot)^t}  
        & B\mathcal{M}(\R)^{op_1} \ar[r]^{\Gamma} & B\M,}\]
and we have already established that $\Gamma$ strictly commutes with
the pairings, we only have to establish the effect of coordinatewise
involution and subsequent transposition. Both functors strictly
commute with the direct sum of matrices, thus induce functors
on the delooping bicategories, but in \ref{transmultopp} we saw that
transposition fully reverses the monoidal structure given by tensor
product. So we have to consider the following situation. For emphasis
I suppress the commutativity of the multiplication and call it a 
bimonoidal bicategory.

{\prop{Given a bimonoidal bicategory $(\C,+,\cdot)$ and its 
    multiplicative opposition $(\C,+,\circ)$ the induced 
    monoidal structure on $\C^+$ by $\circ$ is strictly
    naturally isomorphic to the opposite monoidal structure on
    $\C^+$ induced by $\cdot$.
    \begin{proof}
    By retracing the construction in \ref{bipzumult} we find on 
    objects that $\circ$ induces 
    \[(c_1,\ldots,c_n)\circ_*(d_1,\ldots,d_m)
    =(c_i\circ d_j)_{\omega(i,j)}=(d_j\cdot c_i)_{\omega(i,j)}.\]
    Thus the isomorphism is given by using the smash
    symmetry $\chi$ on $\Fi$ to exchange the 
    indices: \[\xymatrix{
    (c_1,\ldots,c_n)\circ_*(d_1,\ldots,d_m)=(d_j\cdot c_i)_{\omega(i,j)}
    \ar[r]^-{(\chi_{n,m},\id)}&(d_j\cdot c_i)_{\omega(j,i)}=d\cdot_*c
    =c\cdot_*^{op}d.}\] \end{proof}
}}

In particular we find that the involution and subsequent transposition
strictly oppose the multiplicative structure on the delooping 
bicategories.

{\cor{\label{indinvSP}
    The induced involution is a functor $I\colon (\C,+,\cdot)
    \rightarrow (\C^{op_1},+,\cdot^{op})$, thus composition with
    $\Gamma$ induces a map of $E_\infty$-spectra:
    \[\Gamma\circ|NI|\colon(H\C,\mu)\rightarrow (H\C,\mu^{opp}).\]}}
 
\chapter{Partial Uniqueness Results} \label{conjun}
Evidently, I am interested if the multiplication on the delooping 
I describe in chapter \ref{multbidel} describes a new structure or the 
known $E_\infty$-structure on the $K$-theory of an $E_\infty$-ring 
spectrum. 

Given this I can avoid the complicated constructions of the 
trace map described for example in \cite{BHMtr, S tr} (compare also 
\cite{DGM12}). By \cite{BGT2013} there is an essentially unique
map of symmetric multifunctors from $K$-theory to any other 
(symmetric) multifunctor $F$ that satisfies $F(\S)=\S$. For
topological Hochschild homology the equality $THH(\S)=\S$ is
immediate from the definitions, thus there is a unique $E_\infty$-map
$K\rightarrow THH,$ which by \cite[Theorem 1.9]{BGT2013} is the trace map.
Specifically the essential uniqueness proven in \cite{BGT2013} entails
that the trace map is a point in the space of $E_\infty$-maps $K\rightarrow THH$,
which is contractible.

However, for the multiplicative uniqueness I have to concede conjecture
status, while the additive uniqueness, i.e., the fact that there is
essentially only one delooping of the nerve of a symmetric monoidal 
category by its $E_\infty$-structure is classical \cite{MT1978}, 
which can be rephrased nicely with the results of \cite{GGN}.

\section*{Convention: The Language of $\infty$-Categories}
To conveniently state and prove results about uniqueness of 
$E_\infty$-structures we evidently need some organisational
language in which to compare them. As shown in \cite{EM} 
multicategories can be used in absence of 
symmetric monoidal structures on a category to define
operad-structures. However, comparison theorems in this setting seem 
too restrictive to expect. Since $E_\infty$-structures
are a coherently weakened notion of commutativity, we would
not expect different $E_\infty$-structures from potentially
different $E_\infty$-operads to be comparable by strict 
multifunctors. This would probably prove to be even worse for
general operads. 

Given the recent popularisation of $(\infty,1)$-categories for instance
by \cite{Bergn1, Bergn2, Joyal,JoyT,Lu1,Lu2} 
advancing the theory of quasi-categories as a convenient model for 
$(\infty,1)$-categories, the authors have made such comparison results easier
to state and prove in a satisfactory manner. Hence following my
principal sources \cite{GGN,BGT2013} for the uniqueness results
I use the language of $(\infty,1)$-categories, and refer to 
specific results about quasi-categories in \cite{Lu1, Lu2} where 
I need them. 
For a nice survey I specifically recommend \cite{Bergn2} and 
an accompanying talk \cite{Bergn3}.

\section{Uniqueness of the Spectrum $\R\mapsto H\R$}
I can directly (partially) quote this result from 
\cite{GGN} with the only alteration that, as in \cite{BDRR2011},
I refer to the associated spectrum of a permutative category 
$\C$ as its Eilenberg-MacLane-spectrum $H\C$, for instance given
by the delooping of \cite{EM}.
{\thm[First part of Prop. 8.2. in \cite{GGN}]{The 
Eilenberg-MacLane-spectrum functor $H\colon \SymMonCat \rightarrow Sp$ 
is lax symmetric monoidal.}}

This statement does not obviously incorporate $\infty$-categories. Their
use is implicit in the symmetric monoidal structure on $\SymMonCat$, which
probably does not exist, when we consider $\SymMonCat$ as a $1$-category 
with objects small symmetric monoidal $1$-categories and symmetric 
monoidal functors as morphisms -- cf. \cite{K} for a specific
construction of a product on symmetric monoidal categories with strictly
symmetric monoidal functors, which fails to have a unit object, and
does not produce bimonoidal/bipermutative categories as its monoids.
However, if we relax to considering $\SymMonCat$ as a $2$-category with
the same objects and $1$-cells, but additionally monoidal natural 
transformations as $2$-cells, we can see in \cite{Schm} that $\SymMonCat$
does in fact support the structure of a ``symmetric monoidal bicategory''. 
But his construction has the wrong ($E_\infty$) monoids, thus at most
serves as a proof that $\SymMonCat$ can equipped with a symmetric monoidal
bicategory structure at all. 

I strongly conjecture that a symmetric monoidal product on the 
bicategory of symmetric monoidal categories with the right monoids can 
be constructed:	Specifically the construction of a ``classifying 
pseudofunctor'' as in \ref{BCfunctor}, which for $(\C,+)$ symmetric 
monoidal gives a pseudofunctor $B\C\colon \Fi\rightarrow Cat$. The 
additive Grothendieck construction as in \ref{c+} is the Grothendieck 
construction over this functor. When we restrict $B\C$ to the source 
category with just surjections $\Ep$ however, the Grothendieck 
construction $\C^{+,epi}$  over this functor admits a natural adjunction 
\[\xymatrix{\C^{+,epi} \ar@<0.5ex>[r] & \ar@<0.5ex>[l] \C}\]
and thus the classifying spaces are homotopy equivalent. 
We can easily consider the product $B\C\times B\D$, and the Grothendieck
construction over this functor yields a candidate for a ``smash product''
of permutative categories. However I do not know, which of the specific
``averaging'' processes described by the Grothendieck construction 
yields the most convenient smash product. We could consider the
product $B\C\times B\D$ and pull back by the diagonal $\Fi\rightarrow
\Fi\times\Fi$. Alternatively we could directly consider the Grothendieck
construction restricted to $\Ep\times\Ep$. Since this becomes unwieldy 
quite fast, I have not established the properties this product might have.
I am quite sure it is unital and associative ``up to adjunction'', symmetric
up to isomorphism, however, one would have to establish the appropriate
coherences of adjunctions and isomorphisms.

The theorem quoted above reframes the results in
\cite{EM} that the associated Eilenberg-MacLane-spectrum is a symmetric
multifunctor from permutative categories to spectra, where in
\cite{EM} the multicategory-structure is precisely the concession needed
for the fact that the $1$-category of symmetric monoidal categories 
(probably) does not support a symmetric monoidal product.

The results of \cite{GGN} make it unnecessary to construct a symmetric
monoidal product on $\SymMonCat$ on the $2$-categorical level.

Recall from \cite{Lu2,GGN} that given an $\infty$-category, we have naturally
associated to it its category of pointed objects $\C_*$, its category
of $E_\infty$-monoids $\mathrm{Mon}_{E_\infty}\C$, modelled as $\Gamma$-objects
in $\C$, its category of grouplike $E_\infty$-monoids $\mathrm{Grp}_{E_\infty}\C$,
and its stabilisation, for instance given as the category of spectrum objects
in $\C$, denoted $Sp(\C)$. Moreover, each pointed object $c\in\C$ gives rise to a free $E_\infty$-monoid $hc$
given by $hc(n_+)=c^{\times n}$ with the evident maps induced by pointed maps in $\Fi=\Gamma$.
Group completion provides a free grouplike object associated to an arbitrary
$E_\infty$-monoid in the sense that it is a left-adjoint to the forgetful
functor in the opposite direction. Finally, a grouplike $\Gamma$-object gives
rise to an associated spectrum for instance by insertion of simplicial spheres 
$\S^n$ (cf. \ref{insertS1}).

{\thm[Theorem 5.1 \cite{GGN}]{Let $\C^\otimes$ be a closed symmetric
monoidal presentable $\infty$-category $\C$. The 
$\infty$-categories $\C_*$, $\mathrm{Mon}_{E_\infty}\C$, $\mathrm{Grp}_{E_\infty}\C$, 
$Sp(\C)$ all admit closed symmetric monoidal structures, which are uniquely
determined by the requirement that the respective free functors from $\C$
are symmetric monoidal. Moreover each of the functors \[\C_*\rightarrow 
\mathrm{Mon}_{E_\infty}\C\rightarrow \mathrm{Grp}_{E_\infty}\C\rightarrow 
Sp(\C)\] uniquely extends to a symmetric monoidal functor.}}

By \cite{GGN} the $\infty$-category of symmetric monoidal categories
thus has a symmetric monoidal $\infty$-category-structure by the following 
argument of Theorem 5.1 in \cite{GGN} specialised for $\C$ the $\infty$-category of 
$1$-categories with cartesian product as its closed symmetric monoidal structure.
The $\infty$-category of symmetric monoidal categories is identified 
as a smashing $\infty$-localisation with the $E_\infty$-monoids 
in spaces $\mathrm{Mon}_{E_\infty}(Top)$ \cite[Theorem 4.6]{GGN}. So they get 
\[\SymMonCat=\mathrm{Mon}_{E_\infty}(Cat_1)\simeq Cat_1\otimes 
\mathrm{Mon}_{E_\infty}(Top),\] 
where $\otimes$ denotes the tensor-product of presentable $\infty$-categories as
defined in \cite{Lu2}. Hence arguing the same way as for Bousfield
localisations of symmetric monoidal model categories, one only has to establish
that $\_\otimes \mathrm{Mon}_{E_\infty}(Top)$ respects local equivalences with
respect to $\mathrm{Mon}_{E_\infty}$. This is trivial for a smashing localisation 
in $\infty$-categories, because of the equivalence 
$\_\otimes \mathrm{Mon}_{E_\infty}(Top)\otimes 
\mathrm{Mon}_{E_\infty}(Top) \simeq \_\otimes \mathrm{Mon}_{E_\infty}(Top),$ 
which makes the smashing functor idempotent up to a chosen coherent
equivalence, hence a localisation. The 
equivalence $\mathrm{Mon}_{E_\infty}(\mathrm{Mon}_{E_\infty}(\_))\simeq \mathrm{Mon}_{E_\infty}(\_)$
is just an elaboration on the Eckmann-Hilton-argument, two 
compositions which are homomorphisms with respect to each other and 
have neutral elements are equal.

Replace in \cite[p. 19]{GGN} the expression ``algebraic $K$-theory'' with 
``Eilenberg-MacLane spectrum''; this 
identifies the Eilenberg-MacLane spectrum functor as the composition of functors
\[\xymatrix{\SymMonCat \ar[r]^{(\cdot)^{iso}} &\SymMonCat 
\ar[r]^{N(\cdot)} & \mathrm{Mon}_{E_\infty}(\mathcal{T})
\ar[r] & \mathrm{Grp}_{E_\infty}(\mathcal{T}) \ar[r] & Sp,}\]
for $\mathcal{T}$ some $\infty$-category of spaces
and $Sp$ modelled by any model category
of spectra as seen in \cite{MMSS} or directly by stabilisation of
an $\infty$-category of spaces as in \cite[Section 1.4.3]{Lu2}.
Additionally by \cite[p. 624]{Lu2} the $\infty$-category
$Sp$ admits an essentially unique (i.e., parametrised by a contractible
Kan complex) symmetric monoidal structure characterised by \begin{itemize}
\item[1.] $Sp\times Sp\rightarrow Sp$ preserves colimits in each argument,
\item[2.] the unit is the sphere spectrum $\S$. \end{itemize}
Even more drastically, we have by \cite[Corollary 6.3.2.19]{Lu2} that each
stable presentable $\infty$-category $\C$ with a symmetric monoidal product,
which preserves colimits in each argument, admits a unique (up to 
contractible choice) symmetric monoidal functor $Sp\rightarrow \C$ (in particular
pointed $\S\rightarrow 1_\C$), which preserves small colimits.

Thus $Sp$ is not only equipped with a unique symmetric monoidal structure, but
it is initial among stable presentable $\infty$-categories with
a symmetric monoidal product preserving colimits. Hence $Sp$ is unique 
with these properties.

Returning to the Eilenberg-MacLane spectrum we find that the functors
\[\xymatrix{\mathrm{Mon}_{E_\infty}(sSet) \ar[r] & \mathrm{Grp}_{E_\infty}(sSet) 
\ar[r] & Sp,}\] are uniquely symmetric monoidal by \cite[Theorem 5.1]{GGN}, 
thus we can reduce multiplicative uniqueness to the claim \[\xymatrix{\SymMonCat 
\ar[r]^{(\cdot)^{iso}} &\SymMonCat \ar[r]^{N(\cdot)} & 
\mathrm{Mon}_{E_\infty}(sSet),}\] is uniquely symmetric monoidal. Passing
from a category to its maximal subgroupoid $(\cdot)^{iso}$ is 
strictly symmetric monoidal on the level of $1$-categories, and it is the unique 
such structure on $\infty$-categories by \cite[Corollary 5.5 (i)]{GGN} 
on the subdiagram for $\C = Cat_1$ the $\infty$-category of small $1$-categories 
with cartesian product as symmetric monoidal structure and $\D = Gpd$ the 
$\infty$-category of small $1$-groupoids with the same symmetric monoidal structure:
\[\xymatrix{Cat_1\ar[d]_{(\cdot)^{iso}}\ar[r] & \mathrm{Mon}_{E_\infty}Cat_1
\ar@{-->}[d]\\Gpd \ar[r] & \mathrm{Mon}_{E_\infty}Gpd.}\]
Here the corollary precisely says that the unique symmetric monoidal structure
given on $\mathrm{Mon}_{E_\infty}\C$ for a symmetric monoidal $\infty$-category
$\C$ allows a unique symmetric monoidal extension of the dashed arrow in the diagram.
(Compare however Warning 5.2 in \cite{GGN}: The $E_\infty$-monoids are defined
with respect to the cartesian product. Their monoidal 
product is the extension of a potentially different monoidal structure on $\C$.)
The identification $\mathrm{Mon}_{E_\infty}Cat_1 = \SymMonCat$ gives uniqueness
for $(\cdot)^{iso}$.

The same argument for $\C = Gpd$ and $\D=sSet$ yields the unique extension of the
nerve $N(\cdot)$ to a symmetric monoidal functor, because the nerve is strictly
product-preserving, hence extends uniquely to a symmetric monoidal functor on
the respective $E_\infty$-monoids. Since the Eilenberg-MacLane spectrum $H\C$ has
zero-level $H\C(\S^0)=N\C,$ we find that it is this unique extension.

Thus by these arguments we get the following classical (cf. \cite{MT1978}) theorem:
{\thm{There is a unique functor extending the classifying space construction
    $|N\cdot|\colon Cat_1\rightarrow Top$ to their $E_\infty$-monoids, thus
    inducing a commutative diagram, where the horizontal arrows are the
    respective free functors: \[\xymatrix{ Cat_1\ar[d]_{|N\cdot|}\ar[r]& 
    Mon_{E_\infty}(Cat_1)\ar@{=}[r]\ar[d]&\SymMonCat\\Top \ar[r]&Mon_{E_\infty}(Top)
    \ar@{=}[r]&Top^\Gamma.}\]}}

Note that the equality below is by definition, while the equality above is the
identification of symmetric monoidal $1$-categories with $\Gamma$-objects in $1$-categories,
which can be found originally in \cite{Th1,Th2}.

\section{Conjectural Multiplicative Uniqueness}
The theorems of \cite{GGN} establish the functor $H$ as the unique
functor from symmetric monoidal groupoids to connective spectra with the
group completion property as classically identified in \cite{MT1978}. 

Apart from the claim in \cite{May2009} on page 321 in a footnote ``I now have 
a sketch proof that looks convincing.'' I could not find a result immediately 
stating the fact that the constructions in \cite{EM,May2009,GGN} each produce the 
same $E_\infty$-structure on $H\R$. Unfortunately I could not prove 
this uniqueness, i.e., that the multifunctor-structure 
of \cite{EM} is essentially the only multifunctor-structure on $H(\cdot)$. I suspect
the essential uniqueness can analogously to \cite{BGT,BGT2013,GGN} be specified to
say the space of multifunctor-structures on $H(\cdot)$ is actually contractible.

The following elaboration on \cite[p. 19]{GGN} would identify ``my'' 
$E_\infty$-structure on $H\M$, as described in the previous chapter, 
with the $E_\infty$-structure on the Eilenberg-MacLane spectrum 
of the category of matrices with coefficients in the Eilenberg-MacLane
spectrum $H(\MM(H\R))$ given by using \cite{EM} twice.

In the sequence
\[\xymatrix{\SymMonCat \ar[r]^{(\cdot)^{iso}} &\SymMonCat 
\ar[r]^{N(\cdot)} & \mathrm{Mon}_{E_\infty}(\mathcal{T})
\ar[r] & \mathrm{Grp}_{E_\infty}(\mathcal{T}) \ar[r] & Sp,}\]
the first, third and fourth functor admit unique multifunctor-structures by \cite{GGN}. 

Thus the appropriate conjecture establishing multiplicative uniqueness of $H$ is:
{\conj{The passage from symmetric monoidal categories to $\Gamma$-spaces has a unique 
    multifunctor-structure.}}

I have tried two avenues, which can probably be made to work:
Given the symmetric monoidal structures on $\Gamma$-spaces, and the
unique symmetric monoidal structure on $\SymMonCat$ extending the product on $Cat$
asserted by \cite{GGN}, one should be able to make the identification 
of $Mon_{E_\infty}(N\cdot)\colon \SymMonCat\rightarrow Mon_{E_\infty}(sSet)$ 
as the tensor-unit in a symmetric monoidal sub-$\infty$-category of the functors 
$Fun(\SymMonCat,Mon_{E_\infty}(sSet))$, but I simply do not know how to approach
finding the appropriate subcategory systematically.

Furthermore each nerve $N$ is defined as a right adjoint.
Specifically choose a cosimplicial category. Then functors from this
category associate a simplical object to categories. Considering 
instead $Ex^2\circ N$ one can promote this to the right adjoint
in a Quillen equivalence of categories with the Thomason model 
structure to simplicial sets in the Kan model structure \cite{Th3}.

But the symmetric monoidality established in \cite{GGN} is canonical only for 
left-adjoint functors, thus the nerve is not easily a 
canonical extension. This might feel trivial, but since the canonical 
structures in \cite{GGN} are established by specific 
left-adjoint functors, their compatibility with 
right-adjoints is not straight-forward.

This is parallel to the transfer of model structures: Presentable $\infty$-categories 
are nerves of combinatorial simplicial model categories. In particular these 
are cofibrantly generated with generating (trivial) cofibrations satisfying 
compactness with respect to their category. The dual notion of 
``cosmall/cocompact'' turns out to be too trivial. For instance 
in $Sets$ the only cosmall objects are the empty set and any one-point-set. 
Thus most categories of the type ``sets with structure'' do not have many candidates for 
generating (trivial) fibrations, much less a model structure determined by them. 
On presentable categories we see this on objects: The opposite of a presentable
category is not presentable in general.

In conclusion: The first avenue is quite ambitious, since in particular it
yields a comparison of the $E_\infty$-structures produced by \cite{EM,May2009,GGN}.

The modest approach runs into a similar problem, 
i.e., a map in the ``wrong'' direction. In principle it 
should not be too hard to produce an $E_\infty$-map of symmetric 
spectra $H(\M)\rightarrow H(\MM(H\R))$ with the $E_\infty$-structures of chapter \ref{multbidel} 
and \cite{EM}. I modelled the delooping in chapter \ref{multbidel} explicitly 
as a careful generalisation of \cite{EM}. It would be 
conceptual to pass through strictification to $2$-categories:
The canonical monoidality map of strictification
is an inclusion $(\C\times\D)^{st}\rightarrow \C^{st}\times\D^{st}$ given by
including words of pairs to pairs of words of equal length. It
is an equivalence. Any inverse would need to be 
coherently associative and symmetric with respect to the product, which
is not possible without introducing an adequate 
tricategory-structure on permutative bicategories. 
This hinges on the fact that the canonical map is only pseudonatural, 
and the anchor equivalence $\C^{st}\rightarrow \C$ is only lax 
monoidal up to an invertible transformation. 

Consider instead strictification by Yoneda embedding into $Fun(\C^{op},Cat)$. 
This produces a similar problem in addition to several new ones. Canonically we get a
map in the wrong direction. One could try to resolve this by a Day-type
convolution, but this introduces new problems. What is a small
diagram over which to take the colimit? In addition the target 
$2$-category as well as the source tricategory make a variety
of lax colimits conceivable. Finally given a Day-type convolution, 
does it make the Yoneda-embedding an appropriate multifunctor?

Another problem, which I assume is much simpler to resolve, is the fact
that strictification cannot respect the functor-strictnesses of
addition and tensor product the way I axiomatised it for permutative and
symmetric monoidal bicategories. Thus even if one could pass to a 
bimonoidally equivalent $2$-category, the result would not have addition
and tensor product given by strict $2$-functors. Thus some clever 
construction of an equivalent $2$-category, for which these functors are
strict would yield a canonical map to the construction of \cite{EM}, but I did not 
find such a $2$-category. So here the essential question is how a 
bipermutative bicategory can strictify to a bipermutative $2$-category, 
thus a particular example of a simplicially enriched bipermutative 
$1$-category suited to the machine of \cite{EM}.

Finally, I want to state a guess, why these problems arise: 
$K$-theory constructions involve passing to categories of isomorphisms first. 
By \cite[Proposition 8.14]{GGN} there is a structural reason for this: 
Group completion adds objectwise monoidal inverses,
but consequently turns every morphism into an equivalence by an Eckmann-Hilton-type argument.
I have not found a good way to impose
invertibility of morphisms productively, but it has to enter in an essential way.

Furthermore this seems to conflict heavily with the principal example $\M$. There the equivalence
$1$-cells are automatically isomorphisms, since the $2$-cells are only products
of isomorphisms. For instance for $Ob\R = \mathbb{N}$ we only get permutation
matrices as equivalence $1$-cells, which only incorporates the endomorphism-objects 
$\R(0,0)$ and $\R(1,1)$ as $2$-cells. This ``delooping'' does not
satisfy the comparison of \cite{BDRR2011}, i.e., it does not deloop the appropriate
space in order to be equivalent to $K(H\R)$. I have no idea how to resolve this
conflict. I have to concede however
that the conflict might be my illusion, and \cite[Proposition 8.14]{GGN} only enforces
isomorphism $2$-cells in $\M,$ which is consistent with the assumption in
\cite{BDRR2011}. In particular this perspective makes the assumption in
\cite{BDRR2011} that each translation functor $X\oplus\_$ be a faithful functor
appear as an inessential peculiarity of the Grayson-Quillen-completion,
while discarding all non-invertible morphisms is essential to the construction.
 \chapter{$THH$ and the Trace Map}\label{THHundtr}
Since my construction of the delooping of a bimonoidal bicategory 
naturally yields a symmetric spectrum and I do not want to switch 
contexts too much, I refer the reader to \cite{ShTHH} for Topological 
Hochschild Homology in symmetric spectra, as well as
\cite{AnR} for a careful study of the algebraic 
structure present for nice input spectra. I also want to mention 
the nicely model independent paper \cite{BFV}, which states 
in general that $THH$ of an $E_n$-ring spectrum is an $E_{n-1}$-ring 
spectrum by careful analysis of the involved operads. They do not 
need to fix their context of a model category of spectra, since any 
category tensored over topological spaces or simplicial sets will do, 
and they all are.

The generalisation of Hochschild homology to ring spectra by way of ``functors
with smash products'' is due to B\"okstedt and part of a paper, which is
notorious for its preprint-stage \cite{B1}. It is however highly influential, in particular,
since B\"okstedt subsequently fully calculated $THH(\Z/p)$ as well as 
$THH(\Z)$ in \cite{B2}, he provided the foundations for many subsequent 
calculations of topological Hochschild homology.

\section{$THH$ with Coefficients}
The algebra $V(1)_*THH(ku)$ that Christian Ausoni describes in 
\cite{AuTHH} is quite unwieldy --- in particular it is of good
use to have descriptions of easier objects with clear relations
to $THH(ku)$ the way he describes it in \cite{AuTHH}. 
By introducing the appropriate analog of logarithmic structures 
on ring spectra, Sagave and Schlichtkrull provided a localisation 
sequence, which makes the extension $\ell\rightarrow ku$ tamely 
ramified in a well-defined way \cite{SaSch14}. In particular, they 
describe the algebra $V(1)_*THH(ku)$ as a square-zero extension 
of the $V(1)$-homotopy of $THH^{log}(ku)$. However, since I 
introduce $THH$ purely for analysis of $V(1)_*K(ku)$ by trace 
methods along the lines of \cite{AuTHH,AuKku}, the introduction of 
logarithmic structures and the resulting cofibre sequence on 
topological Hochschild homology seems a drastic detour, which 
I do not want to present here.

Even before it was known that there are symmetric monoidal categories
of spectra, people have studied $THH$ in that then hypothetical 
category \cite{MS} or like B\"okstedt in auxiliary categories, which
turned out to be equivalent to the model categories of spectra \cite{MMSS}. 
In particular, given the construction of Hochschild homology 
for discrete algebras as presented in \cite{Loday} for instance, 
the generalisation to spectra is straightforward. 

{\defn{\label{THHblubbAM}\cite[Definition IX.2.1]{EKMM}
Given an associative $\S$-algebra $A$ with product $\mu\colon 
A\wedge A\rightarrow A$ and unit $\eta\colon\S \rightarrow A$, and an
$A$-bimodule $M$ with actions $M\wedge A \rightarrow M$ and
$A\wedge M\rightarrow M$, which I denote by $\mu$ as well, let 
$THH_\bullet(A,M)$ be the following simplicial spectrum: In degree $n$ 
we have the $(n+1)$-fold smash product:\[THH_n(A,M):=M\wedge A^{\wedge 
n},\]with face maps\[d_i\colon M\wedge A^{\wedge n}\rightarrow M\wedge
A^{\wedge n-1}\]defined as\[d_i=\begin{cases}id^{\wedge i-1}\wedge\mu
\wedge id^{\wedge n-i}&0\leq i\leq n\\(\mu\wedge id^{n-1})\circ t & 
i=n+1,\end{cases}\]for $t$ the symmetry of the smash product that 
exchanges factors as follows:\[\xymatrix{M\wedge A_1\wedge A_2\wedge
\ldots\wedge A_n \ar[r]^-t& A_n\wedge M\wedge A_1\wedge A_2\wedge
\ldots\wedge A_{n-1}.}\] The degeneracies are given by insertion of 
units at all places after the module $M$: $s_i\colon M\wedge A^{\wedge
n}\rightarrow M\wedge A^{\wedge n+1}$ for $0\leq i\leq n$:\[s_i=id^{i
+1}\wedge \eta \wedge id^{n-i},\] where I have notationally suppressed
the unit isomorphism $A\cong \S\wedge A.$ Call this the simplicial 
Hochschild spectrum of $A$ with coefficients in $M$.

For $M$ the algebra itself we set $THH_\bullet(A):=THH_\bullet(A,A)$ and call the
resulting simplicial spectrum the simplicial Hochschild spectrum of $A$.}}

This construction can be defined for arbitrary $\S$-algebras $A$ and $A$-bimodules
$M$. However, for it to be of topological significance, in particular for 
$THH_\bullet(A,M)$ to be homotopy invariant, we impose a technical condition.

{\rem{If the unit of the algebra $\S\rightarrow A$ is a cofibration in the
    model structure on a chosen model category of spectra, and $A$ and $M$
    are cofibrant $\S$-modules, the simplicial spectrum $THH_\bullet(A,M)$ 
    is \emph{proper} \cite[Theorem VII.6.7]{EKMM}. This means that for each simplicial
    degree $n$ the inclusion \[sTHH_n(A,M)\rightarrow THH_n(A,M)\] of $sTHH_n(A,M)$ the
    image of all degeneracies with target degree $n$ is a cofibration \cite[p. 182]{EKMM}.

    This in particular implies that a weak equivalence $M\rightarrow M'$ of
    $A$-bimodules and a weak equivalence $A\rightarrow A'$ of algebras, with $M'$ and
    $A'$ again cofibrant $\S$-modules with the unit $\S\rightarrow A'$ a cofibration,
    gives rise to a levelwise weak equivalence $THH_\bullet(A,M)\rightarrow 
    THH_\bullet(A',M')$, and since both spectra are proper this induces an
    equivalence on the realisations as well.\label{cofibTHH}}}

For a nicely short exposition of this compare to section 7 of \cite{BLPRZ}.

{\defn{Let $A$ be an associative $\S$-algebra, which is a cofibrant
$\S$-module, and for which the unit $\S\rightarrow A$ is a cofibration. Let
furthermore $M$ be an $A$-bimodule, which is a cofibrant $\S$-module. The
\emph{topological Hochschild homology} of $A$ with coefficients in $M$ 
is defined as the realisation of the (proper) simplicial
spectrum $THH_\bullet(A,M)$. To give this unambiguous meaning, understand
this as the coend
\[THH(A,M):=|THH_\bullet(A,M)|=\int^\Delta THH_q(A,M)\wedge (\Delta_q)_+,\]
where we use the tensored structure of a model category of spectra
over topological spaces to form $\_\wedge (\Delta_q)_+$ and then
form the coend.

Analogously, with $A=M$ define $THH(A):=|THH_\bullet(A)|=|THH_\bullet(A,A)|.$}}

We care how multiplicative opposition affects this construction
and find that it opposes the simplicial structure as expected.
{\rem{For an associative $\S$-algebra $A$ denote its multiplicative opposition $A^\mu$.
    In particular, I do not want to use the notation $A^{op}$, since this notation
    implies the wrong idea for the Eilenberg-MacLane-spectrum of a bipermutative
    (bi)category $H\C$: No opposition of morphisms is involved, only the symmetry
    of the smash product in spectra.}}

{\prop{(cf.\cite[5.2.1]{Loday}, \cite[Chapter 5]{L2011}) 
For an $\S$-algebra $A$
and an $A$-bimodule $M$ we see that simplicial
opposition on the Hochschild spectrum is isomorphic to multiplicative opposition, i.e.
\[\iota\colon THH_\bullet(A^\mu,M^\mu)\cong \widetilde{THH}_\bullet(A,M)\]
where on the left we oppose the multiplication on $A$ and exchange
the left- and right-action on $M$, while on the right we reverse
the simplicial direction as in \ref{catoppnerv}, i.e., $\widetilde{THH}_\bullet(A,M)
= THH_\bullet(A,M)\circ r$ for $r\colon \Delta\rightarrow \Delta$ the reversion
functor. \begin{proof}
The relevant isomorphism in discrete algebra reads 
$m\otimes a_1\otimes\ldots\otimes a_n\mapsto  m\otimes a_n\otimes \ldots\otimes a_1.$ 
This can be generalised to spectra by the appropriate twists of the smash product.\end{proof}}}

The lemma \ref{simpoppTop} literally applies, because we used the 
tensored structure over $\mathit{Top}$. This gives the following isomorphism
of spectra: {\prop{We have the identification given by reversing
the simplex coordinates in realisations:
\[\Gamma\colon|\widetilde{THH(A)}|\rightarrow |THH(A)|.\]}}

Given these two structural maps, we can easily induce an involution 
on $THH$ of a ring spectrum with anti-involution.
{\defn{\label{THHiota} Let $(A,\mu,\tau)$ be an associative $\S$-algebra with 
anti-involution $\tau$, i.e., a self-inverse $\S$-algebra map:
$\tau\colon (A,\mu)\rightarrow (A,\mu^{opp})$. Then we call the following 
sequence of maps: \[\xymatrix{THH(A)\ar[r]^\tau & THH(A^\mu)
\ar[r]^\iota & \widetilde{THH(A)} \ar[r]^\Gamma & THH(A)}\] the induced 
involution of $\tau$ on $THH(A).$}}

\section{The B\"okstedt Spectral Sequence}
The B\"okstedt spectral sequence, calculating topological from algebraic
Hochschild homology, is the essential tool B\"okstedt uses in \cite{B2}
to calculate $THH(\Z)$ and $THH(\Z/p)$. For a published reference see 
\cite[Theorem IX.2.9]{EKMM}.

The induced maps of $\tau$ and $\iota$ are simplicial by the results
before, so for the spectral sequence associated to the simplicial 
filtration, which yields the B\"okstedt spectral sequence under 
flatness assumptions, we can deduce the following result. 
For technical convenience assume we have arranged for $THH_\bullet(A,M)$
to be proper by the conditions mentioned above in Remark \ref{cofibTHH}.
{\thm{Let $h$ be a generalised homology theory, then the simplicial 
filtration of $|THH(A)|$ yields a spectral sequence \[E^1_{*,n}=
C^{cell}_*(THH_n(A),THH_{n-1}(A),h_*)\Rightarrow h_*(THH(A)).\]If $h$ 
satisfies the K\"unneth-formula on $A$, i.e., $h_*(A\wedge A)\cong 
h_*(A)\otimes_{h_*(\S)} h_*(A)$, then we can identify the $E^2$-term 
with the algebraic Hochschild-homology of $h_*(A)$, i.e.: \[E^2\cong 
HH(h_*(A)).\]

The induced involution given above is compatible with the simplicial
filtration, thus induces a map of spectral sequences.
In particular we find on $E^2$-terms: \[\xymatrix{HH(h_*A)
\ar[r]^-\tau & HH(h_*(A^\mu)) = HH((h_*A)^\mu) \ar[r]^-{\iota} & 
\widetilde{HH}(h_*A)\ar[r]^{\Gamma} & HH(h_*A).}\]}}

{\rem{\label{Gammasign}
Do note that $\tau$ and $\iota$ can be induced on the simplicial
level, while $\Gamma$ is a map of chain complexes given by 
introducing the adequate sign associated to the map 
\[\Delta^n/\partial\Delta^n \rightarrow \Delta^n/\partial\Delta^n\]
with $[t_0,t_1,\ldots,t_n]\mapsto [t_n,t_{n-1},\ldots,t_0],$
which is given by the sign of the permutation that fully inverts
the set $\{0,1,\ldots,n\}$, i.e.,\[(0~ n)(1~ n-1)\ldots (\floor*{\frac{n}2}
\ceil*{\frac{n}2}),\] which is $(-1)^{\frac{n(n+1)}2}.$}}

The B\"okstedt spectral sequence generalises to the case with 
coefficients in a bimodule as well by the same filtration:
{\thm{Let $h$ be a generalised homology theory, $A$ an associative
$\S$-algebra, $M$ an $A$-bimodule; additionally assume that $h$ 
satisfies the K\"unneth-isomorphisms $h_*(A\wedge M) = h_*(A)
\otimes_{h_*(\S)}h_*(M),$ $h_*(M\wedge A)=h_*(M)\otimes_{h_*(\S)}h_*(A),$ and
$h_*(A\wedge A)=h_*(A)\otimes_{h_*(\S)}h_*(A)$, then we have a B\"okstedt
spectral sequence of the form: \[HH(h_*(A),h_*(M))\Rightarrow h_*(THH(A,M)).
\] If additionally for $M_*=h_*(M)$ and $A_*=h_*(A)$ the algebra
$A_*$ is projective over $k=h_*\S$ then we can understand the $E^2$-term
as a derived functor: \[Tor^{A_*\otimes_k{A_*}^{op}}(M_*,A_*).\] }}

{\rem{I have suppressed convergence discussions in spectral sequences,
because the homology theories and spectra I consider only 
give modules and algebras with non-negative grading, thus we have 
strongly convergent first quadrant spectral sequences.}}

\section{The Multiplicative Structure of the Involution}
\subsection*{Structural Example: The Product for the Commutative Case}
Most sources discuss the product on Hochschild homology for the chain
complex associated to the simplicial module formed by the Hochschild
complex (cf. \cite[Lemma 1.6.11]{Loday} or \cite[Theorem VIII.8.8]{McL}). 
However this introduces the complications associated to dealing with
the sum of shuffles in the Eilenberg-Zilber map, so I prefer the simplicial 
structure of the product in algebraic modules as a template for spectra.

{\rem{For $A$ a commutative $k$-algebra let $M,N$ be $A$-modules and 
    consider the simplicial modules $C_n(M,A)=M\otimes A^{\otimes n}$ and
    $C_n(N,A)= N\otimes A^{\otimes n}$. In general we have the pairing
    $M\otimes A^{\otimes n}\otimes N\otimes A^{\otimes n} \rightarrow 
    M\otimes N\otimes (A\otimes A)^{\otimes n},$ which is given by 
    reordering the tensor factors according to the two-rail rail 
    fence cipher, i.e., by pairing $M$ with $N$ and the $i$th $A$-factor
    of the first $A^{\otimes n}$ with the $i$th $A$-factor in the second
    $A^{\otimes n}$. This assembles to a simplicial isomorphism:
    \[C_\bullet(M,A)\otimes C_\bullet(N,A)\rightarrow C_\bullet(M\otimes N,
    A\otimes A).\]

    Hochschild homology is a functor in the module variable for arbitrary maps
    linear over the appropriate algebra. Furthermore it is a functor in the
    algebra argument for algebra maps. For commutative $A$ we have that
    $\mu\colon A\otimes A \rightarrow A$ is an $A$-linear map of $A$-modules 
    as well as a map of $k$-algebras, thus for $M=N=A$ we get a map
    \[C_\bullet(A,A)\otimes C_\bullet(A,A)\rightarrow C_\bullet(A\otimes A,
    A\otimes A) \rightarrow C_\bullet(A,A).\]

    Analogously, we can define for $A$ a commutative $\S$-algebra a pairing
    of the simplicial spectrum $THH_\bullet A$ with itself as
    \[THH_\bullet(A)\wedge THH_\bullet(A)\rightarrow THH_\bullet(A\wedge A)
    \rightarrow THH_\bullet(A).\]

    Since this map is simplicial, it is obviously compatible with the filtration
    giving the B\"okstedt spectral sequence, and thus introduces the structure
    of a spectral sequence of differential graded algebras.}}

{\rem{To use \cite{BGT} I need to diverge into spectral categories, i.e., categories
    enriched over a symmetric monoidal model category of spectra: 
    The above discussion applies to spectral categories as follows. Consider 
    two small spectral categories $\C,\D$, and set their smash product to be 
    the category with objects $Ob\C\times Ob\D$ with smash-product on morphism spectra 
    $\C\wedge \D((c_1,d_1),(c_2,d_2))=\C(c_1,c_2)\wedge\D(d_1,d_2).$ 

    For a small spectral category with cofibrant morphism spectra $\C$ define
    its simplicial Hochschild spectrum as (compare \cite[p. 73]{BGT2013}):
    \[THH_q(\C):=\bigvee \C(c_{q-1},c_q)\wedge \C(c_{q-2},c_{q-1})\wedge 
    \ldots\wedge \C(c_0,c_1)\wedge \C(c_q,c_0).\]
    Take special note of the last factor, making it a ``circle of degree $q$'' in
    $\C$. The analogous face- and degeneracy-maps make it a simplicial spectrum,
    call its realisation the topological Hochschild homology of $\C$.

    The indicated rail-fence shuffle above induces an isomorphism of simplicial 
    spectra: \[THH_\bullet(\C)\wedge THH_\bullet(\D)\rightarrow THH(\C\wedge
    \D),\] which by coherence of the symmetry of $\wedge$ in spectra is strongly 
    symmetric monoidal in the sense that the following diagram strictly 
    commutes: \[\xymatrix{THH_\bullet(\C)\wedge THH_\bullet(\D)\ar[r]
    \ar[d]_{c_\wedge} & THH_\bullet(\C\wedge\D)\ar[d]^{c_\wedge}\\
    THH_\bullet(\D)\wedge THH_\bullet(\C)\ar[r] & THH_\bullet(\D\wedge\C).}\]

    Since realisation and smash product commute by naturality of the product
    isomorphism $|X\times Y|\cong |X|\times|Y|$ in compactly generated Hausdorff 
    spaces for $X,Y$ simplicial spaces, this descends to the same 
    multifunctoriality after realisation.}}

{\prop{Consider the category of small spectral categories $Cat_{Sp}$ enriched
    over a chosen symmetric monoidal model category of spectra. By \cite{Tab}
    we know that by cofibrantly replacing a small spectral category in the
    model structure on $Cat_{Sp}$ we obtain a weakly equivalent small 
    spectral category with cofibrant morphism spectra. 

    On the category of small spectral categories with cofibrant morphism 
    spectra consider topological Hochschild homology $THH_\bullet(\cdot).$
    This is a strong symmetric monoidal and simplicially enriched functor
    \[THH_\bullet(\cdot)\colon Cat_{Sp}\rightarrow sSp.\] In particular, 
    it is a multifunctor by considering the multicategory-structures
    induced by the symmetric monoidal products on $Cat_{Sp}$ and $sSp$.

    By passing to realisations we get a strong symmetric monoidal and
    simplicially enriched functor, hence a multifunctor: \[THH(\cdot)\colon
    Cat_{Sp}\rightarrow Sp.\]

    \begin{proof}The structure of a multifunctor on $THH$ is an elaboration on 
    the transformation $THH(\C)\wedge THH(\D)\rightarrow THH(\C\wedge\D),$ which
    is strictly symmetric, strictly unital, and coherently associative, hence 
    yields a multifunctor of the described type.\end{proof}}}

{\rem{I strongly recommend the preprint of Bj\o rn Ian Dundas establishing multifunctoriality
    of topological cyclic homology as well, and moreover proving that the trace maps are
    also natural transformations for these multifunctor-structures \cite{D2}.}}

In absence of an explicit symmetric monoidal product on symmetric monoidal
categories (cf. however \cite{GGN} for a monoidal structure on the $\infty$-category
of symmetric monoidal categories) the authors in \cite{EM} describe a
multicategory-structure on permutative categories instead, which has bipermutative
categories as its $E_\infty$-monoids, and establish that the Eilenberg-MacLane spectrum
of permutative categories (which \cite{EM} call ``$K$-theory'') can be
given the structure of a multifunctor. Thus by the natural equivalence $K(H\R)=
H(\M)$ established in \cite{BDRR2011} we can consider an
induced multifunctor-structure on $H(\MM_{\_})= H\circ \MM_{\_}.$
In particular, in absence of a comparison of the multiplicative structure
I describe in chapter \ref{multbidel}, I refer to the $E_\infty$-structures induced
by the multifunctor-structure as \textbf{the} $E_\infty$-structure. 

I summarise the appropriate identifications of \cite{BDRR2011,BGT} in the following 
theorem -- to summarise the known results.

{\thm{The algebraic $K$-theory space $B\M$ of a permutative category $\R$ is naturally 
    equivalent to the algebraic $K$-theory space of its associated Eilenberg-MacLane
    spectrum $K(H\R)$ \cite{BDRR2011}. 

    The algebraic $K$-theory functor from small spectral categories to 
    spectra can be given the structure of a (symmetric) multifunctor in an 
    essentially unique way \cite[Theorem 1.5]{BGT}.

    There is an essentially unique natural transformation
    of (symmetric) multifunctors $K\Rightarrow THH,$ which by \cite[Theorem 6.3]{BGT}
    is the \emph{trace map} from algebraic $K$-theory to topological Hochschild
    homology \cite[Theorem 1.9]{BGT}. Thus in particular we get a unique natural 
    multiplicative map $K(H\R)\rightarrow THH(H\R)$ for Eilenberg-MacLane spectra.}}

{\rem{As seen in \ref{conjun} I have to concede that I do not know if the uniqueness
    of \cite{BGT} forces the $E_\infty$-structure of chapter \ref{multbidel} on $H(\M)$ 
    to agree with the canonical structure on $K(H\R)$ asserted by
    \cite{BGT}. If, however, one were able to prove that the Eilenberg-MacLane spectrum
    functor admits an essentially unique multifunctor-structure, or more modestly
    to produce $E_\infty$-equivalences $K(H\R)\rightarrow H\M$ and 
    $H\M\rightarrow K(H\R)$, then either of these would imply that $H\M$ has 
    the unique multifunctor-structure given by composition of
    the unique structure on $K\colon Cat_{Sp}\rightarrow Sp$ and the conjecturally
    unique structure on $H\colon PermCat\rightarrow Sp$, while the second approach
    obviously directly gives the claimed equivalence.}}

Since I introduce the trace map $K\rightarrow THH$ by its multiplicative
universality as proven in \cite{BGT}, I want to emphasise the equivalence of 
$E_\infty$ and commutative structures in most model categories of spectra.

{\rem{\label{commeinf}
    In orthogonal and symmetric spectra with their positive stable model
    structure as well as in the category of $\S$-modules we have that commutative
    ring spectra model all $E_\infty$-ring spectra -- \cite[Lemma 15.5]{MMSS} 
    and also \cite[Theorem 5.1, Chapter III]{EKMM}. However, since I already use the
    reference \cite{EM} often in preceding chapters, I follow the setup of their
    Theorem 1.4; in particular I restrict to the case of symmetric and orthogonal
    spectra in the positive stable model structure for this remark.

    Considering the multicategory-structure on symmetric (or orthogonal) spectra 
    induced by the smash-product, it is meaningful to speak of multifunctor-categories
    $MFun(\mathbb{M},Sp^\Sigma),$ where we consider a simplicially enriched
    multicategory $\mathbb{M}$ and symmetric spectra with their natural simplicial
    mapping spaces. Given an enriched multifunctor $f\colon \mathbb{M}'\rightarrow
    \mathbb{M}$ we have an induced restriction functor: \[f^*\colon 
    MFun(\mathbb{M}',Sp^\Sigma)\rightarrow MFun(\mathbb{M},Sp^\Sigma)\]
    by precomposition with $f$, which by \cite[Theorem 1.4]{EM} is the right adjoint
    in a Quillen adjunction -- with the left-adjoint given by extension in the 
    appropriate manner (cf. p.56 of \cite{EM}).

    If $f$ is an equivalence of simplicially enriched multicategories, i.e., $\pi_0f$ is
    an equivalence of ordinary $1$-categories, and for each set of objects we have a
    weak equivalence of simplicial sets $M(a_1,\ldots,a_n;b)\rightarrow M'(fa_1,\ldots,fa_n;fb),$
    then Theorem 1.4 of \cite{EM} furthermore yields that this adjunction is a Quillen equivalence 
    (compare also \cite{Bergn1}). To my knowledge it has not been established that these
    equivalences of multicategories are weak equivalences in a model structure on 
    small multicategories, which preferably would extend the Bergner model structure 
    on simplicially enriched categories.

    When we consider the Barratt-Eccles operad $E\Sigma_*$ as a one-point 
    multicategory and map it to the terminal multicategory $Com$, we have
    an underlying isomorphism of (one-point) $1$-categories. Since each 
    multimorphism-category
    of $E\Sigma_*$ is equivalent to the one-point category we get a weak equivalence 
    of its nerve to a point as well, giving a Quillen equivalence:
    \[\xymatrix{\mathbb{P}\colon MFun(E\Sigma_*,Sp^\Sigma)\ar@<0.5ex>[r] & MFun(Com,Sp^\Sigma)
    \ar@<0.5ex>[l] \colon U,}\] where I consider the functors given by extension and 
    restriction as a prolongation functor $\mathbb{P}$ and a forgetful functor $U$. In particular
    for a cofibrant $E_\infty$-ring spectrum $A$ in $Sp^\Sigma$ we have a natural $E_\infty$-map
    $\eta\colon A\rightarrow UP(A)$ to a stably equivalent commutative symmetric ring 
    spectrum.}}

\subsection{The Opposite $E_\infty$-Structure}
Following Section 9 of \cite{EM} define the following map of operads.
{\defn{For each $k\geq 0$ set $r_k\colon \{1,\ldots,k\}\rightarrow \{1,\ldots,k\}$ to be $r_k(j)=k+1-j$, i.e.,
the map consisting of only transpositions $(1~ k)$, $(2~ k-1)$ until the centre. In other words $r_k$ fully
reverses the set $\{1,\ldots,k\}$.}}
{\lem{The symmetric sequences $(\Sigma_n)_{n\in\mathbb{N}}$ and $(E\Sigma_n)_{n\in\mathbb{N}}$ of 
categories, where $\Sigma_n$ is the discrete category on objects $\sigma\in\Sigma_n$, while $E\Sigma_n$
is the translation category of $\Sigma_n$ with objects $\sigma\in\Sigma_n$ and a unique morphism between
each pair of objects, form the associative and the Barratt-Eccles-operad.

Both maps $op\colon \Sigma_*\rightarrow \Sigma_*$, $op\colon E\Sigma_*\rightarrow E\Sigma_*$ defined
as $op(\sigma)=r_k\circ \sigma$ for $\sigma\in\Sigma_k$, and extended to a covariant functor in the unique
way on $E\Sigma_*$, are maps of operads. Specifically, $op$ respects units, is equivariant with respect to the
obvious right $\Sigma_*$-action by functors, and preserves multicomposition, i.e., block sum followed by
permutation of blocks.
\begin{proof}
The claim is explicit after Definition 9.1.11 of \cite{EM}, however, the proof is ``left to the reader''. Since
in particular the compatibility with multicompositions can be confusing, I want to elaborate on that.

Unitality of $op$ is obvious, since $r_1=id_{\{1\}}$. Equivariance is obvious as well, since we defined
$op$ by a left-action and the equivariance-condition involves the right-action of $\Sigma_n$ on itself.

For compatibility with multicompositions I first reduce the condition we have
to show drastically:
For $\Theta$ the multicomposition on $\Sigma_*$ and $E\Sigma_*$ we can
reduce an expression $\Theta(r_n\circ \sigma_n; r_{k_1}\circ \sigma_{k_1},\ldots, r_{k_n}\circ \sigma_{k_n})$
by using the right-$\Sigma_*$-action on the resulting product:
$\Theta(r_n\circ \sigma_n; r_{k_1}\circ \sigma_{k_1},\ldots, r_{k_n}\circ \sigma_{k_n})
= \Theta(r_n\circ\sigma_n; r_{k_1},\ldots,r_{k_n}).(\sigma_{k_1}\oplus\ldots\oplus\sigma_{k_n}).$
Thus we can, without loss of generality, consider a multiproduct $\Theta(r_n\circ\sigma_n; r_{k_1},\ldots,r_{k_n})$.
But by the ``inner'' equivariance condition on an operad, we can replace $\sigma_n$ on the left by
the appropriate permutation on the right, giving an expression $\Theta(r_n;r_{l_1},\ldots,r_{l_n})$.
As the final reduction, note that by multiassociativity of the multicomposition $\Theta$ it suffices
to show $\Theta(r_2;r_n,r_m)=r_{n+m},$ which yields the higher compatibilities by an easy induction.

Note that $\Theta(r_2;\sigma_n,\sigma_m)=\chi^+_{n,m}\circ(\sigma_n\oplus \sigma_m)$, for $\chi^+_{n,m}$
the symmetry of the sum in $\mathrm{Fin}$ and $\sigma_i\in\Sigma_i$. Thus we have to show 
$\chi^+_{n,m}\circ(r_n\oplus r_m)=r_{n+m}.$ This is an easy calculation, recall 
\[\chi^+_{n,m}(i)=\begin{cases}i+m,~ & 1\leq i\leq n,\\ i-n,~ & n+1\leq i\leq n+m,\end{cases}\]
and for convenience \[(r_n\oplus r_m)(i)=\begin{cases}r_n(i),~&1\leq i\leq n,\\r_m(i-n)+n,~&n+1\leq i\leq n+m.\end{cases}\]
The block sum trivially preserves the condition on $i$, thus the composite $\varphi:=\chi^+_{n,m}\circ(r_n\oplus r_m)$
is given by
\[\varphi(i)=\chi^+_{n,m}\circ(r_n\oplus r_m)(i)=\begin{cases}r_n(i)+m,~&1\leq i\leq n\\r_m(i-n),~&n+1\leq i\leq n+m.\end{cases}\]
Clearly we have $\varphi(1)=r_n(1)+m = n+m$ and $\varphi(n+m) = r_m(m) = 1$, so $\varphi=(1 ~n+m)\circ \bar\varphi$ and
$\bar\varphi$ is of the form $\Theta(r_2;r_{n-1},r_{m-1})$, so we are done by induction.\end{proof}}}

In particular we can define what the opposite algebra for an associative and an $E_\infty$-algebra are,
when we restrict to $E\Sigma_*$ as the $E_\infty$-operad.
{\defn{An associative algebra in spectra determines and is determined by a 
multifunctor $\Sigma_*\rightarrow Sp$, which sends the unique object of the
multicategory $\Sigma_*$ to the algebra. In particular, we define its opposite
algebra as the composition $\Sigma_*\rightarrow \Sigma_*\rightarrow Sp$ with 
the first map being the opposition $op$ above. This multifunctor determines
and is determined by the algebra with opposed multiplication, which is associative
if and only if the original multiplication is associative.

Analogously an $E_\infty$-algebra in spectra is a multifunctor $E\Sigma_*\rightarrow Sp,$
which is simplicially enriched with respect to the usual simplicial structure on spectra
and $E\Sigma_*$ made simplicial by arity-wise application of the nerve. Since $op$ is
in particular a map of sets in each arity, it extends uniquely to a functor $E(op)\colon
E\Sigma_*\rightarrow E\Sigma_*$ in each arity, thus to a simplicially enriched multifunctor.
The opposite $E_\infty$-algebra is given by precomposition with $E(op)$.

In particular the underlying associative algebra of an $E\Sigma_*$-algebra is given by
restriction along $\Sigma_*\rightarrow E\Sigma_*$ the inclusion of objects, and the underlying
associative algebra of the opposed $E_\infty$-structure is the opposed associative algebra.}}

We can thus define what an anti-involution on an $E\Sigma_*$-spectrum is.
{\defn{An anti-involution $\tau\colon A\rightarrow A$ for $A$ an $E\Sigma_*$-algebra is an
$E\Sigma_*$-map with respect to the $E\Sigma_*$-structure on the source, and the opposed
$E\Sigma_*$-structure on the target, with $\tau^2=\id$.}}

{\rem{Recall the Quillen-equivalence of $E_\infty$-ring spectra and commutative ring
spectra for instance in the positive stable model structure on symmetric spectra \cite{EM,MMSS},
or the model structure on $\S$-modules as exhibited in \cite{EKMM}. With the notation as in
Remark \ref{commeinf} we find that for $A$ a 
cofibrant $E\Sigma_*$-algebra the prolonged involution $\mathbb{P}\tau\colon \mathbb{P}A\rightarrow\mathbb{P}A^\mu$
is a map of commutative spectra. Since $\mathbb{P}(A^\mu)=\mathbb{P}(A)^\mu=\mathbb{P}A$ by
strict commutativity, we find that $\mathbb{P}\tau$ is a self-inverse algebra map of
commutative algebras, which by $\mathbb{P}(A)^\mu=\mathbb{P}A$ becomes an endomorphism.}}

\subsection{Induced Multiplications on $THH$ and the Involution}
I introduced the trace by its multiplicative universality as proven in \cite{BGT}, thus
we need to see that the induced involution of \ref{THHiota} opposes multiplication to find
that the trace map commutes with the involutions on $K$-theory and topological Hochschild 
homology.

The identification $N\C^{op}=\widetilde{N\C}$ extends to the cyclic nerve defining
$THH$ as we see above, and is strictly symmetric monoidal.
{\prop{The natural isomorphism
    \[\iota\colon THH(A^\mu,M^\mu)\rightarrow \widetilde{THH(A,M)}\]
    is strictly symmetric monoidal with respect to the smash product.
    \begin{proof}
    The rail-fence isomorphism indicated above and $\iota$
    are instances of the symmetry of the smash product, 
    thus the coherence of the smash-symmetry yields the claim.\end{proof}}}

The natural homeomorphism $\Gamma$ is symmetric monoidal as well.
{\prop{The natural homeomorphism $\Gamma\colon |\cdot|\Rightarrow |\widetilde{~\cdot~}|$ is
    symmetric monoidal with respect to cartesian as well as smash-product.
    \begin{proof} This is a bit easier to see by considering the symmetric
    monoidal structure of realisation oplax, i.e., $|X\times Y|\cong |X|\times |Y|$. The
    natural map in this case is given by realisation of $pr_X$ and $pr_Y$ respectively.
    In particular it is induced on simplicial objects. The natural transformation
    $\Gamma$, however, operates on the realisation coordinates, thus the transformations
    strictly commute.\end{proof}}}

These propositions assemble to the following:
{\thm{For an $E_\infty$-ring spectrum $A$ with anti-involution $T\colon (A,\mu)
\rightarrow (A,\mu^{opp})$, consider the internal involution induced on $THH$ by
\[\xymatrix{  THH(A)\ar[r]^-T & THH(A^\mu)\ar[r]^\iota & \widetilde{THH(A)} 
\ar[r]^\Gamma &THH(A).}\] This is a natural $E_\infty$-map with respect to the
induced $E_\infty$-structure on the source and the opposed $E_\infty$-structure
on the target $THH(A)$.
\begin{proof} This is just assembling the last three propositions, where we have
analysed each of the maps individually. In particular, the multiplicative opposition
induced from $T$ is not changed by $\iota$ and $\Gamma,$ thus follows the claim.\end{proof}}}

We can reuse \cite{BGT} to establish that the trace map commutes with the induced
involutions. {\thm{The unique natural transformation of (simplicially) enriched 
    multifunctors $tr\colon K\Rightarrow THH$ commutes with the involution on $K$ induced
    as in \ref{indinvSP} and induced on $THH$ as above.\label{trinv}\begin{proof} In the diagram
    \[\xymatrix{K(A) \ar[d]^{T_*}\ar[r] & THH(A)\\ K(A)\ar[r] & THH(A)\ar[u]^{T_*}}\]
    both the upper horizontal map as well as the map given by composing
    the induced involutions with the trace give an $E_\infty$-map $K(A)\rightarrow 
    THH(A)$. The first $E_\infty$-structure is directly asserted by \cite{BGT}, 
    the second follows from the fact, that the $E_\infty$-structure
    is opposed twice by the respective involutions. Thus by uniqueness of the
    multiplicative trace \cite{BGT} we get that the threefold composite describes
    the trace as well.\end{proof}}}

{\rem{With just a conjectural identification of the $E_\infty$-structures 
on $H(\MM(H\R))$ and $H(\M)$ the reference to \ref{indinvSP} in the theorem
is more informal than I intended. Formally, one could, however, set up exactly the same
procedure I describe in chapter \ref{multbidel} to establish that the involution
opposes the $E_\infty$-structure on $H(\MM(H\R))$ as well. This amounts
to mostly rewriting \cite{EM} the way I describe in \ref{pcatneu} but 
explicitly establishing the multiplicative structure for topologically
enriched permutative categories the way I do for bicategories in chapter
\ref{multbidel}. This is not an immediate specialisation of chapter \ref{multbidel},
because I implicitly assume discrete sets of $1$-cells in this thesis, 
but I expect no essential difficulties in generalising chapter \ref{multbidel} 
to bicategories with morphism categories internal to topological spaces.}}

{\rem{I think, the calculations in chapter \ref{calc} are more easily readable if I
declare for the reader how I think about the three maps involved in inducing
the involution on $THH,$ and the analogous sequence on $K$-theory \ref{indinvSP}, i.e.
the classifying spaces of the bicategory of matrices:
\[\xymatrix{THH(A)\ar[r]^-T & THH(A^\mu)\ar[r]^\iota & \widetilde{THH(A)} 
\ar[r]^\Gamma &THH(A),}\]
\[\xymatrix{B\M \ar[r]^{T} & 
        B\mathcal{M}(\mathcal{R}^\mu) \ar[r]^{(\cdot)^t}  
        & B\mathcal{M}(\R)^{op_1} \ar[r]^{\Gamma} & B\M.}\]
In both cases we directly induce a multiplicatively opposing map using
the involution. The transposition on matrices and the map $\iota$ 
allow to identify the simplices in the
nerve of the multiplicative opposition with simplices in the simplicially
opposed nerve. Finally $\Gamma$, in both cases, modifies the simplices
in the realisation by a map of degree $\pm 1$ only depending on the
simplicial degree of the simplex, which internalises the involution.

In summary: $\Gamma\circ\iota$ as well as $\Gamma \circ (\cdot)^t$ consist
of degrees and a preferred identification of simplices, thus are usually
easily analysed.}}

\section{A Useful Subspectrum of $THH$}
In section 3 of \cite{MS} the authors identify a simplicial subspectrum of
$THH_\bullet(A)$ which is naturally included for any $A$. Since then the
functor homology interpretations of Hochschild homology (cf. for instance
\cite{Loday,PRi}) via the Loday functor $\mathcal{L}(A,M)\colon \Fi\rightarrow 
k\mathrm{-Mod}$ have been established, providing a nicely clean, natural
interpretation of this subspectrum. I want to elaborate on this in this
section.

This section deserves an emphasised \emph{special 
acknowledgement}:
Since Stephanie Ziegenhagen and I have been close ever since our own
modest beginnings in Algebraic Topology, I have also observed her
conception of her thesis \cite{Z} in quite some detail. If I had not
been a test-case for numerous ``functor co*homology''-talks provided 
in her own trademark-clarity, I am quite sure I would never have 
understood and probably not even bothered to understand that context.
Thus the essential clarifications in this section rest firmly on her
shoulders.

\subsection*{Non-Commutative Sets}
The existence of the subspectrum identified in \cite[Section 3]{MS} 
does not depend on any additional structure on an associative ring 
spectrum $A$. In particular, it is not relevant if $A$ happens to be 
commutative or not. 

To properly identify the simplicial topological 
Hochschild homology spectrum for an associative algebra object 
however, one obviously needs to keep track of the order with respect 
to which one multiplies. The category of non-commutative sets $\Fia$ 
does just that. I follow the exposition in the sections 1.2-1.4 in 
\cite{PRi}, but instead immediately consider pointed sets, 
called $\Gamma(as)$ in \cite{PRi}.

{\defn{(cf. \cite[Section 1.2]{PRi})\label{fiass}
    The category $\Fia$ has objects pointed finite sets $n_+=\{*,1,\ldots,n\}$,
    and morphisms pointed maps $f\colon n_+\rightarrow m_+$ with chosen total
    orderings on the fibres $f^{-1}j$ for every $j\in m_+$ (including the
    basepoint). For maps composable in finite pointed sets, i.e., $f\colon n_+\rightarrow m_+$, 
    $g\colon m_+\rightarrow l_+$ the underlying map is the composite in finite
    sets $gf$ with total orderings on the fibres given as indicated by:
    \[(gf)^{-1}i=f^{-1}g^{-1}i=\coprod_{j\in g^{-1}i} f^{-1}j.\]
    Explicitly, the ordering of elements $j\in g^{-1}i$ provides an ordering
    of the summands, while each summand is ordered with the order chosen for $f$.}}

{\lem{\cite[Lemma 1.1]{PRi}\label{dec}
    Any morphism $f\colon n_+\rightarrow m_+$ in $\Fia$ has a unique decomposition 
    $\Delta_f\circ \sigma_f,$ where $\Delta_f\colon n_+\rightarrow m_+$ is
    order-preserving and pointed, and $\sigma_f\colon n_+\rightarrow n_+$ is 
    a bijection, usually not pointed.
    \begin{proof}The proof of \cite{PRi} is for the unpointed case, so I want
    to retrace the decomposition for pointed maps. Given a map $f\colon n_+
    \rightarrow m_+$ we find a unique order-preserving map $\Delta_f$ isomorphic 
    to it over $m_+$, i.e., with 
    \[\xymatrix{n_+\ar[r]^f \ar[d]_{\sigma_f}& m_+\\ n_+.\ar[ur]_{\Delta_f}}\]
    More explicitly $\Delta_f$ is the unique order-preserving map with the same
    fibre-cardinalities as $f$, i.e., $|\Delta_f^{-1}\{i\}|=|f^{-1}\{i\}|$ for 
    every $i\in m_+$. In particular there is a unique bijection 
    $n_+\rightarrow n_+$ which makes the fibres of $f$ into intervals in 
    $n_+$ while order preserving on the fibres. The total ordering chosen 
    on the fibres of $f$ then fixes a unique fibre-wise 
    bijection $n_+\rightarrow n_+$ of the order induced by $f$ to the order
    induced by the total order on $n_+$. The composite is the unique bijection
    $\sigma_f$.

    In particular we find that $\sigma_f$ is pointed if and only if the base-point
    is minimal in the chosen order of $f^{-1}\{*\}.$\end{proof}}}

{\rem{Since $\sigma_f$ is not pointed in general the
    decomposition is not internal to pointed sets. However, the bijection 
    necessarily still satisfies $\sigma_f(*)\in\Delta_f^{-1}(*),$ because 
    the composite is a pointed map.}}

This category makes it possible to define a Loday functor 
$\mathcal{L}(A,M)\colon \Fia\rightarrow Sp$ for an associative $\S$-algebra
$A$ and $A$-bimodule $M$. I want to specifically elaborate on the
dependence of $\mathcal{L}$ on the symmetric monoidal structure with
respect to which it is defined. Hence I consider a general symmetric
monoidal category $(\C,\otimes,\mathbbm{1},c_\otimes)$ an associative
$\otimes$-algebra $A$ in $\C$ and an $A$-bimodule $M$, keeping the
designations usual for monoids and bimodules in $(Sp,\wedge,\S,c_{\wedge}).$

{\defn{(cf. \cite[Section 1.3]{PRi})\label{Lod}
    The Loday functor $\mathcal{L}(A,M)\colon \Fia\rightarrow \C$ is given on
    objects as $n_+\mapsto M\otimes A^{\otimes n}$. For a morphism $f\colon 
    n_+\rightarrow m_+$ consider its unique decomposition $f=\sigma\circ \delta$.
    Then $\sigma$ describes a unique symmetry of the monoidal structure:
    \[c_\sigma\colon M\otimes A^{\otimes n} \rightarrow 
    A^{\otimes |f^{-1}\{*\}< *|}\otimes M \otimes A^{\otimes |f^{-1}\{*\}> *|}
    \otimes A^{\otimes |f^{-1}\{1\}|}\otimes\ldots \otimes A^{\otimes |f^{-1}\{m\}|}.\]
    From this reordered object we consider the map
    \[A^{\otimes |f^{-1}\{*\}< *|}\otimes M \otimes A^{\otimes |f^{-1}\{*\}> *|}
    \otimes A^{\otimes |f^{-1}\{1\}|}\otimes\ldots \otimes A^{\otimes |f^{-1}\{m\}|}
    \rightarrow M\otimes A^{\otimes m},\]
    composed of the left- and right-action of $A$ on $M$, i.e., 
    $A^{\otimes |f^{-1}\{*\}< *|}\otimes M \otimes A^{\otimes |f^{-1}\{*\}> *|}
    \rightarrow M$ and $|f^{-1}i|$-fold
    products $A^{\otimes |f^{-1}\{i\}|}\rightarrow A$. Call this map $\delta_*,$
    then define $f_*=\delta_*\circ c_\sigma.$ This makes $\mathcal{L}(A,M)$ a
    functor by uniqueness of the decomposition \ref{dec}.}}

To relate this functor to the simplicial topological Hochschild homology spectrum
I need to elaborate on the simplicial circle a bit more. More specifically, we need to
know that it is in fact a simplicial associative pointed set. 

{\prop{(cf. \cite[Section 1.4]{PRi})\label{ptdassS1}
    Recall the simplicial set $\Delta_1=\Delta(\_,[1])$ with its boundary 
    $\partial\Delta^1$ identified as the constant maps $f\colon [n]\rightarrow [1].$
    The quotient $\S^1=\Delta_1/\partial\Delta_1$ is a pointed simplicial set by 
    \ref{pointedS1}, which can be promoted to a pointed associative simplicial
    set $\S^1_{As}\colon \Delta^{op}\rightarrow \Fia$.}}

Given an associative $\S^1$, we can interpret topological Hochschild homology
for associative algebras. Recall the spectral
Loday functor for an associative $\S$-algebra $A$ and $A$-bimodule $M$:
\[\mathcal{L}(A,M)\colon \Fia\rightarrow Sp.\]
More generally for an associative algebra and bimodule in a 
symmetric monoidal category $(\C,\otimes,\mathbbm{1},c_\otimes)$ with
the assignment on objects: $\mathcal{L}(A,M)(n_+)=M\otimes A^{\otimes n}$.
The total order on the fibres precisely makes this a well-defined functor
for associative (as opposed to commutative) objects. We can easily identify
the simplicial Hochschild spectrum of $A$ with coefficients in $M$ 
as the composite of the Loday functor $\mathcal{L}(A,M)\colon \Fia\rightarrow Sp$
with the pointed associative circle \ref{ptdassS1}: $\S^1_{As}\colon \Delta^{op}
\rightarrow \Fia$.

{\prop{(cf. \cite[p. 213]{Loday}) The composite:
    \[\xymatrix{\Delta^{op}\ar[r]^-{\S^1_{As}}&\Fia\ar[rr]^-{{\mathcal{L}}(A,M)}&&Sp}\]
    is strictly equal to the simplicial Hochschild spectrum of $A$ with
    coefficients in $M$ as defined in \ref{THHblubbAM}. In particular, the
    identification is natural in maps of algebras and bimodules.
    \begin{proof}Objectwise the identification is clear, the example above 
    should convince the reader that I fixed the choices of orderings and 
    basepoints in \ref{ptdassS1} just so that one can identify the face 
    maps and degeneracies in the standard complex easily with the maps 
    induced on the pointed associative circle.\end{proof}}}

With these considerations in place we can find that the subspectrum
identified in \cite[Section 3]{MS} is inherent to the Loday-functor.

{\prop{\cite[Section 3]{MS}
    Let $A$ be an associative algebra, and $M$ an $A$-bimodule with
     a map of $A$-bimodules
    $A\rightarrow M.$ In particular both have a fixed unit map from 
    the sphere spectrum $\S\rightarrow A \rightarrow M.$

    Then we have two factorisations of the identity at $M$:
    \[\xymatrix{M\cong M\wedge\S\ar[rr]\ar[d]&& M\wedge A \ar[d]\\
    M\vee A \ar[r]& M\vee M=(M\wedge\S)\vee(\S\wedge M)\ar[r]& M,}\] 
    with the analogous factorisations 
    holding for the left-module action and the algebra structure of $A$. 

    Thus in particular for $A$ an $\S$-algebra, and $M$ an $A$-algebra, 
    we can define the Loday-functor with respect to coproducts
    $\L^\vee(A,M)(n_+)=M\vee A^{\vee n}$. By universal property of the
    coproduct to describe a map $\L^\vee(A,M)(n_+)=M\vee A^{\vee n}
    \rightarrow M\wedge A^{\wedge n}=\L(A,M)(n_+),$ we need to describe
    it on each summand. Thus for $M$ consider $M\cong M\wedge \S^{\wedge n}
    \rightarrow M\wedge A^{\wedge n},$ while the $i$th $A$-summand is
    mapped to the $i$th smash factor by the analogous description.

    The factorisation above gives that this is a natural transformation 
    \[\L^\vee(A,M)\Rightarrow \L(A,M),\]
    which is moreover natural in the algebra $A$ and the $A$-algebra $M$
    appropriately.}}

{\prop{(cf. \cite[Lemma 3.3]{MS}) The coproduct Loday-functor on an 
    $\S$-algebra $A$, and an $A$-algebra $M$, evaluated on the 
    associative circle is naturally isomorphic to the simplicial 
    spectrum $M\vee ((\S^1)\wedge A)$, for $\S^1=\Delta_1/\partial\Delta_1$ 
    the simplicial circle. 

    In particular, the geometric realisation of the natural transformation 
    above yields a natural map: \[|\L^\vee(A,M)(\S^1_{As})|=|M\vee 
    ((\S^1)\wedge A)| = M\vee \Sigma A \rightarrow THH(A,M).\]}}

I used the description of topological Hochschild homology as a Loday
functor on the associative circle to facilitate the following identification:
{\thm{The Loday functor evaluated on the opposite associative 
    circle $\S^1_{As}\circ r\colon \Delta^{op}\rightarrow \Fia$ 
    is naturally isomorphic to the Loday functor on
    the opposite algebra and opposed bimodule:
    \[\L(A,M)(\S^1_{As}\circ r)\cong \L(A^\mu,M^\mu)(\S^1_{As}).\]
    In particular, for $M$ an $A$-algebra, the natural transformation
    \[\L^\vee(A,M)\Rightarrow \L(A,M),\]
    commutes with this isomorphism, as does its geometric realisation.}}

We can draw the following corollary, which I use repeatedly in the following chapter.
To not confuse opposition of module actions with opposition of multiplications I only
assume an appropriate map $A\rightarrow M,$ which we need for the map $\L^\vee(A,M)\rightarrow \L(A,M)$.
{\thm{Given an associative $\S$-algebra $A$ with anti-involution 
    $T\colon A\rightarrow A^\mu$, and an left- and right-$A$-linear map 
    $A\rightarrow M$ we have the following commutative diagram:
    \[\xymatrix{ M\vee \Sigma A\ar[d]\ar[r]&M\vee \Sigma A\ar[d]\ar[r]
    \rruppertwocell<10>^{1\vee(-1)}{\omit}& M\vee \widetilde{\Sigma} A\ar[d]\ar[r] 
    & M\vee \Sigma A\ar[d]\\  THH(A,M)\ar[r]^{(T,\id_M)}&THH(A^\mu,M^\mu)\ar[r]^\iota 
    &|\widetilde{THH(A,M)}|\ar[r]^\Gamma &THH(A,M).}\]

    More explicitly: The bimodule includes at simplicial degree $0$, thus each of the lower
    three maps restricts to the identity. The suspended algebra is the realisation of the
    simplicial spectrum $\S^1\wedge A,$ hence includes at degree $1$, so that $\iota$ and
    $\Gamma$ together induce a sign.}}

{\rem{I want to place the appropriate emphasis on this subspectrum. Despite the fact that these
    are the ``obvious'' classes in $THH(A,M)$, they are usually not trivial. Instead one can
    usually use the suspension $\Sigma A\rightarrow THH(A,M)$, inducing a map 
    $h_*A\rightarrow h_{*+1}THH(A,M)$, and multiplicative structures on $THH$ to exhaust 
    the classes of interest. Good examples of this are \cite{MS} and \cite{AuTHH}, the second
    of which we study in detail in the next chapter.}}
 \chapter{The Involution on $V(1)_*K(ku_p)$}\label{calc}
In this chapter we can finally use the results of all preceding
chapters to revisit the calculations of Christian Ausoni to analyse
them along the induced involutions - primarily the ones in \cite{AuTHH,AuKku}. 
For this I go through many details of the 
calculations and recall the ones, which I need for establishing
the involution on classes in homology or homotopy groups. However
I have written this chapter under the assumption that the reader
has the sources \cite{AuTHH,AuKku} close.
In particular the effect of the involution has clearer emphasis,
when I consider the hard calculations of \cite{AuTHH,AuKku} as given.

Inherent to the calculations of \cite{AuTHH,AuKku} is the restriction to
odd primes, for a partial picture at $p=2$ see \cite{AnHL} computing the
homotopy groups $\pi_*THH(\l)$ and $\pi_*THH(ko)$ locally at $2$.

\section{The Involutions on $\ell$ and $ku$}
\subsection*{Preliminaries}
The model provided by algebraic 
$K$-theory of an algebraic closure of a finite field $K(\barF_q)$,
which comes with a ring map given by the Brauer lift $K(\barF_q)
\rightarrow KU$, as well as the connective cover given by $H\MM_\CC
=ku\rightarrow KU$ yield the same $E_\infty$-structure, when 
completed at $p$ for which $q\in(\Z/p^2)^\times$ is a generator, by
\cite{BaRi08}. 
The homology theories in \cite{AuTHH,AuKku} are insensitive to 
$p$-completion, because the involved homology theories are 
$H\F_p$-local (in the sense of Bousfield-localisation), hence I
switch between the models for $ku$ whenever convenient for a 
clearer exposition. 

Do note that by the homotopy limit involved in the definition of 
topological cyclic homology and the fact that completion can for instance
be described as a homotopy colimit we cannot expect 
completion and cyclic homology to commute. 

On $K$-theory we can
trace the analogous failure back to the fact that a ring usually
has fewer units than its $p$-completion, thus analysing $BGL(R_p^{\wedge})$
and $BGL(R)_p^{\wedge}$ are usually two different problems. However 
topological Hochschild homology does commute with colimits (given
cofibrant spectra, as we assumed above \ref{cofibTHH}), thus as long as
I rely on \cite{AuTHH} for the determination of $V(1)_*THH(ku)$ and do
not refer to $TC(ku)$ and $K(ku)$, it is not ambiguous, if I do not
specify, if $ku$ denotes its integral, $p$-local or $p$-completed 
version.

By the description of $ku$ given above \ref{ltolzeta} we know that
at an odd prime $p$ the spectrum $ku$ has a direct summand $\ell$ 
called the Adams summand, first identified
by Adams through operations on vector bundles - see Lecture 4 of \cite{Ad69}. 
The inclusion of fields
$L\rightarrow L(\zeta_p)$ induces a map of $E_\infty$-ring spectra
$i\colon K(L)\rightarrow K(L(\zeta_p)),$ and I fix these completed at
$p$ as a model for the inclusion $i\colon \ell_p\rightarrow ku_p$ \ref{einfapprox}.

By basic obstruction theory (cf. \cite[Proposition 3.1]{EKMM},
\cite[p. 36, Lemma 2.12]{MayEinf})
we can realise the map $ku\rightarrow H\Z = H(\pi_0ku)$ 
as a map of $E_\infty$-ring spectra, as well as the map 
$ku\rightarrow H\Z\rightarrow H\Z/p$. This induces in particular
a map of $E_\infty$-ring spectra $\ell\rightarrow H\Z/p$, which by
\cite[Lemma 16.8]{Ad74} realises the inclusion:
\[H_*(\ell;\Z/p)=\Z/p[\xi_1,\xi_2,\ldots]\otimes E(\tau_2,\tau_3,\ldots)\]
\[\phantom{blubbelblubbel}\rightarrow A_*=H_*(H\Z/p,\Z/p) = 
\Z/p[\xi_1,\xi_2,\ldots]\otimes E(\tau_0,\tau_1,\ldots)\]
of the indicated subalgebra of the dual Steenrod algebra at $p$
(cf. for instance \cite[pp. 51-53]{Kochm}), with generators in degrees
$|\xi_i|=2p^i-2$ and $|\tau_i|=2p^i-1$. Do note that the map $H\Z_p\rightarrow
H\Z/p$ induces an injection on $H\Z/p$-homology as well 
with image the full polynomial algebra, and all of the
exterior algebra apart from $\tau_0$, i.e., the dual of the Bockstein
element in the Steenrod-algebra, which is the ${H\F_p}_*$-Bockstein
map.

Since the map $\ell\rightarrow ku\rightarrow \Z$ can be realised on
bipermutative categories as $\MM_L\rightarrow \MM_{L(\zeta_p)}\rightarrow 
\mathbb{N}^\delta,$ with $\N=\N^\delta$ considered as a discrete bipermutative 
category with its obvious rig structure, we see that the involutions
are compatible as follows:
\[\xymatrix{\ell\ar[d]^{\tau_\ell} \ar[r] & ku\ar[d]^{\tau_{ku}}\ar[r] 
		& H\Z\ar@{=}[d]\\\ell\ar[r]&ku\ar[r]&H\Z,}\]
independently of what involutions we chose compatibly on $ku$ and $\ell$.

In particular we find that any involution on $\ell$ given by an involution
on $\MM_L$ induces the identity on its homology:
{\prop{Let $(T,t)\colon \MM_L\rightarrow \MM_L^\mu$ be an involution of
	bipermutative categories with $H\MM_L=\ell,$ then the involution 
	induced	on homology with $\Z/p$-coefficients is trivial.
	\begin{proof}Let me repeat the core of the argument: If the 
	involution arises on bipermutative categories, then we can map 
	to the discrete rig $\mathbb{N},$ and this induces the 
	monomorphism $H_*\ell\rightarrow H_*H\Z$. Since $\mathbb{N}$ is 
	discrete it only supports the trivial involution.\end{proof}}}

Recall \cite[Theorem 2.5]{AuTHH}: 
{\thm{There is an isomorphism of $A_*$-comodule algebras: \[H_*(ku,\F_p) =
	H_*(\ell,\F_p)\otimes P_{p-1}(x)\] with $x\in H_2(ku,\F_p)$ the 
	Hurewicz-image of $u\in\pi_2ku$, where $P_{p-1}(x)$ denotes the 
	polynomial algebra on $x$ truncated by $x^{p-1}.$ 

	Furthermore $H_*\ell$ can be identified as the inclusion of $\Z/(p-1)$-
	fixed points under the action by the Galois group of $L\rightarrow
	L(\zeta_p)$, hence in particular we have: \[\ell\simeq ku^{h\Z/(p-1)}.\]

	\begin{proof}Each of the statements is found on pp. 1268--1269 of 
    \cite{AuTHH}. Compare also for the fixed point statement the corresponding
    statements on $THH$ and $K$-theory on p. 1307 of \cite{AuTHH}. 
    While the fixed point statement is
	immediate from the homological fixed point spectral sequence (cf. for instance
	\cite{BrRo05}) which collapses at $E^2$, because the relevant group homology
	is acyclic, since the order of the group is a unit $p-1=-1\in\F_p^\times$ 
    -- cf. \cite[p. 156]{Rot2009}. \end{proof}}}

{\thm{\label{invku}
	The involution on $H_*ku$ is completely determined by the effect on
	$u\in\pi_2ku,$ thus also on its image under the Hurewicz map $x\in H_2(ku,\F_p)$.

	Explicitly: For the map of commutative $\S$-algebras $\tau\colon ku\rightarrow ku$ 
	induced	by strictifying $H\V_\CC$ and complex conjugation along the Quillen 
	equivalence	of commutative and $E_\infty$-ring spectra (recalled in 
	\ref{commeinf}), we get	\[\tau_*(u^n)=(-1)^nu^n.\]

	Since on $\V_\CC$ complex conjugation and transposition-inversion agree, this
	is also the effect of the involution induced by the identity. \begin{proof}
	We have the canonical map $ku\rightarrow KU$, and by a classical result of Snaith \cite{Sn}
	we know, that we can obtain $KU$ as the suspension spectrum of the infinite complex
	projective space by inverting the Bott class $u\in \pi_2\Sigma^{\infty}_+\CC P^\infty.$
	For a modernised account in motivic spectra compare Gepner-Snaith \cite{GSn}.

	But this class arises as the suspension of $u\colon \S^2\rightarrow\CC P^\infty$ on
	space level. In particular we can choose to realise it as the inclusion 
	$\Sigma U(1)\rightarrow BU(1)\simeq \CC P^\infty,$ where complex conjugation evidently
	acts on $\Sigma U(1)\cong \S^2$ by a reflection along one equator, hence has
	degree $-1$.\end{proof}}}

We have seen at \ref{Invandmultappr} that the involution induced \ref{indinvSP} 
on $\ell$ is strictly equal to the identity. In particular for commutative models 
inverting and transposing agrees with complex conjugation, thus I consider 
the effect of complex conjugation as fundamental. 
{\cor{Transposition-inversion induces the identity on $(H\F_p)_*\ell$, 
	thus on $H\F_p$-homology of $ku$ it is given
	as $x\mapsto -x$ and the identity on $H_*\ell$.	\begin{proof}
	I already presented above that the fact that $\ell\rightarrow H\F_p$
	induces a monomorphism on homology, forces any self-map on bipermutative
	categories to be visible on $\mathbb{N}^\delta,$ thus trivial.
	\end{proof}}}

{\cor{\label{invkucoeff}Complex conjugation on $\pi_*ku=ku_*\cong \Z[u]$ induces the map:
	$u^n\mapsto (-1)^nu^n$ by \ref{invku}. Thus for a prime $p\geq 3$ 
	the map $\ell\rightarrow ku$ realising the inclusion 
	$\Z_{(p)}[v]\mapsto\Z_{(p)}[u]$ with $v\mapsto u^{p-1},$ identifies 
	the effect of conjugation on $\ell$ as the identity.}}

Proceeding in following \cite{AuTHH} we consider topological Hochschild homology
of $ku$ with coefficients in $H\Z_p,$ which is a $ku$-module by the canonical
maps $ku\wedge H\Z_p\rightarrow H\Z_p\wedge H\Z_p\rightarrow H\Z_p$.

{\thm[{\cite[pp. 1282--1287, Proposition 5.6]{AuTHH}}]{There is an isomorphism of $A_*$-comodule \label{hthhkuzp}
	algebras \[H_*(THH(ku,H\Z_p),\F_p)= 
	H_*(H\Z_p,\F_p)\otimes E([\sigma x],[\sigma \xi_1])\otimes P([y]).\]
	with degrees $|\sigma x|=|x|+1=3$, $|\sigma\xi_1|=2p-2+1=2p-1$, and $|y|=2p$.

	The B\"okstedt spectral sequence for $THH(ku,H\Z_p)$:
	\[E^2=HH^{\F_p}(H_*(ku,\F_p),H_*(H\Z_p,\F_p))\Rightarrow H_*(THH(ku,H\Z_p);\F_p)\]
	has $E^2$-term:
	\[HH_*(H_*ku,H_*H\Z_p)\cong H_*H\Z_p\otimes E(\sigma x, \sigma \xi_1,\sigma \xi_2,\ldots)
	\otimes \Gamma(y,\sigma \tau_2,\sigma \tau_3,\ldots),\]
	where $\Gamma$ denotes the divided power algebra over $\F_p$ on the given generators.
	The spectral sequence collapses at $E^{2p}$ and thus has $E^\infty$-term:
	\[E^{2p}=E^\infty=H_*H\Z_p\otimes E(\sigma x,\sigma \xi_1)\otimes 
	P_p(y,\sigma\tau_2,\sigma\tau_3,\ldots)\]
	with multiplicative extensions $[y]^p=[\sigma\tau_2]$ and $[\sigma\tau_i]^p=[\sigma\tau_{i+1}]$.
	\begin{proof}
	This is all explicit in \cite{AuTHH} at the given pages.
	\end{proof}}}

{\rem{Note in particular that the divided power algebra $\Gamma(y)$ in $E^2$-terms
	of the B\"okstedt spectral sequence actually gives rise to a polynomial
	algebra in \ref{hthhkuzp}.}}

For the Adams summand the analogous computational result was already published in 
1991 by McClure and Staffeldt. In absence of the complications introduced by the
truncated polynomial algebra $P_{p-1}(x)$ I can directly quote the result for 
$THH(\ell)$.
{\thm[{\cite[Proposition 4.2; p.22]{MS}, cf. also \cite[Theorem 5.9]{AuTHH}}]{For any prime $p\geq 3$ 
    there is an isomorphism of $\F_p$-algebras:
    \[H_*(THH(\ell),\F_p)=H_*(\ell,\F_p)\otimes E([\sigma\xi_1],[\sigma\xi_2])\otimes 
    P([\sigma\tau_2]).\]}}

Since the computation of $THH(\ell)$ involves fewer complications than
the analogous one for $THH(ku)$ I can immediately determine the full effect
of the involution here.
{\prop{\label{invHell}
	An endomorphism of $\ell$, which induces the identity on homology ${H\F_p}_*\ell$,
	induces the identity on the tensor factor ${H\F_p}_*\ell$ of
	${H\F_p}_*THH(\ell)$ as well. On the suspension classes
	the homeomorphism $\Gamma$ induces the sign $-1$.
	\begin{proof}It is easily seen that the tensor factor $(H\F_p)_*\ell$ 
	stems from simplicial degree $0$, thus $\iota$ and $\Gamma$ are 
	identities there. The suspended classes $[\sigma x]$ can be 
	represented by classes $1\otimes x$ in the B\"okstedt spectral sequence. 
	In particular we see that since these classes are of simplicial degree 
	$1$, the simplicial	inversion is the identity, while the homeomorphism 
	$\Gamma$ introduces	the sign of one transposition, thus $-1$.\end{proof}}}

I opted to present these results before the calculational cornerstone of \cite{AuTHH},
which is fundamental to Ausoni's calculations, hence also to mine. 
In \cite{MS} the choice of $\ell$ as their object of focus facilitates the calculations
performed by McClure and Staffeldt, particular the absence 
of the truncated polynomial algebra in $H\F_p$-homology of $\ell$. 
In \cite{AuTHH} Christian Ausoni traces the effect of this algebra in ${H\F_p}_*ku$
on homology carefully, isolating its Hochschild homology in \cite[Proposition 3.3]{AuTHH}, 
which I reduce here to isolating the acyclic resolution and the resultant cycles. More generally 
for $k$ a commutative ring with unit the Hochschild homology of algebras $A=k[X]/f$ with $f$ a
monic polynomial was quite generally calculated in \cite{GGRSV}, however I am following the
exposition of Ausoni specialised to $f=X^n$ and $k=\F_p$. As far as I know the original
source for this resolution is \cite[Section 3]{MN}.
{\prop{Let $A=P_h(x)$ be the polynomial algebra truncated by the ideal $x^h$ over $k=\F_p$,
	then there is an acyclic resolution of $A$ as an $A\otimes A^{op}=A^e$-module with
    underlying graded module: \[X=A^e\otimes E(\sigma x)\otimes \Gamma(\tau)\]
	with $A^e$ in resolution degree $0$, $\sigma x$ of degree $(1,|x|)$, $\tau$ of
	degree $(2,h|x|)$.
	For $h$ a unit in $\F_p$ the cycles in $A\otimes_{A^e}X$ consist of the 
	$A$-submodule $\Gamma(\tau)\otimes\{\sigma x\}\oplus (\tau)\otimes\{x\}$
	for $(\tau)\subset \Gamma(\tau)$ the ideal of positive divided powers in
	$\Gamma(\tau).$\label{hhtrPol}
	\begin{proof}This is all explicit in \cite[Proposition 3.3]{AuTHH}. Note that $X$ 
	in fact describes a $\Z$-resolution of $\Z[x]/x^h$, 
	identifying the cycles in $A\otimes_{A^e}X$ however is less clean 
	for $k=\Z.$ \end{proof}}}

{\prop{Consider on $A=P_h(x)$ the morphism of commutative $k=\F_p$-algebras given
	by $x\mapsto -x.$ On generators in Hochschild homology it induces:
	$x\mapsto -x, \sigma x\mapsto -\sigma x, \tau\mapsto \tau.$
	\begin{proof} In the acyclic resolution given above with $A^e=P_h(x)\otimes 
	P_h(y)$ we have	$d(\sigma x) = x-y, d(\tau)=\frac{x^h-y^h}{x-y}\sigma x$ 
	\cite[Proposition 3.3]{AuTHH}. Thus a lift of the given map is given by 
	$\sigma x\mapsto -\sigma x$ and $\tau\mapsto \tau,$ giving the claimed 
	effect.\end{proof}}}

The analogous consideration works immediately for ${H\F_p}_*THH(ku,H\Z_p).$
{\thm{Complex conjugation on $ku$ induces the identity on the tensor factor
	${H\F_p}_*(H\Z_p)\subset {H\F_p}_*(THH(ku,H\Z_p))
	={H\F_p}_*H\Z_p\otimes E(\sigma x,\sigma \xi_1)\otimes P(\tau).$ 
		
	The involution induced by $THH(A,M)\rightarrow 
	\widetilde{THH(A,M)}\rightarrow THH(A,M),$ with first map the anti-involution
	composed with simplicial reversion, and second map the homeomorphism $\Gamma$
	on realisations -- cf. \ref{THHiota} -- induces the following maps:
	$\sigma x\mapsto \sigma x, \sigma \xi_1\mapsto -\sigma\xi_1, \tau\mapsto -\tau.$

	In more detail: The map induced by $x\mapsto -x$ gives $\sigma x\mapsto -\sigma x$
	and the identity on $\sigma\xi_1$ and $\tau$, while the homeomorphism $\Gamma$
	induces a sign $-1$ on all three classes.
	\label{invthhkuzp}
	\begin{proof} We see in the resolution $X$ chosen above that $\iota$
	can be represented as the identity, since it is also a resolution of $A$ 
	as an $(A^e)^{op}=A^{\otimes2}$-module, since $A$ is commutative. 
	Thus only $\Gamma$ introduces an additional effect as a sign dependent 
	on resolution degree, which is $1$ for the suspensions, and $2$ for 
	$\tau$, thus we get $-1$ in both cases \ref{Gammasign}.\end{proof}}}

\section{Increasing Chromatic Complexity -- Reduction by $p$}
As indicated I use the modules, which are easiest to describe, whenever
possible. For the next calculational steps of \cite{AuTHH} I thus need
to introduce ``mod $p$ homotopy''. The idea is quite simple, instead of
considering the prime $p$ as a self-map on $H\Z$ or $H\Z_p$, we can
consider it as a self-map of the sphere spectrum, giving the cofibre
sequence of spectra:
\[\xymatrix{\S\ar[r]^p&\S\ar[r]&V(0)\ar[r]&\Sigma\S.}\]
In particular one could hope that $V(0)$ gives a better approximation
to homotopy groups than $H\F_p$, thus its homology theory is
usually called \emph{mod $p$ homotopy}. In other words the spectrum
$V(0)$ is the two-cell spectrum $\S^0\cup_p \mathbb{D}^1=V(0)$.

It is classical that the spectrum $V(0)$ at a prime $p\geq 3$ admits
a multiplication, which is part of an $A_{p-1}$-structure, which
cannot be extended to $A_p$. For $p=2$ we do not
have a multiplication, while for $p=3$ the multiplication is 
not associative even up to homotopy. For a good survey of
this I refer to \cite{SchV0}. In particular in \cite[Theorem 2.5]{SchV0} we see the
obstruction to extending the $A_{p-1}$-structure to an $A_p$-structure.

The construction \cite[Definition 2.1]{SchV0} 
works by introducing levels of extended powers 
$D_nX=X^{\wedge n}\wedge_{\Sigma_n} {E\Sigma_n}_+$. As a consequence, 
a coherent $M=\S^2\cup_p\mathbb{D}^3$-module structure as defined in 
\cite[Definition 2.1]{SchV0} entails degrees of commutativity as well. Specifically in 
\cite[Example 2.4]{SchV0} the ``tautological'' coherent module structure on 
$V(0)$ is defined up to degree $p-1$. So for $p\geq 5$, the second
extended power $D_2V(0)=V(0)^{\wedge 2}\wedge_{\Sigma_2}{E\Sigma_2}_+$ is part
of the module structure. Thus the multiplication admits the following factorisation:
\[\xymatrix{V(0)\wedge V(0)\ar[d]_\mu \ar[r]^{tw}& V(0)\wedge V(0)\ar[d]_\mu\\
D_2V(0)\ar[r]^{(\_\wedge (12))} & D_2V(0),}\]
where the transposition $(12)$ acts on the factor $E\Sigma_2$, and thus
is canonically simplicially homotopic to the identity. Thus the multiplication
on $V(0)$ is homotopy commutative for $p\geq 5$, by analogously considering
the third extended power, which is also part of the module structure for
$p\geq 5$, we get homotopy associativity as well.

For the classical interpretations in particular of the obstruction class I
defer to the references of \cite{SchV0}, in particular the three Toda 
references, and the reference to Ravenel.

\subsection{The Involution on $V(0)_*THH(ku,H\Z_p)$}
Referring to \cite[Proposition 10.1]{AuTHH} I focus exclusively on $ku$ from
here. The appropriate restriction for $\ell$ follows by the observation, that
the Galois group acts by maps on coefficients of the approximating 
bipermutative categories. In particular the action strictly commutes with
the involution on $ku$, thus the induced involution for $\ell$ can be
recovered by restricting to the submodule of fixed points under the action
of the Galois group by \cite[Proposition 10.1]{AuTHH}.

Introducing $V(0)$-coefficients makes the topological Hochschild homology
of $H\Z_p$ easier to understand. By the equivalence $V(0)\wedge H\Z_p\simeq 
H\F_p$ the resulting module admits an $\F_p$-algebra structure for all odd 
primes $p\geq 3$.
{\prop{\cite[Theorem 5.7]{AuTHH} For any prime $p\geq 3$ there is an isomorphism
of $\F_p$-algebras \[V(0)_*THH(H\Z_p)\cong E(\lambda_1)\otimes P(\mu_1),\]
with degrees $|\lambda_1|=2p-1$ and $|\mu_1|=2p$.
Moreover the Hurewicz homomorphism is an injection
\[V(0)_*THH(H\Z_p)\rightarrow (H\F_p)_*(V(0)\wedge THH(H\Z_p))\]
with $\lambda_1\mapsto[\sigma\xi_1]$ and 
$\mu_1\mapsto [\sigma\tau_1]-\tau_0[\sigma\xi_1].$}}

The fact that the Hurewicz is an injection immediately yields the following corollary.
{\cor{The involution induced on $V(0)_*THH(H\Z_p)$ by the identity as
	in \ref{THHiota} is $\lambda_1\mapsto -\lambda_1$ and $\mu_1\mapsto -\mu_1$.
	\begin{proof}Arguing as in the propositions for $\ell$ we identify the
	claimed effects as the effect of $\iota$ and $\Gamma$ on suspension classes.
	\end{proof}}}

For $THH(ku,H\Z_p)$ the equivalence $V(0)\wedge H\Z_p\simeq H\F_p$ yields
an $\F_p$-algebra structure on the mod $p$ homotopy for all $p\geq 3$, giving
the result:
{\prop{\cite[Theorem 6.8]{AuTHH} There is an isomorphism of $\F_p$-algebras
for any prime $p\geq 3$:
\[V(0)_*THH(ku,H\Z_p)\cong E(z,\lambda_1)\otimes P(\mu_1)\]
with degrees $|z|=3, |\lambda_1|=2p-1$ and $|\mu_1|=2p.$

The Hurewicz homomorphism is an injection
\[V(0)_*THH(ku,H\Z_p)\rightarrow {H\F_p}_*(V(0)\wedge THH(ku,H\Z_p))\]
with $z\mapsto [\sigma x], \lambda_1\mapsto [\sigma\xi_1],$ and
$\mu_1\mapsto [\tau]-\tau_0[\sigma\xi_1]$.}}

{\rem{Let me note in particular that the $0$th Postnikov section 
	$j\colon ku\rightarrow H\Z_p$ induces a map of $H\Z_p$-algebras,
	which on mod $p$ homotopy gives $j_*(\sigma x)=0, j_*(y)=\sigma\tau_1,$
	and $j_*(\sigma\xi_1)=\sigma\xi_1$. Thus the only class we have 
	not analysed with respect to the involution is $\sigma x$.}}

{\cor{The involution on $V(0)_*THH(ku,H\Z_p)\cong E(z,\lambda_1)\otimes P(\mu_1)$
	is given as follows: $z\mapsto z, \lambda_1\mapsto -\lambda_1, 
	\mu_1\mapsto -\mu_1$. \label{v0thhkuzp}\begin{proof}
	The Hurewicz homomorphism, as well as the map induced by $j\colon ku\rightarrow 
	H\Z_p$ give monomorphisms in the degrees relevant to $\lambda_1$ and $\mu_1$,
	giving the claim for them. For $z$ simply note that it is the mod $p$
	reduction of the integral class $\sigma x$ considered before.
	\end{proof}}}

\section{Reducing by $\alpha_1\colon \Sigma^{2p-2}V(0)\rightarrow V(0)$}
In \cite[Sections 7+8]{AuTHH} Ausoni proceeds to identify the mod $p$
homotopy of $THH(ku)$ by considering a Bockstein spectral sequence
associated to the Bott class $u\colon ku\rightarrow ku$. In mod $p$
coefficients the resulting algebra $V(0)_*THH(ku)$ however has
infinitely many generators and infinitely many relations for any
presentation \cite[Corollary 7.11]{AuTHH}. Thus even for purely presentational
reasons it is convenient to introduce one further reduction here.

Recall that the obstruction to extending the $A_{p-1}$-structure on $V(0)$ to
an $A_p$-structure is the Adams self-map usually called 
$\alpha_1\colon \Sigma^{2p-2}V(0)\rightarrow V(0)$. In particular in the
coherent module structures as considered in \cite{SchV0} it appears as a
non-trivial obstruction to unitality of a non-existent $A_p$-structure on
the Moore spectrum (cf. \cite[Theorem 2.5]{SchV0}).

Consider the cofibre sequence defining $V(1)$:
\[\xymatrix{\Sigma^{2p-2}V(0)\ar[r]^-{\alpha_1}& V(0)\ar[r]
& V(1)\ar[r] & \Sigma^{2p-1}V(0).}\] Compare specifically to
page 58 of \cite{Toda}. In particular \cite[Theorem 4.1]{Toda} fixes
$V(1)$ as the unique (up to homotopy equivalence) spectrum with
$4$ cells with attaching maps as indicated:
$V(1)=(\S\cup_pC\S)\cup_{\alpha_1}C(\S^{2p-2}\cup_p C\S^{2p-2}).$
It is usual to call its homology theory \emph{$V(1)$-homotopy}, 
I do so as well in what follows.

By considering $V(0)$ and $V(1)$ as part of a family of spectra
$V(a)$ with inclusions $V(a)\rightarrow V(b)$ for $a<b$ Toda 
identifies multiplicative pairings of the form $V(a)\wedge V(b)
\rightarrow V(c)$ with $a,b\leq c$. In particular for $V(1)$ 
included into $V(1\frac12),V(2\frac14),V(2\frac34),$ and $V(3)$,
we find that the first three cases of \cite[Theorem 4.4]{Toda} give
a multiplication on $V(1)$ for $p\geq 11, p=7$ and $p=5$ respectively.
Furthermore \cite[Theorem 6.3]{Toda} gives in particular that for
$p=3$ such a multiplication cannot exist, while \cite[Theorem 6.1]{Toda}
explicitly establishes that a spectrum of the analogous type of
$V(1)$ does not exist at $p=2$ at all. In particular since this
consideration has naturally led us to primes with $p\geq 5$ we
can apply \cite{Oka} to find that the obstruction to 
homotopy-commutativity is always $2$-torsion, while the obstruction
to homotopy-associativity is always $3$-torsion. Thus both vanish
in the coefficients of $V(1)$.

Since I wanted to introduce $V(1)$ as late as possible with respect
to the calculations in \cite{AuTHH}, I only partially quoted the 
result of \cite{MS} regarding $THH(\ell).$ Here it can thus serve to
convince the reader that $V(1)$ simplifies the modules considerably.

{\prop{(\cite{MS}, cf. \cite[Theorem 5.9]{AuTHH}) For any prime $p\geq 3$
there is an isomorphism of $\F_p$-algebras:
\[(H\F_p)_*(THH(\ell))\cong (H\F_p)_*\ell\otimes E(\sigma\xi_1,\sigma\xi_2)
\otimes P(\sigma\tau_2).\]
The $V(1)$-homotopy $V(1)_*THH(\ell)$ maps by an injective Hurewicz 
homomorphism to $(H\F_p)_*(V(1)\wedge THH(\ell)),$ with image generated
as an algebra by $[\sigma\xi_1],[\sigma\xi_2]$ and 
$[\sigma\tau_2]-\tau_0[\sigma\xi_2]$, yielding an isomorphism of $\F_p$-algebras:
\[V(1)_*THH(\ell)\cong E(\lambda_1,\lambda_2)\otimes P(\mu_2),\]
with degrees $|\lambda_1|=2p-1, |\lambda_2|=2p^2-1,$ and $|\mu_2|=2p^2$, with
generators defined as the preimages of $[\sigma\xi_1],[\sigma\xi_2]$ and 
$[\sigma\tau_2]-\tau_0[\sigma\xi_2]$ respectively.}}

In particular since the generators are preimages of suspension classes by an
injective Hurewicz homomorphism, the involutions are determined by \ref{invHell},
giving the following cleaner statement.
{\cor{For the $V(1)$-homotopy of $THH(\ell)$: \label{v1thell}
	\[V(1)_*THH(\ell)\cong E(\lambda_1,\lambda_2)\otimes P(\mu_2),\]
	the homeomorphism $\Gamma$ (cf. \ref{simpoppTop}) induces a 
	sign $-1$ on each generator. Thus
	the induced involution \ref{THHiota} on $V(1)_*THH(\l)$ is given as:
	$\lambda_1\mapsto -\lambda_1, \lambda_2\mapsto -\lambda_2,
	\mu_2\mapsto -\mu_2.$}}

\subsection{The Homology and $V(1)$-Homotopy of $THH(ku)$}
People familiar with the computations in \cite{AuTHH} know that the
algebras on homology ${H\F_p}_*THH(ku)$ and $V(1)_*THH(ku)$ contain  
big subalgebras $\Omega_*$ and $\Xi_*$ respectively 
on $(p-1)^2+1=p^2-2p$ generators with quite a few relations 
- cf. \cite[Definition 9.9, Definition 9.13]{AuTHH}. Essentially this stems from the 
truncated polynomial algebra in ${H\F_p}_*ku$ \cite[Proposition 2.3]{AuTHH}.

In order to understand the involution on $H_*THH(ku)$ and $V(1)_*THH(ku)$
however I do not need to display the relations, it suffices to understand
the effect on the cycles 
$(x)\otimes (\tau)\oplus (\sigma x)\otimes \Gamma(\tau)
\rightarrow HH_*(P_{p-1}(x))$.

To determine the involution on $V(1)_*THH(ku)$ we shall use the topological
Hochschild homology of $ku$ with coefficients in $H\Z_p$, i.e., 
$V(1)_*THH(ku,H\Z_p)$ as an anchor.
{\lem{(\cite[p. 1305]{AuTHH})
	The isomorphism on mod $p$ homotopy $V(0)_*THH(ku,H\Z_p)\cong 
	E(z,\lambda_1)\otimes P(\mu_1)$ implies the algebra isomorphism:
	\[V(1)_*THH(ku,H\Z_p)\cong E(z,\lambda_1,\varepsilon)\otimes 
	P(\mu_1)\] with $\varepsilon$ of degree $2p-1$.	\begin{proof}
	The Hurewicz from $V(0)_*THH(ku,H\Z_p)$
	to ${H\F_p}_*(V(0)\wedge THH(ku,H\Z_p))$
	is a monomorphism \cite[Theorem 6.8]{AuTHH}. So the Adams map $\alpha_1\colon 
	\Sigma^{2p-2}V(0)\rightarrow V(0)$ induces the zero map on 
	$V(0)_*THH(ku,H\Z_p)$, because it induces the trivial map
	in $H\F_p$-homology. Thus the $V(1)$-homotopy of $THH(ku,H\Z_p)$
	consists of two shifted copies of its mod $p$ homotopy, which
	we can parametrise by a formal generator $\varepsilon$ 
	of degree $2p-1$.\end{proof}}}

{\prop{The induced involution \ref{THHiota} on $V(1)_*THH(ku,H\Z_p)$ is
	given on generators as:
	\[z\mapsto z, \lambda_1\mapsto -\lambda_1, \mu_1\mapsto -\mu_1,
	\varepsilon\mapsto \varepsilon.\]
	\begin{proof}
	Since $\varepsilon$ in essence describes the connecting homomorphism
	of the cofibre sequence:
	\[\Sigma^\bullet V(0)\rightarrow V(0)\rightarrow V(1)\rightarrow 
	\Sigma^\bullet V(0),\]
	and thus acts on the coefficients it commutes with the involution induced
	on $THH$. On the other elements the involution is the one given in 
	\ref{v0thhkuzp}	\end{proof}}}

As seen above I do not need to display the relations of the algebras 
$\Xi_*$ and $\Omega_*$ defined in \cite[Definition 9.9, Definition 9.13]{AuTHH}, thus
I use the following simplified description -- which amounts to ignoring
the relations introduced by boundaries in the Hochschild complex.
{\prop{The $P_{p-1}(u)$-algebras $\Xi_*$ and $\Theta_*$ admit
	a surjection of the $P_{p-1}(u)$-module 
	$\Gamma(\tau)\{\sigma u\}\oplus (\gamma_1\tau)\{u\}\oplus 
	\F_p\otimes\{\mu_2\}$
	with $(\gamma_1\tau)\subset \Gamma(\tau)$ the ideal of positive
	divided powers in $\Gamma(\tau)$.

	Specifically for $\Xi_*$ we assign 
	$\gamma_i\tau\cdot\sigma u\mapsto \bar z_i$ and $\gamma_i\tau\cdot u
	\mapsto \bar y_j$.
	\begin{proof}
	The generators of $\Xi_*$ and $\Omega_*$ arise from the $E^2$-term
	of the B\"okstedt spectral sequence 
	$HH_*({H\F_p}_*ku)\Rightarrow {H\F_p}_*THH(ku),$
	which are images of the generators given by the inclusion
	$HH_*(P_{p-1}(u))\rightarrow HH_*({H\F_p}_*ku).$ For Hochschild
	homology of this truncated polynomial algebra we have determined
	the claimed surjection in \ref{hhtrPol}. The class $\mu_2$ maps
	to $[\sigma \tau_2]$ in ${H\F_p}_*THH(ku)$ and to the
	class with the same name in $V(1)_*THH(ku)$.\end{proof}}}

This description is sufficient to determine the effect of the maps
defining the involution on $THH(ku)$. 

{\thm{For the isomorphism of \cite[Proposition 9.10]{AuTHH} 
	\[{H\F_p}_*THH(ku)\cong {H\F_p}_*\ell\otimes E([\sigma\xi_1])\otimes \Xi_*\]
	and the analogous isomorphism in $V(1)$-homotopy of 
	\cite[Theorem 9.15]{AuTHH}: $V(1)_*THH(ku)=E(\lambda_1)\otimes \Theta_*$
	we can determine the involutions as follows.
	On $\mu_2$ we have $\mu_2\mapsto -\mu_2$ as visible in 
	$THH(\l)$ \ref{invHell},\ref{v1thell}.
	On $a_i\in\Omega_*$ we have $a_i\mapsto (-1)^ia_i$, analogously on
	$\bar z_i\in\Xi_*$: $\bar z_i\mapsto (-1)^i\bar z_i$.
	For $b_i\in\Omega_*$ we have $b_i\mapsto (-1)^{i+1}b_i$, analogously
	for $\bar y_i\in\Xi_*$: $\bar y_i\mapsto (-1)^{i+1}\bar y_i$.
	\label{invTHHku}\begin{proof}
	The classes $a_i$ and $\bar z_i$ arise as infinite cycles in the
	B\"okstedt spectral sequence represented by the classes $\sigma x\gamma_i\tau$
	giving the claim by Theorem \ref{invthhkuzp}. Analogously the classes $b_i$
	and $\bar y_i$ are cycles associated to the classes $x\gamma_i\tau$ for
	$i\geq 1$, thus giving the claim again by Theorem \ref{invthhkuzp}.
	\end{proof}}}

\section{Results on the Involution on $V(1)_*K(ku_p)$}
Since the calculations in \cite{AuKku}, as well as \cite{AuKl1,AuKl2} rely
on trace methods, my intended approach was to determine the involution
on $K(ku_p)$ and $K(\l_p)$ by using the trace $tr\colon K\Rightarrow THH$. As
shown above it commutes with the involutions \ref{trinv} defined
on $K$ as in \ref{indinv} and on $THH$ as in \ref{THHiota}. A few of 
the classes allow more direct approaches, so I prefer these for the 
exposition below.

From this point on I explicitly denote the completions at $p$ of $\l$ and $ku$.
In particular since the computations in \cite{AuKl1,AuKku} rely partly on
comparison to the integers, thus on the results of \cite{BHMtr} and more
specifically \cite{BMTCZ} it would not be reasonable to expect global information
given our limited state of knowledge about $K(\Z)$.
Instead we rely on the computations for $K(\Z_p)$.

\subsection{The Module $V(1)_*K(ku_p)$ and its Traces}
For reference in the sections below I directly quote the main result
of \cite{AuKku}. I examine a few classes individually below, so I 
quote only the identification of the module given in \cite[Theorem 8.1]{AuKku}.
{\thm{\cite[Theorem 8.1]{AuKku}}{There is an isomorphism of $P(b)$-modules
\[\begin{aligned} \label{dasmonster}
V(1)_*K(ku_p)\cong &P(b)\otimes E(\lambda_1,a_1)\\
\oplus &P(b)\otimes E(\lambda_1)\otimes \F_p\{\sigma_n|~1\leq n\leq p-2\}\\
\oplus &P(b)\otimes \F_p\{\partial\lambda_1,\partial b,
	\partial a_1,\partial\lambda_1a_1\}\\
\oplus &P(b)\otimes E(a_1)\otimes \F_p\{t^d\lambda_1|~0<d<p\}\\
\oplus &P(b)\otimes E(\lambda_1)\otimes \F_p\{t^{p^2-p}\lambda_2\}\\
\oplus &\F_p\{s\}.\end{aligned}\]}}

With regard to the traces of the classes, I draw the following corollary,
which is less precise than the determination in \cite[Theorem 8.1]{AuKku} 
but sufficient for determining the involution on the first direct summand.
{\cor{to \cite[Theorem 8.1]{AuKku}}{
	In the direct sum decomposition above, the third to
	sixth summand are contained in the kernel of the map induced
	by the trace $V(1)_*K(ku_p)\rightarrow V(1)_*THH(ku_p).$

	Furthermore the traces of the classes $\sigma_n$ are contained in
	the $P_{p-1}(u)$-subalgebra of $V(1)_*THH(ku_p)$ generated
	by $a_0\in V(1)_3THH(ku_p)$, while the traces of the first
	summand are part of the $P_{p-1}(u)$-subalgebra generated by
	$a_1,b_1,$ and $\lambda_1$.}}

We can use basic linear algebra over $\F_p$ to actually find that
the trace can be described as a direct sum of maps with ``orthogonal''
images in low degrees. Again this follows directly from reading
\cite[Theorem 8.1]{AuKku} and \cite[Theorem 9.15]{AuTHH} appropriately.
{\cor{to \cite[Theorem 8.1]{AuKku}}{
	The trace corresponds to a direct sum of maps with respect
	to the direct sum decomposition of $V(1)_*K(ku_p)$ above
	and the direct sum decomposition of $V(1)_*THH(ku_p)$ (as a 
	$P_{p-1}(u)$-module) given
	by the generators $\lambda_1,a_i,b_i,\mu_2$.

	In particular, in low degrees it is injective on the classes
	$\lambda_1,a_1$ of the first summand, with image having
	trivial intersection with the other summands.}}

\subsection{The Involution on $E(\lambda_1,a_1)\subset V(1)_*K(ku_p)$.}
By the corollary above we only need to determine the involution on 
$\lambda_1,$ and $a_1$ in $V(1)_*THH(ku)$, which implies we have
the same effect on $V(1)_*K(ku_p)$ by injectivity in those degrees.
So I isolate the part of \ref{invTHHku} relevant to $K(ku_p)$ here.

{\thm{The involution on the $P(b)$-subalgebra \[E(\lambda_1,a_1)
    \subset	V(1)_*K(ku_p)\] is given by $\lambda_1\mapsto -\lambda_1$ 
    and	$a_1\mapsto -a_1$. \begin{proof} This follows directly
    from the theorems on $V(1)_*THH(ku)$. Specifically recall
    that $\lambda_1$ stems from the class of $\sigma\xi_1$ in
    ${H\F_p}_*THH(H\Z_p)$ with $\xi_1$ the generator in the
    dual of the Steenrod algebra. Thus the involution induced
    by conjugation is trivial, $\Gamma$ induces a sign $-1$.

    The class $a_1$ can be traced back to the class 
    $\sigma x\gamma_1\tau=\sigma x\tau$ in the B\"okstedt spectral
    sequence. The class $\sigma x$ has simplicial degree $1$, $\tau$
    has simplicial degree $2$. The map induced by conjugation induces
    a sign $-1$ on $\sigma x$ and the identity on $\tau$, the homeomorphism
    $\Gamma$ induces a sign $-1$ on both, yielding the claim.\end{proof}}}

{\rem{We have the involution determined on this full subalgebra after 
	determining	the effect on $b$ as well below.}}

\subsection{The Suspended Bott Classes $\sigma_n$}
By restricting the isomorphism of \cite[Proposition 5.2]{AuKku} we get
an inclusion identifying the classes $\sigma_n\in V(1)_*K(ku_p).$ In fact
they are global, i.e., $\sigma_n\in V(1)_*K(ku).$

Specifically they arise as follows: We have the inclusion 
$BBU_\otimes\subset BGL_1ku,$ where the units of an $E_\infty$-ring spectrum
are defined as the (homotopy) pullback:
\[\xymatrix{
GL_1A\ar[r]\ar[d]&\Omega^\infty A\ar[d]\\
GL_1(\pi_0A)\ar[r]&\pi_0A,}\]
which is a sufficient notion of ``units'' for our purposes. For $ku$ we have
$GL_1\pi_0ku= GL_1\Z=\{\pm 1\}$, hence $GL_1ku=BU\times\{\pm1\}.$
Restricting to the ``index'' $+1$, we get the
inclusion $BBU\rightarrow BGL_1ku$. By delooping the
topologically enriched permutative category on objects natural
numbers with endomorphisms $GL_nku$ along the lines of \cite{EM}
recalled in \ref{pcatneu} we get a model for $K$-theory of $ku$.
In particular its underlying infinite loop space is the group completion:
$\Omega B(\coprod_nBGL_nku)=\Omega^\infty K(ku).$ The canonical
inclusion $GL_1ku\rightarrow \coprod_nGL_nku$ fits in the
sequence of maps
\[BBU\rightarrow BGL_1ku\rightarrow \coprod_nBGL_nku\rightarrow 
\Omega^\infty K(ku).\]

Considering units in a stricter setting, for instance
with $ku$ as a symmetric ring spectrum, we can find a strictly
associative model for $BU_\otimes$. Thus
we have a one-point suspension category $\Sigma BU_\otimes$.
Embed this into its free permutative category $\P\Sigma BU_\otimes
= \coprod_n E\Sigma_n\times_{\Sigma_n}(\Sigma BU_\otimes)^n.$ The induced
map of permutative categories $\P\Sigma BU_\otimes\rightarrow \coprod_nGL_nku$
delooped as in \ref{pcatneu} induces a map:
\[H\P\Sigma BU_\otimes\rightarrow H(\coprod_nGL_nku),\]
which we can identify as
\[\omega\colon \Sigma^\infty BBU_\otimes\rightarrow K(ku)\]
as given in \cite[p. 627]{AuKku}.

Trivially the map $BU\rightarrow BU\times \Z$ identifying $BU$ as a connected
cover of $BU\times\Z$ is an isomorphism in positive degrees, thus we have classes
$y_n\in\pi_{2n}BU$ for $n\geq 1$ which map to the powers of the Bott class $u^n$.

Suspending these once gives classes $\sigma y_n\in\pi_{2n+1}\Sigma BU$ giving classes:
\[\Sigma^\infty \S^{2n+1}\rightarrow \Sigma^\infty\Sigma BU\rightarrow
\Sigma^\infty BBU\rightarrow K(ku).\]

Call these $\sigma_n$ in agreement with \cite[Definition 3.2]{AuKku}.
Then we have:
{\prop{(cf. \cite[Proposition 5.2]{AuKku})
	The classes $\sigma_n\in \pi_{2n+1}K(ku)$ are non-trivial for
	$1\leq n\leq p-2$.}}

{\rem{I have to concede that I am not certain about non-triviality 
    for the higher $\sigma_n$, however the traces of the $\sigma_n$
    are $u^{n-1}a_0$ by \cite[Theorem 8.1]{AuKku}. 
	Thus one should probably consider the upper limit $p-2$ as an artefact 
	of the relations for $\Theta_*$ and
	$\Xi_*$ in $V(1)_*THH(ku)$. At least on the subspectrum
	$\Sigma(ku)\rightarrow THH(ku)$ one can identify these globally in homotopy 
	groups, thus in particular removing the bound $p-2$. However establishing
    their non-triviality would entail 
    non-trivial calculations in the homotopy groups of $THH(ku)$.}}

We can identify the involution on these suspended classes as follows:
{\thm{The involution on the classes $\sigma_n\in\pi_{2n+1}K(ku)$ for $n\geq 1$ 
	is given as $\sigma_n\mapsto (-1)^{n+1}\sigma_n$. 
	\begin{proof}
	Consider part of the inclusion into $\Omega^\infty K(ku)$:
	\[BBU\rightarrow BGL_1ku\rightarrow \coprod_nBGL_nku.\]
	Then the classes of the $\sigma_n$ are given by $\Sigma \S^{2n}\rightarrow
	BBU\rightarrow BGL_1ku$. Thus conjugation acts as it does on $u^n$,
	giving a sign $(-1)^nu^n$. Transposition has no effect on $GL_1ku$.
	Finally $\Gamma$ acts on simplicial degree $1$ here, thus reverses the
	signs to give $(-1)^{n+1}\sigma_n.$	\end{proof}}}

\subsection{The Higher Bott Class $b\in V(1)_{2p+2}K(ku)$}
The class of major interest in $K(ku)$ is a class in degree $2p+2$ of
the $V(1)$-homotopy of $K(ku)$, which is a non-trivial root of $v_2\in\pi_*V(1),$
thus in particular establishing $K(ku)$ as the representing spectrum of
a homology theory of chromatic type $2$.

Moreover: By the calculations of Ausoni in \cite{AuKku}, in particular Theorem 8.1
as recalled above in Theorem \ref{dasmonster}, we 
know that apart from a sporadic class the module 
$V(1)_*K(ku)$ is a free module over the polynomial
algebra on $b$. 

{\rem{To be consistent in denoting $H\F_p$-homology on the left, I refer
    to classes in degree $n$ as $z\in H\F_{p,n}X$ in the following proposition.}}

Recall the construction of the element $b$:
{\prop{Consider the homology algebra of $\CC P^\infty\simeq K(\Z,2)$, which
is a divided power algebra $\Gamma(y)\cong {H\F_p}_*K(\Z,2)$.

Then in the spectral sequence in $H\F_p$-homology associated to the bar filtration of
$K(\Z,3)=B(K(\Z,2))$ the class $y^{p-1}\otimes y$ is an infinite non-bounded cycle 
of degree $(2,2p)$. By \cite[Lemma 2.3]{AuKku} we have in particular an
exact sequence ${H\F_p}_{,5}(K(\Z,3))\rightarrow V(1)_{2p+2}(K(\Z,3))
\rightarrow {H\F_p}_{,2p+2}(K(\Z,3))\rightarrow {H\F_p}_{,0}(K(\Z,3)).$
Since ${H\F_p}_{,5}(K(\Z,3))=0$, and $(P^1)^*(\gamma_{p-1}(y))=0$ the
last map and the first group are zero, so we have a unique class 
$V(1)_{2p+2}K(\Z,3)$ which maps to the class of $y^{p-1}\otimes y$ by the
Hurewicz $V(1)\rightarrow H\F_p$. Finally by considering the
embedding $K(\Z,3)=BBU(1)\rightarrow BBU\rightarrow \coprod_nBGL_n(ku)$ as
before we get the higher Bott element $b\in V(1)_{2p+2}K(ku)$.}}

{\thm{The involution on the higher Bott element is trivial, specifically
	the algebra map $ku\rightarrow ku$ induced by conjugation acts as
	a sign $-1$, and the homeomorphism $\Gamma$ acts as a sign $-1$.
	\begin{proof} By \cite[p. 623]{AuKku} we know 
	$\Sigma K(\Z,2)\rightarrow B_2\rightarrow \Sigma^2(K(\Z,2)^{\wedge 2})$
	induces an injective map in degree $2p+2$:
	$V(1)_{2p+2}B_2\rightarrow V(1)_{2p+2}(\Sigma^2(K(\Z,2)^{\wedge 2}))$
	with $B_2\subset K(\Z,3)$ the image of the $2$-skeleton.

	In particular the map $B_2\rightarrow K(\Z,3)\rightarrow BBU
	\rightarrow\coprod_nBGL_n(ku)$ again is a class associated to
	$GL_1ku$, thus transposition has no effect. Furthermore it
	is a class of simplicial degree $2$ by definition, thus $\Gamma$
	acts as a sign $-1$. Finally we have to determine how complex
	conjugation acts on $b$. For this consider the representative
	of the bar spectral sequence $y^{p-1}\otimes y$. On homology
	of $K(\Z,2)=BU(1)$ complex conjugation acts by a group homomorphism
	on $U(1)$ as $y^n\mapsto (-1)^ny^n$.
	In particular we get: $y^{p-1}\otimes y \mapsto (-1)^py^{p-1}\otimes y=-y^{p-1}\otimes y.$

	In summary $\Gamma$ and the conjugation cancel out, which is visible
	on $B_2$.\end{proof}}}

\subsection{Summary of the Induced Involution}
Here I want to summarise the above results. 
Recall the isomorphism of \cite{AuKku}:
\[\begin{aligned}
V(1)_*K(ku_p)\cong &P(b)\otimes E(\lambda_1,a_1)\\
\oplus &P(b)\otimes E(\lambda_1)\otimes \F_p\{\sigma_n|~1\leq n\leq p-2\}\\
\oplus &P(b)\otimes \F_p\{\partial\lambda_1,\partial b,
	\partial a_1,\partial\lambda_1a_1\}\\
\oplus &P(b)\otimes E(a_1)\otimes \F_p\{t^d\lambda_1|~0<d<p\}\\
\oplus &P(b)\otimes E(\lambda_1)\otimes \F_p\{t^{p^2-p}\lambda_2\}\\
\oplus &\F_p\{s\}.\end{aligned}\]

In the section above we have established that $b$ is invariant under the induced
involution. In the first section we have determined the involution on $\lambda_1$ 
and $a_1$ to each be given by a sign. In the second section we found that the
involution induces $\sigma_n\mapsto (-1)^{n+1}\sigma_n$.
\subsection{The $TC$-Classes}
For the third to the sixth summand in the above decomposition,
one would have to establish, if there is an involution on topological cyclic 
homology, which is compatible with the cyclotomic trace. Specifically the classes 
$t^d\lambda_1$ and $t^{p^2-p}\lambda_2$ are composites of the eponymous classes 
$\lambda_i\in V(1)_*THH(ku)$ and powers of the classes $t\in H_*(\Z/p^n)\rightarrow H_*(\S^1)$ 
arising from the homotopy fixed point spectral sequences from $THH$ to $TC$. In particular, 
one would have to establish how the circle action on $THH$ behaves with respect to the involution. 

Regrettably I have to say that this is beyond the scope of this thesis. In particular I cannot
even offer a conjecture on what map the involution induces on the other summands. Specifically
I do not know the effect of the involution on $\partial, s$ and $t$, which are each part
of the remaining classes, and each most transparently appear in $V(1)_*TC(ku_p)$.

I hope to get back to a full description of the involution on $V(1)_*K(ku_p)$ in future work.
Most probably the description of an involution on $TC$ can be bypassed for
this case, since the class $\partial$ is an artifact of $p$-adic completion,
which is already present in $V(1)_*K(\Z_p)$, and the classes $t^d$ are
visible in the homotopy fixed point spectral sequence for $THH(ku)^{h\S^1}$.
 \clearpage \begin{thebibliography}{9999999}
\bibitem[Ad1]{Ad74} J. Adams, \emph{Stable Homotopy and Generalised Homology,} Univ. of Chicago Press, 1974, Reissued Edition 1995 
\bibitem[Ad2]{Ad69} J. Adams, \emph{Lectures on Generalised Cohomology,} Lec. Notes Math. \textbf{99}, 1969
\bibitem[AnHL]{AnHL} V. Angeltveit, M. Hill, T. Lawson, \emph{The Topological
Hochschild Homology of $\l$ and $ko$}, Am. J. Math. \textbf{132}(2), 2010, \arxiv{0710.4368}
\bibitem[AnR]{AnR} V. Angeltveit, J. Rognes, \emph{Hopf Algebra Structure on Topological 
Hochschild Homology}, Algebr. Geom. Topol. \textbf{5}, 2005, 1223–-1290, \arxivold{0502195}
\bibitem[A-Kku]{AuKku} C. Ausoni, \emph{On the Algebraic $K$-Theory of the Complex $K$-Theory 
Spectrum}, Inventiones mathematicae \textbf{180}(3), 2010, 611--668, \arxiv{0902.2334}
\bibitem[A-THH]{AuTHH} C. Ausoni \emph{Topological Hochschild Homology of Connective Complex 
$K$-Theory}, Am. J. Math. \textbf{127}(6), 2005, 1261--1313 %notarxiv
\bibitem[AR1]{AuQku} C. Ausoni, J. Rognes, \emph{Rational Algebraic $K$-Theory of 
Topological $K$-Theory}, Geom. Topol. \textbf{16}, 2012, 2037--2065, \arxiv{0708.2160}
\bibitem[AR2]{AuKl1} C. Ausoni, J. Rognes, \emph{Algebraic $K$-Theory of Topological $K$-Theory,}
Acta Math. \textbf{188}, 2002, 1--39
\bibitem[AR3]{AuKl2} C. Ausoni, J. Rognes, \emph{Algebraic $K$-Theory of the First Morava
$K$-Theory,} J. Eur. Math. Soc. \textbf{14}, 2011, 1041--1079, \arxiv{1006.3413}
\bibitem[ADR]{ADR} C. Ausoni, B. Dundas, J. Rognes, \emph{ Divisibility of the Dirac Magnetic 
Monopole as a Two-Vector Bundle over the Three-Sphere,} Doc. Math. \textbf{13}, 2008, 795--801
\bibitem[BDR]{BDR2004} N. Baas, B. Dundas, J. Rognes, \emph{Two-Vector Bundles and Forms of 
Elliptic Cohomology}, {London Mathematical Society Lecture Note Series
\textbf{308}}, 2004, 18--44, \arxivold{0306027}
\bibitem[BDRR1]{BDRR2011} N. Baas, B. Dundas, B. Richter, J. Rognes, \emph{Stable Bundles over 
Rig Categories}, {Journal of Topology \textbf{4}(3)}, 2011, 
623--640, \arxiv{0909.1742}
\bibitem[BDRR2]{BDRR2013} N. Baas, B. Dundas, B. Richter, J. Rognes, \emph{Ring Completion of 
Rig Categories}, {Journal f{\"u}r die reine und angewandte Mathematik 
(Crelles Journal) \textbf{674}}, 2013, 43--80, \arxiv{0706.0531}
\bibitem[BR1]{BaRi05} A. Baker, B. Richter, \emph{$\Gamma$-Cohomology of Rings of Numerical
Polynomials and $E_\infty$-Structures on $K$-Theory}, Comment. Math. Helv.
\textbf{80} (4), 2005, 691--723, \arxivold{0304473}
\bibitem[BR2]{BaRi08} A. Baker, B. Richter, \emph{Uniqueness of {$E_\infty$}-Structures for
Connective Covers}, Proc. Amer. Math. Soc. \textbf{136}, (2008), 707--714, \arxivold{0506422}
\bibitem[Ba]{Bar} C. Barwick, \emph{Multiplicative Structures on Algebraic $K$-Theory}, 
preprint \arxiv{1304.4867}, (2013)
\bibitem[Ben]{Bena} J. B\'enabou, \emph{Introduction to Bicategories}, in Reports of the Midwest
Category Seminar, Springer Berlin, 1967
\bibitem[Ber1]{Bergn1} J. Bergner, \emph{A Model Category Structure on the Category of Simplicial
Categories}, Trans. Amer. Math. Soc. \textbf{359}, 2007, 2043--2058, \arxivold{0406507}
\bibitem[Ber2]{Bergn2} J. Bergner, \emph{A Survey of $(\infty,1)$-Categories}, \arxivold{0610239}
\bibitem[Ber3]{Bergn3} J. Bergner, \emph{Models for Homotopical Higher Categories}, Talk at
the MSRI (Jan. 2014), \href{https://www.youtube.com/watch?v=q8J0bbaZnRM}{https://www.youtube.com/watch?v=q8J0bbaZnRM}
\bibitem[BGT1]{BGT2013} A. Blumberg, D. Gepner, G. Tabuada, \emph{A Universal Characterization
of Higher Algebraic $K$-theory}, {Geometry and Topology \textbf{17}}, 
2013, 733--838, \arxiv{1001.2282}
\bibitem[BGT2]{BGT} A. Blumberg, D. Gepner, G. Tabuada, \emph{Uniqueness of the 
Multiplicative Cyclotomic Trace}, to appear in Advances in Mathematics, \arxiv{1103.3923}, (2012)
\bibitem[BV]{BV} J. Boardman, R. Vogt, \emph{Homotopy Invariant Algebraic Structures on Topological
Spaces,} Springer Verlag Berlin, 1973
\bibitem[BLPRZ]{BLPRZ} I. Bobkova, A. Lindenstrauss, K. Poirier, B. Richter, I. Zakharevich,
\emph{On the Higher Topological Hochschild Homology of $\F_p$ and Commutative $\F_p$-Group
Algebras,} Proceedings of the BIRS workshop \emph{WIT: Women in Topology}, 2014
\bibitem[B1]{B1} M. B\"okstedt, \emph{Topological Hochschild Homology,} Unpublished
\bibitem[B2]{B2} M. B\"okstedt, \emph{The Topological Hochschild Homology of $\Z$ and $\Z/p$},
Unpublished
\bibitem[BM]{BMTCZ} M. B\"okstedt, I. Madsen, \emph{Topological Cyclic Homology of the
Integers,} Ast\'erisque \textbf{226}, 1994, 57--143
\bibitem[BHM]{BHMtr} M. B\"okstedt, W. Hsiang, I. Madsen, \emph{The Cyclotomic Trace and 
Algebraic $K$-Theory of Spaces}, Inventiones mathematicae \textbf{111}(1), 1993, 465--539 %noarxiv
\bibitem[BFV]{BFV} M. Brun, Z. Fiedorowicz, R. Vogt, \emph{On the Multiplicative 
Structure of Topological Hochschild Homology}, Algebraic \& Geometric 
Topology \textbf{7}, 2007, 1633–-1650, \arxivold{0410367}
\bibitem[BrRo]{BrRo05} R. Bruner, J. Rognes, \emph{Differentials in the Homological Homotopy Fixed
Point Spectral Sequence,} Alg. Geom. Topol. \textbf{5}, 2005, 653--690, \arxivold{0406081}
\bibitem[CCG]{CCG2010} P. Carrasco, A. Cegarra, A. Garz\'on, \emph{Nerves and Classifying 
Spaces for Bicategories}, Algebraic and Geometric Topology \textbf{10} 
(1), 2010, 219--274, \arxiv{0903.5058}
\bibitem[D1]{D98} B. Dundas, \emph{The Cyclotomic Trace for Symmetric Monoidal Categories,}
Geometry and Topology \AA rhus 1998, Contemp. Math. \textbf{258}, 2000, AMS, 121--143
\bibitem[D2]{D2} B. Dundas, \emph{The Cyclotomic Trace Preserves Operad Actions,} 2014, preprint
\bibitem[DGM]{DGM12} B. Dundas, T. Goodwillie, R. McCarthy, \emph{The Local Structure of 
Algebraic $K$-Theory}, Springer London, 2012 %noarxiv
%\bibitem[DK]{DK} W. Dwyer, D. Kan, \emph{Function Complexes in Homotopical Algebra},
%Topology \textbf{19}, 1980, 427--440
\bibitem[EKMM]{EKMM} A. Elmendorf, I. Kriz, M. Mandell, J.P. May, \emph{Rings, Modules,
and Algebras in Stable Homotopy Theory}, AMS Surveys and Monographs \textbf{47}, 1995
\bibitem[EM]{EM} A. Elmendorf, M. Mandell, \emph{Rings, Modules, and Algebras in 
Infinite Loop Space Theory}, Adv. Math. \textbf{205}, 2006, 163--228, \arxivold{0403403}
\bibitem[Gau\ss]{Gauss} C.F. Gau\ss, \emph{Disquisitiones Arithmeticae}, 
1801, Latin \href{http://edoc.hu-berlin.de/ebind/hdok2/h284\_gauss\_1801/pdf/h284\_gauss\_1801.pdf}{
http://edoc.hu-berlin.de/ebind/hdok2/h284\_gauss\_1801/pdf/h284\_gauss\_1801.pdf},\\ English
revised translation 1986, Springer New York 
\bibitem[GGN]{GGN} D. Gepner, M. Groth, T. Nikolaus, \emph{Universality of Multiplicative
Infinite Loop Space Machines}, preprint, \arxiv{1305.4550}
\bibitem[GSn]{GSn} D. Gepner, V. Snaith, \emph{Motivic Spectra Representing Cobordism and $K$-Theory,}
Documenta Math. \textbf{14}, 2009, 359--396, \arxiv{0712.2817}
\bibitem[GHOsT]{GHOsT} P. Goerss, M. Hopkins, \emph{Moduli Spaces of Commutative Ring Spectra,}
London Math. Soc. Lecture Notes \textbf{315}, 2004, 151--200
\bibitem[GJ]{GJ} P. Goerss, J. Jardine, \emph{Simplicial Homotopy Theory}, Birkh{\"a}user Basel, 2009 reprint %noarxiv
\bibitem[GGRSV]{GGRSV} Jo. Guccione, Ju. Guccione, M. Redondo, A. Solotar, O. Villamayor,
\emph{Cyclic Homology of Algebras with One Generator,} $K$-Theory \textbf{5}, 1991, 51--69
\bibitem[GJOs]{GJOs} N. Gurski, N. Johnson, A. Osorno, \emph{$K$-Theory for
$2$-Categories}, preprint, \arxiv{1503.07824}
%\bibitem[GPS]{GPS} R. Gordon, A. Power, R. Street, \emph{Coherence for Tricategories,} Mem. Amer. Math. Soc. \textbf{117} 
%(558), 1995
%\bibitem[GS]{GSch} P. Goerss, K. Schemmerhorn, \emph{Model Categories and Simplicial Methods,}
%Contemp. Math. \textbf{436}, 2077, 3--49, \arxivold{0609537}
\bibitem[J]{Joyal} A. Joyal, \emph{Quasi-Categories and Kan Complexes}, 
J. Pure Appl. Algebra \textbf{175}, 2002, 207--222 %noarxiv
\bibitem[JT]{JoyT} A. Joyal, M. Tierney, \emph{Quasi-Categories vs. Segal Spaces},
Contemp. Math. \textbf{431}, 2007, 277--326, \arxivold{0607820}
\bibitem[KV]{KV} M. Kapranov, V. Voevodsky, \emph{Braided Monoidal $2$-Categories
and Manin-Schechtman Higher Braid Groups,} J. Pure Appl. Algebra \textbf{92}, 1994, 241--267
\bibitem[Ko]{Kochm} S. Kochman, \emph{Bordism, Stable Homotopy and Adams Spectral Sequences,} Fields Institute Monographs, AMS, 1996
\bibitem[K]{K} H. K\"onig, \emph{The Segal Model as a Ring Completion and a Tensor 
Product of Permutative Categories}, PhD thesis, 2011, 
\href{http://ediss.sub.uni-hamburg.de/volltexte/2011/5032}{http://ediss.sub.uni-hamburg.de/volltexte/2011/5032}
\bibitem[LP]{LP2008} S. Lack, S. Paoli, \emph{2-Nerves for Bicategories},
{Journal of $K$-theory \textbf{38(2)}}, 2008, 153--175, \arxivold{0607271}
\bibitem[Lan]{L2011} M. Lange, \emph{Examples of Involutions on Algebraic $K$-Theory of
Bimonoidal Categories}, 2011, Diploma Thesis, available at\\
\href{http://www.math.uni-hamburg.de/home/richter/DA-Lange.pdf}{http://www.math.uni-hamburg.de/home/richter/DA-Lange.pdf}
\bibitem[Lap]{Lap} M. Laplaza, \emph{Coherence for Categories with Associativity, 
Commutativity and Distributivity}, Bull. AMS \textbf{78}, 1972, 220--222 %noarxiv
%\bibitem[Law]{Law} T. Lawson, \emph{Commutative $\Gamma$-Rings do not Model All 
%Commutative Ring Spectra}, Homology, Homotopy \& Applications \textbf{11.2}, 2009, 189--194, %noarxiv
\bibitem[Le]{Lei} T. Leinster, \emph{Basic Bicategories}, \arxivold{9810017}, 1998
\bibitem[Lo]{Loday} J.-L. Loday, \emph{Cyclic Homology}, Grundlehren \textbf{301}, Springer, 1992 %noarxiv
\bibitem[Lu1]{Lu1} J. Lurie, \emph{Higher Topos Theory}, Princeton University Press 
\textbf{170}, 2009, \arxivold{0608040}
\bibitem[Lu2]{Lu2} J. Lurie, \emph{Higher Algebra}, (August 2012) Preprint at\\ 
\href{http://www.math.harvard.edu/\textasciitilde lurie/papers/higheralgebra.pdf}{
http://www.math.harvard.edu/\textasciitilde lurie/papers/higheralgebra.pdf}
\bibitem[Lu]{LuStab} J. Lurie, \emph{Stable Infinity Categories}, \arxivold{0608228}
\bibitem[McL]{McL} S. Mac Lane, \emph{Homology,} 1995 reprint of the 1975 edition
\bibitem[MMSS]{MMSS} M. Mandell, J.P. May, S. Schwede, B. Shipley \emph{Model Categories of
Diagram Spectra}, Proceedings of the London Mathematical Society, \textbf{82}(2), 2001, 441--512 %noarxiv
\bibitem[MN]{MN} T. Masuda, T. Natsume, \emph{Cyclic Cohomology of Certain Affine Schemes,}
Publ. RIMS, Kyoto Univ. \textbf{21}, 1985, 1261--1279
\bibitem[May{$E_\infty$}]{MayEinf} J.P. May, \emph{$E_\infty$-ring spaces and $E_\infty$-ring spectra}, Lect. Notes Math. \textbf{577}, Springer Heidelberg, 1977 
\bibitem[May1]{MaySoat} J.P. May, \emph{Simplicial objects in algebraic topology}, Chicago Lect. in Math., Univ. Chicago Press, 1967
\bibitem[May2]{May2009} J.P. May, \emph{The Construction of $E_\infty$-Ring Spaces from 
Bipermutative Categories}, Geometry \& Topology Monographs \textbf{16}, 2009, 283--330, \arxiv{0903.2818}
\bibitem[MT]{MT1978} J.P. May, R. Thomason, \emph{The Uniqueness of Infinite Loop Space 
Machines}, {Journal of Topology \textbf{17}(3)}, 1978, 205--224%noarxiv
\bibitem[MSm]{MSm} J. McClure, J. Smith, \emph{Operads and Cosimplicial Objects: An Introduction,} 2004, \arxivold{0402117}
\bibitem[MS]{MS} J. McClure, R. Staffeldt, \emph{On the Topological Hochschild Homology
of} $bu$ I, Am. J. Math., 1993, 1--45 %noarxiv
\bibitem[MT]{MT} M. Mimura, H. Toda, \emph{Topology of Lie Groups I}, Transl. of Math. Mon. AMS, 2000
\bibitem[Oka]{Oka} S. Oka, \emph{Multiplicative Structure of Finite Ring Spectra and Stable
Homotopy of Spheres,} Alg. Top. \AA rhus 1982, Lecture Notes in Math. \textbf{1051}, 1984, 418--441
\bibitem[Os]{Os} A. Osorno, \emph{Spectra Associated to Symmetric Monoidal
Bicategories}, Algebraic and Geometric Topology, \textbf{12}(1), 2012, 307--342, \arxiv{1005.2227}
\bibitem[PR]{PRi} T. Pirashvili, B. Richter, \emph{Hochschild and Cyclic Homology via Functor Homology,} 
$K$-Theory \textbf{25} (1), 2002, 39--49 %noarxiv
\bibitem[Q1]{Q1971} D. Quillen, \emph{The Adams Conjecture}, {Journal of Topology \textbf{10}(1)}, 1971, 67--80 %noarxiv
\bibitem[Q2]{Q1972} D. Quillen, \emph{On the Cohomology and $K$-theory of the General Linear 
Groups over a Finite Field}, {Annals of Mathematics}, 1972, 552--586 %noarxiv
\bibitem[Q3]{Q1973} D. Quillen, \emph{Higher algebraic $K$-theory: I}, {Higher $K$-theories}, 1973, 85--147 %noarxiv
\bibitem[R]{Ri2010} B. Richter, \emph{An Involution on the $K$-theory of Bimonoidal 
Categories with Anti-Involution}, Algebraic and Geometric Topology \textbf{10}, (2010), 315--342, \arxiv{0804.0401}
\bibitem[Ro]{Rog14} J. Rognes, \emph{Chromatic Redshift,} \arxiv{1403.4838}
\bibitem[RSS]{RSS14} J. Rognes, S. Sagave, C. Schlichtkrull, \emph{Logarithmic Topological 
Hochschild Homology of Topological $K$-Theory Spectra}, preprint, \arxiv{1410.2170}, (2014)
\bibitem[Ros]{Ros1994} J. Rosenberg, \emph{Algebraic $K$-Theory and its Applications}, 1994 %noarxiv
\bibitem[Rot]{Rot2009} J. Rotman, \emph{An Introduction to Homological Algebra}, Springer New York, second edition 2009
\bibitem[Ru]{Rudy} Y. Rudyak, \emph{On Thom Spectra, Orientability, and Cobordism}, Springer Berlin Heidelberg, 1998
\bibitem[SaS]{SaSch14} S. Sagave, C. Schlichtkrull, \emph{Localization Sequences for
Logarithmic Topological Hochschild Homology}, preprint, \arxiv{1402.1317}, (2014) 
\bibitem[S]{S tr} C. Schlichtkrull, \emph{The Cyclotomic Trace for Symmetric Ring 
Spectra}, Geometry \& Topology Monographs \textbf{16}, 2009, 545--592, \arxiv{0903.3495}
\bibitem[Schm]{Schm} V. Schmitt, \emph{Tensor Product of Symmetric Monoidal Categories}, 2007, \arxiv{0711.0324}
\bibitem[SP]{SP09} C. Schommer-Pries, \emph{The Classification of Two-Dimensional Extended Topological Field Theories,}
PhD thesis, UC Berkeley 2009, \arxiv{1112.1000}
\bibitem[Schw1]{SchV0} S. Schwede, \emph{The Stable Homotopy Category is Rigid,} Ann. Math. \textbf{166}, 2007, 837--863
\bibitem[Schw2]{SchSym} S. Schwede, \emph{An untitled book project about symmetric spectra (version 2012)}, 
\href{http://www.math.uni-bonn.de/people/schwede/SymSpec.pdf}{http://www.math.uni-bonn.de/people/schwede/SymSpec.pdf}
\bibitem[Se]{Seg} G. Segal, \emph{Categories and Cohomology Theories}, Topology \textbf{13}(3), 1974, 293--312 %noarxiv
\bibitem[Se]{Serre} J.-P. Serre, \emph{Linear Representations of Finite Groups}, Springer New York, 1977%noarxiv
\bibitem[Sh]{ShTHH} B. Shipley, \emph{Symmetric Ring Spectra and Topological Hochschild 
homology}, $K$-Theory \textbf{19}(2), 2000, 155-–183, \arxivold{9801079}
\bibitem[Sn]{Sn} V. Snaith, \emph{Algebraic Cobordism And $K$-Theory,} Mem. Amer. Math. Soc. \textbf{221}, 1979
\bibitem[Str1]{Str} R. Street, \emph{Categorical Structures}, in Handbook of Algebra \textbf{1},
ed. M. Hazewinkel, Elsevier, 1996
\bibitem[Str2]{Str2} R. Street, \emph{The Algebra of Oriented Simplexes}, J. Pure Appl Alg. \textbf{49}(3), 1987, 283--335
\bibitem[Sw]{Sw} R. Switzer, \emph{Algebraic Topology - Homology and Homotopy}, Classics in Math., 
Springer Berlin Heidelberg, 2002 reprint %noarxiv
\bibitem[Tab]{Tab} G. Tabuada, \emph{Homotopy Theory of Spectral Categories}, Adv. Math.
\textbf{221(4)}, 2009, 1122--1143, \arxiv{0801.4524}
\bibitem[Th1]{Th1} R. Thomason, \emph{Symmetric Monoidal Categories Model All Connective 
Spectra}, Theory and Applications of Categories \textbf{1.5}, 1995, 78--118 %noarxiv
\bibitem[Th2]{Th2} R. Thomason, \emph{Homotopy Colimits in the Category of Small 
Categories}, Math. Proc. Cambridge Philos. Soc. \textbf{85.1}, 1979, 91--109 %noarxiv
\bibitem[Th3]{Th3} R. Thomason, \emph{$Cat$ as a Closed Model Category,} Cahiers Topologie G\'eom.
Diff\'erentielle \textbf{21}(3), 1980, 305--324
\bibitem[Toda]{Toda} H. Toda, \emph{On Spectra Realising Exterior Parts of the Steenrod Algebra,}
Topology \textbf{10}, 1971, 53--65
\bibitem[W]{WeiK} C. Weibel, \emph{The $K$-Book: An Introduction to Algebraic $K$-Theory},
Graduate Studies in Math. \textbf{145}, AMS, 2013
\bibitem[Z]{Z} S. Ziegenhagen, \emph{$E_n$-Cohomology as Functor Cohomology and
Additional Structures,} PhD thesis, 2014, \href{http://ediss.sub.uni-hamburg.de/volltexte/2014/6950}{
http://ediss.sub.uni-hamburg.de/volltexte/2014/6950}
\end{thebibliography}

\section*{Summary}\pagestyle{empty}
In this thesis I investigate the interaction of multiplicative and involutive structures
on algebraic $K$-theory of $E_\infty$-ring spectra. Algebraic $K$-theory is 
associated classically to discrete rings by a definition of Quillen \cite{Q1973} with
preceding approaches by Grothendieck, Bass and Milnor, who defined $K_0R$, $K_1R$ and $K_2R$ respectively.
Quillen identified algebraic $K$-theory as the homotopy groups of a space naturally associated
to a ring, unifying the first three definitions, and providing a
definition of $K_nR$ for all natural numbers. One approach to computations is the generalisation
of $K$-theory to more ringlike objects, yielding more induced structures
on $K$-theory, for instance multiplicative structures as described in \cite{EM,GGN,BGT,May2009} and 
involutions as defined in \cite{Ri2010}.

Driven by the main example, algebraic $K$-theory of the connective complex $K$-theory spectrum $ku$, and
building on the computations of Christian Ausoni in \cite{AuKku,AuTHH}, I focus on the induced multiplicative
structure on $K(ku)$ in chapters 1 to 3. Building on the delooping of permutative bicategories as developed
by Ang\'elica Osorno in \cite{Os}, I exhibit a tensor product on a bicategory of matrices $\M$ associated
to a bipermutative category $\R$ in chapter 2. By modifying 
the delooping of Osorno in chapter 3 I find an induced $E_\infty$-ring spectrum structure on $K(ku)$ by
identifying this spectrum as the Eilenberg-MacLane-spectrum associated to the bicategory of matrices
over finite-dimensional complex vector spaces. The involution as defined in \cite{Ri2010} easily
generalises to this setting, so I can exhibit the interaction of the involution with the multiplication
easily.

Since the calculations in \cite{AuKku,AuTHH} rely on trace methods, i.e., are obtained by careful
comparison of $K(ku)$ to topological Hochschild homology $THH(ku)$ along the trace map, I can use
the compatibility of the trace map with the multiplicative structures defined on both as a universal
property by the results of \cite{BGT2013,BGT}. This in particular implies that the trace map is
compatible with the involution defined on $K$-theory as in \cite{Ri2010} and on topological
Hochschild homology analogous to \cite{Loday}, giving the main result of chapter 5.

Finally in chapter 6 I investigate the involution on mod $(p,v_1)$ homotopy groups of $K(ku)$ as 
calculated in \cite{AuKku}. Specifically there is a subalgebra of $V(1)_*K(ku)$, which can be
understood purely in terms of the trace map $K(ku)\rightarrow THH(ku)$, and there are special
classes $\sigma_n$ as well as the ``higher Bott element'' $b$. For all of these I describe
the effect induced by complex conjugation on $ku$, and thus the induced involution on 
algebraic $K$-theory on $V(1)_*K(ku)$. 

\section*{Zusammenfassung}
In dieser Dissertation untersuche ich Multiplikationen und Involutionen auf algebraischer
$K$-Theorie von $E_\infty$-Ringspektren. Klassisch ist algebraische $K$-Theorie
eine Invariante diskreter Ringe definiert von Quillen in \cite{Q1973}. Quillen
vereinheitlicht Definitionen von Grothendieck, Bass und Milnor der Gruppen $K_iR$ f\"ur
$i=0,1,2$ in dieser Reihenfolge. Er definiert algebraische $K$-Theorie als Homotopiegruppen
eines nat\"urlich zu einem Ring $R$ assoziierten Raumes $BGL(R)^+$. Diese Definition vereinheitlicht
zugleich die vorher genannten Definitionen und gibt eine Definition f\"ur alle
nat\"urlichen Zahlen. Ein Zugang zu Berechnungen ist die Verallgemeinerung
von $K$-Theorie auf ringartige Objekte, was insbesondere Aufschluss gibt \"uber
induzierte Strukturen auf $K$-Theorie wie etwa Multiplikationen \cite{EM,GGN,BGT,May2009}
und Involutionen \cite{Ri2010}.

In den Kapiteln 1 bis 3 untersuche ich die multiplikative Struktur von $K(ku)$, die
durch die Identifikation entlang \cite{BDRR2011} als Delooping einer Bikategorie von 
Matrizen $\M$ induziert wird.
Genauer definiere ich in Kapitel 2 ein Tensorprodukt, das eine Multiplikation auf
$\M$ induziert, die sich mit der additiven Struktur, die Ang\'elica Osorno \cite{Os}
definiert, vertr\"agt. In Kapitel 3 beschreibe ich eine Variante ihres Deloopings
\cite{Os}, die durch das Tensorprodukt die Struktur eines $E_\infty$-Ringspektrums erh\"alt. 
Die von Birgit Richter in \cite{Ri2010} beschriebene Involution l\"asst sich
auf $\M$ erweitern, und wir erhalten, dass die Involution diese Multiplikation opponiert.

Die Berechnungen der mod $(p,v_1)$ Homotopiegruppen von Christian Ausoni in \cite{AuKku,AuTHH} basieren
grundlegend auf Spurmethoden, also einem sorgsamen Vergleich algebraischer $K$-Theorie mit topologischer
Hochschildhomologie. Die vergleichende Abbildung ist die Spur $K(ku)\rightarrow THH(ku)$, die ich 
mithilfe der Resultate aus \cite{BGT2013,BGT} als die universelle multiplikative nat\"urliche
Transformation $K\Rightarrow THH$ in Kapitel 5 definieren kann. Aus dieser Universalit\"at folgt
das Hauptresultat von Kapitel 5, dass sich die Spurabbildung auch mit der induzierten
Involution auf $K$-Theorie wie in \cite{Ri2010} beschrieben und der auf topologischer 
Hochschildhomologie zu der in \cite{Loday} analogen Involution vertr\"agt.

In Kapitel 6 untersuche ich die Involution auf mod $(p,v_1)$ Homotopiegruppen von $K(ku)$, wie sie aus
der Berechnung von \cite{AuKku} hervorgehen. Es gibt eine Unteralgebra in $V(1)_*K(ku)$, die sich vollst\"andig
\"uber die Spur in $V(1)_*THH(ku)$ verstehen l\"asst, sowie spezielle Elemente $\sigma_n$ und das
``h\"ohere Bott-Element'' $b$ in diesem Modul. F\"ur diese Klassen beschreibe ich die 
Involution auf $V(1)_*K(ku)$.

\newpage
\section*{Publications/Publikationen}
I have not published anything based on this thesis.\\
Es gibt keine Ver\"offentlichungen, die aus dieser Arbeit hervorgegangen sind.

\section*{Eidesstattliche Versicherung} Ich versichere an Eides statt, dass ich die Arbeit selbstst\"andig verfasst und 
keine anderen als die angegebenen Hilfsmittel und Quellen benutzt habe.\\[10ex]\flushright$\overline{\mathtt{Marc~Lange,}~
~~~~ \mathtt{Hamburg, ~24.7.2015}}$\end{document}
\grid
