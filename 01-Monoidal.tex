\chapter{Permutative Categories and Connective Spectra}
\pagestyle{headings} 
This chapter is just a summary of known techniques to combinatorially
model (connective) spectra by permutative categories. In particular
until \ref{pcatneu} there is nothing original in this chapter. 
The examples are borrowed from \cite[pp. 160-167]{MayEinf} and 
\cite[pp. 337+338]{Ri2010}. However I deviate quite a bit
from May's notation. Furthermore I want to warn the reader that I am 
close to an erroneous sequence of lemmas in \cite{MayEinf} 
(VI.2.3, VI.2.6, VI.4.4) (cf. \cite[p. 321]{May2009}). 

The claimed result in \cite{MayEinf} can be stated informally as: 
Bipermutative categories yield maximally homotopy commutative ring-spectra. 
The error is combinatorial, in how the multiplication and 
addition ought to interact, governed by the notion of an 
``operad pair''. But the claimed ``operad pair'' in \cite{MayEinf} is
not an operad pair as defined there. The result still holds \cite{May2009} 
(and its accompanying papers), \cite{EM}, but the techniques employed differ 
quite a bit from the planned proof in \cite{MayEinf}.

\section{Delooping Permutative Categories}
Permutative categories seen through a modern eye are a categorified
version of abelian groups with just as much strictness as generality
would allow - compare the classical strictification result \ref{strict1}.
{\defn{A \textbf{permutative category} $(\A,+,0,c_+)$ is a category $\A$ 
together with a functor $+\colon \A\times\A\rightarrow \A$, a strict 
additive unit $0\in\A$, and a twist natural transformation (for $T\colon \A\times\A\rightarrow
\A\times\A$ the exchange of factors):
\[c_+\colon (+\circ T)\Rightarrow +,\] satisfying the following 
conditions:\begin{enumerate}\item $+$ is strictly associative:
\[+\circ (+\times id)=+\circ (id\times +),\]
\item The unit $0$ is a strict unit for $+$:\[0+\_=\_+0=id_\A,\]
\item The twist is trivial at $0$: For every object $a\in\A$ we have the
identity \[c_+=id: a=0+a\rightarrow a+0=a,\]
which is natural in $a$.
\item The twist is its own inverse: For every two objects $a,b\in \A$ we 
have the commutative diagram:
\[\xymatrix{a+b\ar[dr]_{c_+} \ar@=[rr]&& a+b \\& b+a.\ar[ur]_{c_+}}\]
\item The twist is associative: For each triple of objects $a,b,c\in\A$
we have the commutative diagrams:
\[\begin{array}{cc}\xymatrix{a+b+c\ar[r]^{c_+}\ar[dr]_{id+c_+} & c+a+b\\
& a+c+b\ar[u]_{c_++id},}~~~
\xymatrix{a+b+c\ar[r]^{c_+}\ar[dr]_{c_++id} & b+c+a\\
& b+a+c\ar[u]_{id+c_+}.}\end{array}\]\end{enumerate}}}

I have no need in this thesis for the most general symmetric monoidal 
categories given for instance by module categories. Nonetheless the 
following statement shows that the structure of permutative 
categories is sufficiently general:

{\thm{\label{strict1} For any symmetric monoidal category 
$\C$ there is a symmetrically monoidally 
equivalent permutative category $Str(\C)$ with a natural
equivalence $Str(\C)\rightarrow \C$.}}

There are many ways to obtain this result. A brute force way is to
consider words in objects, have the empty word be the strict unit, and
add in morphisms accordingly 
\cite[Prop VI.3.2,cf. pp.155-157]{MayEinf}. This proof has as a 
corollary that a small symmetric monoidal category of cardinality 
$\aleph$ yields a permutative category of size smaller than 
$\aleph^\omega$ (for $\omega=|\N|$). In particular the theorem stays 
true with ``small'' added in everywhere.

The high-tech way to show this result is given by the Yoneda Lemma
in bicategories \cite[2.3]{Lei}. One considers a monoidal category 
$\C$ as a one-point bicategory $\Sigma\C$, embeds it into the equivalent 
one-point 2-category given by the essential image of the Yoneda 
embedding \[Y\colon \Sigma\C \rightarrow Fun(\Sigma\C^{op},Cat_1)\]
and the symmetry just comes along. Then sizes are limited by the
bicategorical Yoneda Lemma.

Regarding enrichments we find that a
topological, simplicial, etc. symmetric monoidal category strictifies to a 
permutative category of the same kind.

\subsection{Bimonoidal Categories} Since this thesis is about 
multiplicative structures, I want to introduce the types of 
multiplication on permutative categories right away. Furthermore, I 
prove a convenient lemma, so that I do not have to bore the 
reader with pages of coherence diagrams. The following two concepts 
are directly copied from \cite{EM} for instance, although a look into
\cite{MayEinf} shows that at least the $E_\infty$-version (i.e., 
bipermutative categories) was known to be a fruitful concept for much 
longer. Like the concept of permutative categories these concepts
are strictified versions of general ringlike objects in $1$-categories.
Laplaza has shown in 1972 \cite{Lap} that the analogous strictification
result to \ref{strict1} above holds for these structures. Thus the
following definitions represent no loss of generality.

{\defn{\label{bim1}A \textbf{ring category} $(\R,+,\cdot,0,1,c_+)$ is given by a 
permutative structure $(\R,+,0,c_+)$ and a strictly associative and 
strictly unital monoidal structure $(\R,\cdot,1)$, which interact by 
two natural isomorphisms: \[\lambda \colon ab+ab'
\rightarrow a(b+b'),~~ \rho \colon ab+a'b\rightarrow (a+a')b,\]
 such that the following properties hold:
 \begin{enumerate} \item $0$ is a strict zero for multiplication $\cdot$:
\[0\cdot a = a\cdot 0 = 0~~~\forall a\in \R,\]
\item $+$-associativity of distributors:
\[\lambda\circ(\lambda+ \id)=\lambda\circ(\id+ \lambda),~~
\rho\circ(\rho+ \id)=(\id+ \rho)\circ \rho,\]
\item additive symmetry of distributors:
\[(c_+ \cdot \id)\circ\lambda = \lambda\circ c_+,~~
(\id \cdot c_+)\circ\rho = \rho\circ c_+,\]
\item $\cdot $-associativity of distributors:
\[\lambda = (\id\cdot \lambda)\circ\lambda,~~
\rho = (\rho \cdot \id)\circ\rho,\]
\item middle $\cdot$-associativity of distributors:
\[(\id\cdot \rho)\circ\lambda=(\lambda\cdot \id)\circ\rho,\]
\item mixed associativity of distributors:
\[\lambda\circ(\rho+\rho)= \rho\circ(\lambda+\lambda)
	\circ(\id+c_++\id).\]\end{enumerate}}}

It is nice to have a multiplicative structure on any given object, but
it is genuinely hard to produce ring categories, which are not also 
commutative up to some degree or an infinity of degrees. In particular,
I do not investigate plain ring categories in this thesis. So I define 
the $E_\infty$-multiplicative version next, also directly following 
\cite{MayEinf,EM}, but explicitly with no strictness assumptions on
either distributor. This type of category is the central object of study
in the first three chapters.

{\defn{A \textbf{bipermutative category} $(\R,+,\cdot,0,1,c_+,c_\cdot)$ is a 
ring category, where the multiplicative category $(\R,\cdot,1)$ is 
also permutative with twist $c_\cdot$, and where the distributors are 
interrelated via the multiplicative twist as follows:
\[\xymatrix{ab+ab'\ar[r]^{\lambda} \ar[d]^{c_\cdot+c_\cdot} & a(b+b') 
						\ar[d]^{c_\cdot}\\
ba+b'a \ar[r]^{\rho} & (b+b')a.}\]}}

{\rem{\label{IsoIsoIso} Do note that although I am following 
\cite{EM} as well, I want the distributivity transformations to be 
isomorphisms! This is essential in defining the bicategory of matrices
in Chapter \ref{modulbicat}. 

I do not fix either distributor to be the identity intentionally: 
Since I want to investigate multiplicative structures 
interacting with involutions, having both distributors general 
isomorphisms not forced to be identities, makes it meaningful to speak 
of the multiplicatively opposite bimonoidal/bipermutative category.}}

\subsection{Bipermutative Structures on Finite Sets}
As an illustration that bipermutative categories are natural 
things to consider, I give a lemma which applies in a variety of 
cases, where the category is a skeletal version of some category 
with coproducts and products (or tensor products). I have denoted the
following lemma analogous to the second monoidal structure being the
product. However, I want to explicitly emphasise that I do not assume
$\pi$ to be a product-functor or the unit $*\in\C$ to be terminal.

{\lem{\label{bipermallthethings} Let $\C$ be a small category with 
coproducts, which is furthermore a permutative closed category with
monoidal structure \[\pi\colon \C\times\C\rightarrow \C,\] 
which is strictly associative and has unit $*\in\C$. Assume $\pi$ has a
right adjoint: \[Hom(-,-)\colon \C^{op}\times\C\rightarrow 
\C.\] Assume furthermore a chosen functor representing coproducts:
\[\sqcup\colon \C\times \C \rightarrow \C,\] which is strictly 
associative, and a chosen representative $\emptyset$ for the
initial object, thus the unit for $\sqcup$.

Then this can be endowed with unique natural transformations 
$c_\sqcup,c_\pi,\lambda,\rho$, such that the resulting tuple 
$(\C,\sqcup,\pi, \emptyset,*,c_\sqcup,c_\pi)$ is a bipermutative 
category.

\begin{proof} The proof is just a repeated application of universal 
properties. Because $\pi(-,c)$ has a right adjoint, it commutes with 
coproducts, in particular we have $\pi(\emptyset,c)=\emptyset, ~~
\forall c\in\C$. Consider the additive symmetry: For $T\colon \C\times
\C\rightarrow \C\times \C$ the symmetry of the product on categories
both $\sqcup$ and $\sqcup\circ T$ represent a coproduct-functor on $\C$, 
hence by the universal property of the coproduct, we get a unique natural 
transformation \[c_\sqcup\colon \sqcup\circ T\Rightarrow \sqcup.\]
By uniqueness of the natural isomorphism $c\sqcup d\rightarrow c\sqcup d$, for
every pair $c,d\in\C$, we also get that $c_\sqcup$ is a 
symmetry: \[id_{\sqcup}=c_\sqcup^2\colon \sqcup = 
\sqcup\circ T^2\Rightarrow \sqcup.\]
The other natural isomorphisms are constructed much the same way.

For the interaction of the natural isomorphisms consider for
instance the relation $\lambda=c_\pi\circ\rho\circ(c_\pi\sqcup c_\pi)$.
Both are natural isomorphisms between the functors:
\[\sqcup\circ (\pi\times\pi)\circ (id\times T\times id)
\circ (\Delta\times id) \Rightarrow \pi\circ (id\times \sqcup),\]
hence again by uniqueness of those natural isomorphisms we have 
equality. Every other diagram commutes by the same reasoning.
\end{proof}}}

This lemma illustrates well, why ring categories which are just
associative but admit no multiplicative symmetries, are a bit
harder to come by. The easy (closed) monoidal constructions usually
come from symmetric universal properties: product, tensor product, 
smash product, etc.. What follows are the prototypical examples 
which provide the structural morphisms in my examples of interest.

{\ex{\label{Fin} In everything ringlike that follows, the category 
with objects the non-negative integers $\{\mathbf{n}=\{1,\ldots,n\}|~~
n\in\N_0\}$ and with 
morphisms the symmetric groups $\Sigma_*(\textbf{n},\textbf{n}) = 
\Sigma_n$, i.e., $\Sigma_* = \coprod_n \Sigma_n$, features 
prominently. In particular categories and bicategories of the 
form ``free modules over $\R$'' have their 
structural natural transformations given by permutations.

Define the following functors: \[+,\cdot\colon \Sigma_*\times\Sigma_*
\rightarrow \Sigma_*\] on objects: \[\textbf{n}+\textbf{m} := \{1,\ldots,n+m\}, 
\textbf{n}\cdot \textbf{m} := \{1,\ldots,nm\},\]
and more interestingly on morphisms: \[(f+g)(i):=\begin{cases} f(i), ~~
& i\leq n,\\ g(i-n)+n, ~~& i\geq n+1, \end{cases}~~~ \mathrm{for~~} 
f\in \Sigma_n, g\in\Sigma_m,\] and  \[(fg)((i-1)m+j) := (f(i)-1)m+g(j) 
~~~ (i=1,\ldots,n; j=1,\ldots m).\] Note that this implicitly fixes a 
choice of bijections $n\times m\rightarrow nm$, in this case given by 
$(i,j) \mapsto (i-1)m + j$. Two easy calculations show that $+$ and 
$\cdot$ defined this way are strictly associative. To use Lemma
\ref{bipermallthethings} we need to exhibit $+$ as representing 
coproducts, so consider the embedding \[\Sigma_* \rightarrow \mathrm{Fin}.\]
That is, we embed the skeletal category of finite sets and
bijections into the skeletal category of finite sets and all maps.
Together with the canonical injections $\textbf{n}\rightarrow \textbf{n}+\textbf{m}$
and $\textbf{m}\rightarrow \textbf{n}+\textbf{m}$, which are part of the
category $\mathrm{Fin}$, we have that $+$ represents coproducts.
Hence $\mathrm{Fin}$ is a bipermutative category by Lemma \ref{bipermallthethings}.
The structural maps we get are all isomorphisms, so restricting to
$\Sigma_*$ again makes $\Sigma_*$ a bipermutative category. Furthermore
we can restrict to its subcategory on all objects with just epimorphisms
and get the bipermutative category $\mathrm{Epi}$. Also we can restrict
to the subcategory on all objects with just injections to get the bipermutative
category $\mathrm{Inj}$.

The induced additive symmetry $c_+\colon n+m\rightarrow m+n$ is given 
by: \[c_+(i) :=\begin{cases}i+m, ~~& i\leq n,\\ i-n, ~~&n+1\leq i,
	\end{cases}\] and we have the multiplicative symmetry $c_\cdot
\colon nm \rightarrow mn$ given by:\[c_\cdot((i-1)m+j) = (j-1)n + i.\]

Recall that the distributivity transformations of bipermutative 
categories determine each other \[\xymatrix{ab+ac\ar[r]^{\lambda} 
\ar[d]_{c+c} & a(b+c) \ar[d]^c\\ ba+ca\ar[r]^\rho & (b+c)a.}\]
May shows in full generality \cite[p. 155, Proposition 3.5]{MayEinf} that one
can always strictify one distributivity to be the identity. For finite
sets this corresponds to ordering a product of finite sets either 
lexicographically or anti-lexicographically. This way we find for 
general bipermutative categories two one-sidedly strict cases:
\[\lambda = id \Rightarrow \rho_{b,c;a} = c_{a,b+c}\circ 
(c_{b,a}+c_{c,a}),\] \[\rho = id \Rightarrow \lambda_{a;b,c} 
= c_{b+c,a}\circ (c_{a,b}+c_{a,c}).\] The choice of bijection for 
products considered above forces $\lambda=id$ and a non-trivial right 
distributivity, so the first case. This makes $\Sigma_*$ into a 
bipermutative category. The opposite choice with $\rho=id$ is given
for instance in \cite[p. 161, Example 5.1]{MayEinf}.}}

{\ex{\label{Finp}We can just repeat the argument above to find 
the pointed analogues of the above categories, hence we get 
$\mathrm{Inj}_+,\Ep,\Fi$. For definiteness let me reemphasise 
the bipermutative structure on these categories:

The coproduct (for $\mathrm{Inj}_+$) is the pointed sum, hence 
we can define a strictly associative functor representing it by:
\[ \mathbf{n}_++\mathbf{m}_+:= \mathbf{n}_+\vee \mathbf{m}_+ = 
(\mathbf{n}+\mathbf{m})_+,\] with the obvious extension to morphisms.

Fixing a choice of associative bijections $\bar\omega_{n,m}\colon
\mathbf{n}\times \mathbf{m}\rightarrow \mathbf{nm}$ induces associative 
bijections for the smash product by: \[\mathbf{n}_+\wedge \mathbf{m}_+ 
= (\mathbf{n}\times \mathbf{m})_+ \rightarrow (\mathbf{nm})_+ = \mathbf{nm}_+.\]

Hence the symmetries and distributors are given by adjunction of 
basepoints to the relevant morphisms above. In other words, given 
the bijections $\omega$, I fix the pointed structures such that
\[(\cdot)_+ \colon \mathrm{Fin}\rightarrow \mathrm{Fin}_+\]
becomes a strictly bipermutative functor with respect to disjoint
union and pointed sum, and cartesian product and smash product.}}

\subsection{Bicategories - Notation for this thesis}
In \ref{pcatneu} I use functors between bicategories with non-trivial
coherence 2-cells, so I fix the notations and conventions for 
bicategories here.
{\defn{A small \textbf{bicategory} with strict identities $\C$ is given by 
a set of objects $\C_0=Ob\C$, a set of 1-categories $\C_1=Mor\C$, 
and the following maps:\begin{itemize}
\item source and target \[s,t\colon \C_1\rightarrow \C_0,\]
where we call objects $a,b\in\C_1$ with $t(a)=s(b)$ composable,
\item identity objects \[\id_\_\colon \C_0\rightarrow \C_1,\]
with $s\circ \id_\_ = t\circ \id_\_=id_{\C_0},$
\item a composition functor \[\boxempty\colon \C_1\times_{\C_0}\C_1
\rightarrow \C_1,\]where the pullback is to be understood as 
\emph{composable pairs} in the sense described above,
\item a natural associativity isomorphism
\[\alpha\colon (\_\boxempty\_)\circ(\_\boxempty\_\times id_{\C_1})
\Rightarrow (\_\boxempty\_)\circ(id_{\C_1}\times\_\boxempty\_).\]
\end{itemize}These satisfy:\begin{itemize}
\item The identity 1-cells are strict units, i.e., we have strict equalities
of functors ($\C_0\times\C_1\rightarrow \C_1$ or $\C_1\times\C_0
\rightarrow \C_1$ respectively):\[(\_\cp\_)\circ (\id_{\_}\times 
id_{\C_1}) =(\_\cp\_)\circ (id_{\C_1}\times \id_{\_})=pr_{\C_1},\]
where $pr_{\C_1}$ denotes the respective projection onto the $\C_1$-factor.
\item The transformation $\alpha$ is the identity, if any factor is
$\id_\_$. \item The transformation $\alpha$ satisfies Mac Lane's 
pentagon identity, i.e., we have a unique associator on fourfold 
$\cp$-composites, hence by induction on composites of arbitrary 
length.\end{itemize}}}

{\rem{Most of the definition is standard apart from the fact that for
this thesis I can get away with strict identities, so I took them as 
part of the definition.

For clarity: I stick to the convention that the associator always
transforms expressions with left-biased bracketing $(ab)c$ into
expressions with right-biased bracketing $a(bc)$.

Further notation: I will refer to morphisms in (the disjoint union of)
the categories $\C_1$ as 2-cells, and refer to them globally as $\C_2$.
Furthermore I denote the component of $\C_1$, which is $(s,t)$-over 
objects $a,b\in\C_0$ as $\C_1(a,b)=:\C(a,b),$ i.e., the full subcategories 
of $\C_1$ with objects: \[\C(a,b)_0 = \{f\in\C_1| s(f)=a \mathrm{~and~} 
t(f)=b\}.\] Moreover I will always refer to the objects of the bicategory
as objects, the objects of the morphism categories as 1-cells, thus in particular
I refrain from calling 1-cells ``objects'' of their respective morphism categories.

I only stick to the notation $\cp$ here to stress the 
difference between the composition functor and composition in $\C_1$. 
In the example of interest $\M$ the 1-cells are given by matrices, and
hence the composition functor $\cp$ is matrix multiplication, which I
denote by $\cdot$ or drop from notation altogether, while the 
composition of 2-cells is just composition in a product $1$-category,
so I refer to that as $\circ$.
Also just for this section I denote identity $1$-cells by $\id_a$ to 
stress that they are usually in fact natural basepoints for the
categories $\C(a,a)$, but usually are not defined as maps of the 
type $x\mapsto x$. Again in $\M$ these are the identity matrices, but
the way these matrices represent ``linear maps of modules'' is at the
very least not obvious. After this section I trust the context is
sufficient to infer which type of identities I refer to.}}

{\ex{A category enriched in categories, i.e., a 2-category, is a 
bicategory with $\alpha=id$. In particular, every 1-category is a
bicategory with discrete morphism categories and hence trivial
associator as well.

\label{sigmasymmmon}
Any monoidal category $(\C,\otimes,1)$ can be 
understood as a one-point bicategory $\Sigma\C$: set 
$\Sigma\C_0=\{*\}$ and $\Sigma\C_1
=\Sigma\C(*,*)=\C$. Composition is given by $\otimes$,
the associator hence by the associator for $\otimes$.

Conversely the endomorphism category of any object $a\in\A_0$ in a
bicategory $\A$ yields the monoidal category $\A(a,a)$, occasionally
denoted by $\Omega_a\A$. In particular we have the trivial equality
\[\Omega_*\Sigma(\C,\otimes)=(\C,\otimes),\]
and for each object $a\in\A_0$ the strict inclusion functor
\[\eta_a\colon \Sigma\Omega_a\A=\Sigma\A(a,a)\rightarrow \A\]
with $\eta_a(*)=a.$}}

Given two bicategories there are adequate 1-cells between them, but
the designations in the literature vary quite a bit - specifically 
compare the classical ``Introduction to Bicategories'' of B\'enabou 
\cite{Bena} with the more recent overview in \cite{CCG2010}. On
pages 9 and 10 of \cite{CCG2010} the authors provide an excellent 
dictionary of the common terms for morphisms. A more detailed
exposition can be found in Ross Street's ``Categorical Structures'' 
\cite{Str}, in particular section 9. Furthermore its references are
a nice guide to the literature up to 1993.

I fix my notation here, and only define the types of functors
that appear in this thesis.

{\defn{\label{pseudo} For two bicategories $\C,\D$ a 
\textbf{pseudofunctor} $F\colon \C\rightarrow \D$ consists of the 
following maps: \begin{itemize} \item A map on objects $F_0\colon\C_0
\rightarrow\D_0.$\item For each pair of objects $a,b\in\C_0$ a functor
\[F_1\colon \C(a,b)\rightarrow\C(F_0a,F_0b),\] that is pointed at 
identities, i.e., for every $a\in\C_0$: \[F_1(\id_a)=\id_{F_0a}.\] 
\item For each triple of objects $a,b,c\in\C_0$ a natural isomorphism, 
which I refer to as \emph{compositor 2-cell}, \[F_2\colon (\_ 
\cp_{\D} \_)(F_1\times F_1)\Rightarrow F_1\times (\_ \cp_{\C} \_),\] 
which is trivial at identities, i.e., $F_2=id_{pr_{\D_1}}$ at: 
\[pr_{\D_1}= (\_ \cp_{\D}\_)\circ
(F_1\times F_1)\circ(id_{\C_1}\times\id_\_)\Rightarrow F_1\circ(
\_\cp_{\C}\_)\circ(id_{\C_1}\times\id_\_)=pr_{\D_1},\] and similarly
for the other argument.\end{itemize}The compositor satisfies 
associativity, i.e., for every composable triple $f,g,h$ of $1$-cells 
in $\C_1$ the diagram \[\xymatrix{(F_1f\cp_{\D} F_1g)\cp_{\D} F_1h 
\ar[d]_{\alpha_\D} \ar[rr]^{F_2\cp_{\D}id_{\D_1}} && F_1(f\cp_{\C}g)
\cp_{\D}F_1h\ar[r]^{F_2} &F_1((f\cp_{\C}g)\cp_{\C}h) 
\ar[d]^{F_1(\alpha_\C)}\\F_1f\cp_{\D}(F_1g\cp_{\D}F_1h) 
\ar[rr]_{id_{\D_1}\cp_{\D}F_2}&&F_1f\cp_{\D}(F_1(g\cp_{\C}h)) 
\ar[r]_{F_2}&F_1(f\cp_{\C}(g\cp_{\C}h))}\]commutes.

A pseudofunctor is called strict, if furthermore $F_2=id$ as
morphisms in $\D_1$, i.e., $F_1(fg)=F_1fF_1g$ and $F_1\alpha=\alpha$ 
and for every pair of composable 1-cells $f,g\in\C_1$ we have \[F_2=
id_{F_1(fg)}\colon F_1f\cp_\D F_1g \Rightarrow F_1(f\cp_\C g).\]}}

{\rem{Occassionally -- cf. \cite[pp. 9--10]{CCG2010} --
one refers to pseudofunctors as \emph{strong normal} functors, where strong
refers to the fact that the involved $2$-cell is an isomorphism and
normality refers to strictly fixing identity $1$-cells, which I carry
through this thesis as a permanently standing assumption.

For emphasis I occassionally refer to functors, which strictly respect
identity $1$-cells and composition of $1$-cells, as \emph{strict normal}
functors. \label{strictn}}}

{\rem{I only study bicategories arising from 
``finitely generated free module''-constructions, and thus by 
standard assumptions for $K$-theory only consider isomorphism 
subcategories of permutative $1$-categories. For a clarifying mathematical
reason to restrict to isomorphisms see in particular 
\cite[Proposition 8.14]{GGN}. Since I can restrict to exclusively
isomorphism $2$-cells, I have no need for less rigid functors between 
bicategories.}}

Pseudofunctors satisfy a preservation property on $1$-cells, 
which (op)lax functors of bicategories do not satisfy in general.
{\prop{A pseudofunctor $F\colon \C\rightarrow\D$
sends equivalence $1$-cells of $\C$ to equivalences in $\D$. In 
particular, isomorphism $1$-cells in $\C$ are sent to equivalences
in $\D$.\begin{proof} Let $f\colon a\rightarrow b$ be an equivalence
in $\C$, i.e., there is a $1$-cell $g\colon b\rightarrow a$ and 
isomorphism $2$-cells $\varepsilon\colon gf\Rightarrow \id_a,$
$\eta\colon \id_b \Rightarrow fg$. Then $Ff$ is an equivalence
inverted by $Fg$ by the following diagrams:
\[\begin{array}{lcr}{\xymatrix{\rrlowertwocell<-7>_{\id}{<-3.5>~F_2}
\rruppertwocell<1>^{F(gf)}{<1.5>~~F(\varepsilon)}
\rrcompositemap<7>_{Fg}^{Ff}{\omit}&&}}
&\mathrm{~and~}&{{\xymatrix{\rruppertwocell<9>^{\id}{<2>~~~F_2^{-1}}
\rrlowertwocell<2>_{<-0.25>F(fg)}{<-2.5>~~F(\eta)}
\rrcompositemap<-7>_{Ff}^{Fg}{\omit}&&.}}}\\\end{array}\]
Hence we need the compositor two-cell and its inverse, and find that
$Ff$ is an equivalence. In particular, if $F$ is not a strict functor,
$F$ sends isomorphisms to equivalences with $F_2$ as a non-trivial
isomorphism to the identity.

(Furthermore note that I have implicitly used the assumption that
$F$ is normal in the diagrams by $\id=F\id$.)\end{proof}}}

{\defn{Let $F,G$ be two pseudofunctors of bicategories:
\[(F_0,F_1,F_2),(G_0,G_1,G_2)\colon \C\rightarrow\D.\] Then a 
(strong) \textbf{pseudonatural transformation} $\sigma\colon F\Rightarrow G$ 
consists of chosen $1$-cells \[\sigma^0\colon \C_0\rightarrow \D_1\]
with $s(\sigma^0(a))=F_0a$ and $t(\sigma^0(a))=G_0a,$ as well as 
coherence isomorphism $2$-cells chosen for every pair $a,b\in\C_0$ 
and every 1-cell $f\in\C(a,b)$: \[\xymatrix{F_0a
\drtwocell<\omit>{~~\sigma^1_f} \ar[r]^{F_1f} \ar[d]_{\sigma^0a}& 
F_0b\ar[d]^{\sigma^0b}\\ G_0a \ar[r]_{G_1f}& G_0b,}\] which are 
appropriately natural and are compatible with the compositors, i.e.
we have for all objects $a,b,c\in\C_0$ and all $1$-cells $f,g\in\C_1$: 
\[\begin{array}{lcccr}{\xymatrix{F_0a\ar[d]_{\sigma_0}
\rruppertwocell<10>^{F_1(gf)}{<-3>\;\;\;\;F_2^{-1}} \ar[r]|{F_1f} 
\drtwocell<\omit>{~~\sigma^1_f} &F_0b\ar[d]|{\sigma_0}\ar[r]|{F_1g} 
\drtwocell<\omit>{~~\sigma^1_g}& F_0c\ar[d]^{\sigma_0} \\ G_0a
\rrlowertwocell<-10>_{G_1(gf)}{<3>\;\;\;G_2}\ar[r]|{G_1f} &G_0b
\ar[r]|{G_1g}&G_0c}} && = && {\xymatrix{F_0a \ar[d]_{\sigma_0}
\ar[r]^{F_1(gf)} \drtwocell<\omit>{~~\sigma^1_{gf}} & F_0c
\ar[d]^{\sigma_0}\\ G_0a \ar[r]_{G_1(gf)} & G_0c.}}\end{array}\]
Furthermore for bicategories with strict units we want 
$\sigma^1_{\id_a}=id_{\sigma^0_a}$ for each object $a\in\C_0$.
If in addition the two-cells $\sigma^1$ are identities we call
$\sigma$ a strict natural transformation.}}

{\rem{Since I restrict attention to bicategories with only isomorphism
two-cells or at least functors and transformations with isomorphism
two-cells, I do not need to introduce the concepts of lax and oplax
natural transformations.}}

It is classical for $1$-categories that a natural transformation
$\eta\colon F\Rightarrow G$ is the same thing as a functor 
$\C\times I\rightarrow \D$ for $I = [0<1]$ the interval
category. The analogous statement for bicategories holds true as well, which
I want to isolate into a proposition for emphasis and reference.

{\prop{A pseudonatural transformation $\eta$ of functors $F,G\colon 
	\C\rightarrow\D$ consists of the same data as a pseudofunctor
	$\eta\colon \C\times I\rightarrow \D,$ while the coherence
	of $2$-cells is equivalent to the pseudonaturality of the
	transformation.}}

Given sufficient experience with $1$-categories one would expect that 
I introduced all types of morphisms, but it should not be surprising 
that the extra-level of morphisms in bicategories (i.e., two-cells)
introduces a higher type of morphisms between ``natural 
transformations'', called modifications.

{\defn{\label{modis}
Given two pseudonatural transformations $\sigma,\tau$ between
the same two pseudofunctors $F,G$, a \textbf{modification} $\xi\colon \sigma
\Rightarrow \tau$ consists of a choice of isomorphism two-cells 
\[\xi\colon \C_0\rightarrow \D_2,\] making the following two 
diagrams of two-cells equal:
\[\xymatrix{~~~\ar[rr]^{Ff}\ddtwocell_{<2>{\tau^0a}}^{<2>\sigma^0a}<6>{\xi_a} 
\ddrrtwocell<\omit>{\sigma^1}& &\ar[dd]^<<<<{\sigma^0b}&\ar[rr]^{Ff}
\ddrrtwocell<\omit>{\tau^1~~~~~~~}\ar[dd]_>>>>>>{\tau^0a}&&~~~ 
\ddtwocell^{<2>{\sigma^0b}}_{<2>{\tau^0b}}<6>{_\xi_b}\\&&~~~~\ar@{{}={}}[r]&~~~~&\\
~~~\ar[rr]_{Gf}&&&\ar[rr]_{Gf}&&~~~.}\]}}

{\rem{Do note that on the level of $2$-cells the diagrams above are
only commutative squares resembling naturality in the context of 
$1$-categories.}}

With the relevant morphisms in place I can introduce 
equivalences of bicategories. {\defn{A pseudofunctor of 
bicategories $F\colon \C\rightarrow \D$ is an \textbf{equivalence of 
bicategories}, if there is a pseudofunctor $G\colon\D\rightarrow\C$
and there are two pseudonatural equivalences $\eta\colon FG\Rightarrow 
\id_\D$ and $\zeta\colon GF\Rightarrow \id_\C$.}}

The following proposition is particularly useful in chapter \ref{multbidel}, 
nonetheless its appropriate context is abstract nonsense about 
bicategories, so this section. I repeat the proof in particular to convince
the reader that it holds for bicategories with enriched morphism categories.
{\prop{\label{meinCzuZF}A pseudonatural transformation of pseudofunctors of 
small bicategories is an equivalence if and only if all its 
$1$-cells are equivalences.\begin{proof} It is clear that an inverse
equivalence establishes each component $1$-cell as an equivalence in
the target category. So we have to establish that having all $1$-cells 
equivalences is sufficient.

Let $\eta\colon F\Rightarrow G$ be a pseudonatural transformation
comprised of $1$-cells $\eta^1\colon \C_0\rightarrow \D_1$, and for 
each pair of objects $a,b$ in $\C$ a natural transformation:
\[\xymatrix{\C(a,b)\rrtwocell<5>^{{\eta^1_b}_*\circ F_1}_{{\eta^1_a}^*
\circ G_1}{~~~\eta^2_{a,b}}&& \D(Fa,Gb).}\]

By assumption each $1$-cell $\eta^1_a$ is an equivalence, so by the
axiom of choice choose for each $a$ an inverse equivalence $\zeta^1_a$
and isomorphism two-cells $\sigma\colon \zeta^1\eta^1 \rightarrow \id$
and $\tau\colon \id \rightarrow \eta^1\zeta^1$.

With these choices in place we can make $\zeta$ into a pseudonatural
transformation by choosing its two-cells as indicated by the 
following diagram:\[\xymatrix{\zeta GA= \zeta GA \id \ar[r]^{\tau} &
(\zeta GA)\cdot(\eta\zeta)\ar[d]^\alpha& (\zeta\eta)(FA\zeta) 
\ar[r]^\sigma & \id FA\zeta=FA\zeta.\\&\zeta ((GA\eta)\zeta) 
\ar[r]^{\eta^2} &\zeta ((\eta FA)\zeta) \ar[u]^\alpha & }\]

Since $\tau$ and $\sigma$ are chosen objectwise, we get a natural 
transformation. Since each arrow is an isomorphism $2$-cell, the 
$1$-cells $\zeta$ along with the two-cells indicated above compromise 
a pseudonatural transformation. It is compatible with the 
compositors of $G$ and $F$ because $\eta$ is, and hence $\zeta$ is an 
inverse equivalence to $\eta$. The relevant modifications are by 
construction given by $\tau$ and $\sigma$.\end{proof}}}

{\rem{For this proposition bicategories
are much nicer than the stricter theory of $2$-categories. 
Even if the pseudo-natural transformation strictly satisfies 
naturality, its inverse equivalence might have non-trivial $2$-cells.}}

{\rem{Do note that despite the fact this proposition is the analogue
to the $1$-category statement that a natural transformation is a 
natural isomorphism if and only if each component is an isomorphism,
its truth is (ZF-)axiomatically equivalent to the statement that a 
functor is an equivalence of categories if and only if it is 
essentially surjective and fully faithful. Hence it is
stronger because of the missing uniqueness for the inverse 
$1$-cells.}}

The following proposition is an immediate generalisation from
the context of $1$-categories. I want to exhibit
that the proof works in the context I define above. 
Thus it reassures us that the definitions are consistently chosen.

{\prop{A pseudofunctor $F=(F_0,F_1,F_2)$ of (small) bicategories is an 
equivalence of bicategories if and only if $F_0$ is surjective up to 
equivalence and each functor $F_1$ is an equivalence of 
$1$-categories.\begin{proof}Given a pseudofunctor $F\colon\C
\rightarrow\D$ that satisfies the conditions, we can by the axiom of 
choice find a map $G_0\colon Ob\D=\D_0\rightarrow \C_0$ such that 
there is an equivalence $\eta_d\colon F(G_0(d))\rightarrow d$ for 
each $d\in\D_0$. Fix that equivalence and an inverse $\kappa^d$ 
together with the isomorphisms $\eta_d\kappa^d\cong \id$ and $\kappa^d\eta_d
\cong \id$ for each $d\in\D_0$, it is the system of $1$-cells for the 
natural equivalence $FG \simeq \id_\D$ we need.

By assumption we have for each pair $d_1,d_2$ an equivalence of 
categories \[F_1\colon \C(G_0d_1,G_0d_2)\rightarrow \D(FG_0d_1,
FG_0d_2).\] Fix an inverse equivalence for each such pair $G^{d_1,d_2}$,
then we make $G$ into a functor of bicategories by the following
assignment on morphism categories:
\[\xymatrix{ \D(d_1,d_2) \ar[r]^{\eta_{d_1}^*} 
& \D(FG_0d_1,d_2)\ar[r]^{\kappa^{d_2}_*} 
& \D(FG_0d_1,FG_0d_2)\ar[r]^{G^{d_1,d_2}} & \C(G_0d_1,G_0d_2).}\]
Without loss of generality make $G$ into a functor pointed
at the identity $1$-cells.

This is an inverse equivalence to $F$ by construction; it is a 
pseudofunctor with compositor $2$-cell given by the isomorphisms 
$\eta\kappa\cong\id$ chosen above and with the natural equivalence
on one side given by $\eta$ with inverse $\kappa$ and on the other
by $G\eta$ and $G\kappa$.\end{proof}}}

It is reassuring to know that bicategories can still be strictified
to 2-categories. (This is wrong for tricategories!)

{\lem{(cf. \cite{Lei}) Each bicategory is equivalent to a 2-category.}}

{\rem{\label{nota}From here I drop the properly emphasised but 
clumsy notation, and denote $\cp$ in a bicategory by $\cdot$ or do not
denote it at all, while composition of $2$-cells is denoted by 
$\circ$, as it is usually the composition in some product of $1$-categories
in my examples.

For functors I refer to $(F_0,F_1)$ generically as $F$ and I denote 
the compositor in uppercase greek letters $\Phi$, thus referring to a
pseudofunctor $(F,\Phi)$ for instance.

I stick to the following conventions for elements in 
a general bicategory: objects are denoted by lowercase 
latin letters $a,b,c,\ldots\in\C_0$, which in $\M$ are natural 
numbers, but I do not want to restrict to that case unnecessarily. 
In $\M$ the $1$-cells are matrices, hence I denote $1$-cells by 
uppercase latin letters $A,B,C,\ldots\in\C_1$, and finally $2$-cells
by lowercase greek letters $\varphi,\psi,\ldots$ which are morphisms 
in products of the coefficient category $\R$ for $\M$.}}

{\rem{I refer to bicategories $\C$ as \emph{enriched in} topological 
spaces, simplicial sets, categories,\ldots if the morphism categories
$\C_1$ are enriched in these monoidal categories.}}

{\ex{A category enriched in topologically or simplicially enriched
categories is a bicategory enriched in topological spaces or simplicial 
sets respectively. A topological/simplicial monoidal category gives 
rise to a one-point bicategory enriched in topological 
spaces/simplicial sets. Do note that the propositions before 
work in the enriched cases as well, i.e. for enriched bicategories, and
enriched pseudofunctors, since the axiom of choice was only 
involved objectwise.}}

The rest of this chapter -- apart from the section \ref{kumodels} -
can safely be skipped by the reader familiar
with the delooping in \cite{EM}. For ease of reference I rewrite
their delooping in the following sections, so that the delooping in
bicategories \ref{multbidel} can be read in close analogy with the
case in 1-categories.

\subsection{A Delooping of Permutative Categories} 
\label{pcatneu}  Permutative categories 
provide a classical useful tool to model connective 
spectra, hence are valuable in stable homotopy theory. Even 
more than that Thomason proved \cite{Th1} that ``Symmetric monoidal
categories model all connective spectra''. Thomason was also driven
by the desire to provide a nice model for a smash product of spectra:
``[I]n June 1993 [\ldots] I used this alternate model of stable 
homotopy to give the first known construction of a smash product 
which is associative and commutative up to coherent natural 
isomorphism in the model category.''
Since \cite{MMSS} showed ``all'' models for symmetric monoidal 
categories of spectra yield (Quillen-)equivalent results, Thomason's
construction of a smash product has lost attention.

I elaborate on the construction $\C^+$ in \cite{Th2} in excessive detail, 
so I can refer back to its details for the analogous construction in 
bicategories \ref{multbidel}. Warning on notation: The notational 
conventions for bicategories described in \ref{nota} do 
not apply here, because they would clash with the natural interpretations.

The following results are each found in section 4 of \cite{Th2}. In particular,
I repeat his results and definitions in order to fix the notations I mimic
for bicategories in chapter \ref{multbidel}.

{\defn{\label{fastocat}Let $(\C,+,0,c_+)$ be a permutative category and 
$f\colon n_+\rightarrow m_+$ a map of pointed sets $n_+=\{0,1,\ldots,n\}$.
Define the following functor: \[\begin{aligned} f_*\colon& 
  \C^{\times n} &\rightarrow& \C^{\times m}\\
&(c_1,\ldots,c_n) &\mapsto& (\sum_{i\in f^{-1}j} c_i)_{j=1,\ldots,m},
\end{aligned}\] where the empty sum is defined to be the zero object (and 
its identity). Note that we have to
use the induced ordering $f^{-1}j\subset (n,\leq)$ and sum the $c_i$ 
accordingly.}}

{\rem{Note in particular that this gives a left-action of the 
symmetric groups on the respective powers: 
$\sigma_*(c_1,\ldots,c_n)_j = \sum_{i\in \sigma^{-1}j}c_i 
= c_{\sigma^{-1}j}.$}}

The fact that we have to choose an ordering on the fibres of $f$ is 
precisely what breaks the strictness of the functor $(\cdot)_*\Fi\rightarrow 
Cat$, which on morphisms is the assignment $f\mapsto f_*$ according to
the above definition. I isolate this fact into the following lemma.

{\lem{\label{symmundstrong}Given pointed maps $f\colon n_+\rightarrow m_+$ and $g
\colon m_+\rightarrow l_+$ there is a natural isomorphism of functors 
$\varphi_{g,f}\colon(gf)_*\Rightarrow g_*f_*$. \begin{proof} We can 
consider this componentwise, so without loss of generality let $g
\colon m_+\rightarrow 1_+ = \{0,1\}$ the unique map with $g^{-1}0=
\{0\}$. Then the summation according to $g_*f_*$ looks as follows:
\[(g_*f_*)(c)=\sum_{i=1}^m (f_*c)_i = 
\sum_{i=1}^m\sum_{j\in f^{-1}i}c_j,\]whereas the summation of $(gf)_*$
is according to the linear order on $n$ given by \[(gf)_*(c)=
\sum_{i\in (gf)^{-1}1}c_i = \sum_{i=1}^n c_i.\] Then there is a unique
 additive symmetry giving the isomorphism: \[\xymatrix@1{ ((gf)_*c)_j
=\sum_{k\in (gf)^{-1}j}c_k\ar[r]^-{c^+_{g,f}} 
&\sum_{i\in g^{-1}j}\sum_{k\in f^{-1}i}c_k=\sum_{i\in g^{-1}j}(f_*c)_i 
= g_*(f_*(c)),}\] which yields a natural isomorphism of functors: 
\[(gf)_*\Rightarrow g_*f_*.\]\end{proof}}}

{\rem{\label{troublemaker} Take special note of the following 
composites, which reappear in the delooping constructions and the 
distributivity axioms of bimonoidal categories. Define \[f\colon 
4_+\rightarrow 2_+: f(1)=f(3)=1, f(2)=f(4)=2\] and \[g\colon 4_+
\rightarrow 2_+: g(1)=g(2)=1, g(3)=g(4)=2\]. We have the unique map 
$q\colon 2_+\rightarrow 1_+$ with $q^{-1}0=\{0\}$ for $n_+=\{0,1,\ldots,n\}$ pointed at $0$.
We also have a unique map $c\colon
4_+\rightarrow 1_+$ with $c^{-1}0=\{0\}$. Then we have $qf=qg=c$, and
hence for a permutative category a unique isomorphism:
\[q_*g_*=c_*=(qf)_*\Rightarrow q_*f_*, \] which is given by the 
symmetry: \[1+c^++1\colon a+b+c+d \rightarrow a+c+b+d.\] In more 
detail: The compositor for $q$ and $g$ is the identity: 
$\varphi_{q,g}=id$, and for $q$ and $f$ is the symmetry 
$\varphi_{q,f}=1+c^++1$.

More generally: Since the action of the symmetric groups on the
respective powers of $\C$ is strict, we get $\varphi_{\sigma_1,
\sigma_2}=id$, for each $n\in \N$ and $\sigma_1,\sigma_2\in \Sigma_n$.
Consider the composite of a permutation $\sigma\in\Sigma_n$ with the 
unique map $q\colon n_+\rightarrow 1_+$, with $q^{-1}0=\{0\}$. Then 
we get: $q_*(c_1,\ldots,c_n)=\sum_{i=1}^nc_i,$ and
$q_*\sigma_*(c_1,\ldots,c_n)=\sum_{i=1}^nc_{\sigma^{-1}i},$ hence
$\varphi_{q,\sigma}=c^+_{\sigma}$, for $c^+_\sigma$ the unique natural 
additive symmetry in $\C$ between these sums.}}

We can define a ``classifying'' pseudofunctor into the $2$-category 
$Cat$ for a permutative category.


{\prop{\label{BCfunctor} Given a permutative category $(\C,+,0,c_+)$, the assignment
\[\begin{array}{rcl}B_{\C}\colon Fin_+ &\rightarrow& Cat \\  n_+ 
&\mapsto &\C^{\times n}, \\  f\colon n_+\rightarrow m_+ &\mapsto 
&f_*\colon \C^{\times n}\rightarrow \C^{\times m} \end{array}\] 
defines a pseudofunctor of $2$-categories, where we consider 
$Fin_+$ as a $2$-category with discrete morphism categories.
\label{THEpseudofunctor} 
\begin{proof} We have to prove the coherence: \[\xymatrix{ (hgf)_* 
\ar[r]^{\varphi_{hg,f}} \ar[d]^{\varphi_{h,gf}} & (hg)_*f_* 
\ar[d]^{\varphi_{h,g}id}\\ h_*(gf)_* \ar[r]^{id\varphi_{g,f}} 
& h_*g_*f_*,}\] but we know by \cite{Lap} that the additive symmetry 
defining the transformation \[(hgf)_*\Rightarrow h_*g_*f_*\] is 
uniquely determined by the ordering of summands the composite 
$h_*g_*f_*$ induces, so the diagram commutes. Furthermore we 
obviously have $id_*=id$, so $B_\C$ is a normal functor.\end{proof}}}

{\rem{Observe that giving a covariant pseudofunctor
$\mathrm{Epi}\rightarrow Cat$, which on objects is the assignment $n\mapsto 
\C^n$, already defines a symmetric monoidal product on $\C$, which is 
strictly associative and includes a symmetry but does not have a unit or 
unitors. In order to take care of the zero object we need a strong
normal functor $\mathrm{Fin}\rightarrow Cat$. Choosing pointed sets as an index
category induces projections in the additive Grothendieck construction below
associated to the maps: \[\rho^i\colon n_+\rightarrow 1_+\] with
\[\rho^i(j)= \begin{cases}0 & j\neq i\\ 1 & j=i,\end{cases}\]
which is easily identified as: $(\rho^i)_*=pr_i\colon \C^{\times n} 
\rightarrow \C.$}}

This association of a monoidal category to a functor is sufficiently
natural for the following lemma to hold:
{\lem{A (pointed) functor $F\colon (\C,+)\rightarrow (\D,+)$ together 
with a natural transformation $\lambda\colon F(\_)+F(\_)\Rightarrow 
F(\_+\_)$ is strongly symmetrically monoidal if and only if the
induced map of pseudofunctors $B_\C\Rightarrow B_\D$ is a 
pseudonatural transformation. For $\C$ and $\D$ permutative
$F$ is a strictly additive functor if and only if the induced
map is a strict natural transformation $B_\C\Rightarrow B_\D$.

Additionally a natural transformation of symmetric monoidal functors
is symmetrically monoidal if and only if it induces a modification
of the respective induced transformations.
\begin{proof}
That a strong symmetric functor defines a pseudonatural transformation
is given in \cite[Paragraph 4.1.4]{Th2}. The converse follows from the observation
that we can reconstruct $F$ by restriction to
$U$-level $1$: $F=B_F|_{\{1_+\}}\colon \C\cong B_\C(1_+)\rightarrow 
B_\D(1_+)\cong \D.$ 
The pseudonaturality $2$-cell of $B_F$ for the unique map 
$q\colon 2_+\rightarrow 1_+$ with $q^{-1}(0)=0$, gives the
diagram:
\[\xymatrix{B_\C(2_+)\ar[r]^{B^1_F}\ar[d]_q
	\drtwocell<\omit>{B^2_F}& B_\D(2_+)\ar[d]^q\\
	B_\C(1_+)\ar[r]^{B^1_F}& B_\D(1_+),}\] 
which identifies $\lambda$ as the pseudonaturality $2$-cell $B^2_F$.
It is coherently associative and symmetric because of the appropriate
pseudonaturality diagrams in higher degrees.

Comparing the diagram above with the diagram in definition \ref{modis}
yields the properly analogous claim for monoidal natural transformations. 
\end{proof}}}

I do not give the full generality of Grothendieck constructions, but 
only define the resulting category of the Grothendieck construction 
on $B_\C$ with respect to the permutative category $(\C,+,0,c_+)$. It
destills the complexity of the functor into an ordinary 1-category, 
so I do not refer to bicategories again until \ref{modulbicat}. For
a compatible general exposition of the Grothendieck construction as
considered by Grothendieck compare pages 47--49 in \cite{Bena}.

{\defn{\label{c+}Define the category $\C^+$ as follows: Its objects 
are \[Ob\C^+ := \coprod_n \C^{\times n},\] its morphisms:
\[\C^+((c_1,\ldots,c_n),(d_1,\ldots,d_m)) =
\coprod_{f\in \Fi(n_+,m_+)}\C^m(f_*c,d).\]
The identities are given by the identities in $\Fi$ and $\C^m$, the 
composition is given as follows: $c=(c_1,\ldots,c_n),
d=(d_1,\ldots,d_m),e=(e_1,\ldots,e_l)$: \[\xymatrix{
\C^+(d,e)\times \C^+(c,d) \ar@{=}[d]\\
\coprod_{g,f}\C^l(g_*d,e)\times \C^m(f_*c,d) 
	\ar[d]^{\coprod id\times g_*}\\
\coprod_{g,f}\C^l(g_*d,e)\times \C^l(g_*f_*c,g_*d)
	\ar[d]^{comp_{\C^l}}\\
\coprod\C^l(g_*f_*c,e) \ar[d]^{\varphi_{g,f}^*}\\
\coprod\C^l((gf)_*c,e) \subset \C^+(c,e).}\]
It is associative precisely because $\varphi$ is, and if 
$(\C,+,0,c_+)$ carries an enrichment such that $+$ is an enriched 
functor, then $\C^+$ is enriched over the same category.

Call this the \textbf{additive Grothendieck construction} on a
permutative category $(\C,+)$.}}

{\rem{The construction $\C^+$ is already
given by Thomason in \cite[Definition 2.1.2]{Th2}. Let me summarise the 
idea of the construction. Given a monoidal product on a category: 
$\otimes\colon \C\times\C\rightarrow \C$, we can define for each map 
$f\colon n_+\rightarrow m_+$ of finite sets a functor $\C^{\times n}
\rightarrow \C^{\times m}$ (in the same direction). This assembles 
into a pseudofunctor $\Fi\rightarrow Cat$. The Grothendieck 
construction associated to it is $\C^+$. I find it useful to
describe the delooping constructions of \cite{EM,Os,Seg} in this
one syntax.

Note that by the description of the composition given above we have
an enriched additive Grothendieck construction for an enriched permutative
category - specifically if the monoidal functor $+\colon \C\times\C\rightarrow\C$
is enriched, then the enrichment carries over to $\C^+$.}}

{\ex{Consider an ordinary category, that is one enriched in sets, 
then we can write morphisms between tuples as pairs \[(f, (a_1,\ldots,
a_m))\colon (c_1,\ldots,c_n)\rightarrow (d_1,\ldots,d_m),\] where 
$f\colon n_+\rightarrow m_+$ and $a_i \colon \sum_{j\in f^{-1}i}c_j
\rightarrow d_i,$ where we understand the empty sum as the zero object.

Then composition looks as follows (with $f\colon n\rightarrow m, 
g\colon m\rightarrow l$):\[(g,(b_1,\ldots,b_l))\circ(f,(a_1,
\ldots,a_m))=(gf,(b_1,\ldots,b_l)\circ g_*(a_1,\ldots,a_m)\circ
\varphi_{g,f}).\]

I consider the two extreme cases: The composites of symmetries are 
strict. So we have trivial compositors here, and thus (with $\bar a^i
= (a^i_1,\ldots,a^i_n)$ ~$n$-tuples of morphisms in $\C$):
\[(\sigma,\bar a^3)\circ (\tau,\bar a^2)\circ (\rho,\bar a^1)
= (\sigma\tau\rho, \bar a^3\circ \sigma.\bar a^2\circ 
		(\sigma\tau).\bar a^1).\]
The other extreme case is given 
for the composite of any symmetry $\sigma\in \Sigma_n$ with a map 
$q\colon n_+\rightarrow 1_+$ with $q^{-1}0=\{0\}$, there is a unique
additive symmetry $c^+_\sigma$ in $\C$: \[c^+_\sigma\colon 
q_*(c_1,\ldots,c_n)=\sum^n_{i=1}c_i\rightarrow 
q_*(\sigma.(c_1,\ldots,c_n))=\sum^n_{i=1}c_{\sigma^{-1}i}.\]
Thus we get: \[\begin{array}{ll} (q,a)\circ(\sigma,(b_1,\ldots,b_n)) 
      &= (q\sigma, a\circ q_*(b_1,\ldots,b_n)
				\circ \varphi_{\sigma,q})\\
      &= (q, a\circ (\sum_i b_i) \circ c^+_{\sigma})\\
\multicolumn{2}{c}{\text{and by the naturality of the additive twist we can identify this with:}}\\ 
      &= (q, a\circ c^+_{\sigma}\circ (\sum_ib_{\sigma^{-1}i}))\\
      &= (q, a\circ c^+_\sigma \circ q_*(\sigma.(b_1,\ldots,b_n)))\\
      &= (q, a\circ c^+_\sigma) \circ (id, \sigma.(b_1,\ldots,b_n)).
\end{array}\] In particular this Grothendieck construction 
incorporates structural morphisms for permutations between $n$-tuples 
that project down to the ordinary additive symmetry in $\C$.}}

{\rem{\label{disccomp}Note that in particular the category $\C^+$ has structural 
morphisms $(f,id)\colon c\rightarrow f_*c$, for $c=(c_1,\ldots,c_n)$ 
and $f\colon n_+\rightarrow m_+$. We call these morphisms the 
\emph{discrete component} of morphisms $c\rightarrow d$. This 
is a forgetful functor $U\colon\C^+\rightarrow\Fi$, 
which is important for the delooping construction.}}

{\ex{Building on the previous example let us reconsider the maps from 
\ref{troublemaker}. We set\[f\colon 4_+\rightarrow2_+: f(1)=f(3)=1,f(2)
=f(4)=2,\]\[g\colon4_+\rightarrow2_+: g(1)=g(2)=1,g(3)=g(4)=2,\]\[q
\colon2_+\rightarrow1_+: q(2)=q(1)=1,\]and write $c\colon4_+\rightarrow
1_+$ for the unique map with $c^{-1}0=0$. Then the compositor 
$\varphi_{q,f}$ is given by $1+c_++1$, whereas the compositor 
$\varphi_{q,g}$ is the identity. So we find:\[\begin{aligned}(q,a)
\circ(g,(b_1,b_2))&=(qg,a\circ q_*(b_1,b_2))\\&=(c,a\circ(b_1+b_2))\\
(q,a)\circ(f,(b_1,b_2))&=(qf,a\circ(b_1+b_2))\\&=(c,a\circ(b_1+b_2)
\circ(1+c_++1)).\label{fuerdiehohenReds}\end{aligned}\]

Hence the following diagram represents a commutative square in $\C^+$:
\[\xymatrix{(c_1,c_2,c_3,c_4) \ar[d]^{(g,\id)}\ar[rr]^{(f,\id)} 
&& (c_1+c_3,c_2+c_4) \ar[d]^{(q,\id)}\\
(c_1+c_2,c_3+c_4) \ar[r]^{(q,\id)} & (c_1+c_3+c_2+c_4)
&\ar[l]_{\varphi_{q,f}} (c_1+c_2+c_3+c_4),}\] because $\varphi$ is 
part of the composition law.}}

The additive Grothendieck construction is naturally associated 
to the pseudofunctors $B_\bullet$, thus leading to the 
following naturality:
{\lem{A strong symmetric monoidal functor $(F,\mu)\colon
(\C,+)\rightarrow (\D,+)$ induces a canonical functor $F^+$ on
additive Grothendieck constructions as follows: It assigns tuples
componentwise $F^+(c_1,\ldots,c_n)=(Fc_1,\ldots,Fc_n)$.
For morphisms consider the case with morphism sets. 
The map \[(f,(\varphi_1,\ldots,\varphi_m))\colon (c_1,\ldots,c_n)
\rightarrow (d_1,\ldots,d_m)\] is sent to:
\[\xymatrix{(Fc_1,\ldots,Fc_n)\ar[d]^{(f,id)}\\
(\sum_{j\in f^{-1}i}Fc_j)_i\ar[r]^{(id,(\mu)_i)}&
(F(\sum_{j\in f^{-1}i}c_j))_i \ar[d]^{(id,(F(\varphi_1),
\ldots,F(\varphi_m)))}\\& (Fd_1,\ldots,Fd_m). }\]
This is a functor precisely because $(F,\mu)$ is symmetrically 
monoidal.

A monoidal natural transformation $\eta\colon (F,\mu)\Rightarrow 
(G,\nu)$ induces a natural transformation $\eta^+\colon F^+\Rightarrow
G^+$. \begin{proof}I only comment on the natural transformation. Again 
consider the case with morphism sets, then we set $\eta^+$ as tuples
with the appropriate instances of $\eta$ and no discrete component.
This is natural in morphisms $(f,(\mu)_i)$ trivially, because $\eta$
is monoidal, and it is natural in morphisms $(id,(F(\varphi_1),\ldots,
F(\varphi_m)))$ because it is a product of natural transformations
$\eta\colon F\Rightarrow G$.

Do note that it is not meaningful for a natural transformation to be
symmetrically monoidal; there is no additional compatibility for 
$\eta$ to be satisfied.\end{proof}}}

Given any symmetric monoidal category $\C$ we can restrict to its
subcategory of isomorphisms $\C^{iso}$, which is a symmetric monoidal
subcategory of $\C$, so this gives an inclusion on their Grothendieck
constructions. \label{isorest}{\cor{There is a natural inclusion 
$I\colon(\C^{iso})^+\rightarrow \C^+$.}}

The following definition describes the index categories relevant for
the delooping. Specifically, we consider the comma category of finite
pointed sets under $A_+$, for $A_+$ not necessarily an object of $\Fi$.

{\defn{\label{commaindex}For an arbitrary finite pointed set $A_+$ 
define the index category $A_+\downarrow 
\Fi$ as follows: Objects are pointed maps $p\colon A_+
\rightarrow n_+$ and morphisms $f\colon p\rightarrow q$ are 
commutative triangles under $A_+$:~~~~~~ \xymatrix{A_+ \ar[r]^p \ar[dr]_q
& n_+ \ar[d]^f\\ &m_+.}}}

{\defn{Call the morphisms $\rho^i$ with $\rho^i(j)=*$ for $j\neq i$ and $\rho^i(i)=1$. 
They fit into the diagram:\[\xymatrix{ A_+ \ar[r]^f \ar[dr]_{\chi_{f^{-1}i}} 
& n_+ \ar[d]^{\rho^i}\\ & 1_+,}\] for each $f$ and each non-empty
preimage $f^{-1}i\neq \emptyset.$}}

{\rem{Note that the functors $(\rho^i)_*\colon \C^n\rightarrow \C$ are
strictly equal to the projection onto the $i$th factor.}}

{\rem{We have a target functor, i.e., $T\colon A_+\downarrow
\Fi\rightarrow \Fi$. \label{target1}

For $f\colon A_+\rightarrow B_+$ a pointed map
we have a (covariant) functor $f^*\colon \Bix\rightarrow\Aix$, which
is a functor over $T$, i.e., the diagram
\[\xymatrix{\Bix \ar[dr]_T\ar[r]^{f^*} & \Aix\ar[d]^T\\&\Fi}\]
commutes.}}

{\rem{The categories $\Aix$ have a ``full'', actually discrete, 
subcategory on objects the characteristic maps $\chi_a\colon A_+\rightarrow 1_+$
with $\chi_a(x)=+$ for $x\neq a$ and $\chi_a(a)=1$. This yields a
functor $j_A\colon A^\delta \rightarrow \Aix,$ which includes the
unpointed set $A$ as a discrete subcategory $A^\delta$ into $\Aix$,
by identifying each element $a$ with its characteristic map $\chi_a$.}}

It is not strictly necessary to define the delooping construction 
for $n=1$ (\ref{ca-n!}), but the essential mathematics happen here.
The cases for $n>1$ are then reductions to this case.

{\defn{Given a permutative category $(\C,+)$ and a finite pointed set
$A_+$ define its \textbf{first delooping category} $\C(A_+,1)$ as the 
category of functors lifting $T\colon \Aix\rightarrow \Fi$ through 
$U\colon (\C^{iso})^+\rightarrow \Fi$, i.e., the dashed arrows in the
diagram: \[\xymatrix{	& (\C^{iso})^+\ar[d]^U\\
\Aix \ar@{-->}[ur] \ar[r]_T & \Fi,}\]
where $T$ is the target functor given in \ref{target1}, and $U$ is the
functor sending each morphism to its discrete component as in \ref{disccomp}.
Its morphisms are the natural transformations of the functors pushed
forward with $I\colon (\C^{iso})^+\rightarrow \C^+,$ i.e., natural 
transformations but with arbitrary components, not just isomorphisms.}}

{\rem{The delooping of a permutative category given in 
\cite{EM} can be understood as a categorified version of the 
usual classifying space construction for abelian groups 
(cf. \cite[pp.87+88 and Theorem 23.2]{MaySoat}). 
For the functoriality of the delooping construction in finite
pointed sets and arbitrary maps it is more convenient to consider
all maps of finite pointed sets as structural morphisms. But one should
think of the structural $\Ep$-morphisms as fundamental, whereas 
non-surjective morphisms just keep track of zeroes.}}

{\rem{By contravariant functoriality of the indexing categories
over $T$ we get that $\C(\_,1)$ defines a covariant 
functor $\Fi\rightarrow Cat$. Furthermore restriction along
$j_A\colon A^\delta \rightarrow \Aix$ 
is a functor $R\colon \C(A_+,1)\rightarrow \C^{\times A}$.}}

{\prop{Every functor $F\colon \Aix\rightarrow \C^+$ lifting $T$ through $U$
	is isomorphic to a unique \emph{strict representative}. i.e., a functor,
	which assigns to commutative triangles of $\Aix$ only morphisms with 
    discrete components and additive symmetries and appropriate 
    identities in the second component.
	In particular any two functors restricting to the same $A$-tuple along 
	$R$ are naturally isomorphic. \begin{proof} To build the 
	\emph{strict representative} $F^{st}$ proceed as follows: Choose a bijection 
	$\sigma_A\colon A_+\rightarrow |A|_+$ and set $F^{st}(\sigma_A)
	=(F(\chi_a\colon A_+\rightarrow 1_+))_{a\in A}$. Any other 
	object of $\Aix$ has a unique morphism from $\sigma_A$. 
	So for $p\in \Aix$ set $F^{st}(p)=(p\sigma_A^{-1})_*(F^{st}(\sigma_A))
	\in\C^{\times|T(p)|}\subset\C^+$. I drop the $\sigma_A^{-1}$ from the notation
	immediately, since it is only there to make coherent choices for all
	maps of finite sets at once. For a commutative triangle under $A_+$:
	\[\xymatrix{A_+\ar[r]^{p}\ar[dr]_{qp} & n_+\ar[d]^q\\&m_+}\]	we need 
	a morphism $F^{st}(p)=p_*(F^{st}(\sigma_A)) \rightarrow 
	q_*p_*(F^{st}(\sigma_A))\rightarrow (qp)_*(F^{st}(\sigma_A))=F^{st}(qp),$ 
	which we can choose to be $(q,\varphi^{q,p})$; in particular it only 
	has the claimed components. Furthermore it obviously projects down 
	to $q$ in $\Fi$ by the forgetful functor $U\colon \C^+\rightarrow \Fi,$ 
	so it is a lift of $T$ through $U$. Since the second component is always 
	a symmetry we also trivially have a functor $F^{st}\colon\Aix\rightarrow 
	(\C^{iso})^+$, hence an object of $\C(A_+,1)$.

	For the isomorphism first consider the following diagram:
	\[\xymatrix{A_+\ar[r]^{\sigma_A}\ar[dr]_{\chi_a}& |A|_+.\ar[d]^{\rho^a} 
	\\ & 1_+}\] By definition $F$ sends $\rho^a$ to a morphism ($U$-)over 
	$\rho^a$, hence of the form $(\rho^a,f_a)$ with $f_a\colon 
	(\rho^a)_*F\sigma_A=F(\sigma_A)_a\rightarrow F(\chi_a)$ an isomorphism
	in $\C$. These assemble to an isomorphism $F\sigma_A\rightarrow 
	(F\chi_a)_{a\in A}$ in $\C^{\times A}$.

	We can uniquely write each $Fq\colon Fp\rightarrow F(qp)$ as
	\[\xymatrix{Fp\ar[rr]^-{(q,\id)} && q_*Fp \ar[rr]^-{(\id,
	F^{\C}(q))} && F(qp),}\] with $F^\C$ a morphism in a product 
	category $\C^k$ for $q\colon A_+\rightarrow k_+$.

	In particular we get a canonical morphism:
	\[\xymatrix{Fp \ar[rr]^-{(F^{\C}(p\sigma_A^{-1}))^{-1}} && p_*F\sigma_A
	\ar[rrr]^-{(\id,p_*((F\chi_a)_{a\in A}))} &&& p_*F^{st}\sigma_A = F^{st}(p),}\]
	which assembles to a natural transformation $F\Rightarrow F^{st}$
	whose components are isomorphisms by the assumption on $F$.	So we 
	have a canonical isomorphism for each functor to its strict
	representative, and the strict representative only depends on the
	restriction of $F$ along $j_A$.\end{proof}}}

This immediately has the following corollary, which is useful for constructing
such lifting functors more easily.
{\cor{Each functor $F$ lifting $T$ through $U$ is uniquely determined up to natural
	isomorphism by its restriction to $A_+\downarrow \Ep$. In particular we
	can assume without loss of generality that for each $p\colon A_+\rightarrow k_+$
	with $k>|A|$ the object $Fp$ is given by padding with zeroes from a
	bijection $F\sigma_A$, and the morphisms accordingly only have identities
	in zero-components.}}

{\rem{\label{warumEpi}Restricting to epimorphisms is extremely convenient. 
Given any finite set $A_+$ we know a priori that the index-categories
$A_+\downarrow Epi_+$ are finite, i.e., have finitely many objects.

Since for finite $A_+$ there is always a non-unique maximal surjection, i.e.
a bijection $A_+ \rightarrow |A|_+$, any other object of $A_+ 
\downarrow \Fi$ can be written relative to a chosen bijection. 
In particular for $n>|A|$ there is an injection $|A|_+ \rightarrow n_+,$ 
which is unique, if we choose the injection strictly monotonous 
and with minimal maximal element.}}

Let me reemphasise the uniqueness clause of the strict representative to
the canonical delooping statement:
{\prop{For $(\C,+)$ a permutative category the delooping category $\C(A_+,1)$
	is naturally equivalent to a product category by restriction along 
	$j_A$. I.e., we have an equivalence:\[\C(A_+,1)\simeq \C^A.\]\begin{proof}
	The construction of the strict representative given above can also be
	used to promote each $A$-tuple to a lifting functor, which gives the
	inverse equivalence to restriction along $j_A$. 
	The natural isomorphism on the left was given above, on $\C^A$ these
	functors strictly compose to the identity.	\end{proof}}}

{\rem{It is true, but inessential and uninstructive to prove, that the 
	delooping category $\C(A_+,1)$ is actually naturally \textbf{isomorphic}
	to the Segal construction on a permutative category $\C$ as defined in 
	\cite[Construction 4.1, Theorem 4.2]{EM}.

	The isomorphism $\C(A_+,1)\rightarrow 
	\C^{Seg}$ is given by sending a	lifting functor $F$ to its 
	restriction on characteristic functions for subsets 
	$\chi_S\colon A_+\rightarrow 1_+$. The additors $\rho_{S,T}$ are given 
	by the maps associated to the factorisation in $\Aix$: \[\xymatrix{A_+
	\ar[r]^{\chi_{(S,T)}} \ar[dr]_{\chi_{S\cup T}}& 2_+\ar[d] \\ &1,}\]
	for $S,T$ disjoint subsets of $A$.	

	The associativity of the additors given in the diagrams of 
	\cite[Construction 4.1]{EM} follows from the fact that the map
	$\chi_{(S,T,U)}\colon A_+\rightarrow 3_+$ in particular
	has maps in $\Aix$ to $\chi_{(S\cup T,U)}$ and $\chi_{(S,T\cup U)},$
	which both map to $\chi_{S\cup T\cup U}$, giving a commutative
	square in $\Aix$, thus one in $\C^+$ for each lifting functor.}}

\subsection{The Construction $\C(A_+,n)$ for Permutative $1$-Categories}
\label{ca-n!} I want to describe the delooping
construction in \cite[Construction 4.4]{EM} in the same way that I 
just described the Segal construction $\C(A_+,1)$. To that 
end I consider the Segal construction with $n$-fold products 
of finite sets and maps flattened into the additive Grothendieck 
construction $\C^+$, such that the case before is $n=1$.

As in \ref{Finp} fix a smash product functor on $\Fi$.
Then we can define a symmetric monoidal structure on $\C^+$ when 
given a bimonoidal structure on $\C$.

{\prop{
	Consider the additive Grothendieck construction $\C^+$ for
	a bimonoidal category $(\C,+,\cdot)$, then
	we have an induced monoidal structure on $\C^+$,
	which makes the forgetful functor $U\colon \C^+ \rightarrow \Fi$
	strictly monoidal with respect to the induced
	multiplication on $\C^+$ and the smash-product functor on
	$\Fi$. If moreover the multiplication of $(\C,+,\cdot)$ makes
	$\C$ a bipermutative category, the induced monoidal structure
	is symmetric, and the functor $U$ is strictly symmetric
	monoidal.
	\begin{proof}
	Given a smash product functor on $\Fi$ we have fixed pointed 
	bijections $(n\times m)_+=n_+\wedge m_+\rightarrow nm_+,$
	hence also $\omega_{n,m}\colon n\times m\rightarrow nm,$ which 
	are associative. For two objects $c=(c_1,\ldots,c_n),
	d=(d_1,\ldots,d_m)$ in $\C^+$
	set their product to be: \[c\boxtimes d = (c_id_j)_{\omega(i,j)},\]
	where I have written points in the indexing set 
	$\{1,\ldots,nm\}$ as images $\omega(i,j)$.

	The essential subtlety is the fact that this can be made into
	a functor. Consider the following two objects in $\C^+$:
	\[(f\wedge g)_*(c\boxtimes \bar c)_{\omega(i,j)}
	=\sum_{\omega(k,l)\in(f\times g)^{-1}(\omega(i,j))}c_k\bar c_l,\]
	and analogously: \[((f_*c)\boxtimes(g_*\bar c))_{\omega(i,j)}
	= (f_*c)_i\cdot(g_*\bar c)_j = \left(\sum_{k\in f^{-1}i}c_k\right)
	\left(\sum_{l\in g^{-1}j}\bar c_l\right).\]

	By \ref{bim1} and \cite{Lap} we find that there is a unique composite
	of distributors and additive symmetries comparing these objects. For
	instance by first reducing summands on the left, then on the right,
	we get: \[(f\wedge g)_*(c\boxtimes \bar c) = \sum_{k,l}c_k\bar c_l
	\rightarrow \sum_k\left(c_k\left(\sum_l\bar c_l\right)\right)\]
	\[~~~~~~~
	\rightarrow \left(\sum_{k\in f^{-1}i}c_k\right)
	\left(\sum_{l\in g^{-1}j}\bar c_l\right) = f_*c\boxtimes g_*\bar c.\]
	Hence there is a unique structural morphism $D^{f,g}$ determined by
	the summations $f$ and $g$ induce. Thus for two
	maps in $\C^+$ in the case of hom-sets with structure:
	\[(f,(a_1,\ldots,a_{m_1}))\colon c=(c_1,\ldots,c_{n_1})
	\rightarrow (d_1,\ldots,d_{m_1})=d,\]\[(g,(b_1,\ldots,b_{m_2}))\colon 
	\bar c=(\bar c_1,\ldots,\bar c_{n_2})\rightarrow 
	(\bar d_1,\ldots,\bar d_{m_2})=\bar d,\]
	we set their product to be the composite:
	\[\xymatrix{c\boxtimes \bar c = (c_i\bar c_j) \ar[r]^-{(f\wedge g)_*}&
	\left(\sum c_k\bar c_l\right)\ar[r]^-{D^{f,g}}&	\left(\sum c_k
	\right)\left(\sum\bar c_l\right) \ar[r]^-{a_ib_j}&(d_i\bar d_j).}\]
	Analogously define the product for general enriched bimonoidal categories as
    follows: Consider $D^{f,g}\circ (f\wedge g)_*$ on the $(f,g)$-component
	of the morphisms on the product of the additive Grothendieck construction
	on $\C$, i.e., $(\C^+\times\C^+)((c,\bar c),(d,\bar d))=
	\coprod_{(p,q)}\C^{|\bar c|}(p_*c,\bar c)\times \C^{|\bar d|}(q_*d,\bar d)$,
	and postcompose with the monoidal product $\cdot$ of $\C$, which in its
	$\omega(i,j)$th component pairs the $i$th factor in the first product 
	with the $j$th factor in the second product.

	This assignment evidently sends identities to identities, and it
	respects composites, because $\cdot$ is part of a bimonoidal/bipermutative
	structure on $\C$, hence we can always interchange distributors $D^{f,g}$
	and tuples of genuine $\C$-morphism $(a_ib_j)$ by summing up and reducing
	the appropriate components.
	In summary we obtain a functor:
	\[\boxtimes\colon \C^+\times\C^+\rightarrow \C^+.\]

	Since we have chosen $\wedge$ to be a strictly unital functor 
	on $\Fi$ and $\cdot$ strictly unital on $\C$, the $1$-tuple 
	$(1)\in\C^+$ with entry the multiplicative unit of $\C$ is a 
	strict unit for $\boxtimes$. Since $\wedge$ is strictly associative
	and $\cdot$ is strictly associative, $\boxtimes$ is strictly associative
	as well.

	Finally for $\cdot$ not just a monoidal, but a braided or symmetric
	monoidal structure with symmetry $c^\cdot$, consider the symmetry 
	in $\Fi$ for $\wedge$, and call	it $\chi$. Then a multiplicative 
	symmetry for $\boxtimes$ on $\C^+$ is given	by:
	\[\xymatrix{c\boxtimes d = (c_id_j)_{\omega(i,j)} \ar[r]^-\chi
	& (c_id_j)_{\omega(j,i)} \ar[r]^-{c^\cdot} & (d_jc_i)_{\omega(j,i)}
	=d\boxtimes c.}\]
	It squares to the identity if $c^\cdot$ does, and satisfies the braiding
	coherence diagrams for triple products that $c^\cdot$ satisfies. Hence
	yields a braided/symmetric monoidal structure, if $(C,\cdot)$ is
	braided/symmetric monoidal and bimonoidal as $(C,+,\cdot)$.	We have
	evidently constructed the symmetric monoidal structure just so that
	$U$ becomes a strictly (braided/symmetric) monoidal functor.
	\end{proof}}}

I do not intend to get back to this multiplicative structure until chapter
\ref{multbidel}, but wanted to explicitly state it for $1$-categories.
It emphasises that the multiplicative structure exhibited in 
chapter \ref{multbidel} can be built easily here as well.

To define the higher delooping categories $\C(A_+,n)$ we need to consider
target functors for the product categories $(\Aix)^n$.
{\prop{\label{T-n}
	There is a forgetful functor $T_n\colon (\Aix)^n\rightarrow \Fi$ for each
	$n\geq 1$, which is faithful away from the basepoint, and can moreover 
	be chosen to be associative, i.e., the diagram \[\xymatrix{(\Aix)^n\times
	(\Aix)^m\times(\Aix)^l \ar[rr]^-{T_{n+m}\times \id}\ar[d]_{\id\times 
	T_{m+l}}&&\Fi\times (\Aix)^l \ar[d]^{\id\times T_l}\\(\Aix)^n\times\Fi 
	\ar[d]_{T_n\times \id}&& \Fi\times\Fi \ar[d]^{\wedge}\\\Fi\times\Fi 
	\ar[rr]^{\wedge} && \Fi,}\]	commutes. Moreover the functors $T_n$ can 
	be chosen symmetric, i.e., the functors
	\[\xymatrix{(\Aix)^n\times (\Aix)^m \ar[rr]^-{T_n\times T_m} &&
	\Fi\times \Fi\ar[r]^-{\wedge} & \Fi	}\]	and	\[\xymatrix{(\Aix)^{n+m}
	 \ar[r]^-{tw^\times_{n,m}}& (\Aix)^{m+n} \ar[r]^-{T_m\times T_n} 
	& \Fi\times\Fi\ar[r]^-\wedge&\Fi}\]	are naturally isomorphic by exchanging 
	priority of the smash components, i.e., $\chi^\wedge$ on $\Fi$.
	\begin{proof}Simply set $T_n$ to be the following functor:\[\xymatrix{
	(\Aix)^n\ar[r]^-{(T)^n} & \Fi^n	\ar[r]^{\wedge} & \Fi,}\]where $T$ is 
	the target functor of $\Aix$, and $\wedge$ is the 
	$n$-fold smash, which is defined because $\wedge$ is strictly 
	associative. Then the $T_n$ inherit associativity and symmetry as claimed, 
	and are just as faithful as $T$ and $\wedge$. Hence for maps with 
	$f^{-1}+=\{+\}$	we get injectivity on hom-sets.\end{proof}}}

These functors should give the reader a reasonable hunch how I define
$\C(A_+,n)$ such that $\C(A_+,1)$ considered above trivially becomes the
case $n=1$.
{\defn{The \textbf{higher delooping category} of a permutative category $\C(A_+,n)$ for
	$n\in\mathbb{N}$ and $A_+$ a finite pointed set, is given as the
	category of functors lifting $T_n$ through $U$, i.e., the dashed arrows
	in the diagram:
	\[\xymatrix{& (\C^{iso})^+\ar[d]^U \\ (\Aix)^n \ar@{-->}[ur]\ar[r]^-{T_n} 
	& \Fi.}\]Its morphisms are the natural transformations of functors pushed forward
	with the inclusion $(\C^{iso})^+\rightarrow \C$, i.e., natural transformations
	with arbitrary components, not just isomorphisms.}}

{\ex{\label{kommKub} Consider again (\ref{troublemaker}) the prototypical 
surjections $f,g\colon 4_+\rightarrow 2_+$, with $f(1)=f(3)=1$,$f(2)=f(4)=2$, 
$g(1)=g(2)=1$,$g(3)=g(4)=2$,and $q\colon 2_+\rightarrow 1_+$ again. 
Choosing the bijection $\omega$ for the smash product
gives the following commutative cube: \[\xymatrix{ (2\times 2)_+ 
\ar[rr]^<<<<<{id\times q}\ar[dd]_<<<<<{q\times id} \ar[dr]^\omega && 
(2\times 1)_+\ar[dr]^\omega \ar[dd]^<<<<<{q\times id}\\ &4_+
\ar[rr]^<<<<<{f}\ar[dd]_<<<<<{g} && 2_+\ar[dd]^>>>>>q\\ (1\times 2)_+ 
\ar[rr]^<<<<<{id\times q} \ar[dr]_\omega && (1\times 1)_+ 
\ar[dr]^\omega\\ &2_+\ar[rr]_>>>>>q&&1_+.}\]

Flattening this cube makes a pentagon with the additional arrow given
by $1+c^++1$, which appears for instance in axiom (5) of 
\cite[Construction 4.4]{EM}.}}

{\prop{\label{symmSpekt!!} Each permutation $\sigma\in\Sigma_n$ 
induces a functor \[\sigma\colon (A_+\downarrow \Fi)^{\times n} 
\rightarrow (A_+\downarrow \Fi)^{\times n}.\] Precomposing with this
permutation of the components and post-composing a functor with
the symmetry $\chi$ of the smash-product on $\Fi$, induces a natural
$\Sigma_n$-action on $\C(A_+,n)$ .
\begin{proof}The statement concerning the symmetry warrants some 
explanation. By \ref{T-n} we know that we can choose the target
functors $T_n$ as $(\Aix)^n\rightarrow \Fi^n\rightarrow \Fi$, hence
the symmetry isomorphism from $T_n$ to $\sigma^*T_n$, for 
$\sigma\in\Sigma_n$ can be pushed forward to $\Fi$ by the appropriate
symmetry $\chi_{\sigma}^{\wedge}$ of the smash-product on $\Fi$. Then
by pushing forward the permuted functor in $\C^+$ with the same symmetry
we get a functor lifting $T_n$ again.\end{proof}}}

{\rem{To state the following proposition conveniently I introduce a standard
    simplifying assumption. For $\C$ a permutative category we can without 
    loss of generality assume that $0\in\C$ is an isolated object, i.e., it
    has at most non-trivial endomorphisms, but no maps in $\C$ from or to
    different objects. This makes $\C\setminus\{0\}\sqcup\{0\}$ a decomposition
    of $\C$ by full subcategories. 

    For $\C$ with isolated zero $0$ as above we can define the smash product of $\C$
    with a finite pointed set $A_+=A\sqcup\{*\}$ as:
    \[A_+\wedge \C := End(0)\coprod_{x\in A} (\C\setminus\{0\}).\]

    The assumption is without loss of generality since we can attach to each
    category $\C$ a disjoint basepoint $\C_+:=\C\sqcup\{*\}$. For $\C$ permutative
    we extend the monoidal functor by letting $*$ act as a strict unit, thus in 
    particular the object $0\in\C$ is not neutral in $\C_+$. On classifying spaces
    we get $|N\C_+|=|N\C|\sqcup \{*\}$, i.e., we only added a disjoint basepoint
    to the classifying space as well.

    Finally note that for $\C$ permutative with an isolated additive unit $0$, any
    finite pointed set $A_+$ and any natural number $n$, the
    higher delooping categories have an isolated zero as well:
    The category $\C(A_+,n)$ has a basepoint given by the 
    functor $$O\colon (A_+\downarrow Epi_+)^{\times n}\rightarrow \C^+$$ 
    with $O((f_1,\ldots,f_n)\colon (A_+,\ldots,A_+) \rightarrow (k^1_+,
    \ldots,k^n_+))=0$ the $k^1\cdot\ldots\cdot k^n$-tuple consisting 
    only of the additive unit, and each morphism is sent to its appropriate 
    discrete component with $\C^{\times \bullet}$-components only 
    identities. If the additive unit in $\C$ is isolated, then this zero functor $O$ is 
    an isolated basepoint of $\C(A_+,n)$ as well, and we can identify the smash as:
    \[A_+\wedge \C(A_+,n) 
    =End(O) \sqcup \coprod_{x\in A} (\C(A_+,n) \setminus\{O\})\times 
    \{x\}.\] Take note that the disjoint union is over all non-basepoints 
    in $A_+$, hence all elements of $A$. The fact that $O$ is isolated 
    ensures that each $\C(A_+,n)\setminus\{O\}$ forms a (sub)category.}}

The extension functors are a bit obscured by the fact that I chose to 
reduce the arguments in the delooping-construction of \cite{EM}
to $n$ equal inputs $A_+$ only, but it coalesces nicely. 

{\prop{We have a natural inclusion of categories: \[e\colon A_+
\wedge\C(A_+,n)\rightarrow \C(A_+,1+n).\] Furthermore this inclusion
is $\Sigma_n$-equivariant, where on $\C(A_+,1+n)$ the 
action is given by restriction along the inclusion $\Sigma_n=\Sigma_1
\times\Sigma_n\rightarrow \Sigma_{1+n}$, i.e., letting $n$-permutations 
act on the indices $\{2,\ldots,n+1\}$.
\begin{proof} 
As above we see if the additive unit in $\C$ is isolated, then 
the zero functor $O$ is an isolated basepoint of $\C(A_+,n)$, 
and we can identify the smash as: \[A_+\wedge \C(A_+,n) 
=End(O) \sqcup \coprod_{x\in A} (\C(A_+,n) \setminus\{O\})\times \{x\}.\]
Hence I can describe the extension functor on each component 
seperately. We set $e(O)=O$ and send an endomorphism of 
$O\in\C(A_+,n)$ to the endomorphism of $O\in \C(A_+,1+n)$, which we obtain 
by appropriately extending with identities.

More interestingly consider a summand $\C(A_+,n)\setminus\{0\}\times 
\{x\}$. Then we have to define $e(F,x)(p_1,\ldots,p_{n+1})$ for each 
$(n+1)$-tuple of maps $p_i\colon A_+\rightarrow k_i$. We set: 
\[e(F,x)(p_1,\ldots,p_{n+1})=\begin{cases} 0 &~~p_{1}\neq (\rho^x
\colon A_+\rightarrow 1_+),\\F(p_2,\ldots,p_{n+1}) &~~p_{1}=\rho^x.
\end{cases}\] Accordingly, $e(F,x)$ is the identity on the additive unit
for all $(n+1)$-tuples of morphisms, which do not have $id_{1_+}=
id_{\rho^x}$ as its first component. On the tuples, where the 
last component is $id_{1_+}$ and target and source have last 
component $\rho^x$ we can use $F$ on the first $n$ maps.

This is obviously a functor, natural in $\C$ and $A_+$.
Equivariance follows by our choice of singling out the first component. 
Changing the marked component changes the inclusion $\Sigma_n\rightarrow 
\Sigma_{1+n}$ but still yields equivariance as claimed.\end{proof}}}

Again we have the result making $\C(A_+,n)$ a delooping of $\C$.
{\thm{For $(\C,+)$ a permutative category we have a natural equivalence
	of categories \[\C(A_+,n)\simeq Set(A,\C(A_+,n-1)),\]
	with the map $\C(A_+,n)\rightarrow Set(A,\C(A_+,n-1))$ given by
	restriction of functors along $(\id_{n-1},j_A)\colon
	(\Aix)^{n-1}\times A^\delta\rightarrow (\Aix)^n.$
	Inductively we get a natural equivalence:
		\[\C(A_+,n)\simeq \C^{A^{\times n}}.\]\begin{proof}
	The start of the induction is the case $n=1$ displayed above. The same
	argument with \emph{strict representatives} can be made to prove the
	equivalence above by making a functor in $\C(A_+,n)$ only consist of
	discrete components and additive symmetries one component at a time.
	\end{proof}}}

I give the construction of the Eilenberg-Mac Lane spectrum based on
this delooping in chapter \ref{multbidel} in the maximal generality
I need it. The case of permutative $1$-categories then follows by 
considering them as permutative bicategories with discrete morphism
categories. The maximal generality in this thesis is motivated by the 
principal example of interest $K(ku)$. The next chapter is thus 
concerned with the module bicategory of a bimonoidal category. 
For the spectrum $ku$ I want to fix the relevant models and maps next.

\section{Models for $ku$}\label{kumodels}
Assuming that the delooping given by $\C(A_+,n)$ yields an $E_\infty$ 
symmetric ring spectrum $H\C$, which I prove in chapter \ref{multbidel}, 
we get models for connective $K$-theory with an $E_\infty$-multiplication 
by considering nicely explicit bipermutative categories.

{\ex{\label{modultensor}
	Given a commutative ring $k$, consider its (skeletal)
	category of finitely generated free modules $\MM_k$ on 
	objects: \[\Ob\MM_k = \Ob~\mathrm{Fin} = \{\mathbf{n}| n\in\mathbb{N}\}.\] 
	Consider the unpointed sets $\mathbf{n}$ as ranks of finitely generated
	free modules over $k$. To establish its 
	bipermutative structure first consider the morphism sets in a bigger category
    $\MM^L$ (compare \ref{Fin}):
	\[\MM^L_k(\mathbf{n},\mathbf{m}):= 
	\mathit{Hom}_k(k\{1,\ldots,n\},k\{1,\ldots,m\})\]
	of all $k$-linear maps of free modules on the unpointed sets $\mathbf{n},
	\mathbf{m}$.

	Fixing the direct sum functor as the linear extension of disjoint
	union gives a strictly associative coproduct-functor for $\MM^L$:
	\[k\{1,\ldots,n\}\oplus k\{1,\ldots,m\}:=k\{\mathbf{n+m}\}
	=k\{1,\ldots,n,n+1,\ldots,n+m\},\] with the obvious extension to 
	morphisms by linearly extending the description of $\mathrm{Fin}$ 
	on basis elements.

	The product functor chosen on finite sets (specifically by fixing
	associative bijections $\omega=\omega_{n,m}\colon \mathbf{n}\times 
	\mathbf{m} \rightarrow \mathbf{nm}$) 
	extends to the tensor-product of free modules: \[k\{1,\ldots,n\}\otimes 
	k\{1,\ldots,m\} := k\{1,\ldots,nm\},\] where we define the 
	tensor-product of linear maps represented as quadratic matrices
	$f\in M_n(k), g\in M_m(k)$ as follows:
	\[(f\otimes g)(e_{\omega{(i,j)}}):= \sum_{\omega(s,t)\in nm} 
	f_{si}g_{tj}e_{\omega(s,t)}.\] For the entries of the representing
	matrix for $f\otimes g$ with respect to the ordering on $\mathbf{nm}$ fixed
	by $\omega\colon \mathbf{n}\times\mathbf{m}\rightarrow \mathbf{nm}$ we get:
	\[(f\otimes g)_{\omega(i_1,j_1),\omega(i_2,j_2)}
	=f_{i_1,i_2}\cdot g_{j_1,j_2},\] which is a strictly associative
	representation of the tensor-product. We know it is left-adjoint
	to the $\mathit{Hom}_k$-functor, hence Lemma \ref{bipermallthethings} applies, and
	we get a bipermutative structure on the  $\MM^L_k$.

	Analogous to finite sets we can restrict to 
	injections, surjections and isomorphisms. The case of 
	isomorphisms is the one of primary interest in this thesis, so I 
	define \emph{the} module category $\MM_k$ of a commutative ring
    as:\[\MM_k(n,m)=\begin{cases}GL_n(k) &~~n=m\\ 
	\emptyset &~~n\neq m.\end{cases}\]}}

For $k=\RR,\CC$ we have a topological version of the above example.
{\ex{The same constructions as in the example above describe continuous
     functors with respect to the topologies on $GL_n\RR$ and $GL_n\CC$
     as subspaces of $M_n\RR\cong \RR^{n^2}$ and $M_n\CC\cong \CC^{n^2}$.
     Call the category with objects the natural numbers and morphism
     spaces $GL_nk$ considered as a topologically enriched category $\MM_k^c,$
     where the upper index is a reminder for continuity. Call the 
     analogous discrete category with morphism sets $GL_nk$ and
     their discrete topology $\MM_k^\delta$.

     For real and complex coefficients we can restrict 
     to the respective compact subgroups:
		\[O_n\rightarrow GL_n(\mathbb{R}), \mathrm{~and~}~U_n \rightarrow 
		GL_n(\mathbb{C}).\]
     These inclusions define subcategories of the topological as well as the
     discrete module categories.
     Denote the topological subcategories by $\V^c_k\subset \MM_k^c$ and 
     analogously the discrete subcategories by $\V^\delta_k\subset \MM_k^\delta$.
     Since the symmetries and distributors are unitary morphisms as well, 
     the canonical inclusion functors are bipermutative:
	    \[\V_\mathbb{R}\rightarrow \MM_\mathbb{R},\]
	    \[\V_\mathbb{C}\rightarrow \MM_\mathbb{C},\]
     for the topological as well as the discrete versions.

     A continuous inverse is given by the Gram-Schmidt process, which I
     denote by $r\colon \MM_k\rightarrow \V_k.$ This is compatible
     with direct sum by considering the sum as orthogonal. 
     It can be promoted to a homeomorphism
     as follows (cf. \cite[pp.33-35]{MT}): We have a natural map \[g_n\colon O_nk
     \times H^+_nk\rightarrow GL_nk\] with 
	$g_n(U,B)=UB$ for $O_nk$ the orthogonal group for $k=\RR$ and the 
	unitary group for $k=\CC$, and $H^+_nk$ the space of symmetric/hermitian 
	positive definite matrices. For each $n$ the map $g_n$ is a 
	homeomorphism with inverse:\[h_n\colon GL_nk \rightarrow O_nk\times H^+_nk\] 
	given by: \[h_n(A)= (A\sqrt{(A^*A)}^{-1}, \sqrt{(A^*A)}).\]

	The map $g_n$ is compatible with direct sums. For tensor-products
	consider the following diagram:	\[\xymatrix{
	O_n\times H_n^+\times O_m\times H_m^+\ar[r]^-{g_n\times g_m}\ar[d] 
	& GL_n\times GL_m\ar[dd]^\otimes\\ O_n\times O_m\times H_n^+\times H_m^+
	\ar[d]_{\otimes\times\otimes}\\
	O_{nm}\times H^+_{nm} \ar[r]_{g_{nm}} & GL_{nm}, }	\]
	which needs to commute for $g$ to induce a strictly multiplicative functor 
	$\V_k\rightarrow \MM_k$. Fixing associative bijections $\omega$ 
	write:$(A\otimes B)_{ij}=A_{i_1j_1}\cdot B_{i_2j_2}.$

	Then we have:
	\[\begin{aligned} 
	(g_n(U,B)\otimes g_m(V,C))_{ij}&= g_n(U,B)_{i_1j_1}g_m(V,C)_{i_2j_2}\\
	&=(UB)_{i_1j_1}(VC)_{i_2j_2}=\sum_{k,l} U_{i_1k}B_{kj_1}V_{i_2l}C_{lj_2},
	\end{aligned}\]	while the other side is given by: \[\begin{aligned}
	g_{nm}(U\otimes V,B\otimes C)_{ij}	&= ((U\otimes V)(B\otimes C))_{ij}\\
	&= \sum_p (U\otimes V)_{ip}(B\otimes C)_{pj}= 
	\sum_p U_{i_1p_1}V_{i_2p_2}B_{p_1j_1}C_{p_2j_2}.	\end{aligned}\]
	The terms thus agree by commutativity of addition and multiplication 
	in $k$, so we get a bipermutative inclusion $I\colon \V_k\rightarrow 
	\MM_k$ as well as a bipermutative retraction $R\colon \MM_k\rightarrow 
	\V_k$ given by $g$. Specifically we get topologically enriched  
	functors $R\colon \MM_k^c\rightarrow \V_k^c,$ i.e., functors, which
	are continuous on the morphism spaces, and the same assignments define
	functors on the discrete versions $R\colon \MM_k^\delta\rightarrow 
	\V_k^\delta$.}}

{\rem{The tensor-product structure on categories of the form $\MM_k$ has
    a more natural interpretation: Choose an euclidean/hermitian scalar 
    product in each dimension $n$ for $k=\RR,\CC$ or any 
    non-degenerate bilinear form for an arbitrary field $k$ 
    - say $\langle\cdot,\cdot\rangle$. By basic 
	linear algebra we know that any bilinear form $b$ yields a uniquely 
	determined linear map $f$ such that $\langle f\cdot,\cdot \rangle = 
	b(\cdot,\cdot).$ If moreover the bilinear form $b$ is non-degenerate
	as well, then $f$ is an isomorphism. Hence fixing (any) non-degenerate 
	bilinear form $\langle\cdot,\cdot\rangle$ we get an isomorphism:
    \[Bil_+(V)\cong GL(V)\]
	of non-degenerate bilinear-forms on $V$ and linear automorphisms of $V$,
	which is a homeomorphism when meaningful.

	In light of this consider the following tensor product of bilinear forms:
    For two bilinear forms $b^V\colon V\otimes V\rightarrow k$ and $b^W\colon W\otimes W
	\rightarrow k$, define their tensor product $b^{V\otimes W}\colon (V\otimes W)^{\otimes 2}
	\rightarrow k$ on generators:
	\[b^{V\otimes W}(v_1\otimes w_1,v_2\otimes w_2) = b^V(v_1,v_2)\cdot 
	b^W(w_1,w_2),\]	and extend bilinearly.
	This is non-degenerate if both forms above are non-degenerate. It is
	symmetric/hermitian if both forms are symmetric/hermitian. 

	On representing matrices we get the following: \[b^{V\otimes W}(e_i\otimes 
	e_j,e_k\otimes e_l) = b^V(e_i,e_k)b^W(e_j,e_l) = b^V_{ik}b^W_{jl},\]
	hence precisely the coefficients given in \ref{modultensor} for the
	tensor-product of the linear maps associated to the representing
	matrices.}}

Finally I summarise the canonical functors between the examples:
{\ex{For each commutative ring $k$ we have a canonical inclusion functor
	\[k[\cdot]\colon \mathrm{Fin} \rightarrow \MM^L_k\]
	given by sending each finite set to its associated free module, and
	each map to its linear extension. This is a strictly additive as
	well as multiplicative functor, moreover it strictly respects the
	symmetries and distributors. We can obviously restrict to 
	$\mathrm{Inj}$, $\mathrm{Epi}$, $\Sigma$ and restrict the codomain to 
	the appropriate linear maps, i.e., monomorphisms, epimorphisms, 
	or isomorphisms.
	
	We also have the analogous canonical inclusion functor given by
	reduced free modules:
	\[\tilde k[\cdot]\colon \Fi\rightarrow \MM^L_k\]
	sending each finite set to the associated free module with the 
	basepoint divided out. It is again strictly symmetric monoidal
	with respect to pointed sum and smash product, and respects the
	distributivity transformations as well.

	Finally for $k=\mathbb{R},\mathbb{C}$ these functors even have image
    in the categories $\V_k$, since the described maps are obviously orthogonal/unitary.}}

{\defn{Since we want to model connective complex $K$-theory we consider
$k=\CC$ in the example above, and find that $\V^c_\CC$ also becomes a
topological bipermutative category in that case. The delooping of 
$\V^c_\CC$ is the prototypical model for connective complex $K$-theory. 
In particular I denote by $ku:=H\V^c_\CC$ the delooping spectrum of the
topological category of finitely generated complex vector spaces with
morphism spaces the unitary isomorphisms.}}

\subsection*{Preliminaries on Discrete Models for $ku$}\label{ltolzeta}
For each odd prime $p$ there are in addition ``discrete models'' for $ku$, 
which are a suitable replacement when studying its $H\F_p$-homology.

Fix a prime $p$ for which we want an $H\F_p$-approximation
of $ku$, i.e., a spectrum $E$ with a map $E\rightarrow ku,$ which
induces an isomorphism on $H\F_p$-homology. Following Quillen \cite{Q1971}
we want to approximate $ku$ by algebraic $K$-theory of the algebraic
closure $\bar\F_p$ of the finite field with $p$ elements. In fact it
is sufficient to restrict to a subfield of $\bar\F_p$ which contains
the appropriate roots of unity, which we construct here.

For that we need to choose a prime that is a generator 
of $(\Z/p^2)^\times$. Observe that necessarily the existence of just 
one generator implies that $p$ is odd, because for $p=2$ we have 
$(\Z/2^k)^\times\cong\{\pm 1\}\times\Z/2^{k-2}\Z,$ where for $k\geq 3$ 
the second factor is always generated by the powers of $5$ 
\cite[Art. 91, p. 89 - Latin edn.]{Gauss}. 
For $p$ odd the group of units has a 
decomposition $(\Z/p^k)^\times \cong \Z/(p-1)\times \Z/p^{k-1},$ 
and is thus a product of two cyclic groups of coprime order 
\cite[Art. 84, p. 82 - Latin edn.]{Gauss}. Given an integer $g$ reducing
to a multiplicative generator of the units of $\Z/p$ we know by 
Fermat's little theorem $g^{p-1}=1~ \mathrm{mod}~ p$, so 
$g^{p-1}=1+lp$ for some $l\in\Z$. Then we have in $\Z/p^2$: 
\[(g+p)^{p-1}=g^{p-1}+(p-1)g^{p-2}p \mathrm{~mod~}p^2\neq g^{p-1}
\mathrm{~mod~}p^2,\] so at least one of the integers $\{g,g+p\}$ 
satisfies \[g^{p-1}\neq 1 \mathrm{~mod~}p^2~~~ \mathrm{or}~~~(g+p)^{p-1}\neq 1\mathrm{~mod~} 
p^2.\] An integer of $\{g,g+p\}$ satisfying the above inequality 
(multiplicatively) generates the units of $\Z/p^k$ for each $k\geq 2$, 
in particular it generates the units of the $p$-adic integers $\Z_p$ 
topologically.

We use Dirichlet's Theorem on arithmetic progressions
in the following form:
{\thm[Dirichlet]{For a natural number $n\geq 2$ and a unit 
$a\in(\Z/n)^\times$ consider the class of primes 
$P_a=\{~p\in\N~|~p \mathrm{~prime~ and~~} p = a \mathrm{~mod~} n \}$.
Then each class $P_a$ has ``logarithmic density'' $\frac1{\varphi(n)}$ 
in the set of all primes, for $\varphi(n)$ the number of units in
$\Z/n$.}}

{\rem{I do not need the concept of logarithmic density again, so I 
only give a vague description: The intuition 
is that it is an adapted way to measure subsets of countable sets 
(such as the set of all prime numbers), such that the measure is $0$ 
for finite subsets.}}

One proof of the theorem by complex analysis involves the Dirichlet 
$L$-series associated to a homomorphism $(\Z/n)^\times \rightarrow 
\CC^\times.$ For the trivial homomorphism which sends everything 
to $1\in\CC$ the $L$-series has a singularity in $1$. This forces 
the $L$-series of every non-trivial character to be bounded, 
but non-zero, in $1$. This gives the following comparison of 
divergence around $s=1$: 
\[\sum_{p\equiv a \mod n}p^{-s}=\frac1{\varphi(a)}\log\frac1{s-1}\pm C.\] 
In words: The sum over all primes, which are in $P_a$, 
taken with the exponent $-s$ diverges like $\log\frac1{s-1}$ 
in $1$ (up to a constant $C\in \RR$). In particular there are 
infinitely many such primes.

We can use the theorem in particular to specialise to a 
generator $a\in\Z/p^2$, and find a prime $q$ with $q=a~\mod ~p^2,$ 
which generates the units of each $\Z/p^k$ by the considerations 
before. 

{\ex{I want
to exhibit valid choices for all primes below $100$. I organised the 
table by smallest multiplicative generator $q$ for $\Z/p^2$:\\
\phantom{aaaaaaaaaaaaaaaaaaaaaaaaaaaa} \begin{tabular}{c|r}$q$&$p$\\
\hline$2$&$3,5,11,13,19,29,37,53,59,61,67,83,$\\$3$&$7,17,31,43,79,
89,$\\$5$&$23,47,73,97,$\\$7$&$41,71.$\end{tabular}
\phantom{blubb}\\}}

We want to approximate $ku$ by algebraic $K$-theory applied to 
a suitable tower of field extensions.
Start with the prime field with $q$ elements $\F_q$. Since $q$ 
generates the units of $\Z/p^2$ it generates $\Z/p^\times$
as well. So the cyclotomic polynomial of degree $p-1$ \[\varphi_{p}(X)
=\sum_{i=0}^{p-1}X^i\] is irreducible over $\F_q$. Thus we have the 
extension of fields \[\l_0:=\F_q\rightarrow \F_q[X]/\varphi_p\cong 
\F_{q^{p-1}}=\l_0(\zeta_{p})=:k_0\] for $\zeta_p$ some chosen primitive 
$p$th root of unity.
Since $q$ and $p$ are trivially coprime by the assumptions, each 
element in $\F_q$ has a $p$th root in $\l_0$. However, the units of
$k_0$ have order $q^{p-1}-1$. Because $q$ is a unit in $\Z/p$, we get
\[q^{p-1}-1=0\mathrm{~mod~} p,\]
so taking the $p$th power is not injective, hence not surjective.
So there is an element $a\in k_0$, which does not have a $p$th root.
There is an obvious candidate: $\zeta_p$. Because of our assumption on
$q$ and $p$ we have $q^{p-1}-1=0\mathrm{~mod~} p$, but 
$q^{p-1}-1\neq0\mathrm{~mod~} p^2,$ because $p-1$ is strictly smaller
than the order of the units of $\Z/p^2$, so the units of $\l_0(\zeta_p)$ 
decompose as:\[\l_0(\zeta_p)^\times \cong
\Z/p\langle\zeta_p\rangle\times \Z/s,\] for some $s$ coprime to $p$. 
In particular we find that the kernel of $(\cdot)^p$ is contained 
in the $\Z/p$-summand, while the image is contained in the $\Z/s$-factor,
hence $\zeta_p$ does not have a $p$th root in $k_0=\l_0(\zeta_p)$.
Since $p$ is odd, we get that $f_m(X)=X^{p^m}-\zeta_p$ is irreducible
for each $m\geq 1$ if and only if $\zeta_p$ has no $p$th root in $k_0$,
which we just established. So inductively call $\alpha_i = 
\sqrt[p]{\alpha_{i-1}}$ with $\alpha_0 = \zeta_p$. More explicitly
for each $i\geq 1$ we choose a primitive $p^i$th root of $\zeta_p$, and
call it $\alpha_i$.

\label{myfields}
For exposition let me choose a presentation. We can write $k_i$ for 
$i\geq1$ as:\[\F_q[X,Y_i]/\left(Y_i^{p^i}-X,\sum_{k=0}^{p-1}
X^k\right)\cong\F_q[Y_i]/\left(\sum_{k=0}^{p-1}Y^{p^ik}
=\varphi_p(Y_i^{p^i})\right).\] Then the field $\l_i\subset k_i$ is 
given as the fixed-set under the Galois-action of 
the factor $\Z/p-1$, which stems from the cyclotomic extension.

In the fields $k_i$ we trivially have the inclusions $k_i\rightarrow k_{i+1}$
with $Y_i\mapsto Y^p_{i+1},$ i.e., identifying $Y_{i+1}$ as a $p$th root
of $Y_i$. This yields the following diagram
\[\xymatrix{ k_0\ar[r] & k_1\ar[r] &\ldots \ar[r]& k_i \ar[r]&\ldots& \\
\l_0 \ar[u]\ar[r] & \l_1\ar[r]\ar[u] &\ldots\ar[r] & \l_i\ar[r]\ar[u] &\ldots,&}\] 
where the horizontal arrows signify $\Z/p$-Galois-extensions, while
the vertical arrows are $\Z/(p-1)$-Galois-extensions. Hence on colimits
$K=\colim_i k_i$ and $L=\colim_i \l_i$ we get the $\Z/(p-1)$-extension:
$L\rightarrow K,$ which by the presentations given above is
of the form \[L\rightarrow K=L[X]/\varphi_p =L(\zeta_p).\]

{\rem{Let me emphasise that the prime $p$ defining the degree
of the extension with Galois group $\Z/p-1$ is structurally important, while
$q$ only serves to ensure the existence of this extension and its particular
choice is irrelevant to the construction.}}

{\ex{Building on the previous example the extension
of $L$ by a primitive $p$th root of unity $\zeta_p$, as above
$L\rightarrow L(\zeta_p)$,
induces a map of bipermutative categories $\V_L\rightarrow
\V_{L(\zeta_p)}.$ Hence the delooping of these bipermutative 
categories provides a map \[H(\V_L)\rightarrow H(\V_{L(\zeta_p)}),\]
which is a map of $E_\infty$ symmetric ring spectra. Again referring 
to chapter \ref{multbidel} we see that these spectra are models for the 
algebraic $K$-theory of their respective fields and hence we understand 
this map as:\[K(L)\rightarrow K(L(\zeta_p)).\] Consider the Galois 
group of the extension $L\rightarrow L(\zeta_p)$: 
$G=Gal(L(\zeta_p)/L).$ For any homology theory $h_*$ with $p-1=|G|
=|\Z/p-1|$ a unit in its coefficients the map $K(L)\rightarrow K(L(\zeta_p))$
induces an isomorphism on $h$-homology groups:
\[h_*K(L)=(h_*K(L(\zeta_p)))^{G}.\] In particular for $h = H\F_p$ we 
get an equivalence of $p$-completed spectra: \[K(L)^\wedge_p\simeq
(K(L(\zeta_p))^\wedge_p)^{hG}.\]}}

\subsection*{Comparison of the Models}
One essential insight that led Quillen to the definition of algebraic
$K$-theory \cite{Q1971,Q1972,Q1973} was the fact that he could compute
the full Algebraic $K$-theory of all finite fields by comparison to
fibres of Adams operations on $BU$, so in essence by comparison to
$ku$. I want to exhibit the map involved, which is established by the
Brauer lift, but I shall defer the proofs to the relevant sources.

Again by the construction in \ref{myfields} we can easily fix a
homomorphism:\[\mu\colon L(\zeta_p)^\times \subset \langle\zeta_p
\rangle \times\bigoplus_{l ~\mathrm{prime} \neq p}\Z/l^\infty 
\rightarrow \mathbb{C}^\times\] by setting $\mu(\zeta_p) =
\exp(\frac{2\pi i}{p}).$ For a summand indexed by a prime $l$ choose a 
primitive $l$th root of unity in $\CC^\times$ as 
$\exp(\frac{2\pi i}{l})$ as well as the primitive $l^j$th roots of 
unity $\exp(\frac{2\pi i}{l^j})=\zeta_{l,j}$. This yields coherent 
homomorphisms for each $l$ \[\mu_l\colon \Z/l^j\rightarrow \CC^\times, 
\mathrm{~~hence~on~the~colimit~~} \mu_l\colon\Z/l^\infty \rightarrow 
\CC^\times.\]

With these choices fixed we can use the following theorem (cf.
\cite[p. 283, Theorem 5.3.4]{Ros1994}):
{\thm{Let $G$ be a finite group, and let $\rho\colon G\rightarrow 
GL_n\barF_q$ be a finite-dimensional representation of $G$ over the
algebraic closure of $\F_q$. Let $\{\xi^i_g~|~i=1,\ldots,n\}$ be the
eigenvalues of $\rho_g$ with multiplicities, so that the trace of
$\rho_g$ is given as: $tr(\rho_g)=\sum_i \xi^i_g.$

The function $f^\rho\colon G\rightarrow \CC$ with 
$f^\rho(g)=\sum_i\mu(\xi_g)$ is a class-function, hence by basic 
complex representation theory (cf. \cite[Part I, Chapters 1-3]{Serre}) 
a linear combination of characters of complex $G$-representations. 
Call $f^\rho$ the Brauer character of $\rho$.

In fact we have integral coefficients, so $f$ uniquely determines 
a complex virtual representation of $G$, called the 
\emph{Brauer lift} of $\rho$ - denote it $F(\rho)$. Furthermore 
the Brauer lift is additive, i.e., for a short
exact sequence of $\barF_q[G]$-modules:
\[0\rightarrow U\rightarrow V\rightarrow W\rightarrow 0\]
the lifts satisfy $F(V)=F(U)+F(W)$.}}

{\rem{Observe that the Brauer lift consists of
representations of $G$ that have eigenvalues on the circle 
$\S^1\subset \CC$, because the group has finite order.}}

I want to exhibit the idea of how this induces a comparison map, 
but I gloss over quite a few details, which are in part explained 
in \cite[pp. 284--285]{Ros1994} and much more in the original 
\cite{Q1971,Q1972}.

We want to consider the homomorphism $i_n\colon GL_n(L(\zeta_p))
\rightarrow GL_n(\barF_q)$. Since $L(\zeta_p)$ is a colimit of finite
fields the general linear groups $GL_n(l_i)$ are finite groups, which
yield $GL_n(L(\zeta_p))$ as their colimit. The Brauer Lift is
evidently stable in the colimit over the fields $l_i$, since the 
eigenvalues of the matrices do not change. 

We can determine the virtual dimension of the Brauer Lift by the 
trace of the identity: We get $\xi^i(id_n)=1$ for $1\leq i \leq n$, 
so $\sum_i\mu(\xi^i)=\sum_i\mu(1) = \sum_i1=n$.
This is obviously not stable in $n$; thus subtract the 
trivial $GL_n(\CC)$ representation of $GL_nl_i$,
and consider the lift of $i_n$ minus $n$. Obviously the 
trivial complex representation of $GL_nl_i$ of 
dimension $n$ is a lift for the trivial $GL_nl_i$-representation over
$\barF_q$, so we get $F(i_n-n)=F(i_n)-n$ and hence a stable class
of a virtual representation of dimension zero giving a map 
\[BGL(L(\zeta_p))\rightarrow BGL^\delta(\CC)\rightarrow BGL(\CC)\simeq BU.\]
This map induces homology isomorphisms with $\F_m$-coefficients for
any prime $m$ other than $q$, thus induces an equivalence of 
completed spaces at each prime $m\neq q$:
\[BGL(L(\zeta_p))^{\wedge}_m\rightarrow BU^{\wedge}_m,\]
given as theorems 1.6 and 4.7 by Quillen in \cite{Q1971}.

If the Brauer Lift happened to be not just a virtual representation
but indeed a genuine homomorphism $\Phi\colon GL(L(\zeta_p))
\rightarrow GL(\CC)$ we could try to restrict to the $GL_n$ again,
and induce a map of bipermutative categories
$\V_{L(\zeta_p)}\rightarrow \V_\CC,$ which would give an infinite loop
map, i.e., a map of spectra $K(L(\zeta_p))\rightarrow K(\CC)=ku$, and
furthermore of $E_\infty$-ring spectra. But calculating in low 
dimensions for $GL_2(\F_q)$ shows that the Brauer Lift has a genuine
negative component. I suspect this approach could be repaired
with a ``ring-complete'' version of $\V_\CC$ as given by \cite{BDRR2013},
but the result has been established long before that by other methods.

The additivity directly yields that the Brauer Lift is an 
$E_\infty$-map with respect to the $E_\infty$-structure on 
$BGL(L(\zeta_p))^+$ and $BU^+$ induced by direct sum of matrices. 
Furthermore in ``$E_\infty$ Ring Spaces and $E_\infty$ Ring Spectra'' 
May shows \cite[pp. 212-222]{MayEinf} that the induced map is also an 
$E_\infty$-map for the $E_\infty$-structure induced by tensor-products. 
With all this in place we have an equivalence of 
$E_\infty$ ring spectra at $p$:
\[K(L(\zeta_p))^\wedge_p\rightarrow K(\CC)=ku^\wedge_p.\]
The equivalence of spectra is given in 
\cite[pp. 217+218, Corollary VIII.2.7, Theorem VIII.2.8]{MayEinf}, the compatibility
with the $E_\infty$ structures on these spectra is 
\cite[pp. 219-222, Theorem VIII.2.11]{MayEinf}. \label{einfapprox}

\subsection*{The Involutions}
In this thesis I want to investigate the induced involution on the 
algebraic $K$-theory of $ku$ as well, so as the final comment on the
models I establish which involution is induced on $K(l(\zeta_p))$ by
the Brauer lift.

{\prop{For any multiplicative embedding $\mu\colon l(\zeta_p)^\times
\rightarrow\CC^\times$ we have the following relation for involutions:
\[\mu\circ(\cdot)^{-1}=(\cdot)^{-1}\circ\mu=\overline{(\cdot)}
\circ\mu.\]
That is, we have on $\CC^\times$ that multiplicative inversion $(\cdot)^{-1}$
and complex conjugation $\overline{(\cdot)}$ coincide on the image of the
embedding, and the embedding is a monoid homomorphism, thus compatible with
multiplicative inversion.
\begin{proof} Since we have $\mu(1)=1$ it commutes with inverting 
elements, which is a homomorphism because the involved groups are 
commutative. But since the order of every element of 
$l(\zeta_p)^\times$ is finite, we know that $\mu(l(\zeta_p)^\times)
\subset \S^1$. Hence inverting and complex conjugation coincide.
\end{proof}}}

{\prop{For any representation of a finite group $\rho\colon G
\rightarrow GL_n\barF_q$ and the induced representation
of the group $G^{op}$ with opposed multiplication given by
$\rho\circ(\cdot)^{-1}\colon G^{op}\rightarrow GL_n\barF_q$ we have 
the following relation for the Brauer characters:
\[\overline{f^{\rho}} = f^{\rho\circ(\cdot)^{-1}}.\]
\begin{proof} For $g\in G$ calculate the Brauer character with 
$\xi^i_g$ again the eigenvalues of $\rho_g$ with multiplicities:
\[\overline{\sum_i\mu\xi_g^i}=\sum_i\overline{\mu\xi_g^i} 
=\sum_i(\mu\xi_g^i)^{-1}=\sum_i\mu((\xi_g^i)^{-1})
=\sum_i\mu(\xi_{g^{-1}}^i),\] hence follows the claim.\end{proof}}}

Finally, we would like to induce this Brauer character by some virtual
representation which only explicitly depends on $\rho$ and starts
from the same group $G$ instead of the one with opposed 
multiplication, but we do not want to cancel out the inversion. 
Note that the target is a general linear group. For $G=GL_nR$ the group 
hence comes equipped with a second isomorphism from $G$ to $G^{op}$ 
given by transposition.

{\thm{For any representation of a finite group $\rho\colon 
G\rightarrow GL_n\barF_q$ let $\rho^\dagger$ be the representation 
induced by considering the composition:
\[\xymatrix{G\ar[r]^{(\cdot)^{-1}} & G^{op} \ar[r]^\rho 
& GL_n\barF_q^{op}\ar[r]^{(\cdot)^t}&GL_n\barF_q}\]
Then their Brauer characters satisfy:
\[f^{\rho^\dagger}=\overline{f^\rho},\]
and hence by uniqueness of the associated (virtual) representation
we find
\[F(f^{\rho^\dagger})=\overline{(\cdot)}\circ F(f^\rho).\] 
\begin{proof}Obviously transposing matrices does not change the 
eigenvalues involved in the definition of the Brauer character, so the 
preceding proposition directly yields the claimed result.\end{proof}}}

For ease of reference I summarise Quillen's approximation
\cite{Q1971} by the Brauer lift with respect to its multiplicative
and involutive structure in one theorem.
{\thm{The Brauer lift at any prime $p\geq 3$ is a map of $E_\infty$ ring 
	spectra	$K(L(\zeta_p))=H(\MM(L(\zeta_p)))\rightarrow H(\MM(\CC))=ku,$
	which is an equivalence of $E_\infty$-ring spectra after completion
	at $p$: $K(L(\zeta_p))^\wedge_p\rightarrow ku_p^\wedge.$

	Furthermore the involution on $ku$ given by complex conjugation is
	approximated by $(\cdot)^t\circ(\cdot)^{-1}$ on $L(\zeta_p)$, in 
	particular the involution as induced on $ku$ by \ref{indinvSP}
	from complex conjugation cancels out to give the approximation of
	$E_\infty$-ring spectra with involution:
	\[(K(L(\zeta_p))^\wedge_p,\id)\rightarrow (ku_p^\wedge,
	\overline{(\cdot)}_*).\]}}
\label{Invandmultappr}
