\chapter{Multiplicative Delooping of Bipermutative Bicategories}
\label{multbidel}
In the previous chapters I convinced the reader that bipermutative
bicategories exist, and that they occur when one wants to 
study algebraic $K$-theory of a bipermutative ($1$-)category. 
In particular, the primary example of this thesis $K(ku)$ can be 
described as the Eilenberg-MacLane-spectrum of the bicategory of 
finitely generated free modules of finite-dimensional complex vector 
spaces:\[K(ku)=H\mathcal{M}(\mathcal{V}_\mathbb{C}).\]
To tie this in with the calculations made by Christian Ausoni in the
papers \cite{AuTHH,AuQku,AuKku} we need a combinatorial handle on the 
$E_\infty$-structure on $K(ku)$ induced by the tensor-product on
$\mathcal{M}(\mathcal{V}_\mathbb{C})$. To this end I modify the
delooping given by Ang\'elica Osorno in \cite{Os} in a manner 
analogous to \cite{EM} (cf. in particular the paragraph after
Definition 4.3.) so the resulting construction allows an induced
multiplication by the multiplicative structure of a bipermutative
bicategory.
I do restrict to the case of $E_\infty$-structures, and 
also use a specific $E_\infty$-operad, the Barratt-Eccles-operad 
in a tentative multicategory of permutative bicategories. 

The reader should compare the delooping of this chapter to the
delooping in section 6 of \cite{GJOs}. The authors, however, are
driven by the desire to generalise \cite{Th1} to permutative 
bicategories, thus their emphasis is different from mine. This
makes the deloopings differ in a few ways, which I suspect are
inessential.

\section{The Additive Grothendieck Construction}
This section is where the work in \ref{pcatneu} to rewrite the 
delooping construction of \cite{EM} becomes fruitful. Since the 
additive symmetric monoidal structure on $\M$ as described by
\cite{Os} is sufficiently strict that the Grothendieck construction
\ref{c+} can be used for symmetric monoidal bicategories as well. So
we need to study
which functors are adequate for the analogous construction
of $\C(A_+,n)$ such that the delooping given by Osorno is the
case $n=1$ and such that we get pairings \ref{pairing} \[\C(A_+,n)\times \C(A_+,m)
\rightarrow \C(A_+,n+m),\] which induces an $E_\infty$-multiplication on 
the resulting spectrum \ref{mymult}.

I again use the shorthand $Ob\C=\C_0$ and for $1$-cells when I do not 
want to refer to their source and target I write $Mor\C=\C_1$.

{\rem{As indicated before $\boxplus$, i.e., block sum of matrices
turns $\M$ into a permutative bicategory. The tensor-product does not.}}

Symmetric monoidal bicategories of the strict type of permutative
bicategories allow for the same construction of an associated $\C^+$
as in \ref{c+}. 

{\lem{A permutative bicategory $\C$ has an associated pseudofunctor 
\[B_\C\colon \Fi\rightarrow Bicat,\] given by $B_\C(n_+)=
\C^{\times n}$ and $F_\C(f\colon n_+\rightarrow m_+) = f_*\colon 
\C^{\times n} \rightarrow \C^{\times m},$ where $f_*$ is the strict 
functor $f_*(c_1,\ldots,c_n)_j = \sum_{i\in f^{-1}j}c_i,$ with
compositors given by the additive symmetry: $\varphi^{f,g}\colon 
f_*\circ g_*\Rightarrow (fg)_*$ as a strict natural transformation of 
strict functors. \begin{proof} The proofs in \ref{pcatneu} transfer 
without any problems when I restrict the target to be the $2$-category
of bicategories with strict functors as $1$-cells and strictly natural
transformations as $2$-cells, which works because of the strictness of
permutative bicategories.\end{proof}}}

In particular the construction \ref{c+} translates literally:

{\defn{\label{bic+}Given a permutative bicategory $(\C,+,0,c_+)$ 
define its additive Grothendieck construction $\C^+$ as follows: It 
has objects: \[\C^+_0 = \coprod_{n\geq 0} \C^{\times n}\] and 
morphism categories: \[\C^+((c_1,\ldots,c_n),(d_1,\ldots,d_m))= 
\coprod_{f\in \Fi(n_+,m_+)} \C^{\times m}(f_*(c_1,\ldots,c_n),(d_1,
\ldots,d_m)),\] with composition functors given for a triple of
objects $a=(a_1,\ldots,a_n), b=(b_1,\ldots,b_m), c=(c_1,\ldots,c_l)$ as:
\[\xymatrix{\C^+(b,c)\times\C^+(a,b) \ar@{=}[d]\\ \coprod_{f,g} \C^{
\times l}(g_*b,c)\times\C^{\times m}(f_*a,b)\ar[d]^{\coprod id\times 
g_*}\\\coprod_{f,g}\C^{\times l}(g_*b,c)\times\C^{\times l}(g_*f_*a,
g_*b)\ar[d]^{\coprod comp_{\C^l}}\\\coprod_{f,g}\C^{\times l}(g_*f_*a,
c)\ar[d]^{\varphi^*}\\\coprod_{f,g}\C^{\times l}((gf)_*a,c)\subset
\C^+(a,c).}\]Identities are given by pairs $(id,(id))$ for 
$id\colon n_+\rightarrow n_+$ and $(id)\colon (a_1,\ldots,a_n)
\rightarrow (a_1,\ldots,a_n)$ the $n$-tuple of identities. In 
particular the identities are strict identities because the ones in 
$\C$ are strict.

The associator is given as follows: In each product bicategory $\C^l$ 
we have an associator given by the $l$-tuple with the appropriate 
instances of the $\C$-associator. For this paragraph call this 
$\alpha_l$. Then consider the following two ways of forming a 
three-fold composite for $a,b,c$ as above and $d=(d_1,\ldots,d_k)$:
\[\xymatrix{\C^+(c,d)\times \C^+(b,c)\times \C^+(a,b) \ar[r] \ar[d]
& \C^+(b,d)\times\C^+(a,b)\ar[d]\\\C^+(b,d)\times\C^+(a,b) \ar[r] & 
\C^+(a,d).}\] So we need a natural transformation of the two 
composition-functors: \[\C^+(c,d)\times \C^+(b,c)\times \C^+(a,b)
\Rightarrow \C^+(a,d),\] which is defined on the category $\C^+(a,d)=
\coprod_f \C^k(f_*a,d)$, hence it has components $\alpha_k$.}}

{\rem{Let me issue a warning here: I have no idea what happens, when
one tries to apply the same construction to less strict symmetric
monoidal categories - even of the type I defined in \ref{mdlbic}. I
strongly suspect, it involves a lot more care.

In particular this construction is not naturally set up with respect
to the context of bicategories, in that it is the Grothendieck 
construction of a functor into the $2$-category of bicategories
with strict functors and strict natural transformations as morphisms.
Not every strong functor of symmetric monoidal $1$-categories
can be made strict - consider for instance the strictification functor
$\varepsilon\colon (\C,+)^{st}\rightarrow (\C,+)$, which augments the 
permutative strictification of an arbitary symmetric monoidal category
 \ref{strict1}. Its suspension to bicategories \ref{sigmasymmmon} 
hence is an example of a strong normal functor/pseudofunctor, which is 
not equivalent to a strict functor.}}

{\rem{If the input bicategory $\C$ is in fact a $2$-category,
    i.e., a bicategory with associator-cells only identities, then
    $\C^+$ is a $2$-category as well.}}

{\rem{Do note that the subcategory of morphisms with only discrete
components is a strict $2$-category, because its composition is just
the one in $\Fi$.

Furthermore, just as in the $1$-categorical case, we have that each
morphism $1$-cell in $\C^+$ can be written uniquely as: \[\xymatrix{
{\mathbf{c}}=(c_1,\ldots,c_n) \ar[r]^-{(f,\id)}&
f_*\mathbf{c}\ar[rrr]^-{(\id_m, (A_1,\ldots,A_m))} &&& (d_1,\ldots,
d_m).}\]}

\section{A Multiplicative Delooping for Bicategories}
{\Remi{Because it features prominently in this chapter let me recall
the concept of an \emph{equivalence} in a bicategory. It is a
$1$-cell, say $A\colon a\rightarrow b$, which has a $1$-cell in
the other direction $B\colon b\rightarrow a$ such that $AB\cong \id_b$
and $BA\cong\id_a$. By fixing directions and only demanding morphisms
instead of isomorphism $2$-cells we arrive at the concept of 
\emph{adjoint} $1$-cells, but I do not need that here.}}

{\ex{Adjoint as well as equivalence $1$-cells in our primary example
of interest $\mathcal{M}(\MM_k)$ are just permutation matrices. To see 
this first note that we consider only isomorphism $2$-cells, because
we only have isomorphisms in $\MM_k$, hence being adjoint and being equivalent
is the same. Furthermore since $\MM_k$ is skeletal the existence of 
an isomorphism $AB\rightarrow \id_n$ already gives an equality of the $1$-cells,
i.e., matrices, $AB=\id_n$, analogously for $BA$. Hence $A$ and $B$ are strictly 
invertible matrices, both in $GL(\mathbb{N})$, thus permutation matrices.}}

The natural analogue for bicategories of \ref{isorest} is the 
following proposition:
{\prop{There is a natural inclusion $(\C^{eq_1})^+\rightarrow \C^+$,
where the bicategory $\C^{eq_1}$ is the bicategory with the same
objects as $\C$, and morphism categories on $1$-cells only the
equivalences and all $2$-cells.

Furthermore there is a natural inclusion $(\C^{eq_1,iso_2})^+
\rightarrow (\C^{eq_1})^+\rightarrow \C^+$ where we restrict to just
equivalence $1$-cells and isomorphism $2$-cells.}}

\subsection*{The Construction $\C(A_+,1)$ for Permutative Bicategories}
Many of the next considerations would probably work for bicategories 
with not just isomorphism $2$-cells, if one modifies the results 
appropriately. Most of the time this means changing equivalences into
adjunctions, which probably introduces a lot more thought about when
these $1$-cells compose appropriately. But since my emphasis is on 
delooping to a $K$-theory spectrum, in this chapter I only consider 
bicategories with $\C = \C^{iso_2},$ hence also $\C^+=(\C^{iso_2})^+$. 
This in particular makes every adjunction in $\C$ already an 
equivalence.

To arrange the delooping bicategories as functor bicategories
we need to understand the forgetful functor $\C^+\rightarrow \Fi$ 
again, which yields a stronger assertion for bicategories than
for $1$-categories.
{\prop{The forgetful functor $U\colon \C^+\rightarrow \Fi,$ which assigns
each tuple of objects in $\C^+$, i.e., $(c_1,\ldots,c_n)$ to its
finite pointed set $n_+$ and on morphisms
$(f,(A_1,\ldots,A_m))$ forgets down to  
the discrete component in finite pointed sets $f\colon n_+\rightarrow m_+$, is a 
strict functor of bicategories.\begin{proof}
This is trivially true, since the target is a $1$-category, thus 
does not support non-trivial compositor $2$-cells.\end{proof}}}

Recall the ``comma categories'' introduced in \ref{commaindex}: For
an arbitrary finite set $A$ with added disjoint basepoint $\{*\}\sqcup A=A_+$
consider the category of pointed maps under it, i.e., $\Aix$ 
with objects maps from $A_+$ to a natural number $n_+=\{0,1,
\ldots,n \}=\{*\}\sqcup\{1,\ldots,n\},$ and morphisms commutative 
triangles under $A_+$. 

Recall that the index categories also have forgetful functors 
$T\colon \Aix \rightarrow \Fi$, which send each object to its 
target and forgets the commutativity of triangles under $A_+$. 

This way I can define $\C(A_+,1)$ for permutative bicategories:
{\defn{The bicategory $\C(A_+,1)$ has objects strong normal functors
that lift $T$ through $U$ \[\xymatrix{&(\C^{eq_1,iso_2})^+\ar[d]^{U}\\
\Aix\ar@{-->}[ur]\ar[r]^T & \Fi,}\] i.e., send maps of finite sets 
under $A_+$ to equivalences in $\C^+$.

The category of morphisms between two such lifts $F,G$ is given by
the morphism category $\mathrm{Bicat}(J_*F,J_*G)$, with $J\colon
(\C^{eq_1})^+\rightarrow \C^+$ the natural inclusion. That is,
we consider the morphism category with strong pseudonatural 
transformations, i.e., with isomorphism $2$-cells but arbitrary 
$1$-cells of $\C$, and modifications between those comprised of 
isomorphism $2$-cells.}}

{\rem{Recall that a map of finite based sets $f\colon A_+\rightarrow
    B_+$ induces a map of the indexing categories in the opposite
    direction $f^*\colon \Bix\rightarrow\Aix$, which is a functor
    over $T\colon \Bix\rightarrow \Fi$. By restricting lifting 
    functors from $\Aix$ to $\Bix$ along $f^*$ we thus get lifting 
    functors from $\Bix$, so in summary a strict normal functor 
    $f_*\colon\C(A_+,1)\rightarrow \C(B_+,1)$ in the same direction 
    as $f$.}}

Since in what follows the subbicategory of $\C^+$ with just 
equivalence $1$-cells is the central object, I reduce the notation 
to ${\C^{eq}}^+$ to refer to the additive Grothendieck
construction on the subbicategory of equivalence $1$-cells and 
isomorphism $2$-cells of a permutative bicategory $(\C,+)$.

{\rem{Again consider in $\Aix$ the ``full'', actually
discrete, subcategory given by characteristic functions $\chi_a\colon
A_+ \rightarrow 1_+$ with $\chi_a(x)=*$ for $x\neq a$ and 
$\chi_a(a)=1$. This yields a natural inclusion:
\[\chi_\bullet\colon A^\delta \rightarrow \Aix.\]

On the other hand we have the subcategory of $\Aix$ given by objects
the bijections and morphisms between them. This gives an inclusion
of the translation category associated to the bijections of $A$,
or equivalently pointed bijections of $A_+$\[E\Sigma_A \rightarrow 
\Aix.\]This inclusion embeds the full subcategory of initial objects
of $\Aix$, since each map under $A_+$ can be factored uniquely through 
a bijection. In particular I do not refer to these initial objects 
as initial again, because the isomorphisms between them are 
prominent in the delooping.}}

The delooping is supposed to be a generalisation of the classical
delooping of (topological) abelian groups, so we should expect the
objects to be determined by their ``summands''. The following 
proposition should thus not be surprising, parallel
to the analogous statement in \ref{pcatneu}.

{\prop{Any pseudofunctor lifting $T$ through $U$ has a unique up to
equivalence \emph{strict representative}. More precisely: Any two 
functors with the same restrictions along $\chi_\bullet$ are 
naturally equivalent in $\C(A_+,1)$.\begin{proof}
Choose a total ordering on $A$, hence a bijection $\sigma^A\colon
A_+\rightarrow |A|_+$, and consider a lifting functor 
$F\colon \Aix\rightarrow (\C^{eq})^+.$ 

By the assumption that $F$ sends $1$-cells to equivalences we have
an equivalence in the product bicategory $\C^{\times |A|}$ of the
form $F\sigma_A\rightarrow (F\chi_a)_{a\in A_+}$ given by the
components associated to the diagrams in $\Aix$:
\[\xymatrix{A_+\ar[r]^{\sigma_A}\ar[dr]^{\chi_a} & |A|_+
\ar[d]^{\rho^a} \\	  & 1_+.}\]
So the equivalence is given by $(\id_{|A|},(F^\C\rho_a)_{a\in A})$ in 
$\C^{\times|A|}\subset \C^+$, for $F^\C$ the $\C$-$1$-cells of the 
equivalence without their discrete components $\rho^a$ in $\C^+$.
Choose an inverse to this equivalence in the product category, hence
$\zeta_a F^\C(\rho_a)\cong \id_{F\chi_a}$ with the analogous isomorphism
for the other composition of $\zeta_a$ with $F^\C(\rho_a)$.

Build the \emph{strict representative} as follows: $F^{st}(\sigma_A)
:= (F\chi_a)_{a\in A}$. Any other object of $\Aix$ has a unique 
morphism coming from $\sigma_A$, so for $p\in \Aix$ set 
$F^{st}(p):=(p\circ\sigma_A^{-1})_*(F^{st}(\sigma_A))\in\C^{\times
|Tp|}.$ Again drop $\sigma_A^{-1}$ from the notation for instance by
assuming $A_+$ totally ordered, thus a unique element of $\Fi$ itself.
For a commutative triangle under $A_+$:\[\xymatrix{ A_+\ar[r]^p
\ar[dr]^{qp}& n_+\ar[d]^q\\& m_+}\] we need to have a morphism
\[F^{st}(p)=p_*(F^{st}(\sigma_A))\rightarrow 
q_*p_*(F^{st}\sigma_A) \rightarrow (qp)_*(F^{st}(\sigma_A))=
F^{st}(qp),\] which by construction of $\C^+$ we can take to be 
$(q,\varphi^{q,p}),$ and this is obviously a morphism over $q$ in 
$\Ep$. So we have constructed $F^{st}$ as a lift of $T$ through $U$, 
which sends each commutative triangle in $\Aix$ to morphisms in $\C^+$ 
with just discrete components and additive symmetries. In particular
we can choose $F^{st}$ with identity $2$-cells, and thus have a 
strict normal functor, because the additive symmetries were assumed
to be strictly natural for permutative bicategories.

By the decomposition of $1$-cells in $\C^+$, we can uniquely write the
map $F(p\circ\sigma_A^{-1})\colon F(\sigma_A)\rightarrow F(p)$ as its
discrete component followed by a $1$-cell with discrete component the
identity \[\xymatrix{F\sigma_A \ar[r]^{(p,\id)} &p_*F\sigma_A 
\ar[rrr]^{(\id,F^\C(p\circ\sigma_A^{-1}))} &&& Fp}.\]
So we have in $\C^+$ with the equivalence $1$-cells $\zeta_a$ as 
chosen before: \[\xymatrix{F^{st}p = p_*((F\chi_a)_{a\in A}) 
\ar[rrr]^-{(\id,p_*((\zeta_a)_a))}&&& p_*F\sigma_A 
\ar[rr]^-{(\id,F^\C(p\circ\sigma_A^{-1}))}&& Fp,}\] which we can promote to 
a pseudonatural transformation by choosing as the naturality $2$-cells 
the inverses of the adequate compositor $2$-cells $F$. This 
transformation then trivially commutes with the strict compositor of 
$F^{st}$ and the ones of $F$, and has as $1$-cells equivalences by
construction. So we have established a pseudonatural equivalence 
$F^{st}\simeq F,$ which only depended on data coming from $F,$ 
while $F^{st}$ even only depended on the restriction of $F$ along 
$A^\delta\rightarrow \Aix$, hence is as unique as claimed.
\end{proof}}}

{\rem{At this point let me informally compare this construction to
the one displayed in the proof of Theorem 3.6 in \cite{Os}.
The objects as described there are strict functors $(\Aix)^{op}
\rightarrow \C^+,$ so precisely the chosen inverse equivalences 
$\zeta$ I just described. For notational convenience let me treat 
them as if the functors in \cite{Os} were written down as covariant 
functors $\Aix\rightarrow\C^+$.

The passage to the strict representative as I indicated above shows
that each pseudofunctor $\Aix \rightarrow {\C^{eq}}^+$ lifting $T$ 
through $U$ is naturally equivalent to one that is not just a strict
functor, but also just comprised of discrete components. So we can
include the functors described by Osorno into $\C(A_+,1)$ and find
that the strict representative is of the kind described in \cite{Os}, 
so we get a surjection up to equivalence, which by inspection of
the $1$- and $2$-cells described in that same proof is also 
an equivalence on the morphism categories. (Most specifically she
describes the construction on $\underline{\mathbf{n}},$ which is
$n_+$ in my convention, so a subbicategory of $\C(n_+,1)$.)

I chose the morphism categories in the delooping construction just so 
that this equivalence is true.}}
	
The passage to the strict representative is sufficiently natural
that the typical delooping result is an easy corollary:
{\cor{\label{dell} We have a natural equivalence of bicategories:
	\[(\cdot)^{st}\colon \C(A_+,1)\rightarrow \C^{A}.\]	
\begin{proof} We know that 
$\C^{A}$ is strictly equal to the bicategory of functors $A^\delta 
\rightarrow \C^+$ and we can restrict each functor to its components 
on $(\chi_a)_{a\in A}$, which is precisely $A^\delta$ as a full 
subcategory of $\Aix$. So the inclusion of the
product bicategory by the functor, which sends each tuple to a lifting
functor which is its own strict representative, is an inverse 
equivalence for $(\cdot)^{st}$. On the left we find that the natural 
equivalence to the identity is just the one described at the end of 
the proof before. On the right we have: Making a tuple into a strict 
functor and then restricting to its $\chi_a$-summands is strictly 
equal to the identity functor on $\C^A$.\end{proof}}}

{\rem{This is the initial step in the induction to prove the
analogous equivalence for the higher delooping bicategories 
$\C(A_+,n).$}}

\subsection{The Construction $\C(A_+,n)$ for Permutative Bicategories}
This section is where the rewriting of the delooping constructions 
of \cite{EM} and \cite{Os} in \ref{pcatneu} and the section before
comes to fruition, because this way it easily generalises to using 
$(\Aix)^{\times n}$ as the index category, and letting coherence 
$2$-cells take care of themselves by using pseudofunctors.

Recall from \ref{pcatneu} the target functors $T_n\colon (\Aix)^n
\rightarrow \C_+$ given in \ref{T-n}: We have the analogous 
definition of $\C(A_+,n)$ for bicategories.
{\defn{For a permutative bicategory $(\C,+)$ the delooping bicategory 
	$\C(A_+,n)$ has as objects strong normal functors $F$ lifting
	$T_n$ through $U$:	\[\xymatrix{& (\C^{eq})^+\ar[d]^U\\ 
    (\Aix)^n \ar@{-->}[ur]^F \ar[r]_-{T_n} & \Fi.}\]
	Its morphism bicategories are again the pseudonatural 
	transformations and modifications of the functors after including
	by $(\C^{eq})^+\rightarrow \C^+.$}}

{\rem{Do note that as in the case in \ref{pcatneu} the delooping bicategories
	$\C(A_+,n)$ have a canonical basepoint object given by the functor $O_n$,
	which has as objects the adequate zero-tuples of each degree and morphisms
	consisting of the adequate discrete components with $\id_0$ as its 
	second component.\label{thebasenull}}}

{\rem{Since for $n=1$ we do not have to choose bijections for the smash
    product in $\Fi$ the functor $T_1$ is strictly the same as $T$ in 
    the section before, in particular I described the same bicategory 
    of functors.
    
    It is consistent to set $\C(A_+,0)=\C$, since $\left(\Aix\right)^0=*$ is
    the one-point category.}}

{\rem{Apart from an opposition of the indexing category $\Aix$ this 
    definition would read ``strict normal'' functors in the delooping
    considered by \cite{Os}, which works well there because the 
    additive structure is strict enough. Since I want to induce a 
    multiplicative structure from a symmetric monoidal structure
    as given by $\boxtimes$ in \ref{multbidel}, which prominently
    features a pseudofunctor which is usually not strict, I need to
    consider more generally all strong normal pseudofunctors with
    possibly non-trivial compositor $2$-cells.}}

It is possible to give an explicit construction of the 
delooping bicategories of a permutative bicategory along the lines 
of \cite{EM} and \cite{Os}. However, since the tensor functor I 
describe in \ref{tensor} has a non-trivial isomorphism $2$-cell I 
cannot restrict to strict additors the way Osorno does in 
\cite[Proof of Theorem 3.6, p. 11]{Os}, but have to allow potentially
non-trivial isomorphism $2$-cells. This becomes unwieldy in the 
explicit construction, so I arranged the delooping by functor 
bicategories, analogous to the rewriting of \cite{EM} I present
in \ref{pcatneu} above. In particular the construction runs parallel 
to the $1$-categorical case, with equivalences inserted where there 
are isomorphisms for permutative $1$-categories.

{\rem{At this point an informal comparison to \cite{EM} is convenient.
	For the case of bicategories with discrete morphism categories (or
	actually topological spaces or simplicial sets, which are discrete 
	as $1$-categories, but possibly non-discrete as spaces,) we can 
	compare to Construction 4.4. on page 19 of \cite{EM}. The systems 
	described there are indexed over arbitrary product categories 
	$(A_1\downarrow\Ep)\times\ldots\times(A_n\downarrow\Ep)$. By 
	passing their based systems of subsets $S=(S_1,\ldots,S_n)$ to 
	their unbased components we can associate to each such subset a 
	characteristic map, and thus an object in $A_i\downarrow\Ep$. The 
	additors $\rho$ described there are then given under the condition
	that we can factor a characteristic map over $2_+$, and hence we 
	get an associated map for this $2_+\rightarrow 1_+$-component.

	Condition (1) is then the fact that the wedge at the basepoints of 
	the index-categories $A_i\downarrow \Ep$, which is given by tuples 
	of maps, which have any component mapping to $0_+$, smashes to 
	$0_+$ and is hence sent to $0_+$ by	$T_n$. Condition (2) is the 
	normality of the functor, i.e., strictly respecting identities. 
	Condition (3) expresses the fact that the choice of which subset 
	to map to which element in $2_+$ should not matter. Condition (4)
	is simply the functoriality, as in respecting composites strictly, 
	because the	context in \cite{EM} are enriched $1$-categories, and 
	last condition (5) is by construction of $\C^+$ and by the remark
	\ref{fuerdiehohenReds}, which still applies for the additive
	Grothendieck construction on even general bicategories, also just
	strictly respecting composites. Here $1+c_++1$ is the twist
	needed to express the following trivially commutative diagram
	in $\Ep^{\times 2}:$\[\xymatrix{2_+\wedge 2_+ \ar[r]\ar[d]& 1_+
	\wedge 2_+\ar[d]\\2_+\wedge 1_+ \ar[r] & 1_+\wedge 1_+,}\]by 
	flattening it with the bijections $\omega$ chosen before, such 
	that we have get indexing sets appropriate for summations.
	Morphisms are the appropriate stricter version of the ones 
	considered in \cite{Os} as well, so the same remarks apply.

	In particular do note that $\C(A_+,n)$ could easily be generalised
	to $n$ different indexing categories, but since the resulting
    spectrum is defined by inserting $\S^1$ for each $A_+$, I chose
    to reduce to the case with equal inputs. }}

As indicated at the end of the last section I prove the equivalence
of $\C(A_+,n)$ to the appropriate product bicategory by displaying
the inductive step in constructing the equivalence. For this let me
emphasise that for arbitrary bicategories (possibly enriched) we have
the following simple case of the exponential law, for $A,B$ sets 
considered as discrete categories, hence bicategories:
\[Fun(A,Fun(B,\C))\cong Fun(A\times B, \C)\cong \C^{A\times B}.\]
In particular I can reduce the index juggling quite a bit by proving
this form of the following theorem.
{\thm{For $(\C,+)$ a permutative bicategory we have the following 
	natural equivalence of bicategories:\[\C(A_+,n)\simeq 
	Set(A,\C(A_+,n-1)).\]Inductively we find the natural equivalence:
	\[\C(A_+,n)\simeq \C^{A^{\times n}}.\]\begin{proof}
	The functor $\C(A_+,n)\rightarrow Set(A,\C(A_+,n-1))$ is -- as
    in the case of permutative $1$-categories -- given by restricting
    along $(\Aix)^{n-1}\times A^\delta \rightarrow (\Aix)^n$ with one
    component the inclusion of $A$ as the full discrete subcategory
    of characteristic functions $\chi_a$ in $\Aix$.

    The same reasoning as for $\C(A_+,1)$ before yields for each
    functor in $\C(A_+,n)$ a strict representative, which in this
    case means strict with respect to one of the $\Aix$-factors, for
    instance the last one as described above. So the restriction 
    has an inverse functor given by extending an $A$-tuple of 
    functors in $\C(A_+,n-1),$ by sending the maps in the last 
    factor to the appropriate discrete components in $\C^+$, and 
    thus summing up the functors according to the chosen bijections.
    \end{proof}}}

Furthermore we have the following generalisations of the analogous
results in \ref{pcatneu}.
{\prop{For each $n\in\mathbb{N}$ we have a strictly natural strict 
    $\Sigma_n$-action on $\C(A_+,n),$ given by permuting the inputs 
    and pushing forward with the induced symmetry $\chi^\wedge$ of 
    $\Fi$ in $\C^+$.}}
{\prop{For each pointed finite set $A_+$ we have natural strict extension
    functors \[A_+\wedge \C(A_+,n)\rightarrow \C(A_+,1+n),\] which are
    $\Sigma_1\times\Sigma_n$-equivariant.}}

Both proofs essentially proceed as the case for $1$-categories, where
the strictness of the functors is a consequence of the fact that they
only use the discrete components in $\C^+$, which are part of the
included $2$-category on all objects but just discrete morphisms.

{\rem{The extension functors as well as the $\Sigma_n$-action of the
propositions above strictly respect the basepoint functors $O$. For the
extension this is obvious, since we extend functors by zeroes. For the
symmetric action observe that in particular the assignment on objects
gives constant tuples, which are hence invariant under permutations.}}

The delooping construction $\C(A_+,n)$ is directly comparable to
Osorno's delooping \cite{Os} by restricting the source of the lifting
functors. {\prop{For two finite pointed sets $A_+,B_+$ we have a functor
	\[\Aix\times\Bix\rightarrow (A_+\wedge B_+)\downarrow \Fi,\]which by
	the canonical identification $A_+\wedge B_+\cong (A\times B)_+$ is 
	an indexing category for the construction $\C(\_,1)$.\begin{proof}
	By choosing an identification $k_+\wedge l_+\cong kl_+$, i.e., a
	total ordering on binary products, for instance the lexicographic
	order, we get a smash product functor on $\Fi$. Thus we map a pair
	of morphisms $f\colon k_+\rightarrow l_+$, $g\colon m_+\rightarrow 
	n_+$ to $f\wedge g \colon km_+\rightarrow ln_+$.
	
	Analogously on objects, for a pair $p\colon A_+\rightarrow k_+$ and
	$q\colon B_+\rightarrow l_+$, we can consider their product
	$p\times q\colon A_+\times B_+\rightarrow k_+\times l_+$ composed
	with the canonical projection $k_+\times l_+\rightarrow kl_+$ to
	the smash product. This factors over $(A\times B)_+= A_+\wedge B_+$.
	
	Thus the above assignments define a functor, since the smash
	product on $\Fi$ is a functor.\end{proof}}}

{\cor{For any finite pointed set $A_+$ we have a natural functor
	\[(\Aix)^{\times n}\rightarrow {A_+}^{\wedge n}\downarrow \Fi.\]}}

{\prop{Restricting a functor $F\in\C((A_+)^{\wedge n},1)$ along the 
	functor	$S\colon (\Aix)^{\times n}\rightarrow {A_+}^{\wedge n}
	\downarrow \Fi$	gives an element of $\C(A_+,n)$. \begin{proof}
	Reconsider the definition of $T_n$ as in \ref{T-n}: For 
	$T\colon \Aix\rightarrow \Fi$ the forgetful functor assigning to
	each object $p\colon A_+\rightarrow n_+$ its target $n_+$ and to
	each commutative triangle $f\colon p\rightarrow fp$ the map $f$ we
	set $T_n$ to be \[(\Aix)^n\rightarrow \Fi^n\rightarrow \Fi\]
	with first map $(T)^n$ and second map the $n$-fold smash product.
	
	This factors as: \[\xymatrix{(\Aix)^n \ar[r]^-{(T)^n}\ar[d]_{S}&\Fi^n
	\ar[d]^{\wedge}\\(A_+)^{\wedge n}\downarrow\Fi \ar[r]^-{T} & \Fi.}\]
	
	Thus a functor $F\colon {A_+}^{\wedge n}\downarrow \Fi \rightarrow 
	\C^+$ satisfying $UF = T$ trivially satisfies $UFS = TS$, which by
	the commutative square above gives $UFS = TS = T_n$, so $FS$ is
	an element of $\C(A_+,n)$.\end{proof}}}

Finally to tie the delooping categories in with the delooping 
constructed in \cite{Os} we need the following equivalence:
{\thm{\label{AeqzuOs} Smashing the source category 
    $(\Aix)^n\rightarrow (A_+)^{\wedge n}\downarrow \Fi$ induces a 
    natural strict restriction functor \[S^*\colon \C((A^{\times n})_+,1)
    \rightarrow \C(A_+,n),\] which is a natural equivalence of 
    bicategories. \begin{proof} Recall the statement and proof
    of corollary \ref{dell}: Specifically for the delooping bicategories
    $\C((A^{\times n})_+,1)$ and $\C(A_+,n)$ we find that the natural
    inclusions ${(A^{\times n})_+}^\delta\rightarrow (\Aix)^n$ and
    ${(A^{\times n})_+}^\delta \rightarrow (A_+)^{\wedge n}\downarrow 
    \Fi$ -- each identifying ${(A^{\times n})_+}^\delta$ as a discrete
    subcategory of the respective indexing categories -- fit into a
    commutative triangle:
    \[\xymatrix{
    &{(A^{\times n})_+}^\delta \ar[dl]\ar[dr]&\\
    (\Aix)^n \ar[rr]^S && (A^{\times n})_+\downarrow \Fi.}\]
    Since restriction along the diagonal arrows of this triangle each
    give equivalences of bicategories by \ref{dell}, we find that
    $S^*$ also is an equivalence of bicategories.\end{proof}}}

\section{The Multiplicative Structure on $\C(A_+,n)$}
Let me reiterate that I only consider multiplicative structures as
induced on matrices for a bipermutative coefficient
category $\R$. Hence I restrict to the $E_\infty$-case and by 
choosing the Barratt-Eccles-operad as in 
\cite[p. 16, Theorem 3.7]{EM} I can avoid constructing a 
multi(bi)category-structure for permutative bicategories.

To understand how a bipermutative structure induces a multiplication
on the delooping bicategories $\C(A_+,n)$ I have to fix a 
multiplication on $\C^+$ and compatible target functors
$T_n$. The induced multiplication on $\C^+$
is a direct generalisation from the case of $1$-categories, so I repeat 
the proof to keep track of strictnesses and genuine $2$-cells.

{\thm{\label{bipzumult}
    Consider a bipermutative bicategory (see \ref{bipbic}) 
    $(\C,+,\cdot)$. Fix a smash product on $\Fi$, then we have a 
    symmetric monoidal structure $\boxtimes$ on $\C^+$ making the 
    forgetful functor strictly symmetric monoidal \[U\colon
    (\C^+,\boxtimes)\rightarrow (\Fi,\wedge).\] \begin{proof} 
    Do the same on objects as in \ref{pcatneu}: \[(c_1,\ldots,
    c_n)\boxtimes(d_1,\ldots,d_m)=(c_id_j)_{\omega(i,j)},\] where the 
    multiplication on the right is the multiplicative structure
    of the bipermutative bicategory $(\C,+,\cdot)$. Again the subtlety 
    is the definition on morphisms. For this consider first the following
    objects in $\C^+$: \[(f\times g)_*(c\boxtimes \bar c)_{\omega(i,j)}
    = \sum_{\omega(k,l)\in (f\times g)^{-1}(\omega(i,j))}c_k\bar c_l,\]
    and analogously:\[(f_*c\boxtimes g_*\bar c)_{\omega(i,j)}
    = (f_*c)_i\cdot (g_*\bar c)_j = \left(\sum_{k\in f^{-1}i}c_k\right)
    \left(\sum_{l\in g^{-1}j}\bar c_l\right).\]
    By \ref{bipbic} we find a unique structural map $D^{f,g}$ comprised
    of isomorphism $1$-cells (given for instance here by first all left 
    reductions, then all right reductions) \[(f\times g)_*(c\boxtimes
    \bar c)=\sum_{k,l}c_k\bar c_l \rightarrow \sum_k \left(c_k \left(\sum_l 
    \bar c_l\right)\right)\]
    \[~~~~~~~~\rightarrow\left(\sum_{k\in f^{-1}i}c_k\right)
    \left(\sum_{l\in g^{-1}j}\bar c_l\right)=f_*c\boxtimes g_*\bar c,\]
    which is given by composites of distributors, and uniquely 
    determined by the summations of $f$ and $g$. Hence for two maps in 
    $\C^+$: \[(f,(a_1,\ldots,a_{m_1}))\colon c=(c_1,\ldots,c_{n_1})
    \rightarrow d=(d_1,\ldots,d_{m_1}),\]
    \[(g,(b_1,\ldots,b_{m_2}))\colon \bar c = (\bar c_1,\ldots,
    \bar c_{n_2}) \rightarrow \bar d = (\bar d_1,\ldots,\bar d_{m_2}),\]
    we set their product to be the following composite:
    \[\xymatrix{c\boxtimes \bar c =(c_i\bar c_j)\ar[r]^-{(f\times g)_*} 
    & \left(\sum c_k\bar c_l\right) \ar[r]^-{D^{f,g}}
    & \left(\sum c_k\right)\left(\sum \bar c_l\right)
    \ar[r]^-{a_i\cdot b_j} & (d_i\cdot \bar d_j). }\]

    The map $\boxtimes$ respects identity $1$-cells strictly, because 
    $\cdot$ was assumed to be normal \ref{bipbic}. Since $f,g$ are part of 
    the data of morphisms in $\C^+$, the structural morphism 
    $(f\times g)_*$ is strictly natural. Since the distributors in 
    \ref{bipbic} are strict natural transformations, $D^{f,g}$ 
    strictly commutes with genuine morphisms of $\C$ as well. Thus 
    only the appropriate products of compositor $2$-cells for $\cdot$ 
    yield the compositor for $\boxtimes$ with no additional $2$-cells 
    introduced by either $(f\times g)_*$ or $D^{f,g}$. So we have a 
    strong normal functor \[\boxtimes\colon \C^+\times\C^+\rightarrow 
    \C^+.\]

    Since $1_+$ is a strict unit object for $\wedge$ on $\Fi$, the
    $1$-tuple $(1)\in \C^+$ with entry the multiplicative unit of $\C$
    yields a strict unit object in $\C^+$. 

    The functor $\boxtimes$ is strictly associative: visibly on objects
    precisely because the bijections for the smash-product in $\Fi$
    are chosen that way, and because the multiplication on $\C$ was
    assumed strictly associative. Because of the strict identity of
    functors for triple products that $\cdot$ on $\C$ satisfies by
    assumption, we get strict associativity for $\boxtimes$ as a 
    strict functor identity on $\C^+$.

    Finally the multiplicative symmetry transformation is given as
    follows. Let the symmetry in $\Fi$ with respect to $\wedge$ be 
    $\chi$, then the $1$-cell for the symmetry of $\boxtimes$ is the 
    composite: \[\xymatrix{c\boxtimes d=(c_id_j)_{\omega(i,j)}
    \ar[r]^-{\chi}&(c_id_j)_{\omega(j,i)} \ar[r]^-{c^\cdot_1}
    &(d_jc_i)_{\omega(j,i)}=d\boxtimes c.}\] Since the symmetry $\chi$ 
    introduces no $2$-cell, the $2$-cell for the symmetry of $\boxtimes$ 
    is thus given as the appropriate product of the symmetry $2$-cells 
    of $\cdot$ in $\C$.

    The symmetry squares to the identity strictly, since the symmetries
    of $(\Fi,\wedge)$ and $(\C,\cdot)$ do. It satisfies the two
    diagrams for triple products strictly for the same reason.

    For the final claim we only need to observe that the discrete 
    components of the functor $\boxtimes$ and its symmetry $c^\boxtimes$
    are modelled just so that the forgetful functor $U\colon \C^+
    \rightarrow \Fi$ is strictly symmetric monoidal. \end{proof}}}

{\rem{Since I have established that a bipermutative structure gives
    a functor over the smash-product functor $\wedge\colon \Fi
    \times\Fi\rightarrow\Fi$ I strongly conjecture that this could
    be used to make bimonoidal and bipermutative categories much more
    explicit in the context of $\infty$-categories. Compare this for
    instance to (p. 149) Definition 2.1.3.7 in \cite{Lu2} and more 
    directly to (p. 136) Definition 2.0.0.7, where Lurie defines a
    symmetric monoidal $\infty$-category just so that by adding in
    all morphisms of $\Fi$ into $\C^+$ the map $U\colon \C^+
    \rightarrow \Fi$ exhibits its nerve $N\C^+$ as a symmetric
    monoidal $\infty$-category, and by 2.1.3.7 $\boxtimes$ as a
    symmetric monoidal functor.}}

The multiplication on $\C^+$ induces a pairing of the delooping bicategories.
{\thm{\label{pairing}Given a bipermutative bicategory $(\C,+,\cdot)$ 
    the delooping bicategories $\C(A_+,n)$ have a pairing pseudofunctor: 
    \[\mu_{n,m} \colon\C(A_+,n)\times\C(A_+,m)\rightarrow \C(A_+,n+m),\] 
    which is strictly natural in $A_+$, and 
    $\Sigma_n\times\Sigma_m$-equivariant. 
        
    Furthermore the pairing strictly satisfies $\mu(O_n,\_)=\mu(\_,O_m)
    = O_{n+m}$, i.e., pairing with the zero-functor yields the constant
    map to the zero-functor $O_{n+m}\in\C(A_+,n+m)$.
    \begin{proof}
    By the propositions before we know that we can pair two lifting
    functors $F_n\colon (\Aix)^n\rightarrow \C^+$ and $G_m\colon
    (\Aix)^m\rightarrow \C^+$ by the symmetric monoidal structure
    on $\C^+$ to give $\boxtimes_*(F_n,G_m)\colon (\Aix)^{n+m}
    \rightarrow \C^+\times \C^+\rightarrow \C^+$, which is evidently
    compatible with the $\Sigma_n$-operation on $F_n$ and the $\Sigma_m$-
    operation on $G_m$ independently, thus with $\Sigma_n\times\Sigma_m$
    as a whole.
	Furthermore evidently $\boxtimes_*(O_n,\_)=\boxtimes_*(\_,O_m)=O_{n+m}$,
	since the zero-functor acts as a strict zero for $\boxtimes$, which
	is induced by $\cdot$ on the bipermutative bicategory.

    Since the symmetric monoidal structure of $\C^+$ is defined over the
    forgetful functor $U\colon \C^+\rightarrow \Fi$ such that it becomes
    strictly symmetric monoidal, the resulting functor lifts the map
    \[\xymatrix{(\Aix)^n\times(\Aix)^m\ar[r]^-{T_n\times T_m} &\Fi
    \times\Fi\ar[r]^\wedge & \Fi,}\] which by \ref{T-n} is the same 
    as $T_{n+m}$. Hence the resulting functor is in $\C(A_+,n+m)$.

    The same description applies to $1$- and $2$-cells, since I did not
    need to refer to $1$-equivalences to define the symmetric monoidal
    structure on $\C^+$. Hence we have a strict symmetric monoidal 
    inclusion $(\C^{eq})^+\rightarrow \C^+$, and can extend the product
    to $1$- and $2$-cells. 
    By applying the compositor of $\cdot$ appropriately componentwise
    we get the compositor $2$-cell for $\mu$.

    Strict naturality in the pointed set is a consequence of the fact 
    that a map of pointed finite sets $f\colon A_+\rightarrow B_+$ 
    induces a strict normal \ref{strictn} functor $\C(A_+,n)\rightarrow
    \C(B_+,n)$ by pulling the category $\Bix$ back along $f$ to $\Aix$ 
    and then pulling back functors along this pullback. In particular 
    it is restriction of the source category, hence the 
    multiplication is strictly natural in $A_+$.\end{proof}}}

{\rem{Do note that by convention $\C(A_+,0)=\C$, so for $(\C,+,\cdot)$
	bipermutative we trivially have	a map $\eta_0\colon 
	*\rightarrow \C(A_+,0)$ sending the object to $1$ and its identity.
	
	Furthermore by the extension $A_+\times \C(A_+,0)\rightarrow
	\C(A_+,1)$ we get a map $\eta_1\colon A_+ \cong A_+\times \{1\}
	\rightarrow \C(A_+,1)$, which hence sends a pair $(a,1)$ to the
	functor, which is the tuple $(1)$ at 
	$\rho^a\colon A_+\rightarrow 1_+$ with zeroes
	adequately added everywhere else.
	
	In particular for $(\C,+,\cdot)$ we can rewrite the extension maps
	$A_+\times \C(A_+,n)\rightarrow \C(A_+,1+n)$ as the multiplication 
	with $\eta_1$: \[A_+\times \C(A_+,n) \rightarrow \C(A_+,1)\times
	\C(A_+,n) \rightarrow \C(A_+,1+n).\]}}

The pairing inherits strict associativity from the strict associativity
of $(\C,\cdot)$.
{\prop{For $(\C,+,\cdot)$ a bipermutative bicategory the 
	pairing of \ref{pairing} is strictly associative, i.e.,
	\[\xymatrix{\C(A_+,l)\times \C(A_+,m)\times \C(A_+,n) \ar[r] \ar[d] 
		& \C(A_+,l)\times\C(A_+,m+n)\ar[d]\\ \C(A_+,l+m)\times\C(A_+,n)
		\ar[r] & \C(A_+,l+m+n)	}\]
	is strictly commutative for each pointed finite set $A_+$ and natural
	numbers $l,m,n$. \begin{proof} This follows from the strict
	associativity of the chosen bijections $\omega\colon n\times m 
	\rightarrow nm$ making the monoidal structure on $\C^+$ with 
	$((c_i),(d_j))\mapsto (c_i\cdot d_j)_{\omega(i,j)}$ strictly 
	associative, because $\cdot$ is strictly associative.\end{proof}}}

By rewriting the extension maps as above we get the following corollary:
{\cor{The pairings commute with extension maps strictly, i.e., there
is a unique pairing of the form
\[\xymatrix{\C(A_+,n)\wedge A_+ \wedge \C(A_+,m) \ar[r] & \C(A_+,n+1+m).}\]}}
	
{\defn{For $\sigma\in\Sigma_n$ define the map:
\[\mu_\sigma\colon \C(A_+,k_1)\times\ldots\times\C(A_+,k_n)
\rightarrow \C(A_+,\sum_i k_{\sigma^{-1}(i)})\]
as the composite of the symmetry in $(Bicat,\times)$
\[c^\times_{\sigma}\colon \prod_i \C(A_+,k_i)\rightarrow 
\prod_i \C(A_+,k_{\sigma^{-1}(i)})\]
followed by the $n$-fold pairing (uniquely determined by the proposition
above):\[\prod_i \C(A_+,k_{\sigma^{-1}(i)})\rightarrow
\C(A_+,\sum_ik_{\sigma^{-1}(i)}).\]}}

{\rem{By definition of $\mu_\bullet$ we have a functor
\[\Sigma_n\times\left(\coprod_{k_1+\ldots+k_n=N}\C(A_+,k_1)\times
\ldots \times \C(A_+,k_n)\right)\rightarrow \C(A_+,N)\]
with $\Sigma_n$ considered as a discrete category, which factors as
\[\Sigma_n\times_{\Sigma_n}\left(\coprod_{k_1+\ldots+k_n=N}\C(A_+,k_1)
\times\ldots\times \C(A_+,k_n)\right)\rightarrow \C(A_+,N).\]

Furthermore since we established that binary multiplication becomes the
constant map to the zero-functor, if one parameter is the zero-functor,
we find that each $\mu_\bullet$ becomes the constant map if one
parameter is restricted to the zero-functor.}}

Finally I want to state the $E_\infty$-commutativity in its binary form
for clarity before summarising the $E_\infty$-structure in \ref{mymult}.

{\prop{\label{minieinf} For a bipermutative bicategory $(\C,+,\cdot)$, a 
finite pointed set $A_+$ and two natural numbers $n,m$ the two pairings
$\mu_{\id},\mu_{(12)}\colon \C(A_+,n)\wedge \C(A_+,m)\rightarrow
\C(A_+,n+m)$ are pseudonaturally isomorphic.

Furthermore the pseudonatural isomorphisms inherent the coherence of the
$\cdot$-symmetry in that for each two $\sigma,\tau\in\Sigma_N$ there
is a unique composite pseudonatural isomorphism
$\mu_\sigma \Rightarrow \mu_\tau$ of pairings
\[\C(A_+,n_1)\wedge\ldots\wedge\C(A_+,n_N)\rightarrow \C(A_+,\sum_i n_i).\]
\begin{proof} The $1$-cells of the pseudonatural transformation consist 
of $(\chi_{n,m},c^1_\cdot)$ with $\chi_{n,m}$ the block permutation 
shifting the first $n$ elements of $n+m$ past the last $m$ elements, and
$c^1_\cdot$ the $1$-cell of the pseudonatural symmetry for $(\C,\cdot)$.

In particular by \ref{mdlbic} we already see that each $1$-cell is a 
strict isomorphism, which squares to the identity, thus we only need 
pseudonaturality. The pseudonaturality $2$-cell is given accordingly
by $(\id, c^2_\cdot)$.

Furthermore we see immediately that the coherence of $c_\cdot$ promotes
to the coherence claimed above.\end{proof}}}

{\Remi{For $1$-categories $\C,\D$ that natural 
transformations $\eta\colon F\Rightarrow G$ are in a natural one-to-one 
correspondence with functors $H\colon \C\times I\rightarrow \D$.

Specifically, the functors $F,G$ are restrictions of $H$ to the
objects $0,1\in I$ respectively, while the components
of the natural transformation $\eta$ are the arrows $H(c,0\rightarrow 1)
=\eta_c$. Naturality of $\eta$ is then equivalent to $H$ being a functor,
because $\C\times I$ is a product-category.}}

{\rem{For $\C,\D$ bicategories the above correspondence generalises to
pseudonatural transformations, which are in one-to-one correspondence
with pseudofunctors. However, since for pseudofunctors the compositor 
$2$-cells fill triangles, while the $2$-cell involved in the pseudonaturality
condition fills a square, we actually get (at least) two correspondences
by fixing one or the other triangle in the diagramme \label{quadtotrig}
\[\xymatrix{H(c,0) \ar[r]\ar[dr]\ar[d] & H(d,0)\ar[d]\\
H(c,1) \ar[r] & H(d,1)}\] to be filled with the compositor $2$-cell.

The same correspondence establishes that a pseudonatural
transformation is strictly natural, i.e., has only identity $2$-cells,
if and only if its associated pseudofunctor is a strict functor.}}

To introduce the specific $E_\infty$-coherences for the pairing on 
the delooping bicategories, recall the Barratt-Eccles operad in 
$1$-categories (cf. \cite[p. 15]{EM}).
{\defn{For any discrete set $M$ define its translation category $EM$ as 
follows: Its objects are the elements of $M$, its arrow set is 
$M\times M$, where $(s,t)\colon s\rightarrow t$ with composition
$(t,u)\circ (s,t) = (s,u)$ and identities $(u,u)$ for an object 
$u\in M$.
By definition each object is initial and terminal, hence the classifying
space of $EM$ is contractible for any $M$.

Moreover for $G$ a group we have a canonical action 
$EG\times G\rightarrow EG$ by the assignment $(s,t).g := (sg,tg)$.}}

We can by coherence of the pseudonatural isomorphisms in \ref{minieinf}
extend the map \[\Sigma_n\times_{\Sigma_n}\left(\coprod_{k_1+\ldots+k_n=N}
\C(A_+,k_1)\times \ldots\times \C(A_+,k_n)\right)\rightarrow \C(A_+,N)\]
over $E\Sigma_n$ as follows:
{\cor{The $n$-fold pairing
\[\mu\colon \Sigma_n\times_{\Sigma_n}\left(\coprod_{k_1+\ldots+k_n=N}
\C(A_+,k_1)\times \ldots \times \C(A_+,k_n)\right)
\rightarrow \C(A_+,N)\]
extends to a pairing
\[\mu^E\colon E\Sigma_n\times_{\Sigma_n}\left(\coprod\C(A_+,k_1)\times \ldots
\times \C(A_+,k_n)\right)\rightarrow \C(A_+,N).\]
\begin{proof}
Since we want to extend the assignment $\mu$, we can define $\mu^E$ at
each object of $E\Sigma_n$ by $\mu$. Locally, i.e., for each
arrow $(s,t)\in E\Sigma_n$, we set $\mu^E(\_,(s,t))$ to be the canonical
pseudonatural $1$-cells for $\mu_s\Rightarrow \mu_t$ as established
in \ref{minieinf}.

By analogy with \ref{quadtotrig} fill in the upper right triangle
with the pseudonaturality $2$-cell for the canonical pseudonatural
isomorphism $\mu_s\Rightarrow \mu_t$.

This assignment defines a normal functor (i.e., one pointed at 
identity $1$-cells) because $c^1_\cdot$ strictly squares to the identity
by \ref{mdlbic}. The compositor $2$-cells are coherent, because the
two diagrams:
\[\begin{array}{c}
\xymatrix{\mu^E(s,F_1,\ldots,F_n) \ar[r]\ar[d]\ar[dr] 
& \mu^E(s,G_1,\ldots,G_n) \ar[d]\\
\mu^E(t,F_1,\ldots,F_n) \ar[r]\ar[d]\ar[dr]&
\mu^E(t,G_1,\ldots,G_n) \ar[d]\\
\mu^E(u,F_1,\ldots,F_n) \ar[r]&
\mu^E(u,G_1,\ldots,G_n) }\\\\
\xymatrix{\mu^E(s,F_1,\ldots,F_n) \ar[r]\ar[dd]\ar[ddr] 
& \mu^E(s,G_1,\ldots,G_n) \ar[dd]\\\\
\mu^E(u,F_1,\ldots,F_n) \ar[r]&
\mu^E(u,G_1,\ldots,G_n) }\end{array}\]
with each upper right triangle filled by the pseudonaturality $2$-cells
express that there is a unique $\cdot$-symmetry from an $s$-permuted
input to an $u$-permuted input. In particular, the composite twist 
factored over a $t$-permuted input produces the same multiplicative
twist. Do note that all the other triangles are filled with identities,
including the ones expressing the equalities 
$E((t,u),\_)\boxempty E((s,t),\_)=E((s,u),\_)$, so that the above prism
with base a triangle degenerates to just three (potentially) 
non-trivial $2$-cells.\end{proof}}}

{\Remi{\label{esigmaop}Since the $\Sigma_*$-module $(E\Sigma_n)_n$ in 
fact is an operad, we have
a multiassociative, $\Sigma_*$-equivariant, and unital multicomposition:
\[E\Sigma_N\times E\Sigma_{k_1}\times \ldots\times E\Sigma_{k_N}
\rightarrow E\Sigma_{\sum_i k_i}.\]
I refer to this as \emph{block sum composition} as it is given by 
application of the functor $E$ to the multicomposition 
\[\Sigma_N\times \Sigma_{k_1}\times \ldots\times \Sigma_{k_N}
\rightarrow \Sigma_{\sum_i k_i},\]
which can be described as 
\[(\sigma, \tau_1,\ldots,\tau_N)
\mapsto \sigma\langle k_1,\ldots, k_N\rangle \circ (\tau_1\boxplus
\ldots\boxplus\tau_N)\]
for $\sigma\langle k_1,\ldots,k_N\rangle$ the permutation that permutes
the $N$ blocks of length $k_i$ by exchanging the blocks according to
$\sigma$, and $\boxplus$ a disjoint union functor on finite sets
as in \ref{Fin}.}}

{\prop{The $E\Sigma_*$-extensions of the pairings of 
delooping bicategories for a bipermutative bicategory $(\C,+,\cdot)$ of 
the above corollary make the following multiassociativity-diagram commute
\[\xymatrix{E\Sigma_N\times \left(\prod_i E\Sigma_{k_i}\right)
\times \left(\prod_j \C(A_+,l^i_j)\right)\ar[r]^-\cong\ar[d]
&E\Sigma_N\times \prod_i \left(E\Sigma_{k_i}\times \prod_j \C(A_+,l^i_j)
\right)\ar[d]\\
E\Sigma_{\sum_i k_i} \times \prod_i\prod_j \C(A_+,l^i_j)\ar[dr] & 
E\Sigma_N \times \prod_i \C(A_+,\sum_j l^i_j) \ar[d]\\
& \C(A_+,\sum_{i,j}l^i_j)}\]
for all natural numbers $N, k_i, l^i_j$.
\begin{proof} By the specific structure of $E\Sigma_*$ and the definition 
of the pairings $\mu_\sigma$ by equivariance, we can reduce to the case, 
where each object in $E\Sigma_*$ is the identity, which is just 
the strict associativity of the pairings. Since morphisms in the
$E\Sigma_*$ are uniquely determined by their source and target, and the
pseudonatural isomorphisms of \ref{minieinf} are coherent, this extends
to the morphisms as well.\end{proof}}}

I want to again suppress the operadic context for the $E_\infty$-structure 
and instead display what comprises the algebra structure of 
$\C(A_+,\bullet)$ over $E\Sigma_*$, including its coherences.

{\thm{\label{mymult} Given a bipermutative bicategory $(\C,+,\cdot)$ the resulting 
    pairing of delooping categories from the theorem above:
    \[\mu_{n,m}\colon \C(A_+,n)\times \C(A_+,m)\rightarrow \C(A_+,n+m)\]
    is $E_\infty$ in the following sense (cf. the definition of a commutative
    symmetric ring spectrum as in \cite[p. 9]{SchSym}, as well as
    \cite[pp. 66-68]{MayEinf}):
    \begin{itemize}
    \item It is strictly associative, i.e., we get a well-defined triple
    product for every $n,m,l\in\mathbb{N}$ as a strict identity of 
    strong normal functors: \[\mu_{n,m+l}\circ(\id\times\mu_{m,l})
    =\mu_{n+m,l} \circ(\mu_{n,m}\times \id).\]
    \item The functor $\{*\}\rightarrow 1\in\C\subset\C^+$ considered
    as an element of $\C(A_+,0)$ is a strict unit, turning $\mu_{0,n}
    =\mu_{n,0}=\id_{\C(A_+,n)}$ into a strict identity of functors.
    \item We have a natural central map $\iota_1\colon
    A_+\rightarrow \C(A_+,1)$ given by
    \[\iota_1(a)(p\colon A_+\rightarrow k_+) = \begin{cases}
    1, ~~& ~~ \mathrm{if~~} a\notin p^{-1}+,\\
    0, ~~& ~~ \mathrm{if~~} a\in p^{-1}+,\end{cases}\]
    with structural maps being given either by identities $0+0=0$ or
    $1+0=0+1=1$, hence strict identities. Centrality means that we
    have a strict equality of the functors
    \[A_+\times \C(A_+,n)\rightarrow \C(A_+,1)\times\C(A_+,n)
    \rightarrow \C(A_+,1+n)\]
    and
    \[A_+\times \C(A_+,n)\rightarrow \C(A_+,n)\times A_+
    \rightarrow \C(A_+,n)\times \C(A_+,1)\]
    \[\phantom{blaaaaaaaaaa}\rightarrow \C(A_+,n+1)
    \rightarrow \C(A_+,1+n),\]
    where the final arrow here is the action by $\chi_{1,n}$ which 
    shuffles the last input coordinate to first place without changing
    the order of the other inputs, and pushing forward the functor by
    $\chi$ so that it is a lift \ref{symmSpekt!!}. Do note that
    these maps describe the extension functors $A_+\wedge
    \C(A_+,n)\rightarrow \C(A_+,1+n).$
    \item For each two multiplications of $n$ inputs associated to
    two permutations $\sigma,\tau\in\Sigma_n$ there is a pseudonatural
    isomorphism:
    \[\xymatrix{\C(A_+,m_1)\times\ldots\times\C(A_+,m_n)
    \rrtwocell<7>^{\mu_\sigma}_{\mu_\tau}{~~C_{\overrightarrow{\sigma\tau}}} 
    && \C(A_+,m_1+\ldots+m_n). }\]
    \item The isomorphisms are coherent in that they compose 
    vertically as in $E\Sigma_n$: 
    $C_{\overrightarrow{\sigma\tau}}C_{\overrightarrow{\rho\sigma}}
    = C_{\overrightarrow{\rho\tau}}.$
    \item The isomorphisms have a block sum composition \ref{esigmaop}, 
    which is multiassociative and $\Sigma_*$-equivariant.
    \end{itemize} \begin{proof}
	All the claims are just summaries of the propositions above.
	Note that the coherence of the isomorphisms $C_\_$ follows from the
	coherence of the compositor $2$-cells for $\mu^E$, which itself
	follows from the coherence of the $\cdot$-symmetry for triple 
	products in $(\C,\cdot)$. \end{proof}}}

\section{The Symmetric Spectrum from the Delooping}
The passage from the delooping categories $\C(A_+,n)$ to the 
associated spectrum is fortunately very straight-forward. I fix a
pointed simplicial $\S^1$:

{\defn{\label{pointedS1} Consider the simplicial set 
	$\Delta_1=\Delta(\_,[1])$ and its simplicial subset of constant maps 
	$\partial\Delta_1$, then $\S^1:=\Delta_1/\partial\Delta_1$ is
	a pointed simplicial set.}}

Since the delooping bicategories $\C(A_+,n)$ are strict normal \ref{strictn} functors
in pointed finite sets we get the following:
{\thm{\label{insertS1}The delooping bicategories $\C(A_+,n)$ are strictly simplicial 
    bicategories by insertion of a simplicial set. In particular 
    $\C(\S^1,n)$ is a simplicial bicategory with $\Sigma_n$-action.

    By naturality of the extension functors and understanding $\S^1$
    as a discrete simplicial bicategory we get suspension maps as 
    strict functors of bicategories: \[\S^1\wedge\C(\S^1,n)\rightarrow
    \C(\S^1,1+n),\] which is $\Sigma_n$-equivariant, and assembles to 
    $\Sigma_m\times\Sigma_n$-equivariant maps \[\S^m\wedge\C(\S^1,n)
    \rightarrow \C(\S^1,m+n).\]

    For a bipermutative bicategory $\C$ the pairings from above assemble 
    into strictly associative pairings of simplicial bicategories:
    \[\C(\S^1,n)\wedge\C(\S^1,m)\rightarrow \C(\S^1,n+m),\] with two 
    strictly central unit maps \[\iota_0\colon \{*\}\rightarrow 
    \C(\S^1,0),~~~\iota_1\colon\S^1\rightarrow \C(\S^1,1), \]
    and each two permutations of higher multiplications connected by
    coherent simplicial pseudonatural isomorphisms.

    In summary: Set $HC_n=|N\C(\S^1,n)|,$ then the Eilenberg-Mac Lane
    spectrum $H$ to a permutative bicategory $\C$ inherits a natural 
    $E_\infty$-ring spectrum structure from a bipermutative structure
    on $\C$.}}

By applying the nerve and geometric realisation we get:
{\thm{\label{meinEinf}A bipermutative bicategory $\C$ yields an 
    $E_\infty$-symmetric ring spectrum, which is level-equivalent to 
    the spectrum defined in \cite{Os}. In particular it is 
    semi-stable, because it is equivalent to the symmetric spectrum of 
    a $\Gamma$-space. \begin{proof} By \ref{AeqzuOs} we know that 
    restriction of lifting functors along smashing $(\Aix)^n
    \rightarrow (A^n)_+\downarrow\Fi$ gives a natural equivalence 
    $\C((A_+)^{\wedge n},1)\rightarrow \C(A_+,n),$ which by inserting 
    $\S^1$ yields a level-equivalence: \[\C((\S^1)^{\wedge n},1)=
    \C(\S^n,1)\rightarrow \C(\S^1,n),\] so in particular on 
    realisation of nerves we get a map of symmetric spectra, 
    which is a level-equivalence. \end{proof}}}

\section{The Induced Involution}
We have already seen that an involution on bipermutative 
coefficients $\R$ induces a strictly additive functor on the
bicategories of matrices $\M\rightarrow \MM(\R^\mu)$. By transposition
we can remove the $\mu$-opposition, still strictly additively, but
at the expense of opposing $1$-cells and the tensor-product. 
Finally the comparison $B\C\cong B\C^{op_1}$ is what we need to 
study with respect to additivity and multiplicativity. 

{\lem{Given a bicategory $\C$ and its $1$-opposition, 
    i.e., with respect to $1$-cells, the homeomorphism 
    $\Gamma$ of their nerves is strictly natural with respect
    to functors $F\colon \C\rightarrow \D$. So we have a 
    commutative diagram:
    \[\xymatrix{|N\C|\ar[r]^\Gamma\ar[d]_{|NF|} & |N\C^{op}|
    \ar[d]^{|NF|}\\|N\D|\ar[r]^\Gamma & |N\D^{op}|.}\]
    $\hfill\Box$}}

In particular the homeomorphism is compatible with the pairings
established above.

{\cor{The pairing of delooping bicategories commutes with opposition:
    \[\xymatrix{|N\C(A_+,n)|\times|N\C(A_+,m)|\ar[d]
    \ar[r]^-{\Gamma\times\Gamma}
    &|N\C^{op_1}(A_+,n)|\times|N\C^{op_1}(A_+,m)|\ar[d]\\
    |N(\C(A_+,n)\times\C(A_+,m))|\ar[d]\ar[r]^{\Gamma} 
    &|N(\C^{op_1}(A_+,n)\times\C^{op_1}(A_+,m))|\ar[d]\\
    |N(\C(A_+,n+m))|\ar[r]^\Gamma
    &|N(\C^{op_1}(A_+,n+m))|.}\]
    \begin{proof}
    The only thing left to emphasise is that the homeomorphism 
    $|X|\times|Y|\rightarrow |X\times Y|$ is natural as well (given
    the compactly generated topology on the product).
    \end{proof}}}

{\rem{A minor warning is in order about the notation $\C^{op_1}(A_+,n)$.
    Since $(\C^{op_1})^+\neq (\C^+)^{op_1}$ this is potentially 
    ambiguous, thus I intend to mean the delooping bicategory of
    $\C^{op_1}$. }}

Since the induced involution is defined in \cite{Ri2010} and analogously
in \ref{indinv} as the composite:
\[\xymatrix{B\M \ar[r]^{B\mathcal{M}(T)} & 
        B\mathcal{M}(\mathcal{R}^\mu) \ar[r]^{B(\cdot)^t}  
        & B\mathcal{M}(\R)^{op_1} \ar[r]^{\Gamma} & B\M,}\]
and we have already established that $\Gamma$ strictly commutes with
the pairings, we only have to establish the effect of coordinatewise
involution and subsequent transposition. Both functors strictly
commute with the direct sum of matrices, thus induce functors
on the delooping bicategories, but in \ref{transmultopp} we saw that
transposition fully reverses the monoidal structure given by tensor
product. So we have to consider the following situation. For emphasis
I suppress the commutativity of the multiplication and call it a 
bimonoidal bicategory.

{\prop{Given a bimonoidal bicategory $(\C,+,\cdot)$ and its 
    multiplicative opposition $(\C,+,\circ)$ the induced 
    monoidal structure on $\C^+$ by $\circ$ is strictly
    naturally isomorphic to the opposite monoidal structure on
    $\C^+$ induced by $\cdot$.
    \begin{proof}
    By retracing the construction in \ref{bipzumult} we find on 
    objects that $\circ$ induces 
    \[(c_1,\ldots,c_n)\circ_*(d_1,\ldots,d_m)
    =(c_i\circ d_j)_{\omega(i,j)}=(d_j\cdot c_i)_{\omega(i,j)}.\]
    Thus the isomorphism is given by using the smash
    symmetry $\chi$ on $\Fi$ to exchange the 
    indices: \[\xymatrix{
    (c_1,\ldots,c_n)\circ_*(d_1,\ldots,d_m)=(d_j\cdot c_i)_{\omega(i,j)}
    \ar[r]^-{(\chi_{n,m},\id)}&(d_j\cdot c_i)_{\omega(j,i)}=d\cdot_*c
    =c\cdot_*^{op}d.}\] \end{proof}
}}

In particular we find that the involution and subsequent transposition
strictly oppose the multiplicative structure on the delooping 
bicategories.

{\cor{\label{indinvSP}
    The induced involution is a functor $I\colon (\C,+,\cdot)
    \rightarrow (\C^{op_1},+,\cdot^{op})$, thus composition with
    $\Gamma$ induces a map of $E_\infty$-spectra:
    \[\Gamma\circ|NI|\colon(H\C,\mu)\rightarrow (H\C,\mu^{opp}).\]}}
