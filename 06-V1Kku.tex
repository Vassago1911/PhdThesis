\chapter{The Involution on $V(1)_*K(ku_p)$}\label{calc}
In this chapter we can finally use the results of all preceding
chapters to revisit the calculations of Christian Ausoni to analyse
them along the induced involutions - primarily the ones in \cite{AuTHH,AuKku}. 
For this I go through many details of the 
calculations and recall the ones, which I need for establishing
the involution on classes in homology or homotopy groups. However
I have written this chapter under the assumption that the reader
has the sources \cite{AuTHH,AuKku} close.
In particular the effect of the involution has clearer emphasis,
when I consider the hard calculations of \cite{AuTHH,AuKku} as given.

Inherent to the calculations of \cite{AuTHH,AuKku} is the restriction to
odd primes, for a partial picture at $p=2$ see \cite{AnHL} computing the
homotopy groups $\pi_*THH(\l)$ and $\pi_*THH(ko)$ locally at $2$.

\section{The Involutions on $\ell$ and $ku$}
\subsection*{Preliminaries}
The model provided by algebraic 
$K$-theory of an algebraic closure of a finite field $K(\barF_q)$,
which comes with a ring map given by the Brauer lift $K(\barF_q)
\rightarrow KU$, as well as the connective cover given by $H\MM_\CC
=ku\rightarrow KU$ yield the same $E_\infty$-structure, when 
completed at $p$ for which $q\in(\Z/p^2)^\times$ is a generator, by
\cite{BaRi08}. 
The homology theories in \cite{AuTHH,AuKku} are insensitive to 
$p$-completion, because the involved homology theories are 
$H\F_p$-local (in the sense of Bousfield-localisation), hence I
switch between the models for $ku$ whenever convenient for a 
clearer exposition. 

Do note that by the homotopy limit involved in the definition of 
topological cyclic homology and the fact that completion can for instance
be described as a homotopy colimit we cannot expect 
completion and cyclic homology to commute. 

On $K$-theory we can
trace the analogous failure back to the fact that a ring usually
has fewer units than its $p$-completion, thus analysing $BGL(R_p^{\wedge})$
and $BGL(R)_p^{\wedge}$ are usually two different problems. However 
topological Hochschild homology does commute with colimits (given
cofibrant spectra, as we assumed above \ref{cofibTHH}), thus as long as
I rely on \cite{AuTHH} for the determination of $V(1)_*THH(ku)$ and do
not refer to $TC(ku)$ and $K(ku)$, it is not ambiguous, if I do not
specify, if $ku$ denotes its integral, $p$-local or $p$-completed 
version.

By the description of $ku$ given above \ref{ltolzeta} we know that
at an odd prime $p$ the spectrum $ku$ has a direct summand $\ell$ 
called the Adams summand, first identified
by Adams through operations on vector bundles - see Lecture 4 of \cite{Ad69}. 
The inclusion of fields
$L\rightarrow L(\zeta_p)$ induces a map of $E_\infty$-ring spectra
$i\colon K(L)\rightarrow K(L(\zeta_p)),$ and I fix these completed at
$p$ as a model for the inclusion $i\colon \ell_p\rightarrow ku_p$ \ref{einfapprox}.

By basic obstruction theory (cf. \cite[Proposition 3.1]{EKMM},
\cite[p. 36, Lemma 2.12]{MayEinf})
we can realise the map $ku\rightarrow H\Z = H(\pi_0ku)$ 
as a map of $E_\infty$-ring spectra, as well as the map 
$ku\rightarrow H\Z\rightarrow H\Z/p$. This induces in particular
a map of $E_\infty$-ring spectra $\ell\rightarrow H\Z/p$, which by
\cite[Lemma 16.8]{Ad74} realises the inclusion:
\[H_*(\ell;\Z/p)=\Z/p[\xi_1,\xi_2,\ldots]\otimes E(\tau_2,\tau_3,\ldots)\]
\[\phantom{blubbelblubbel}\rightarrow A_*=H_*(H\Z/p,\Z/p) = 
\Z/p[\xi_1,\xi_2,\ldots]\otimes E(\tau_0,\tau_1,\ldots)\]
of the indicated subalgebra of the dual Steenrod algebra at $p$
(cf. for instance \cite[pp. 51-53]{Kochm}), with generators in degrees
$|\xi_i|=2p^i-2$ and $|\tau_i|=2p^i-1$. Do note that the map $H\Z_p\rightarrow
H\Z/p$ induces an injection on $H\Z/p$-homology as well 
with image the full polynomial algebra, and all of the
exterior algebra apart from $\tau_0$, i.e., the dual of the Bockstein
element in the Steenrod-algebra, which is the ${H\F_p}_*$-Bockstein
map.

Since the map $\ell\rightarrow ku\rightarrow \Z$ can be realised on
bipermutative categories as $\MM_L\rightarrow \MM_{L(\zeta_p)}\rightarrow 
\mathbb{N}^\delta,$ with $\N=\N^\delta$ considered as a discrete bipermutative 
category with its obvious rig structure, we see that the involutions
are compatible as follows:
\[\xymatrix{\ell\ar[d]^{\tau_\ell} \ar[r] & ku\ar[d]^{\tau_{ku}}\ar[r] 
		& H\Z\ar@{=}[d]\\\ell\ar[r]&ku\ar[r]&H\Z,}\]
independently of what involutions we chose compatibly on $ku$ and $\ell$.

In particular we find that any involution on $\ell$ given by an involution
on $\MM_L$ induces the identity on its homology:
{\prop{Let $(T,t)\colon \MM_L\rightarrow \MM_L^\mu$ be an involution of
	bipermutative categories with $H\MM_L=\ell,$ then the involution 
	induced	on homology with $\Z/p$-coefficients is trivial.
	\begin{proof}Let me repeat the core of the argument: If the 
	involution arises on bipermutative categories, then we can map 
	to the discrete rig $\mathbb{N},$ and this induces the 
	monomorphism $H_*\ell\rightarrow H_*H\Z$. Since $\mathbb{N}$ is 
	discrete it only supports the trivial involution.\end{proof}}}

Recall \cite[Theorem 2.5]{AuTHH}: 
{\thm{There is an isomorphism of $A_*$-comodule algebras: \[H_*(ku,\F_p) =
	H_*(\ell,\F_p)\otimes P_{p-1}(x)\] with $x\in H_2(ku,\F_p)$ the 
	Hurewicz-image of $u\in\pi_2ku$, where $P_{p-1}(x)$ denotes the 
	polynomial algebra on $x$ truncated by $x^{p-1}.$ 

	Furthermore $H_*\ell$ can be identified as the inclusion of $\Z/(p-1)$-
	fixed points under the action by the Galois group of $L\rightarrow
	L(\zeta_p)$, hence in particular we have: \[\ell\simeq ku^{h\Z/(p-1)}.\]

	\begin{proof}Each of the statements is found on pp. 1268--1269 of 
    \cite{AuTHH}. Compare also for the fixed point statement the corresponding
    statements on $THH$ and $K$-theory on p. 1307 of \cite{AuTHH}. 
    While the fixed point statement is
	immediate from the homological fixed point spectral sequence (cf. for instance
	\cite{BrRo05}) which collapses at $E^2$, because the relevant group homology
	is acyclic, since the order of the group is a unit $p-1=-1\in\F_p^\times$ 
    -- cf. \cite[p. 156]{Rot2009}. \end{proof}}}

{\thm{\label{invku}
	The involution on $H_*ku$ is completely determined by the effect on
	$u\in\pi_2ku,$ thus also on its image under the Hurewicz map $x\in H_2(ku,\F_p)$.

	Explicitly: For the map of commutative $\S$-algebras $\tau\colon ku\rightarrow ku$ 
	induced	by strictifying $H\V_\CC$ and complex conjugation along the Quillen 
	equivalence	of commutative and $E_\infty$-ring spectra (recalled in 
	\ref{commeinf}), we get	\[\tau_*(u^n)=(-1)^nu^n.\]

	Since on $\V_\CC$ complex conjugation and transposition-inversion agree, this
	is also the effect of the involution induced by the identity. \begin{proof}
	We have the canonical map $ku\rightarrow KU$, and by a classical result of Snaith \cite{Sn}
	we know, that we can obtain $KU$ as the suspension spectrum of the infinite complex
	projective space by inverting the Bott class $u\in \pi_2\Sigma^{\infty}_+\CC P^\infty.$
	For a modernised account in motivic spectra compare Gepner-Snaith \cite{GSn}.

	But this class arises as the suspension of $u\colon \S^2\rightarrow\CC P^\infty$ on
	space level. In particular we can choose to realise it as the inclusion 
	$\Sigma U(1)\rightarrow BU(1)\simeq \CC P^\infty,$ where complex conjugation evidently
	acts on $\Sigma U(1)\cong \S^2$ by a reflection along one equator, hence has
	degree $-1$.\end{proof}}}

We have seen at \ref{Invandmultappr} that the involution induced \ref{indinvSP} 
on $\ell$ is strictly equal to the identity. In particular for commutative models 
inverting and transposing agrees with complex conjugation, thus I consider 
the effect of complex conjugation as fundamental. 
{\cor{Transposition-inversion induces the identity on $(H\F_p)_*\ell$, 
	thus on $H\F_p$-homology of $ku$ it is given
	as $x\mapsto -x$ and the identity on $H_*\ell$.	\begin{proof}
	I already presented above that the fact that $\ell\rightarrow H\F_p$
	induces a monomorphism on homology, forces any self-map on bipermutative
	categories to be visible on $\mathbb{N}^\delta,$ thus trivial.
	\end{proof}}}

{\cor{\label{invkucoeff}Complex conjugation on $\pi_*ku=ku_*\cong \Z[u]$ induces the map:
	$u^n\mapsto (-1)^nu^n$ by \ref{invku}. Thus for a prime $p\geq 3$ 
	the map $\ell\rightarrow ku$ realising the inclusion 
	$\Z_{(p)}[v]\mapsto\Z_{(p)}[u]$ with $v\mapsto u^{p-1},$ identifies 
	the effect of conjugation on $\ell$ as the identity.}}

Proceeding in following \cite{AuTHH} we consider topological Hochschild homology
of $ku$ with coefficients in $H\Z_p,$ which is a $ku$-module by the canonical
maps $ku\wedge H\Z_p\rightarrow H\Z_p\wedge H\Z_p\rightarrow H\Z_p$.

{\thm[{\cite[pp. 1282--1287, Proposition 5.6]{AuTHH}}]{There is an isomorphism of $A_*$-comodule \label{hthhkuzp}
	algebras \[H_*(THH(ku,H\Z_p),\F_p)= 
	H_*(H\Z_p,\F_p)\otimes E([\sigma x],[\sigma \xi_1])\otimes P([y]).\]
	with degrees $|\sigma x|=|x|+1=3$, $|\sigma\xi_1|=2p-2+1=2p-1$, and $|y|=2p$.

	The B\"okstedt spectral sequence for $THH(ku,H\Z_p)$:
	\[E^2=HH^{\F_p}(H_*(ku,\F_p),H_*(H\Z_p,\F_p))\Rightarrow H_*(THH(ku,H\Z_p);\F_p)\]
	has $E^2$-term:
	\[HH_*(H_*ku,H_*H\Z_p)\cong H_*H\Z_p\otimes E(\sigma x, \sigma \xi_1,\sigma \xi_2,\ldots)
	\otimes \Gamma(y,\sigma \tau_2,\sigma \tau_3,\ldots),\]
	where $\Gamma$ denotes the divided power algebra over $\F_p$ on the given generators.
	The spectral sequence collapses at $E^{2p}$ and thus has $E^\infty$-term:
	\[E^{2p}=E^\infty=H_*H\Z_p\otimes E(\sigma x,\sigma \xi_1)\otimes 
	P_p(y,\sigma\tau_2,\sigma\tau_3,\ldots)\]
	with multiplicative extensions $[y]^p=[\sigma\tau_2]$ and $[\sigma\tau_i]^p=[\sigma\tau_{i+1}]$.
	\begin{proof}
	This is all explicit in \cite{AuTHH} at the given pages.
	\end{proof}}}

{\rem{Note in particular that the divided power algebra $\Gamma(y)$ in $E^2$-terms
	of the B\"okstedt spectral sequence actually gives rise to a polynomial
	algebra in \ref{hthhkuzp}.}}

For the Adams summand the analogous computational result was already published in 
1991 by McClure and Staffeldt. In absence of the complications introduced by the
truncated polynomial algebra $P_{p-1}(x)$ I can directly quote the result for 
$THH(\ell)$.
{\thm[{\cite[Proposition 4.2; p.22]{MS}, cf. also \cite[Theorem 5.9]{AuTHH}}]{For any prime $p\geq 3$ 
    there is an isomorphism of $\F_p$-algebras:
    \[H_*(THH(\ell),\F_p)=H_*(\ell,\F_p)\otimes E([\sigma\xi_1],[\sigma\xi_2])\otimes 
    P([\sigma\tau_2]).\]}}

Since the computation of $THH(\ell)$ involves fewer complications than
the analogous one for $THH(ku)$ I can immediately determine the full effect
of the involution here.
{\prop{\label{invHell}
	An endomorphism of $\ell$, which induces the identity on homology ${H\F_p}_*\ell$,
	induces the identity on the tensor factor ${H\F_p}_*\ell$ of
	${H\F_p}_*THH(\ell)$ as well. On the suspension classes
	the homeomorphism $\Gamma$ induces the sign $-1$.
	\begin{proof}It is easily seen that the tensor factor $(H\F_p)_*\ell$ 
	stems from simplicial degree $0$, thus $\iota$ and $\Gamma$ are 
	identities there. The suspended classes $[\sigma x]$ can be 
	represented by classes $1\otimes x$ in the B\"okstedt spectral sequence. 
	In particular we see that since these classes are of simplicial degree 
	$1$, the simplicial	inversion is the identity, while the homeomorphism 
	$\Gamma$ introduces	the sign of one transposition, thus $-1$.\end{proof}}}

I opted to present these results before the calculational cornerstone of \cite{AuTHH},
which is fundamental to Ausoni's calculations, hence also to mine. 
In \cite{MS} the choice of $\ell$ as their object of focus facilitates the calculations
performed by McClure and Staffeldt, particular the absence 
of the truncated polynomial algebra in $H\F_p$-homology of $\ell$. 
In \cite{AuTHH} Christian Ausoni traces the effect of this algebra in ${H\F_p}_*ku$
on homology carefully, isolating its Hochschild homology in \cite[Proposition 3.3]{AuTHH}, 
which I reduce here to isolating the acyclic resolution and the resultant cycles. More generally 
for $k$ a commutative ring with unit the Hochschild homology of algebras $A=k[X]/f$ with $f$ a
monic polynomial was quite generally calculated in \cite{GGRSV}, however I am following the
exposition of Ausoni specialised to $f=X^n$ and $k=\F_p$. As far as I know the original
source for this resolution is \cite[Section 3]{MN}.
{\prop{Let $A=P_h(x)$ be the polynomial algebra truncated by the ideal $x^h$ over $k=\F_p$,
	then there is an acyclic resolution of $A$ as an $A\otimes A^{op}=A^e$-module with
    underlying graded module: \[X=A^e\otimes E(\sigma x)\otimes \Gamma(\tau)\]
	with $A^e$ in resolution degree $0$, $\sigma x$ of degree $(1,|x|)$, $\tau$ of
	degree $(2,h|x|)$.
	For $h$ a unit in $\F_p$ the cycles in $A\otimes_{A^e}X$ consist of the 
	$A$-submodule $\Gamma(\tau)\otimes\{\sigma x\}\oplus (\tau)\otimes\{x\}$
	for $(\tau)\subset \Gamma(\tau)$ the ideal of positive divided powers in
	$\Gamma(\tau).$\label{hhtrPol}
	\begin{proof}This is all explicit in \cite[Proposition 3.3]{AuTHH}. Note that $X$ 
	in fact describes a $\Z$-resolution of $\Z[x]/x^h$, 
	identifying the cycles in $A\otimes_{A^e}X$ however is less clean 
	for $k=\Z.$ \end{proof}}}

{\prop{Consider on $A=P_h(x)$ the morphism of commutative $k=\F_p$-algebras given
	by $x\mapsto -x.$ On generators in Hochschild homology it induces:
	$x\mapsto -x, \sigma x\mapsto -\sigma x, \tau\mapsto \tau.$
	\begin{proof} In the acyclic resolution given above with $A^e=P_h(x)\otimes 
	P_h(y)$ we have	$d(\sigma x) = x-y, d(\tau)=\frac{x^h-y^h}{x-y}\sigma x$ 
	\cite[Proposition 3.3]{AuTHH}. Thus a lift of the given map is given by 
	$\sigma x\mapsto -\sigma x$ and $\tau\mapsto \tau,$ giving the claimed 
	effect.\end{proof}}}

The analogous consideration works immediately for ${H\F_p}_*THH(ku,H\Z_p).$
{\thm{Complex conjugation on $ku$ induces the identity on the tensor factor
	${H\F_p}_*(H\Z_p)\subset {H\F_p}_*(THH(ku,H\Z_p))
	={H\F_p}_*H\Z_p\otimes E(\sigma x,\sigma \xi_1)\otimes P(\tau).$ 
		
	The involution induced by $THH(A,M)\rightarrow 
	\widetilde{THH(A,M)}\rightarrow THH(A,M),$ with first map the anti-involution
	composed with simplicial reversion, and second map the homeomorphism $\Gamma$
	on realisations -- cf. \ref{THHiota} -- induces the following maps:
	$\sigma x\mapsto \sigma x, \sigma \xi_1\mapsto -\sigma\xi_1, \tau\mapsto -\tau.$

	In more detail: The map induced by $x\mapsto -x$ gives $\sigma x\mapsto -\sigma x$
	and the identity on $\sigma\xi_1$ and $\tau$, while the homeomorphism $\Gamma$
	induces a sign $-1$ on all three classes.
	\label{invthhkuzp}
	\begin{proof} We see in the resolution $X$ chosen above that $\iota$
	can be represented as the identity, since it is also a resolution of $A$ 
	as an $(A^e)^{op}=A^{\otimes2}$-module, since $A$ is commutative. 
	Thus only $\Gamma$ introduces an additional effect as a sign dependent 
	on resolution degree, which is $1$ for the suspensions, and $2$ for 
	$\tau$, thus we get $-1$ in both cases \ref{Gammasign}.\end{proof}}}

\section{Increasing Chromatic Complexity -- Reduction by $p$}
As indicated I use the modules, which are easiest to describe, whenever
possible. For the next calculational steps of \cite{AuTHH} I thus need
to introduce ``mod $p$ homotopy''. The idea is quite simple, instead of
considering the prime $p$ as a self-map on $H\Z$ or $H\Z_p$, we can
consider it as a self-map of the sphere spectrum, giving the cofibre
sequence of spectra:
\[\xymatrix{\S\ar[r]^p&\S\ar[r]&V(0)\ar[r]&\Sigma\S.}\]
In particular one could hope that $V(0)$ gives a better approximation
to homotopy groups than $H\F_p$, thus its homology theory is
usually called \emph{mod $p$ homotopy}. In other words the spectrum
$V(0)$ is the two-cell spectrum $\S^0\cup_p \mathbb{D}^1=V(0)$.

It is classical that the spectrum $V(0)$ at a prime $p\geq 3$ admits
a multiplication, which is part of an $A_{p-1}$-structure, which
cannot be extended to $A_p$. For $p=2$ we do not
have a multiplication, while for $p=3$ the multiplication is 
not associative even up to homotopy. For a good survey of
this I refer to \cite{SchV0}. In particular in \cite[Theorem 2.5]{SchV0} we see the
obstruction to extending the $A_{p-1}$-structure to an $A_p$-structure.

The construction \cite[Definition 2.1]{SchV0} 
works by introducing levels of extended powers 
$D_nX=X^{\wedge n}\wedge_{\Sigma_n} {E\Sigma_n}_+$. As a consequence, 
a coherent $M=\S^2\cup_p\mathbb{D}^3$-module structure as defined in 
\cite[Definition 2.1]{SchV0} entails degrees of commutativity as well. Specifically in 
\cite[Example 2.4]{SchV0} the ``tautological'' coherent module structure on 
$V(0)$ is defined up to degree $p-1$. So for $p\geq 5$, the second
extended power $D_2V(0)=V(0)^{\wedge 2}\wedge_{\Sigma_2}{E\Sigma_2}_+$ is part
of the module structure. Thus the multiplication admits the following factorisation:
\[\xymatrix{V(0)\wedge V(0)\ar[d]_\mu \ar[r]^{tw}& V(0)\wedge V(0)\ar[d]_\mu\\
D_2V(0)\ar[r]^{(\_\wedge (12))} & D_2V(0),}\]
where the transposition $(12)$ acts on the factor $E\Sigma_2$, and thus
is canonically simplicially homotopic to the identity. Thus the multiplication
on $V(0)$ is homotopy commutative for $p\geq 5$, by analogously considering
the third extended power, which is also part of the module structure for
$p\geq 5$, we get homotopy associativity as well.

For the classical interpretations in particular of the obstruction class I
defer to the references of \cite{SchV0}, in particular the three Toda 
references, and the reference to Ravenel.

\subsection{The Involution on $V(0)_*THH(ku,H\Z_p)$}
Referring to \cite[Proposition 10.1]{AuTHH} I focus exclusively on $ku$ from
here. The appropriate restriction for $\ell$ follows by the observation, that
the Galois group acts by maps on coefficients of the approximating 
bipermutative categories. In particular the action strictly commutes with
the involution on $ku$, thus the induced involution for $\ell$ can be
recovered by restricting to the submodule of fixed points under the action
of the Galois group by \cite[Proposition 10.1]{AuTHH}.

Introducing $V(0)$-coefficients makes the topological Hochschild homology
of $H\Z_p$ easier to understand. By the equivalence $V(0)\wedge H\Z_p\simeq 
H\F_p$ the resulting module admits an $\F_p$-algebra structure for all odd 
primes $p\geq 3$.
{\prop{\cite[Theorem 5.7]{AuTHH} For any prime $p\geq 3$ there is an isomorphism
of $\F_p$-algebras \[V(0)_*THH(H\Z_p)\cong E(\lambda_1)\otimes P(\mu_1),\]
with degrees $|\lambda_1|=2p-1$ and $|\mu_1|=2p$.
Moreover the Hurewicz homomorphism is an injection
\[V(0)_*THH(H\Z_p)\rightarrow (H\F_p)_*(V(0)\wedge THH(H\Z_p))\]
with $\lambda_1\mapsto[\sigma\xi_1]$ and 
$\mu_1\mapsto [\sigma\tau_1]-\tau_0[\sigma\xi_1].$}}

The fact that the Hurewicz is an injection immediately yields the following corollary.
{\cor{The involution induced on $V(0)_*THH(H\Z_p)$ by the identity as
	in \ref{THHiota} is $\lambda_1\mapsto -\lambda_1$ and $\mu_1\mapsto -\mu_1$.
	\begin{proof}Arguing as in the propositions for $\ell$ we identify the
	claimed effects as the effect of $\iota$ and $\Gamma$ on suspension classes.
	\end{proof}}}

For $THH(ku,H\Z_p)$ the equivalence $V(0)\wedge H\Z_p\simeq H\F_p$ yields
an $\F_p$-algebra structure on the mod $p$ homotopy for all $p\geq 3$, giving
the result:
{\prop{\cite[Theorem 6.8]{AuTHH} There is an isomorphism of $\F_p$-algebras
for any prime $p\geq 3$:
\[V(0)_*THH(ku,H\Z_p)\cong E(z,\lambda_1)\otimes P(\mu_1)\]
with degrees $|z|=3, |\lambda_1|=2p-1$ and $|\mu_1|=2p.$

The Hurewicz homomorphism is an injection
\[V(0)_*THH(ku,H\Z_p)\rightarrow {H\F_p}_*(V(0)\wedge THH(ku,H\Z_p))\]
with $z\mapsto [\sigma x], \lambda_1\mapsto [\sigma\xi_1],$ and
$\mu_1\mapsto [\tau]-\tau_0[\sigma\xi_1]$.}}

{\rem{Let me note in particular that the $0$th Postnikov section 
	$j\colon ku\rightarrow H\Z_p$ induces a map of $H\Z_p$-algebras,
	which on mod $p$ homotopy gives $j_*(\sigma x)=0, j_*(y)=\sigma\tau_1,$
	and $j_*(\sigma\xi_1)=\sigma\xi_1$. Thus the only class we have 
	not analysed with respect to the involution is $\sigma x$.}}

{\cor{The involution on $V(0)_*THH(ku,H\Z_p)\cong E(z,\lambda_1)\otimes P(\mu_1)$
	is given as follows: $z\mapsto z, \lambda_1\mapsto -\lambda_1, 
	\mu_1\mapsto -\mu_1$. \label{v0thhkuzp}\begin{proof}
	The Hurewicz homomorphism, as well as the map induced by $j\colon ku\rightarrow 
	H\Z_p$ give monomorphisms in the degrees relevant to $\lambda_1$ and $\mu_1$,
	giving the claim for them. For $z$ simply note that it is the mod $p$
	reduction of the integral class $\sigma x$ considered before.
	\end{proof}}}

\section{Reducing by $\alpha_1\colon \Sigma^{2p-2}V(0)\rightarrow V(0)$}
In \cite[Sections 7+8]{AuTHH} Ausoni proceeds to identify the mod $p$
homotopy of $THH(ku)$ by considering a Bockstein spectral sequence
associated to the Bott class $u\colon ku\rightarrow ku$. In mod $p$
coefficients the resulting algebra $V(0)_*THH(ku)$ however has
infinitely many generators and infinitely many relations for any
presentation \cite[Corollary 7.11]{AuTHH}. Thus even for purely presentational
reasons it is convenient to introduce one further reduction here.

Recall that the obstruction to extending the $A_{p-1}$-structure on $V(0)$ to
an $A_p$-structure is the Adams self-map usually called 
$\alpha_1\colon \Sigma^{2p-2}V(0)\rightarrow V(0)$. In particular in the
coherent module structures as considered in \cite{SchV0} it appears as a
non-trivial obstruction to unitality of a non-existent $A_p$-structure on
the Moore spectrum (cf. \cite[Theorem 2.5]{SchV0}).

Consider the cofibre sequence defining $V(1)$:
\[\xymatrix{\Sigma^{2p-2}V(0)\ar[r]^-{\alpha_1}& V(0)\ar[r]
& V(1)\ar[r] & \Sigma^{2p-1}V(0).}\] Compare specifically to
page 58 of \cite{Toda}. In particular \cite[Theorem 4.1]{Toda} fixes
$V(1)$ as the unique (up to homotopy equivalence) spectrum with
$4$ cells with attaching maps as indicated:
$V(1)=(\S\cup_pC\S)\cup_{\alpha_1}C(\S^{2p-2}\cup_p C\S^{2p-2}).$
It is usual to call its homology theory \emph{$V(1)$-homotopy}, 
I do so as well in what follows.

By considering $V(0)$ and $V(1)$ as part of a family of spectra
$V(a)$ with inclusions $V(a)\rightarrow V(b)$ for $a<b$ Toda 
identifies multiplicative pairings of the form $V(a)\wedge V(b)
\rightarrow V(c)$ with $a,b\leq c$. In particular for $V(1)$ 
included into $V(1\frac12),V(2\frac14),V(2\frac34),$ and $V(3)$,
we find that the first three cases of \cite[Theorem 4.4]{Toda} give
a multiplication on $V(1)$ for $p\geq 11, p=7$ and $p=5$ respectively.
Furthermore \cite[Theorem 6.3]{Toda} gives in particular that for
$p=3$ such a multiplication cannot exist, while \cite[Theorem 6.1]{Toda}
explicitly establishes that a spectrum of the analogous type of
$V(1)$ does not exist at $p=2$ at all. In particular since this
consideration has naturally led us to primes with $p\geq 5$ we
can apply \cite{Oka} to find that the obstruction to 
homotopy-commutativity is always $2$-torsion, while the obstruction
to homotopy-associativity is always $3$-torsion. Thus both vanish
in the coefficients of $V(1)$.

Since I wanted to introduce $V(1)$ as late as possible with respect
to the calculations in \cite{AuTHH}, I only partially quoted the 
result of \cite{MS} regarding $THH(\ell).$ Here it can thus serve to
convince the reader that $V(1)$ simplifies the modules considerably.

{\prop{(\cite{MS}, cf. \cite[Theorem 5.9]{AuTHH}) For any prime $p\geq 3$
there is an isomorphism of $\F_p$-algebras:
\[(H\F_p)_*(THH(\ell))\cong (H\F_p)_*\ell\otimes E(\sigma\xi_1,\sigma\xi_2)
\otimes P(\sigma\tau_2).\]
The $V(1)$-homotopy $V(1)_*THH(\ell)$ maps by an injective Hurewicz 
homomorphism to $(H\F_p)_*(V(1)\wedge THH(\ell)),$ with image generated
as an algebra by $[\sigma\xi_1],[\sigma\xi_2]$ and 
$[\sigma\tau_2]-\tau_0[\sigma\xi_2]$, yielding an isomorphism of $\F_p$-algebras:
\[V(1)_*THH(\ell)\cong E(\lambda_1,\lambda_2)\otimes P(\mu_2),\]
with degrees $|\lambda_1|=2p-1, |\lambda_2|=2p^2-1,$ and $|\mu_2|=2p^2$, with
generators defined as the preimages of $[\sigma\xi_1],[\sigma\xi_2]$ and 
$[\sigma\tau_2]-\tau_0[\sigma\xi_2]$ respectively.}}

In particular since the generators are preimages of suspension classes by an
injective Hurewicz homomorphism, the involutions are determined by \ref{invHell},
giving the following cleaner statement.
{\cor{For the $V(1)$-homotopy of $THH(\ell)$: \label{v1thell}
	\[V(1)_*THH(\ell)\cong E(\lambda_1,\lambda_2)\otimes P(\mu_2),\]
	the homeomorphism $\Gamma$ (cf. \ref{simpoppTop}) induces a 
	sign $-1$ on each generator. Thus
	the induced involution \ref{THHiota} on $V(1)_*THH(\l)$ is given as:
	$\lambda_1\mapsto -\lambda_1, \lambda_2\mapsto -\lambda_2,
	\mu_2\mapsto -\mu_2.$}}

\subsection{The Homology and $V(1)$-Homotopy of $THH(ku)$}
People familiar with the computations in \cite{AuTHH} know that the
algebras on homology ${H\F_p}_*THH(ku)$ and $V(1)_*THH(ku)$ contain  
big subalgebras $\Omega_*$ and $\Xi_*$ respectively 
on $(p-1)^2+1=p^2-2p$ generators with quite a few relations 
- cf. \cite[Definition 9.9, Definition 9.13]{AuTHH}. Essentially this stems from the 
truncated polynomial algebra in ${H\F_p}_*ku$ \cite[Proposition 2.3]{AuTHH}.

In order to understand the involution on $H_*THH(ku)$ and $V(1)_*THH(ku)$
however I do not need to display the relations, it suffices to understand
the effect on the cycles 
$(x)\otimes (\tau)\oplus (\sigma x)\otimes \Gamma(\tau)
\rightarrow HH_*(P_{p-1}(x))$.

To determine the involution on $V(1)_*THH(ku)$ we shall use the topological
Hochschild homology of $ku$ with coefficients in $H\Z_p$, i.e., 
$V(1)_*THH(ku,H\Z_p)$ as an anchor.
{\lem{(\cite[p. 1305]{AuTHH})
	The isomorphism on mod $p$ homotopy $V(0)_*THH(ku,H\Z_p)\cong 
	E(z,\lambda_1)\otimes P(\mu_1)$ implies the algebra isomorphism:
	\[V(1)_*THH(ku,H\Z_p)\cong E(z,\lambda_1,\varepsilon)\otimes 
	P(\mu_1)\] with $\varepsilon$ of degree $2p-1$.	\begin{proof}
	The Hurewicz from $V(0)_*THH(ku,H\Z_p)$
	to ${H\F_p}_*(V(0)\wedge THH(ku,H\Z_p))$
	is a monomorphism \cite[Theorem 6.8]{AuTHH}. So the Adams map $\alpha_1\colon 
	\Sigma^{2p-2}V(0)\rightarrow V(0)$ induces the zero map on 
	$V(0)_*THH(ku,H\Z_p)$, because it induces the trivial map
	in $H\F_p$-homology. Thus the $V(1)$-homotopy of $THH(ku,H\Z_p)$
	consists of two shifted copies of its mod $p$ homotopy, which
	we can parametrise by a formal generator $\varepsilon$ 
	of degree $2p-1$.\end{proof}}}

{\prop{The induced involution \ref{THHiota} on $V(1)_*THH(ku,H\Z_p)$ is
	given on generators as:
	\[z\mapsto z, \lambda_1\mapsto -\lambda_1, \mu_1\mapsto -\mu_1,
	\varepsilon\mapsto \varepsilon.\]
	\begin{proof}
	Since $\varepsilon$ in essence describes the connecting homomorphism
	of the cofibre sequence:
	\[\Sigma^\bullet V(0)\rightarrow V(0)\rightarrow V(1)\rightarrow 
	\Sigma^\bullet V(0),\]
	and thus acts on the coefficients it commutes with the involution induced
	on $THH$. On the other elements the involution is the one given in 
	\ref{v0thhkuzp}	\end{proof}}}

As seen above I do not need to display the relations of the algebras 
$\Xi_*$ and $\Omega_*$ defined in \cite[Definition 9.9, Definition 9.13]{AuTHH}, thus
I use the following simplified description -- which amounts to ignoring
the relations introduced by boundaries in the Hochschild complex.
{\prop{The $P_{p-1}(u)$-algebras $\Xi_*$ and $\Theta_*$ admit
	a surjection of the $P_{p-1}(u)$-module 
	$\Gamma(\tau)\{\sigma u\}\oplus (\gamma_1\tau)\{u\}\oplus 
	\F_p\otimes\{\mu_2\}$
	with $(\gamma_1\tau)\subset \Gamma(\tau)$ the ideal of positive
	divided powers in $\Gamma(\tau)$.

	Specifically for $\Xi_*$ we assign 
	$\gamma_i\tau\cdot\sigma u\mapsto \bar z_i$ and $\gamma_i\tau\cdot u
	\mapsto \bar y_j$.
	\begin{proof}
	The generators of $\Xi_*$ and $\Omega_*$ arise from the $E^2$-term
	of the B\"okstedt spectral sequence 
	$HH_*({H\F_p}_*ku)\Rightarrow {H\F_p}_*THH(ku),$
	which are images of the generators given by the inclusion
	$HH_*(P_{p-1}(u))\rightarrow HH_*({H\F_p}_*ku).$ For Hochschild
	homology of this truncated polynomial algebra we have determined
	the claimed surjection in \ref{hhtrPol}. The class $\mu_2$ maps
	to $[\sigma \tau_2]$ in ${H\F_p}_*THH(ku)$ and to the
	class with the same name in $V(1)_*THH(ku)$.\end{proof}}}

This description is sufficient to determine the effect of the maps
defining the involution on $THH(ku)$. 

{\thm{For the isomorphism of \cite[Proposition 9.10]{AuTHH} 
	\[{H\F_p}_*THH(ku)\cong {H\F_p}_*\ell\otimes E([\sigma\xi_1])\otimes \Xi_*\]
	and the analogous isomorphism in $V(1)$-homotopy of 
	\cite[Theorem 9.15]{AuTHH}: $V(1)_*THH(ku)=E(\lambda_1)\otimes \Theta_*$
	we can determine the involutions as follows.
	On $\mu_2$ we have $\mu_2\mapsto -\mu_2$ as visible in 
	$THH(\l)$ \ref{invHell},\ref{v1thell}.
	On $a_i\in\Omega_*$ we have $a_i\mapsto (-1)^ia_i$, analogously on
	$\bar z_i\in\Xi_*$: $\bar z_i\mapsto (-1)^i\bar z_i$.
	For $b_i\in\Omega_*$ we have $b_i\mapsto (-1)^{i+1}b_i$, analogously
	for $\bar y_i\in\Xi_*$: $\bar y_i\mapsto (-1)^{i+1}\bar y_i$.
	\label{invTHHku}\begin{proof}
	The classes $a_i$ and $\bar z_i$ arise as infinite cycles in the
	B\"okstedt spectral sequence represented by the classes $\sigma x\gamma_i\tau$
	giving the claim by Theorem \ref{invthhkuzp}. Analogously the classes $b_i$
	and $\bar y_i$ are cycles associated to the classes $x\gamma_i\tau$ for
	$i\geq 1$, thus giving the claim again by Theorem \ref{invthhkuzp}.
	\end{proof}}}

\section{Results on the Involution on $V(1)_*K(ku_p)$}
Since the calculations in \cite{AuKku}, as well as \cite{AuKl1,AuKl2} rely
on trace methods, my intended approach was to determine the involution
on $K(ku_p)$ and $K(\l_p)$ by using the trace $tr\colon K\Rightarrow THH$. As
shown above it commutes with the involutions \ref{trinv} defined
on $K$ as in \ref{indinv} and on $THH$ as in \ref{THHiota}. A few of 
the classes allow more direct approaches, so I prefer these for the 
exposition below.

From this point on I explicitly denote the completions at $p$ of $\l$ and $ku$.
In particular since the computations in \cite{AuKl1,AuKku} rely partly on
comparison to the integers, thus on the results of \cite{BHMtr} and more
specifically \cite{BMTCZ} it would not be reasonable to expect global information
given our limited state of knowledge about $K(\Z)$.
Instead we rely on the computations for $K(\Z_p)$.

\subsection{The Module $V(1)_*K(ku_p)$ and its Traces}
For reference in the sections below I directly quote the main result
of \cite{AuKku}. I examine a few classes individually below, so I 
quote only the identification of the module given in \cite[Theorem 8.1]{AuKku}.
{\thm{\cite[Theorem 8.1]{AuKku}}{There is an isomorphism of $P(b)$-modules
\[\begin{aligned} \label{dasmonster}
V(1)_*K(ku_p)\cong &P(b)\otimes E(\lambda_1,a_1)\\
\oplus &P(b)\otimes E(\lambda_1)\otimes \F_p\{\sigma_n|~1\leq n\leq p-2\}\\
\oplus &P(b)\otimes \F_p\{\partial\lambda_1,\partial b,
	\partial a_1,\partial\lambda_1a_1\}\\
\oplus &P(b)\otimes E(a_1)\otimes \F_p\{t^d\lambda_1|~0<d<p\}\\
\oplus &P(b)\otimes E(\lambda_1)\otimes \F_p\{t^{p^2-p}\lambda_2\}\\
\oplus &\F_p\{s\}.\end{aligned}\]}}

With regard to the traces of the classes, I draw the following corollary,
which is less precise than the determination in \cite[Theorem 8.1]{AuKku} 
but sufficient for determining the involution on the first direct summand.
{\cor{to \cite[Theorem 8.1]{AuKku}}{
	In the direct sum decomposition above, the third to
	sixth summand are contained in the kernel of the map induced
	by the trace $V(1)_*K(ku_p)\rightarrow V(1)_*THH(ku_p).$

	Furthermore the traces of the classes $\sigma_n$ are contained in
	the $P_{p-1}(u)$-subalgebra of $V(1)_*THH(ku_p)$ generated
	by $a_0\in V(1)_3THH(ku_p)$, while the traces of the first
	summand are part of the $P_{p-1}(u)$-subalgebra generated by
	$a_1,b_1,$ and $\lambda_1$.}}

We can use basic linear algebra over $\F_p$ to actually find that
the trace can be described as a direct sum of maps with ``orthogonal''
images in low degrees. Again this follows directly from reading
\cite[Theorem 8.1]{AuKku} and \cite[Theorem 9.15]{AuTHH} appropriately.
{\cor{to \cite[Theorem 8.1]{AuKku}}{
	The trace corresponds to a direct sum of maps with respect
	to the direct sum decomposition of $V(1)_*K(ku_p)$ above
	and the direct sum decomposition of $V(1)_*THH(ku_p)$ (as a 
	$P_{p-1}(u)$-module) given
	by the generators $\lambda_1,a_i,b_i,\mu_2$.

	In particular, in low degrees it is injective on the classes
	$\lambda_1,a_1$ of the first summand, with image having
	trivial intersection with the other summands.}}

\subsection{The Involution on $E(\lambda_1,a_1)\subset V(1)_*K(ku_p)$.}
By the corollary above we only need to determine the involution on 
$\lambda_1,$ and $a_1$ in $V(1)_*THH(ku)$, which implies we have
the same effect on $V(1)_*K(ku_p)$ by injectivity in those degrees.
So I isolate the part of \ref{invTHHku} relevant to $K(ku_p)$ here.

{\thm{The involution on the $P(b)$-subalgebra \[E(\lambda_1,a_1)
    \subset	V(1)_*K(ku_p)\] is given by $\lambda_1\mapsto -\lambda_1$ 
    and	$a_1\mapsto -a_1$. \begin{proof} This follows directly
    from the theorems on $V(1)_*THH(ku)$. Specifically recall
    that $\lambda_1$ stems from the class of $\sigma\xi_1$ in
    ${H\F_p}_*THH(H\Z_p)$ with $\xi_1$ the generator in the
    dual of the Steenrod algebra. Thus the involution induced
    by conjugation is trivial, $\Gamma$ induces a sign $-1$.

    The class $a_1$ can be traced back to the class 
    $\sigma x\gamma_1\tau=\sigma x\tau$ in the B\"okstedt spectral
    sequence. The class $\sigma x$ has simplicial degree $1$, $\tau$
    has simplicial degree $2$. The map induced by conjugation induces
    a sign $-1$ on $\sigma x$ and the identity on $\tau$, the homeomorphism
    $\Gamma$ induces a sign $-1$ on both, yielding the claim.\end{proof}}}

{\rem{We have the involution determined on this full subalgebra after 
	determining	the effect on $b$ as well below.}}

\subsection{The Suspended Bott Classes $\sigma_n$}
By restricting the isomorphism of \cite[Proposition 5.2]{AuKku} we get
an inclusion identifying the classes $\sigma_n\in V(1)_*K(ku_p).$ In fact
they are global, i.e., $\sigma_n\in V(1)_*K(ku).$

Specifically they arise as follows: We have the inclusion 
$BBU_\otimes\subset BGL_1ku,$ where the units of an $E_\infty$-ring spectrum
are defined as the (homotopy) pullback:
\[\xymatrix{
GL_1A\ar[r]\ar[d]&\Omega^\infty A\ar[d]\\
GL_1(\pi_0A)\ar[r]&\pi_0A,}\]
which is a sufficient notion of ``units'' for our purposes. For $ku$ we have
$GL_1\pi_0ku= GL_1\Z=\{\pm 1\}$, hence $GL_1ku=BU\times\{\pm1\}.$
Restricting to the ``index'' $+1$, we get the
inclusion $BBU\rightarrow BGL_1ku$. By delooping the
topologically enriched permutative category on objects natural
numbers with endomorphisms $GL_nku$ along the lines of \cite{EM}
recalled in \ref{pcatneu} we get a model for $K$-theory of $ku$.
In particular its underlying infinite loop space is the group completion:
$\Omega B(\coprod_nBGL_nku)=\Omega^\infty K(ku).$ The canonical
inclusion $GL_1ku\rightarrow \coprod_nGL_nku$ fits in the
sequence of maps
\[BBU\rightarrow BGL_1ku\rightarrow \coprod_nBGL_nku\rightarrow 
\Omega^\infty K(ku).\]

Considering units in a stricter setting, for instance
with $ku$ as a symmetric ring spectrum, we can find a strictly
associative model for $BU_\otimes$. Thus
we have a one-point suspension category $\Sigma BU_\otimes$.
Embed this into its free permutative category $\P\Sigma BU_\otimes
= \coprod_n E\Sigma_n\times_{\Sigma_n}(\Sigma BU_\otimes)^n.$ The induced
map of permutative categories $\P\Sigma BU_\otimes\rightarrow \coprod_nGL_nku$
delooped as in \ref{pcatneu} induces a map:
\[H\P\Sigma BU_\otimes\rightarrow H(\coprod_nGL_nku),\]
which we can identify as
\[\omega\colon \Sigma^\infty BBU_\otimes\rightarrow K(ku)\]
as given in \cite[p. 627]{AuKku}.

Trivially the map $BU\rightarrow BU\times \Z$ identifying $BU$ as a connected
cover of $BU\times\Z$ is an isomorphism in positive degrees, thus we have classes
$y_n\in\pi_{2n}BU$ for $n\geq 1$ which map to the powers of the Bott class $u^n$.

Suspending these once gives classes $\sigma y_n\in\pi_{2n+1}\Sigma BU$ giving classes:
\[\Sigma^\infty \S^{2n+1}\rightarrow \Sigma^\infty\Sigma BU\rightarrow
\Sigma^\infty BBU\rightarrow K(ku).\]

Call these $\sigma_n$ in agreement with \cite[Definition 3.2]{AuKku}.
Then we have:
{\prop{(cf. \cite[Proposition 5.2]{AuKku})
	The classes $\sigma_n\in \pi_{2n+1}K(ku)$ are non-trivial for
	$1\leq n\leq p-2$.}}

{\rem{I have to concede that I am not certain about non-triviality 
    for the higher $\sigma_n$, however the traces of the $\sigma_n$
    are $u^{n-1}a_0$ by \cite[Theorem 8.1]{AuKku}. 
	Thus one should probably consider the upper limit $p-2$ as an artefact 
	of the relations for $\Theta_*$ and
	$\Xi_*$ in $V(1)_*THH(ku)$. At least on the subspectrum
	$\Sigma(ku)\rightarrow THH(ku)$ one can identify these globally in homotopy 
	groups, thus in particular removing the bound $p-2$. However establishing
    their non-triviality would entail 
    non-trivial calculations in the homotopy groups of $THH(ku)$.}}

We can identify the involution on these suspended classes as follows:
{\thm{The involution on the classes $\sigma_n\in\pi_{2n+1}K(ku)$ for $n\geq 1$ 
	is given as $\sigma_n\mapsto (-1)^{n+1}\sigma_n$. 
	\begin{proof}
	Consider part of the inclusion into $\Omega^\infty K(ku)$:
	\[BBU\rightarrow BGL_1ku\rightarrow \coprod_nBGL_nku.\]
	Then the classes of the $\sigma_n$ are given by $\Sigma \S^{2n}\rightarrow
	BBU\rightarrow BGL_1ku$. Thus conjugation acts as it does on $u^n$,
	giving a sign $(-1)^nu^n$. Transposition has no effect on $GL_1ku$.
	Finally $\Gamma$ acts on simplicial degree $1$ here, thus reverses the
	signs to give $(-1)^{n+1}\sigma_n.$	\end{proof}}}

\subsection{The Higher Bott Class $b\in V(1)_{2p+2}K(ku)$}
The class of major interest in $K(ku)$ is a class in degree $2p+2$ of
the $V(1)$-homotopy of $K(ku)$, which is a non-trivial root of $v_2\in\pi_*V(1),$
thus in particular establishing $K(ku)$ as the representing spectrum of
a homology theory of chromatic type $2$.

Moreover: By the calculations of Ausoni in \cite{AuKku}, in particular Theorem 8.1
as recalled above in Theorem \ref{dasmonster}, we 
know that apart from a sporadic class the module 
$V(1)_*K(ku)$ is a free module over the polynomial
algebra on $b$. 

{\rem{To be consistent in denoting $H\F_p$-homology on the left, I refer
    to classes in degree $n$ as $z\in H\F_{p,n}X$ in the following proposition.}}

Recall the construction of the element $b$:
{\prop{Consider the homology algebra of $\CC P^\infty\simeq K(\Z,2)$, which
is a divided power algebra $\Gamma(y)\cong {H\F_p}_*K(\Z,2)$.

Then in the spectral sequence in $H\F_p$-homology associated to the bar filtration of
$K(\Z,3)=B(K(\Z,2))$ the class $y^{p-1}\otimes y$ is an infinite non-bounded cycle 
of degree $(2,2p)$. By \cite[Lemma 2.3]{AuKku} we have in particular an
exact sequence ${H\F_p}_{,5}(K(\Z,3))\rightarrow V(1)_{2p+2}(K(\Z,3))
\rightarrow {H\F_p}_{,2p+2}(K(\Z,3))\rightarrow {H\F_p}_{,0}(K(\Z,3)).$
Since ${H\F_p}_{,5}(K(\Z,3))=0$, and $(P^1)^*(\gamma_{p-1}(y))=0$ the
last map and the first group are zero, so we have a unique class 
$V(1)_{2p+2}K(\Z,3)$ which maps to the class of $y^{p-1}\otimes y$ by the
Hurewicz $V(1)\rightarrow H\F_p$. Finally by considering the
embedding $K(\Z,3)=BBU(1)\rightarrow BBU\rightarrow \coprod_nBGL_n(ku)$ as
before we get the higher Bott element $b\in V(1)_{2p+2}K(ku)$.}}

{\thm{The involution on the higher Bott element is trivial, specifically
	the algebra map $ku\rightarrow ku$ induced by conjugation acts as
	a sign $-1$, and the homeomorphism $\Gamma$ acts as a sign $-1$.
	\begin{proof} By \cite[p. 623]{AuKku} we know 
	$\Sigma K(\Z,2)\rightarrow B_2\rightarrow \Sigma^2(K(\Z,2)^{\wedge 2})$
	induces an injective map in degree $2p+2$:
	$V(1)_{2p+2}B_2\rightarrow V(1)_{2p+2}(\Sigma^2(K(\Z,2)^{\wedge 2}))$
	with $B_2\subset K(\Z,3)$ the image of the $2$-skeleton.

	In particular the map $B_2\rightarrow K(\Z,3)\rightarrow BBU
	\rightarrow\coprod_nBGL_n(ku)$ again is a class associated to
	$GL_1ku$, thus transposition has no effect. Furthermore it
	is a class of simplicial degree $2$ by definition, thus $\Gamma$
	acts as a sign $-1$. Finally we have to determine how complex
	conjugation acts on $b$. For this consider the representative
	of the bar spectral sequence $y^{p-1}\otimes y$. On homology
	of $K(\Z,2)=BU(1)$ complex conjugation acts by a group homomorphism
	on $U(1)$ as $y^n\mapsto (-1)^ny^n$.
	In particular we get: $y^{p-1}\otimes y \mapsto (-1)^py^{p-1}\otimes y=-y^{p-1}\otimes y.$

	In summary $\Gamma$ and the conjugation cancel out, which is visible
	on $B_2$.\end{proof}}}

\subsection{Summary of the Induced Involution}
Here I want to summarise the above results. 
Recall the isomorphism of \cite{AuKku}:
\[\begin{aligned}
V(1)_*K(ku_p)\cong &P(b)\otimes E(\lambda_1,a_1)\\
\oplus &P(b)\otimes E(\lambda_1)\otimes \F_p\{\sigma_n|~1\leq n\leq p-2\}\\
\oplus &P(b)\otimes \F_p\{\partial\lambda_1,\partial b,
	\partial a_1,\partial\lambda_1a_1\}\\
\oplus &P(b)\otimes E(a_1)\otimes \F_p\{t^d\lambda_1|~0<d<p\}\\
\oplus &P(b)\otimes E(\lambda_1)\otimes \F_p\{t^{p^2-p}\lambda_2\}\\
\oplus &\F_p\{s\}.\end{aligned}\]

In the section above we have established that $b$ is invariant under the induced
involution. In the first section we have determined the involution on $\lambda_1$ 
and $a_1$ to each be given by a sign. In the second section we found that the
involution induces $\sigma_n\mapsto (-1)^{n+1}\sigma_n$.
\subsection{The $TC$-Classes}
For the third to the sixth summand in the above decomposition,
one would have to establish, if there is an involution on topological cyclic 
homology, which is compatible with the cyclotomic trace. Specifically the classes 
$t^d\lambda_1$ and $t^{p^2-p}\lambda_2$ are composites of the eponymous classes 
$\lambda_i\in V(1)_*THH(ku)$ and powers of the classes $t\in H_*(\Z/p^n)\rightarrow H_*(\S^1)$ 
arising from the homotopy fixed point spectral sequences from $THH$ to $TC$. In particular, 
one would have to establish how the circle action on $THH$ behaves with respect to the involution. 

Regrettably I have to say that this is beyond the scope of this thesis. In particular I cannot
even offer a conjecture on what map the involution induces on the other summands. Specifically
I do not know the effect of the involution on $\partial, s$ and $t$, which are each part
of the remaining classes, and each most transparently appear in $V(1)_*TC(ku_p)$.

I hope to get back to a full description of the involution on $V(1)_*K(ku_p)$ in future work.
Most probably the description of an involution on $TC$ can be bypassed for
this case, since the class $\partial$ is an artifact of $p$-adic completion,
which is already present in $V(1)_*K(\Z_p)$, and the classes $t^d$ are
visible in the homotopy fixed point spectral sequence for $THH(ku)^{h\S^1}$.
