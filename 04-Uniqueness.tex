\chapter{Partial Uniqueness Results} \label{conjun}
Evidently, I am interested if the multiplication on the delooping 
I describe in chapter \ref{multbidel} describes a new structure or the 
known $E_\infty$-structure on the $K$-theory of an $E_\infty$-ring 
spectrum. 

Given this I can avoid the complicated constructions of the 
trace map described for example in \cite{BHMtr, S tr} (compare also 
\cite{DGM12}). By \cite{BGT2013} there is an essentially unique
map of symmetric multifunctors from $K$-theory to any other 
(symmetric) multifunctor $F$ that satisfies $F(\S)=\S$. For
topological Hochschild homology the equality $THH(\S)=\S$ is
immediate from the definitions, thus there is a unique $E_\infty$-map
$K\rightarrow THH,$ which by \cite[Theorem 1.9]{BGT2013} is the trace map.
Specifically the essential uniqueness proven in \cite{BGT2013} entails
that the trace map is a point in the space of $E_\infty$-maps $K\rightarrow THH$,
which is contractible.

However, for the multiplicative uniqueness I have to concede conjecture
status, while the additive uniqueness, i.e., the fact that there is
essentially only one delooping of the nerve of a symmetric monoidal 
category by its $E_\infty$-structure is classical \cite{MT1978}, 
which can be rephrased nicely with the results of \cite{GGN}.

\section*{Convention: The Language of $\infty$-Categories}
To conveniently state and prove results about uniqueness of 
$E_\infty$-structures we evidently need some organisational
language in which to compare them. As shown in \cite{EM} 
multicategories can be used in absence of 
symmetric monoidal structures on a category to define
operad-structures. However, comparison theorems in this setting seem 
too restrictive to expect. Since $E_\infty$-structures
are a coherently weakened notion of commutativity, we would
not expect different $E_\infty$-structures from potentially
different $E_\infty$-operads to be comparable by strict 
multifunctors. This would probably prove to be even worse for
general operads. 

Given the recent popularisation of $(\infty,1)$-categories for instance
by \cite{Bergn1, Bergn2, Joyal,JoyT,Lu1,Lu2} 
advancing the theory of quasi-categories as a convenient model for 
$(\infty,1)$-categories, the authors have made such comparison results easier
to state and prove in a satisfactory manner. Hence following my
principal sources \cite{GGN,BGT2013} for the uniqueness results
I use the language of $(\infty,1)$-categories, and refer to 
specific results about quasi-categories in \cite{Lu1, Lu2} where 
I need them. 
For a nice survey I specifically recommend \cite{Bergn2} and 
an accompanying talk \cite{Bergn3}.

\section{Uniqueness of the Spectrum $\R\mapsto H\R$}
I can directly (partially) quote this result from 
\cite{GGN} with the only alteration that, as in \cite{BDRR2011},
I refer to the associated spectrum of a permutative category 
$\C$ as its Eilenberg-MacLane-spectrum $H\C$, for instance given
by the delooping of \cite{EM}.
{\thm[First part of Prop. 8.2. in \cite{GGN}]{The 
Eilenberg-MacLane-spectrum functor $H\colon \SymMonCat \rightarrow Sp$ 
is lax symmetric monoidal.}}

This statement does not obviously incorporate $\infty$-categories. Their
use is implicit in the symmetric monoidal structure on $\SymMonCat$, which
probably does not exist, when we consider $\SymMonCat$ as a $1$-category 
with objects small symmetric monoidal $1$-categories and symmetric 
monoidal functors as morphisms -- cf. \cite{K} for a specific
construction of a product on symmetric monoidal categories with strictly
symmetric monoidal functors, which fails to have a unit object, and
does not produce bimonoidal/bipermutative categories as its monoids.
However, if we relax to considering $\SymMonCat$ as a $2$-category with
the same objects and $1$-cells, but additionally monoidal natural 
transformations as $2$-cells, we can see in \cite{Schm} that $\SymMonCat$
does in fact support the structure of a ``symmetric monoidal bicategory''. 
But his construction has the wrong ($E_\infty$) monoids, thus at most
serves as a proof that $\SymMonCat$ can equipped with a symmetric monoidal
bicategory structure at all. 

I strongly conjecture that a symmetric monoidal product on the 
bicategory of symmetric monoidal categories with the right monoids can 
be constructed:	Specifically the construction of a ``classifying 
pseudofunctor'' as in \ref{BCfunctor}, which for $(\C,+)$ symmetric 
monoidal gives a pseudofunctor $B\C\colon \Fi\rightarrow Cat$. The 
additive Grothendieck construction as in \ref{c+} is the Grothendieck 
construction over this functor. When we restrict $B\C$ to the source 
category with just surjections $\Ep$ however, the Grothendieck 
construction $\C^{+,epi}$  over this functor admits a natural adjunction 
\[\xymatrix{\C^{+,epi} \ar@<0.5ex>[r] & \ar@<0.5ex>[l] \C}\]
and thus the classifying spaces are homotopy equivalent. 
We can easily consider the product $B\C\times B\D$, and the Grothendieck
construction over this functor yields a candidate for a ``smash product''
of permutative categories. However I do not know, which of the specific
``averaging'' processes described by the Grothendieck construction 
yields the most convenient smash product. We could consider the
product $B\C\times B\D$ and pull back by the diagonal $\Fi\rightarrow
\Fi\times\Fi$. Alternatively we could directly consider the Grothendieck
construction restricted to $\Ep\times\Ep$. Since this becomes unwieldy 
quite fast, I have not established the properties this product might have.
I am quite sure it is unital and associative ``up to adjunction'', symmetric
up to isomorphism, however, one would have to establish the appropriate
coherences of adjunctions and isomorphisms.

The theorem quoted above reframes the results in
\cite{EM} that the associated Eilenberg-MacLane-spectrum is a symmetric
multifunctor from permutative categories to spectra, where in
\cite{EM} the multicategory-structure is precisely the concession needed
for the fact that the $1$-category of symmetric monoidal categories 
(probably) does not support a symmetric monoidal product.

The results of \cite{GGN} make it unnecessary to construct a symmetric
monoidal product on $\SymMonCat$ on the $2$-categorical level.

Recall from \cite{Lu2,GGN} that given an $\infty$-category, we have naturally
associated to it its category of pointed objects $\C_*$, its category
of $E_\infty$-monoids $\mathrm{Mon}_{E_\infty}\C$, modelled as $\Gamma$-objects
in $\C$, its category of grouplike $E_\infty$-monoids $\mathrm{Grp}_{E_\infty}\C$,
and its stabilisation, for instance given as the category of spectrum objects
in $\C$, denoted $Sp(\C)$. Moreover, each pointed object $c\in\C$ gives rise to a free $E_\infty$-monoid $hc$
given by $hc(n_+)=c^{\times n}$ with the evident maps induced by pointed maps in $\Fi=\Gamma$.
Group completion provides a free grouplike object associated to an arbitrary
$E_\infty$-monoid in the sense that it is a left-adjoint to the forgetful
functor in the opposite direction. Finally, a grouplike $\Gamma$-object gives
rise to an associated spectrum for instance by insertion of simplicial spheres 
$\S^n$ (cf. \ref{insertS1}).

{\thm[Theorem 5.1 \cite{GGN}]{Let $\C^\otimes$ be a closed symmetric
monoidal presentable $\infty$-category $\C$. The 
$\infty$-categories $\C_*$, $\mathrm{Mon}_{E_\infty}\C$, $\mathrm{Grp}_{E_\infty}\C$, 
$Sp(\C)$ all admit closed symmetric monoidal structures, which are uniquely
determined by the requirement that the respective free functors from $\C$
are symmetric monoidal. Moreover each of the functors \[\C_*\rightarrow 
\mathrm{Mon}_{E_\infty}\C\rightarrow \mathrm{Grp}_{E_\infty}\C\rightarrow 
Sp(\C)\] uniquely extends to a symmetric monoidal functor.}}

By \cite{GGN} the $\infty$-category of symmetric monoidal categories
thus has a symmetric monoidal $\infty$-category-structure by the following 
argument of Theorem 5.1 in \cite{GGN} specialised for $\C$ the $\infty$-category of 
$1$-categories with cartesian product as its closed symmetric monoidal structure.
The $\infty$-category of symmetric monoidal categories is identified 
as a smashing $\infty$-localisation with the $E_\infty$-monoids 
in spaces $\mathrm{Mon}_{E_\infty}(Top)$ \cite[Theorem 4.6]{GGN}. So they get 
\[\SymMonCat=\mathrm{Mon}_{E_\infty}(Cat_1)\simeq Cat_1\otimes 
\mathrm{Mon}_{E_\infty}(Top),\] 
where $\otimes$ denotes the tensor-product of presentable $\infty$-categories as
defined in \cite{Lu2}. Hence arguing the same way as for Bousfield
localisations of symmetric monoidal model categories, one only has to establish
that $\_\otimes \mathrm{Mon}_{E_\infty}(Top)$ respects local equivalences with
respect to $\mathrm{Mon}_{E_\infty}$. This is trivial for a smashing localisation 
in $\infty$-categories, because of the equivalence 
$\_\otimes \mathrm{Mon}_{E_\infty}(Top)\otimes 
\mathrm{Mon}_{E_\infty}(Top) \simeq \_\otimes \mathrm{Mon}_{E_\infty}(Top),$ 
which makes the smashing functor idempotent up to a chosen coherent
equivalence, hence a localisation. The 
equivalence $\mathrm{Mon}_{E_\infty}(\mathrm{Mon}_{E_\infty}(\_))\simeq \mathrm{Mon}_{E_\infty}(\_)$
is just an elaboration on the Eckmann-Hilton-argument, two 
compositions which are homomorphisms with respect to each other and 
have neutral elements are equal.

Replace in \cite[p. 19]{GGN} the expression ``algebraic $K$-theory'' with 
``Eilenberg-MacLane spectrum''; this 
identifies the Eilenberg-MacLane spectrum functor as the composition of functors
\[\xymatrix{\SymMonCat \ar[r]^{(\cdot)^{iso}} &\SymMonCat 
\ar[r]^{N(\cdot)} & \mathrm{Mon}_{E_\infty}(\mathcal{T})
\ar[r] & \mathrm{Grp}_{E_\infty}(\mathcal{T}) \ar[r] & Sp,}\]
for $\mathcal{T}$ some $\infty$-category of spaces
and $Sp$ modelled by any model category
of spectra as seen in \cite{MMSS} or directly by stabilisation of
an $\infty$-category of spaces as in \cite[Section 1.4.3]{Lu2}.
Additionally by \cite[p. 624]{Lu2} the $\infty$-category
$Sp$ admits an essentially unique (i.e., parametrised by a contractible
Kan complex) symmetric monoidal structure characterised by \begin{itemize}
\item[1.] $Sp\times Sp\rightarrow Sp$ preserves colimits in each argument,
\item[2.] the unit is the sphere spectrum $\S$. \end{itemize}
Even more drastically, we have by \cite[Corollary 6.3.2.19]{Lu2} that each
stable presentable $\infty$-category $\C$ with a symmetric monoidal product,
which preserves colimits in each argument, admits a unique (up to 
contractible choice) symmetric monoidal functor $Sp\rightarrow \C$ (in particular
pointed $\S\rightarrow 1_\C$), which preserves small colimits.

Thus $Sp$ is not only equipped with a unique symmetric monoidal structure, but
it is initial among stable presentable $\infty$-categories with
a symmetric monoidal product preserving colimits. Hence $Sp$ is unique 
with these properties.

Returning to the Eilenberg-MacLane spectrum we find that the functors
\[\xymatrix{\mathrm{Mon}_{E_\infty}(sSet) \ar[r] & \mathrm{Grp}_{E_\infty}(sSet) 
\ar[r] & Sp,}\] are uniquely symmetric monoidal by \cite[Theorem 5.1]{GGN}, 
thus we can reduce multiplicative uniqueness to the claim \[\xymatrix{\SymMonCat 
\ar[r]^{(\cdot)^{iso}} &\SymMonCat \ar[r]^{N(\cdot)} & 
\mathrm{Mon}_{E_\infty}(sSet),}\] is uniquely symmetric monoidal. Passing
from a category to its maximal subgroupoid $(\cdot)^{iso}$ is 
strictly symmetric monoidal on the level of $1$-categories, and it is the unique 
such structure on $\infty$-categories by \cite[Corollary 5.5 (i)]{GGN} 
on the subdiagram for $\C = Cat_1$ the $\infty$-category of small $1$-categories 
with cartesian product as symmetric monoidal structure and $\D = Gpd$ the 
$\infty$-category of small $1$-groupoids with the same symmetric monoidal structure:
\[\xymatrix{Cat_1\ar[d]_{(\cdot)^{iso}}\ar[r] & \mathrm{Mon}_{E_\infty}Cat_1
\ar@{-->}[d]\\Gpd \ar[r] & \mathrm{Mon}_{E_\infty}Gpd.}\]
Here the corollary precisely says that the unique symmetric monoidal structure
given on $\mathrm{Mon}_{E_\infty}\C$ for a symmetric monoidal $\infty$-category
$\C$ allows a unique symmetric monoidal extension of the dashed arrow in the diagram.
(Compare however Warning 5.2 in \cite{GGN}: The $E_\infty$-monoids are defined
with respect to the cartesian product. Their monoidal 
product is the extension of a potentially different monoidal structure on $\C$.)
The identification $\mathrm{Mon}_{E_\infty}Cat_1 = \SymMonCat$ gives uniqueness
for $(\cdot)^{iso}$.

The same argument for $\C = Gpd$ and $\D=sSet$ yields the unique extension of the
nerve $N(\cdot)$ to a symmetric monoidal functor, because the nerve is strictly
product-preserving, hence extends uniquely to a symmetric monoidal functor on
the respective $E_\infty$-monoids. Since the Eilenberg-MacLane spectrum $H\C$ has
zero-level $H\C(\S^0)=N\C,$ we find that it is this unique extension.

Thus by these arguments we get the following classical (cf. \cite{MT1978}) theorem:
{\thm{There is a unique functor extending the classifying space construction
    $|N\cdot|\colon Cat_1\rightarrow Top$ to their $E_\infty$-monoids, thus
    inducing a commutative diagram, where the horizontal arrows are the
    respective free functors: \[\xymatrix{ Cat_1\ar[d]_{|N\cdot|}\ar[r]& 
    Mon_{E_\infty}(Cat_1)\ar@{=}[r]\ar[d]&\SymMonCat\\Top \ar[r]&Mon_{E_\infty}(Top)
    \ar@{=}[r]&Top^\Gamma.}\]}}

Note that the equality below is by definition, while the equality above is the
identification of symmetric monoidal $1$-categories with $\Gamma$-objects in $1$-categories,
which can be found originally in \cite{Th1,Th2}.

\section{Conjectural Multiplicative Uniqueness}
The theorems of \cite{GGN} establish the functor $H$ as the unique
functor from symmetric monoidal groupoids to connective spectra with the
group completion property as classically identified in \cite{MT1978}. 

Apart from the claim in \cite{May2009} on page 321 in a footnote ``I now have 
a sketch proof that looks convincing.'' I could not find a result immediately 
stating the fact that the constructions in \cite{EM,May2009,GGN} each produce the 
same $E_\infty$-structure on $H\R$. Unfortunately I could not prove 
this uniqueness, i.e., that the multifunctor-structure 
of \cite{EM} is essentially the only multifunctor-structure on $H(\cdot)$. I suspect
the essential uniqueness can analogously to \cite{BGT,BGT2013,GGN} be specified to
say the space of multifunctor-structures on $H(\cdot)$ is actually contractible.

The following elaboration on \cite[p. 19]{GGN} would identify ``my'' 
$E_\infty$-structure on $H\M$, as described in the previous chapter, 
with the $E_\infty$-structure on the Eilenberg-MacLane spectrum 
of the category of matrices with coefficients in the Eilenberg-MacLane
spectrum $H(\MM(H\R))$ given by using \cite{EM} twice.

In the sequence
\[\xymatrix{\SymMonCat \ar[r]^{(\cdot)^{iso}} &\SymMonCat 
\ar[r]^{N(\cdot)} & \mathrm{Mon}_{E_\infty}(\mathcal{T})
\ar[r] & \mathrm{Grp}_{E_\infty}(\mathcal{T}) \ar[r] & Sp,}\]
the first, third and fourth functor admit unique multifunctor-structures by \cite{GGN}. 

Thus the appropriate conjecture establishing multiplicative uniqueness of $H$ is:
{\conj{The passage from symmetric monoidal categories to $\Gamma$-spaces has a unique 
    multifunctor-structure.}}

I have tried two avenues, which can probably be made to work:
Given the symmetric monoidal structures on $\Gamma$-spaces, and the
unique symmetric monoidal structure on $\SymMonCat$ extending the product on $Cat$
asserted by \cite{GGN}, one should be able to make the identification 
of $Mon_{E_\infty}(N\cdot)\colon \SymMonCat\rightarrow Mon_{E_\infty}(sSet)$ 
as the tensor-unit in a symmetric monoidal sub-$\infty$-category of the functors 
$Fun(\SymMonCat,Mon_{E_\infty}(sSet))$, but I simply do not know how to approach
finding the appropriate subcategory systematically.

Furthermore each nerve $N$ is defined as a right adjoint.
Specifically choose a cosimplicial category. Then functors from this
category associate a simplical object to categories. Considering 
instead $Ex^2\circ N$ one can promote this to the right adjoint
in a Quillen equivalence of categories with the Thomason model 
structure to simplicial sets in the Kan model structure \cite{Th3}.

But the symmetric monoidality established in \cite{GGN} is canonical only for 
left-adjoint functors, thus the nerve is not easily a 
canonical extension. This might feel trivial, but since the canonical 
structures in \cite{GGN} are established by specific 
left-adjoint functors, their compatibility with 
right-adjoints is not straight-forward.

This is parallel to the transfer of model structures: Presentable $\infty$-categories 
are nerves of combinatorial simplicial model categories. In particular these 
are cofibrantly generated with generating (trivial) cofibrations satisfying 
compactness with respect to their category. The dual notion of 
``cosmall/cocompact'' turns out to be too trivial. For instance 
in $Sets$ the only cosmall objects are the empty set and any one-point-set. 
Thus most categories of the type ``sets with structure'' do not have many candidates for 
generating (trivial) fibrations, much less a model structure determined by them. 
On presentable categories we see this on objects: The opposite of a presentable
category is not presentable in general.

In conclusion: The first avenue is quite ambitious, since in particular it
yields a comparison of the $E_\infty$-structures produced by \cite{EM,May2009,GGN}.

The modest approach runs into a similar problem, 
i.e., a map in the ``wrong'' direction. In principle it 
should not be too hard to produce an $E_\infty$-map of symmetric 
spectra $H(\M)\rightarrow H(\MM(H\R))$ with the $E_\infty$-structures of chapter \ref{multbidel} 
and \cite{EM}. I modelled the delooping in chapter \ref{multbidel} explicitly 
as a careful generalisation of \cite{EM}. It would be 
conceptual to pass through strictification to $2$-categories:
The canonical monoidality map of strictification
is an inclusion $(\C\times\D)^{st}\rightarrow \C^{st}\times\D^{st}$ given by
including words of pairs to pairs of words of equal length. It
is an equivalence. Any inverse would need to be 
coherently associative and symmetric with respect to the product, which
is not possible without introducing an adequate 
tricategory-structure on permutative bicategories. 
This hinges on the fact that the canonical map is only pseudonatural, 
and the anchor equivalence $\C^{st}\rightarrow \C$ is only lax 
monoidal up to an invertible transformation. 

Consider instead strictification by Yoneda embedding into $Fun(\C^{op},Cat)$. 
This produces a similar problem in addition to several new ones. Canonically we get a
map in the wrong direction. One could try to resolve this by a Day-type
convolution, but this introduces new problems. What is a small
diagram over which to take the colimit? In addition the target 
$2$-category as well as the source tricategory make a variety
of lax colimits conceivable. Finally given a Day-type convolution, 
does it make the Yoneda-embedding an appropriate multifunctor?

Another problem, which I assume is much simpler to resolve, is the fact
that strictification cannot respect the functor-strictnesses of
addition and tensor product the way I axiomatised it for permutative and
symmetric monoidal bicategories. Thus even if one could pass to a 
bimonoidally equivalent $2$-category, the result would not have addition
and tensor product given by strict $2$-functors. Thus some clever 
construction of an equivalent $2$-category, for which these functors are
strict would yield a canonical map to the construction of \cite{EM}, but I did not 
find such a $2$-category. So here the essential question is how a 
bipermutative bicategory can strictify to a bipermutative $2$-category, 
thus a particular example of a simplicially enriched bipermutative 
$1$-category suited to the machine of \cite{EM}.

Finally, I want to state a guess, why these problems arise: 
$K$-theory constructions involve passing to categories of isomorphisms first. 
By \cite[Proposition 8.14]{GGN} there is a structural reason for this: 
Group completion adds objectwise monoidal inverses,
but consequently turns every morphism into an equivalence by an Eckmann-Hilton-type argument.
I have not found a good way to impose
invertibility of morphisms productively, but it has to enter in an essential way.

Furthermore this seems to conflict heavily with the principal example $\M$. There the equivalence
$1$-cells are automatically isomorphisms, since the $2$-cells are only products
of isomorphisms. For instance for $Ob\R = \mathbb{N}$ we only get permutation
matrices as equivalence $1$-cells, which only incorporates the endomorphism-objects 
$\R(0,0)$ and $\R(1,1)$ as $2$-cells. This ``delooping'' does not
satisfy the comparison of \cite{BDRR2011}, i.e., it does not deloop the appropriate
space in order to be equivalent to $K(H\R)$. I have no idea how to resolve this
conflict. I have to concede however
that the conflict might be my illusion, and \cite[Proposition 8.14]{GGN} only enforces
isomorphism $2$-cells in $\M,$ which is consistent with the assumption in
\cite{BDRR2011}. In particular this perspective makes the assumption in
\cite{BDRR2011} that each translation functor $X\oplus\_$ be a faithful functor
appear as an inessential peculiarity of the Grayson-Quillen-completion,
while discarding all non-invertible morphisms is essential to the construction.
